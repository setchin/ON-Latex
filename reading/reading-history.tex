\chapter{冰岛历史选读}
本章对冰岛的历史进行简要的介绍,主要包括冰岛的发现、冰岛的定居情况、基督教的传入以及一些著名的冰岛人的成就——格陵兰岛和“文兰”(Vinland)的发现等。

\section{移民书(Landnámabók)选读}
移民书(Landnámabók, Book of Settlement)是一部冰岛的谱系学著作,它最早由著名冰岛学者Ari Þorgilsson在12世纪早期编写,详细描述了挪威人在9世纪和10世纪挪威人在冰岛的定居情况。

移民书是关于冰岛早期历史的主要来源之一。它共分为5个部分,超过100多章,记录了1400多个定居点和400多个冰岛人的宗谱,包括定居者的名字、家庭、财产和土地所有权。移民书另还包括有关冰岛早期社会和文化的其他信息,例如早期定居者如何建立家庭、养育孩子、种植作物和养殖动物。移民书也涉及冰岛的法律和政治制度。

本章节选了移民书的前言和第一章的部分内容,主要介绍了冰岛的发现过程,部分内容有删改。
\begin{paracol}{2}
    Í \textit{aldarfarsbók} þeiri, er Beda prestr heilagr gerði\footnotemark, er getit eylands þess, er Týli heitir ok á bókum er sagt, at liggi sex dægra sigling í norðr frá Bretlandi. Þar sagði hann eigi koma dag á vetr ok eigi nótt á sumar, þá er dagr er sem lengstr. Til þess ætla vitrir menn þat haft, at Ísland sé Týli kallat, at þat er víða á landinu, er sól skínn um nætr, þá er dagr er sem lengstr, en þat er víða um daga, er sól sér eigi, þá er nótt er sem lengst. En Beda prestr andaðist sjau hundruð þrjátigi ok fimm árum eftir holdgan dróttins várs, at því er ritat er, ok meir en hundraði ára fyrr en Ísland byggðist af Norðmǫnnum.
    \footnotetext{可敬者贝德,又称圣贝德(672-735),英国编年史家及神学家。前文所述的书aldarfarsbók指的可能是其著作De temporum ratione (The Reckoning of Time)或De Temporibus (On Time), 两书中都涉及贝德对宇宙、世界的思考,且都提及过古代欧洲传说中的极北之岛图勒。}
    \switchcolumn

    In the book \textit{On the Condition of Time}, which the Venerable Priest Bede wrote, there is mention of an island called Thule, and it is said in books that it should lie six days' sailing to the north of Britain. He says that day never comes in winter, nor night comes in summer, when the day is longest. This is the reason why the wise men suppose that Thule must be Iceland, for there are many places where the sun shines at night there when the day is longest, but there are also many places where the sun isn't to be seen during the day, when the night is as longest. Bede the Priest breathed his last 735 years after the incarnation of our Lord, according to written sources, and more than 120 years before Iceland was settled by the Norse men.
    \switchcolumn*

    En áðr Ísland byggðist af Nóregi, váru þar þeir menn, er Norðmenn kalla Papa. Þeir váru menn kristnir, ok hyggja menn, at þeir hafi verit vestan um haf, því at fundust eftir þeim bækr írskar, bjǫllur ok baglar\index{bagall!baglar} ok enn fleiri hlutir, þeir er þat mátti skilja, at þeir váru Vestmenn. Enn er ok þess getit á bókum enskum, at í þann tíma var farit milli landanna.
    \switchcolumn

    But before Iceland was settled from Norway there were men there, which the Norwegians called papar. They were Christians and people believe that they must have come across the ocean from the west, because Irish books, bells, croziers, and lots of other things were found after them, which shall clarify the fact that they must have been Irish. And it is recorded in English books that there were sailings between these lands at the time.
\end{paracol}
\begin{translation*}{}
    可敬者贝德在他的著作《论时间》中提到了一个名为图勒的岛屿,据说,从不列颠岛向北航行六天就可以到达那里。贝德说在那里的寒冬终日不见白昼,而盛夏则终日不见黑夜。这就是为什么智者们认为图勒一定是冰岛,因为在冰岛的许多地方,白昼最长之时依旧可在夜晚见到太阳,而黑夜最长之时哪怕白天也无法看到阳光。有文字记载,可敬者贝德在我们的主耶稣降生后735年去世,那时距离挪威人在冰岛定居还有120多年。

    但是在冰岛被挪威人定居之前,那里也有人居住,挪威人称之为“巴帕尔”。这些人是基督徒,人们相信他们是从西方飘洋过海而来,因为在他们之后发现了爱尔兰的书籍、钟、牧杖以及许多其他东西,这些都证明了他们的爱尔兰血统。又据英国的书中记载,当时这些地区之间已有航行。
\end{translation*}
\begin{grammar*}{}
    \begin{enumerate}[leftmargin=*]
        \item er getit eylands þess

              geta接属格名词时除了表示“猜想”外,还有“提及,记录”之义。这里是后者的意思,且使用了被动语态。
        \item ætla vitrir menn þat haft
              haft在这里亦表示因果,即`consider it is the reason',修饰til þess.
    \end{enumerate}
\end{grammar*}
\begin{paracol}{2}
    Svá er sagt, at menn skyldu fara ór Nóregi til Færeyja. Nefna sumir til Naddoð víking. En þá rak vestr í haf ok fundu þar land mikit. Þeir gengu upp í Austfjǫrðum á fjall eitt hátt ok sást um víða, ef þeir sæi reyki eða nǫkkur líkendi til þess, at landit væri byggt, ok sá þeir þat ekki. Þeir fóru aftr um haustit til Færeyja. Ok er þeir sigldu af landinu, fell snær mikill á fjǫll, ok fyrir þat kǫlluðu þeir landit Snæland. Þeir lofuðu mjǫk landit. Þar heitir nú Reyðarfjall í Austfjǫrðum, er þeir hǫfðu at komit. Svá sagði Sæmundr prestr inn fróði.
    \switchcolumn

    It is said that some men wanted to sail from Norway to the Faroes. People mention that it was Naddod the Viking. But they drifted west into the ocean and found there a great land. They went ashore in the Eastfjords and went up to a high mountain, and looked around in all directions to see if they could find smoke or any other sign that the land was settled, but they saw nothing. Around autumn they went back to the Faroes. And when they were sailing away from the island, much snow fell on the mountains, so they called the land Snowland. They praised the land much. The place where they arrived in the Eastfjords is now called Reydar-fell, so said Saemund the Learned Priest.
\end{paracol}
\begin{translation*}{}
    据说曾有人想从挪威出发前去法罗群岛。其中一个被人称作“维京”那多德。可他们被洋流卷向西边,发现了一片广袤的土地。他们从东部峡湾上岸,攀上一座高山,远眺四周,想寻找炊烟或者其他任何能证明这片土地有人居住的迹象,但是他们什么也没看到。到了秋天,他们返回法罗群岛。就在他们离开时,鹅毛大雪覆盖了山峦,于是他们称其为“雪国”,并对这片土地赞不绝口。据牧师赛蒙德所说,他们在东峡湾登陆的地方就是现在的雷扎山。
\end{translation*}
\begin{grammar*}{}
    \begin{enumerate}[leftmargin=*]
        \item nefna sumir til Naddoð víking
              这句话中的Naddoð víking不是til的宾语, 而是nefna的宾语, nefna e-t til e-s `name sth. as sth.'. til后省略了`one of them'.
        \item rak vestr
              reka `drive'的无人称用法,`to be drifted, tossed'.
    \end{enumerate}
\end{grammar*}
\begin{paracol}{2}
    Maðr hét Garðarr Svavarsson, sænskr at ætt. Hann fór at leita Snælands at tilvísun móður sinnar framsýnnar. Hann kom at landi fyrir austan Horn it eystra. Þar var þá hǫfn. Garðarr sigldi umhverfis landit ok vissi, at þat var eyland. Hann var um vetr einn norðr í Húsavík á Skjálfanda\footnote{在今冰岛北部。} ok gerði þar hús.

    \switchcolumn

    A man named Gardar Svavarsson, Swedish by birth, went out in search of Snowland guided by his mother, who was a prophet. He made land east of Eastern Horn where there was a harbor then. Gardar sailed round the land and knew that it was an island. He stayed there over winter in the north of Husa-bay on Skjalfand and built a house there.

    \switchcolumn*

    Um várit, er hann var búinn til hafs, sleit frá honum mann á báti, er hét Náttfari, ok þræl ok ambátt. Hann byggði þar síðan, er heitir Náttfaravík\footnote{同在斯卡尔法德峡湾, 胡萨湾在东岸,纳特法利湾在其正西岸,二者隔海相望。}. Garðarr fór þá til Nóregs ok lofaði mjǫk landit. Eftir þat var landit kallat Garðarshólmr, ok var þá skógr milli fjalls ok fjǫru.

    \switchcolumn

    In the spring when he was ready for sailing, a boat drifted away from him with a man called Nattfari aboard, and a slave and a bondswoman. Nattfari settled down there at a place called Nattfari-bay. Gardar sailed back to Norway and praised the land much. After that the land was called Gardar's Isle and there were woods between the mountains and the shores.
\end{paracol}
\begin{translation*}{}
    有一个名叫加达尔·斯瓦瓦尔松的瑞典人,在他母亲的建议下出发寻找雪国,她的母亲是一位先知。加达尔在岛的东角岬以东的港口登陆。他绕着这片土地航行一周,于是知道这是一个岛屿。他在斯卡尔法德峡湾的胡萨湾北部度过了一个冬天,并在那里修建了房子。

    第二年开春时加达尔准备出海,此时一只船从他的船队里漂走了。船上载有一个男人名叫纳特法利,以及一男一女两个奴隶。纳特法利后来在一个地方定居,那里后来就叫“纳特法利湾”。加达尔返回挪威,对这片土地赞不绝口。此后这片土地就被称为“加达尔岛”,山与海之间有着茂密的丛林。
\end{translation*}
\begin{paracol}{2}
    Flóki Vilgerðarson hét maðr. Hann var víkingr mikill. Hann fór at leita Garðarshólms. Með Flóka var á skipi bóndi sá, er Þórólfr hét, annarr Herjólfr. Faxi hét suðreyskr\footnote{Suðreyjar, `Southern isles', 是维京人对赫布里底群岛的称呼,位于苏格兰西部的大西洋中。} maðr, er þar var á skipi.

    \switchcolumn

    There was a man called Floki Vilgerdarson. He was a great viking. He went to search for Gardar's Isle. With Floki on board was a farmer called Thorolf, another called Herjolf, and also a Hebridean called Faxi.

    \switchcolumn*

    Af Færeyja sigldi hann út í haf með hrafna þá þrjá, er hann hafði blótat í Nóregi. Ok er hann lét lausan inn fyrsta, fló sá aftr um stafn, annarr fló í loft upp ok aftr til skips, inn þriði fló fram um stafn í þá átt, sem þeir fundu landit. Þeir kómu austan at Horni ok sigldu fyrir sunnan landit. En er þeir sigldu vestr um Reykjanes\footnote{雷克雅内斯岬,位于冰岛西南部,雷克雅内斯半岛的末端。nes在古诺尔斯语中就是岬之意。} ok upp lauk firðinum, svá at þeir sá Snæfellsnes\footnote{字面义“雪山岬”,在雷克雅内斯岬以北,海的对岸处。}, þá ræddi Faxi um: "Þetta mun vera mikit land, er vér hǫfum fundit. Hér eru vatnfǫll stór". Síðan er þat kallaðr Faxaóss\footnote{今称Faxaflói, 字面义`Faxi Bay'. 指位于斯奈弗尔斯和雷克雅内斯两个半岛间的海域。法克西找到的地方实际上不是河口,而是一个大港湾。}.

    \switchcolumn

    From Faroe he set out with three ravens which he had consecrated in Norway. When he set the first one free it flew back over the stem, but the second raven flew into the air, and then back to the ship, while the third flew ahead over the stem, and it was in that direction that they found land. They came to the Horn from the east, and sailed along the south coast. As they sailed west around Reykjanes and the fjord opened up wide so they could see Snaefellsnes. Then spoke Faxi: `It must be a big land we've found. Here are big rivers'. After this (the river's mouth) was called Faxi's Mouth.

\end{paracol}

\begin{translation*}{}
    有一个名叫弗洛基·维尔格达尔松的人,是一个伟大航海家。他出发寻找加达尔岛。弗洛基的船上还有一个名叫索罗尔夫的农民,另一个名叫赫约尔夫的人,以及一位名叫法克西的赫布里底人。

    弗洛基从法罗群岛出发,随船携带了三只乌鸦,这些乌鸦是他在挪威祭祀祝福过的。当他放走第一只乌鸦时,它飞回了船头;第二只则飞到空中,然后又飞回了船上;第三只乌鸦径直从船头上向前飞去,他们就沿着这个方向找到了陆地。他们从东边穿过角岬,然后沿着南岸航行。当他们向西驶过雷克雅内斯岬,海湾一下开阔起来,斯奈弗尔斯内斯岬赫然在目。法克西于是说道:“我们一定发现了一片广袤的土地。这里有很宽阔的河流。”此后这个河口就被称为“法克西河口”。
\end{translation*}


\begin{paracol}{2}
    Þeir Flóki sigldu vestr yfir Breiðafjǫrð\footnote{冰岛西部大型浅水湾,在法克西湾以北不远处。} ok tóku þar land, sem heitir Vatnsfjǫrðr við Barðastrǫnd. Þá var fjǫrðrinn fullr af veiðiskap, ok gáðu þeir eigi fyrir veiðum at fá heyjanna, ok dó allt kvikfé þeira um vetrinn. Vár var heldr kalt. Þá gekk Flóki upp á fjall eitt hátt ok sá norðr yfir fjǫllin fjǫrð fullan af hafísum. Því kǫlluðu þeir landit Ísland, sem þat hefir síðan heitit.
    \switchcolumn

    Floki and his men sailed west over Breida-fjord and made land at the place now called Vatns-fjord, against Barda-strand. At that time the fjord was full of fish, and they paid no heed to making hay because of fishing, so all their livestock died in the winter. The following spring was rather cold. Then Floki went up to a high mountain and discovered northwards over the mountain, a fjord full of drift ice. Therefore they called the island Iceland, and it's been called ever since.

    \switchcolumn*

    Þeir Flóki ætluðu brott um sumarit ok urðu búnir lítlu fyrir vetr. Þeim beit eigi fyrir Reykjanes, ok sleit frá þeim bátinn ok þar á Herjólf. Hann tók þar, sem nú heitir Herjólfshǫfn. Flóki var um vetrinn í Borgarfirði\footnote{法克西湾中部峡湾,位于雷卡雅内斯岬以北。} ok fundu þeir Herjólf. Þeir sigldu um sumarit eftir til Nóregs.
    Ok er menn spurðu af landinu, þá lét Flóki illa yfir, en Herjólfr sagði kost ok lǫst af landinu, en Þórólfr kvað drjúpa smjǫr af hverju strái á landinu, því er þeir hǫfðu fundit. Því var hann kallaðr Þórólfr smjǫr.

    \switchcolumn

    Floki and his men intended to sail away in summer, but they were only ready shortly before winter. Their ship could not cruise around Reykjanes, and a boat with Herjolf on board broke loose from them. He made land at a place now called Herjolfs-haven. Floki stayed the winter in Borgar-fjorda and they found Herjolf. They sailed to Norway the next summer. And when men asked about the land, Floki spoke ill of it, but Herjolf told the good and the bad of the land, but Thorolf said that butter dropped from every blade of grass in the land which they had discovered. That's why he was called Thorolf Butter.
\end{paracol}

\begin{translation*}{}
    弗洛基一行向西驶过布雷达湾,在一个叫瓦滕斯峡湾的地方登陆。瓦滕斯峡湾紧挨着巴尔达海滩。当时峡湾里到处都是鱼,他们忙着捕鱼,没有准备好干草,结果所有的牲畜都在冬天都饿死了。接下来的春天也相当寒冷。弗洛基登上一座高山,向北眺望,发现了一片充满浮冰的峡湾。因此他们把这个岛叫做“冰岛”,这个名字流传至今。

    弗洛基和手下准备在夏天离开,可他们直到冬天才做好准备。结果,他们的船无法绕过雷克雅内斯岬。一艘载着赫约尔夫的小船漂走了,他最后成功上岸,他登陆的地方后来叫作赫约尔夫港。弗洛基在博尔加峡湾过冬并找到了赫约尔夫。第二年夏天他们返回了挪威。当人们问起这片土地的情况时,弗洛基尽说了一些坏话,但是赫约尔夫则公正地评价了其利弊,而索罗尔夫说他们发现的土地上每一根草都滴着油珠。因此他被人唤作为“黄油”索罗尔夫。
\end{translation*}

\begin{paracol}{2}
    Bjǫrnólfr hét maðr, en annarr Hróaldr. Þeir váru synir Hrómundar Gripssonar\footnote{传说中的英雄人物,其故事见于萨迦Hrómundar saga Gripssonar, 但其最早的版本今已散佚。}. Þeir fóru af Þelamǫrk\footnote{挪威南部城镇。} fyrir víga sakir ok staðfestust í Dalsfirði á Fjǫlum\footnote{挪威西部的市镇名。达尔峡湾位于其治内。}. Sonr Bjǫrnólfs var ǫrn, faðir þeira Ingólfs ok Helgu, en Hróalds sonr var Hróðmarr, faðir Leifs.
    \switchcolumn

    There was a man called Bjornolf, and another called Hroald. They were sons of Hromund Gripsson. They left Telemark because of manslaughter and took up their abode at Dalsfjord in Fjalar. The son of Bjornolf was Orn, the father of Ingolf and Helga. And the son of Hroald was called Hrodmar, father of Leif.
\end{paracol}

\begin{paracol}{2}
    En þeir Ingólfr ok Leifr fóstbræðr\footnote{Foster Brother除了指养兄弟外,也常指互相结拜的义兄弟。义兄弟用刀刺破手臂使血相融,定为誓约,一人死后另一人会为其复仇。这种兄弟关系在古代北欧十分常见。} bjuggu skip mikit, er þeir áttu, ok fóru at leita lands þess, er Hrafna-Flóki hafði fundit ok þá var Ísland kallat. Þeir fundu landit ok váru í Austfjǫrðum í Álftafirði\footnote{álft的字面意思是“天鹅”,至少有三个冰岛峡湾叫作这个名字。这里因戈尔夫抵达的峡湾在冰岛东面,今Djúpivogur附近。} inum syðra. Þeim virðist landit betra suðr en norðr. Þeir váru einn vetr á landinu ok fóru þá aftr til Nóregs.
    \switchcolumn

    And the blood-brothers Ingolf and Leif prepared a large ship that they possessed and set out to search the land Raven-Floki had discovered, which by that time was called Iceland. They found the land and stayed the first winter at South-Alftafjord in the Eastfjords. The land seemed to them to be better in the south than in the north. They spent one winter in the land and then went back to Norway.
\end{paracol}

\begin{translation*}{}
    又说有一个名叫比约尔诺尔夫的人,还有一个叫赫罗尔德。他们都是是赫罗蒙德·格里普松的儿子。两人因为杀人而离开泰勒马克,定居在菲亚拉尔的达尔峡湾。比约尔诺尔夫的儿子名叫沃尔恩,他生了两个孩子因戈尔夫和赫尔加;赫罗尔德的儿子叫赫罗德马尔,他的儿子名叫莱夫。

    因戈尔夫和莱夫是结义兄弟,他们准备了一艘大船,出发寻找“乌鸦”弗洛基发现的土地。那时候它已经被叫作冰岛。他们找到了冰岛,并在东峡湾的南阿尔夫塔峡湾过了第一个冬天。二人觉得冰岛的南边比北边更好。他们在这片土地上度过了一个冬天,之后便返回了挪威。
\end{translation*}


\begin{paracol}{2}
    Eftir þat varði\index{verja!varði} Ingólfr fé þeira til Íslandsferðar, en Leifr fór í hernað í vestrvíking. Hann herjaði á Írland ok fann þar jarðhús mikit. Þar gekk hann í, ok var myrkt, þar til er lýsti af sverði því, er maðr helt á. Leifr drap þann mann ok tók sverðit ok mikit fé af honum. Síðan var hann kallaðr Hjǫrleifr.
    \switchcolumn

    Later, Ingolf spent their money on an expedition to Iceland, but Leif went on a viking raid to the west. He harried Ireland and found there a large underground chamber. He went inside and it was dark until a light shone from a sword which a man was holding on. Leif killed the man and took the sword and a lot of money from him. Thereafter he was called Hjorleif (Sword-Leif).

    \switchcolumn*

    Hjǫrleifr herjaði víða um Írland ok fekk þar mikit fé. Þar tók hann þræla tíu, er svá hétu: Dufþakr ok Geirrǫðr, Skjaldbjǫrn, Halldórr ok Drafdritr. Eigi eru nefndir fleiri. En eftir þat fór Hjǫrleifr til Nóregs ok fann þar Ingólf, fóstbróður sinn. Hann hafði áðr fengit Helgu Arnardóttur, systur Ingólfs.

    \switchcolumn

    Hjorleif harried wide on Ireland and took much treasure. There he took ten slaves, who are called Dufthak, Geirraud, Skjaldbjorn, Halldor, Drafdrit, more are not named. After that Hjorleif went back to Norway and found Ingolf, his blood-brother. Earlier he had married Helga, Orn's daughter, Ingolf's sister.
\end{paracol}
\begin{translation*}{}
    之后,因戈尔夫投入了一大笔资金前往冰岛,而莱夫则跟随海盗们向西远航,在爱尔兰大肆抢劫。一次,他发现了一个巨大的地下暗室,于是探身进去,里面伸手不见五指。突然一把剑的闪光照在他的脸上,原来是有人持剑在里面守卫。莱夫杀死了那个人,夺走了他的剑和一大笔钱。从此他被称为赫约尔莱夫,意思是“利剑”莱夫。

    赫约尔莱夫将爱尔兰洗劫一空,并且抓走了十个奴隶,他们名叫杜夫萨克、盖尔罗斯、斯卡尔比约恩、哈尔多尔、德拉夫德里特,其他的就不一一列举了。之后,赫约尔莱夫返回挪威,找到了他的兄弟因戈尔夫。此前,莱夫与赫尔加结为夫妻,赫尔加是沃尔恩的女儿,也就是因戈尔夫的妹妹。
\end{translation*}
\begin{grammar*}{}
    \begin{enumerate}[leftmargin=*]
        \item varði Ingólfr fé þeira
              这里的varði是verja而非varða的过去式。verja除了有“防御”的意思之外,还有“包裹”的意思(来自另一个词源),故有引申义“投入(资金)”。
    \end{enumerate}
\end{grammar*}


\begin{paracol}{2}
    Þenna vetr fekk Ingólfr at blóti miklu ok leitaði sér heilla um forlǫg sín, en Hjǫrleifr vildi aldri blóta. Fréttin vísaði Ingólfi til Íslands. Eftir þat bjó sitt skip hvárr þeira mága til Íslandsferðar. Hafði Hjǫrleifr herfang sitt á skipi, en Ingólfr félagsfé þeira, ok lǫgðu til hafs, er þeir váru búnir.

    \switchcolumn

    That winter Ingolf made a great sacrifice to consult the oracles about his destiny, but Hjorleif would never sacrifice to the gods. The oracle directed Ingolf to go to Iceland. After that, each of these brothers-in-law prepared his ship for the expedition to Iceland. Hjorleif had his war-booty on the ship, and Ingolf carried what they held in common, and put out to sea when they were ready.

    \switchcolumn*

    Sumar þat, er þeir Ingólfr fóru til at byggja Ísland, hafði Haraldr hárfagri verit tólf ár konungr at Nóregi. Þá var líðit frá upphafi þessa heims sex þúsundir vetra ok sjau tigir ok þrír vetr, en frá holdgan dróttins átta hundruð ok sjau tigir ok fjǫgur ár.

    \switchcolumn

    The summer when Ingolf and Hjorleif went to settle in Iceland, Harald the Fairhair had been for twelve years King of Norway; there had elapsed from the beginning of the world 6073 years, and from the Incarnation of our Lord 874 years.

\end{paracol}

\begin{translation*}{}
    当年冬天,因戈尔夫大举祭祀,以求从神谕中获知自己的命运,而莱夫则从不愿祭祀神灵。神启示因戈尔夫前往冰岛。于是,兄弟二人各自准备了船只,准备前往冰岛。莱夫在船上载满了他自己的战利品,而因戈尔夫则带了他们共同的财产。一切准备就绪,二人驶向大海。

    因戈尔夫和赫约尔莱夫准备前往冰岛时,“美发王”哈拉尔德已经统治挪威十二年了。那是创世后的6073年,也即耶稣基督降生后的874年。
\end{translation*}
\begin{grammar*}{}
    \begin{enumerate}[leftmargin=*]
        \item leitaði sér heilla um forlǫg sín
              leita的反身式leitask或者leita sér与本身的含义区别不大,leita sér um可以翻译为`explore'. heilla是heill `good luck'的复数属格,此时它尤指从吉兆、神谕中获得的礼物。leita heilla是常见的说法,表示从占卜祭祀中获得启示。
    \end{enumerate}
\end{grammar*}

\begin{paracol}{2}
    Þeir hǫfðu samflot, þar til er þeir sá Ísland. Þá skilði með þeim. Þá er Ingólfr sá Ísland, skaut hann fyrir borð ǫndvegissúlum sínum til heilla. Hann mælti svá fyrir, at hann skyldi þar byggja, er súlurnar kæmi á land. Ingólfr tók þar land, er nú heitir Ingólfshǫfði\footnote{冰岛南部的一块小岬。}, en Hjǫrleif rak vestr fyrir land, ok fekk hann vatnfátt. Þá tóku þrælarnir írsku þat ráð at knoða saman mjǫl ok smjǫr ok kǫlluðu þat óþorstlátt. Þeir nefndu þat minþak\footnote{古爱尔兰语中min是面粉的意思。}. En er þat var tilbúit, kom regn mikit, ok tóku þeir þá vatn á tjǫldum. En er minþakit tók at mygla, kǫstuðu þeir því fyrir borð, ok rak þat á land, þar sem nú heitir Minþakseyrr.

    \switchcolumn

    They sailed together until they saw Iceland, and then they got separated. When Ingolf saw the land, he threw overboard his highseat pillars for an omen. He said that he would settle where the pillars made land. He came ashore where is now called Ingolfs-headland. But Hjorleif drifted west along the coast and ran short of water. Then the Irish slaves made a plan to knead together flour and butter and said it would not give thirst. They called it minthak. But when it was made ready, it started raining heavily, and they collected water on tents. When the minthak began to mould they threw it overboard and it drifted ashore at a place now called Minthakseyr.
\end{paracol}
\begin{translation*}{}
    兄弟二人一起航行,但他们看到冰岛后就分道扬镳了。因戈尔夫将他高椅的支柱扔进海里,以求神示。他决定在支柱登陆的地方定居。他上岸的地方后来被称为“因戈尔夫岬”。而赫约尔莱夫则沿着海岸向西漂流,逐渐喝光了淡水。于是,爱尔兰奴隶们想出了一个办法,他们将面粉和黄油揉在一起,说这种食物不会让人口渴。他们给它起了个名字叫作“闵撒克”。但是,这种食物刚刚做好,天就下起了大雨,他们得以用帐篷接水。闵撒克后来开始长出霉斑,他们就把它扔进了海里。它漂到了岸边,后来这个地方就被称为“闵撒克堤”。
\end{translation*}

\begin{paracol}{2}
    Hjǫrleifr tók land við Hjǫrleifshǫfða\footnote{冰岛南部,在因戈尔夫岬以西约100公里处。}, ok var þar þá fjǫrðr, ok horfði botninn inn at hǫfðanum. Hjǫrleifr lét þar gera skála tvá, ok er ǫnnur tóftin átján faðma\footnote{两臂之长,约合1.8米。}, en ǫnnur nítján. Hjǫrleifr sat þar um vetrinn.

    \switchcolumn

    Hjorleif made land at Hjorleifs-headland, where there was a fjord and its head stretched to the headland. Hjorleif had two houses built there, and one toft was 18 fathoms long and the other 19 fathoms. Hjorleif spent the winter there.

    \switchcolumn*

    En um várit vildi hann sá. Hann átti einn uxa, ok lét hann þrælana draga arðrinn. En er þeir Hjǫrleifr váru at skála, þá gerði Dufþakr þat ráð, at þeir skyldu drepa uxann ok segja, at skógarbjǫrn hefði drepit, en síðan skyldu þeir ráða á þá Hjǫrleif, ef þeir leitaði bjarnarins. Eftir þat sǫgðu þeir Hjǫrleifi þetta. Ok er þeir fóru at leita bjarnarins ok dreifðust í skóginn, þá settu þrælarnir at sérhverjum þeira ok myrðu þá alla jafnmarga sér. Þeir hljópu á brott með konur þeira ok lausafé ok bátinn. Þrælarnir fóru í eyjar þær, er þeir sá í haf til útsuðrs, ok bjuggust þar fyrir um hríð.

    \switchcolumn

    In the spring he wanted to sow. He had one ox and let his slaves to draw the plow. When Hjorleif and his men were at the house, Dufthak gave the advice that they should kill the ox and say a wood-bear had killed it, then they should set upon Hjorleif, if they went to seek the bear. So they told Hjorleif the story, and when they went out to search the bear and spread out in the woods, the slaves set upon every one of them, and murdered them all, as many men as they were themselves. Then they ran away with their women, chattels and the boat. The slaves went to the islands, which they saw at the sea towards the southwest, and prepared themselves to settle for a while.
\end{paracol}
\begin{translation*}{}
    赫约尔莱夫在赫约尔莱夫岬登陆,那里有一个峡湾朝着岬角的方向延申。赫约尔莱夫在那里建造了两座房子,其中一座有18英寻长,另一座有19英寻长。赫约尔莱夫在那里过了一个冬天。

    第二年春天,赫约尔莱夫决定开垦土地。但他只有一头耕牛,于是就让他的奴隶们来拉犁。一天,趁着赫约尔莱夫和他的手下们都在房子里时,杜夫萨克建议奴隶们把牛杀死,然后谎称是有一头棕熊咬死了它。这样,要是赫约尔莱夫去找熊的话,他们就有机会对他们下手。于是,他们就把这件事报告给了赫约尔莱夫。当赫约尔莱夫和手下在森林里四散开来寻找那头熊时,奴隶们一一将他们杀害,杀死的人数和他们自己的人数一样多。然后,他们带着赫约尔莱夫的妻女、财物和船只逃往先前在海上看到的西南方的岛屿,并在那里安顿了一段时间。
\end{translation*}

\begin{grammar*}{}
    \begin{enumerate}[leftmargin=*]
        \item skyldu þeir ráða á þá Hjǫrleif

              ráða接与格时有“承担某事”的含义,ráða á e-n/at e-m于是有“袭击某人”的意思,和英语set upon相似。
    \end{enumerate}
\end{grammar*}

\begin{paracol}{2}
    Vífill ok Karli hétu þrælar Ingólfs. Þá sendi hann vestr með sjó at leita ǫndvegissúlna sinna. En er þeir kómu til Hjǫrleifshǫfða, fundu þeir Hjǫrleif dauðan. Þá fóru þeir aftr ok sǫgðu Ingólfi þau tíðendi. Hann lét illa yfir drápi þeira Hjǫrleifs. Eftir þat fór Ingólfr vestr til Hjǫrleifshǫfða, ok er hann sá Hjǫrleif dauðan, mælti hann: "Lítit lagðist hér fyrir góðan dreng, er þrælar skyldu at bana verða, ok sé ek svá hverjum verða, ef eigi vill blóta."
    \switchcolumn

    Vifil and Karli were the names of the slaves of Ingolf. He sent them west along the shore to look for his highseat pillars. When they came to Hjorleifs-headland, they found Hjorleif dead. Then they turned back to tell Ingolf the tidings. He suffered badly by the death of Hjorleif and his men. After that, he set out west to Hjorleifs-headland, and when he saw Hjorleif dead, he said, "It's a shameful end for a warrior, that slaves should put him to death; but I see it happens to people who won't make sacrifices."

    \switchcolumn*

    Ingólfr lét búa grǫf þeira Hjǫrleifs ok sjá fyrir skipi þeira ok fjárhlut. Ingólfr gekk þá upp á hǫfðann ok sá eyjar liggja í útsuðr til hafs. Kom honum þat í hug, at þeir mundu þangat hlaupit hafa, því at bátrinn var horfinn. Fóru þeir at leita þrælanna ok fundu þá þar, sem Eið\footnote{字面义是地峡的意思。} heitir í eyjunum. Váru þeir þá at mat, er þeir Ingólfr kómu at þeim. Þeir urðu felmtsfullir, ok hljóp sinn veg hverr. Ingólfr drap þá alla. Þar heitir Dufþaksskor\footnote{位于韦斯特曼纳埃亚尔的一座山崖,因山脊上有一条凹痕,故得名。}, er hann lézt. Fleiri hljópu þeir fyrir berg, þar sem við þá er kennt síðan. Vestmannaeyjar heita þar síðan, er þrælarnir váru drepnir, því at þeir váru Vestmenn.

    \switchcolumn

    Ingolf had Hjorleif and his men laid in grave and took over his ship and share of money. Then he went up to the top of the headland and saw some islands lying to the southwest in the sea. It occurred to him that the slaves might have escaped there since the boat was missing. They set out to search the slaves and found them at a place called Eid in the island. They were at meat when Ingolf and his men came to them. They were frightened they fled in all directions. Ingolf killed them all. The place is called Dufthaks-score, where he was killed. Many jumped over a cliff which has been named after them ever since. The place was called Westman Islands where the slaves were killed, because they were from the west.
\end{paracol}
\begin{translation*}{}
    因戈尔夫有两个奴隶,名为维菲尔和卡利。他派这两人沿着海岸向西寻找他高椅的支柱。当他们来到赫约尔莱夫岬时,却发现赫约尔莱夫已经死了。于是,他们立刻回去向因戈尔夫报告了这一消息。因戈尔夫对赫约尔莱夫一行的死感到万分悲痛。接着他前往赫约尔莱夫岬,看到他兄弟的尸体后说道:“一个勇士竟被奴隶杀死,何其可悲啊!但是依我看,这种事情就是会发生在不愿祭祀的人身上。”

    因戈尔夫将死者埋葬,又掌管了他的船和资财。他爬上岬角的高山,远远望见西南边海上的岛屿。联想到赫约尔莱夫的船已经消失不见,因戈尔夫猜想奴隶们或许已经逃到岛上去了。于是,他们出发去找奴隶们报仇,并在岛上一个叫作“艾德”的地方发现了他们。因戈尔夫和手下们杀到时,奴隶们正在吃饭,他们惊慌失措,四散逃跑。因戈尔夫将他们赶尽杀绝。杜夫萨克被杀的地方后来被称为“杜夫萨克之痕”。其他很多奴隶们则从悬崖上跳下,于是这座悬崖也以他们命名。这群奴隶被杀的地方叫作“韦斯特曼纳埃亚尔”,意思是“西方人之岛”,因为这些奴隶来自西方的爱尔兰。
\end{translation*}
\begin{grammar*}{}
    \begin{enumerate}[leftmargin=*]
        \item Lítit lagðist hér fyrir góðan dreng

              leggja的反身式有`set out, go'的意思。lítið leggsk fyrir e-n是一个固定短语,最初的意思是说某人做了很小的抵抗/行动等等。后用于形容被杀的人死得耻辱,`to be easily slain, to come to a shameful end'.
        \item sjá fyrir skipi þeira
              sjá fyrir e-u, 固定短语,意思是“管理”.
        \item lézt
              láta的反身式,意思是“死”.
        \item sem við þá er kennt
              kenna除了基本义“知道”之外,也有“命名,称呼”的意思,并常用短语kenna e-t við e-n `call sth. after sb.' 这里við后的宾语“奴隶”被省略了。
    \end{enumerate}
\end{grammar*}

\begin{paracol}{2}
    Þeir Ingólfr hǫfðu með sér konur þeira, er myrðir hǫfðu verit. Fóru þeir þá aftr til Hjǫrleifshǫfða. Var Ingólfr þar vetr annan, en um sumarit eftir fór hann vestr með sjó. Hann var inn þriðja vetr undir Ingólfsfelli fyrir vestan ǫlfusá. Þau missari fundu þeir Vífill ok Karli ǫndvegissúlur hans við Arnarhvál\footnote{今雷克雅未克附近,提到的沼泽地则是Mosfellsheiði.} fyrir neðan heiði.

    \switchcolumn

    Ingolf and his men took the widows of the men who had been murdered. They went back to Hjorleifs-headland. Ingolf stayed there for another winter. Next summer he sailed westwards along the coast. He passed the third winter at Ingolfs-mount, west of Olfus-river. At that time, Vifil and Karli found his highseat pillars at Arnar-hill, beneath the heath.

    \switchcolumn*

    Ingólfr fór um várit ofan um heiði. Hann tók sér bústað þar, sem ǫndvegissúlur hans hǫfðu á land komit. Hann bjó í Reykjarvík. Þar eru enn ǫndugissúlur þær í eldhúsi\footnote{字面义`Fire house',指的就是房子的正厅,古时在此处生火保暖。}. En Ingólfr nam land milli ǫlfusár ok Hvalfjarðar fyrir útan Brynjudalsá, milli ok ǫxarár, ok ǫll nes út.

    \switchcolumn

    Ingolf traveled down over the moor in the spring. He took up his abode where his highseat pillars had come ashore. He lived in Reykjavik. His highseat pillars are still there in the hall. Ingolf took the vast land between Olfus River and Hvalfjord, south of Brynja-dale, and Oxar Rivers, and all the nesses.

    \switchcolumn*

    Ingólfr var frægastr allra landnámsmanna, því at hann kom hér at óbyggðu landi ok byggði fyrstr landit. Gerðu þat aðrir landnámsmenn eftir hans dæmum.

    \switchcolumn

    Ingolf was the most famous of all the settlers, because he came to the uninhabited land and settled first here. Other settlers came and followed his example.
\end{paracol}
\begin{translation*}{}
    因戈尔夫带上死者的遗孀,回到了赫约尔莱夫岬,在那里度过了第二个冬天。第二年夏天,他沿着海岸向西航行。第三个冬天,他在奥尔弗斯河以西的地方过冬,那里后来被称作“因戈尔夫山”。那年,维菲尔和卡利在阿尔纳丘下发现了因戈尔夫的高椅支柱,阿尔纳丘在一片沼泽地的下方。

    第二年春天,因戈尔夫穿过荒原,在高椅支柱冲上岸的地方驻扎下来。那年他住在雷克雅未克,他高椅的支柱至今还留在那里的厅堂里。因戈尔夫占据了大片土地,东至奥弗斯河,西至布林雅峡谷以南的赫瓦尔峡湾,北达到奥克萨河,并包括了那一带所有的岬角。

    因戈尔夫是所有移民者中最出名的,因为他来到了这片荒无人烟的土地,并第一个在这里定居。其他移民者都效仿他的做法。
\end{translation*}

\section{冰岛人之书(Íslendingabók)选读}
冰岛人之书是由冰岛学者阿里·索吉尔松(Ari Þorgilsson)在12世纪早期编写的一部关于冰岛历史的书籍,它是关于冰岛早期历史的主要来源之一。冰岛人之书主要记载了冰岛的定居、基督教的传入以及一些著名的冰岛主教的情况。冰岛人之书正文共分为10章,第1章介绍冰岛的发现,第2章介绍冰岛法律的由来,第3章介绍冰岛议会的诞生,第4章介绍历法,第5章介绍区划,第6章介绍格林兰的发现,第7章介绍基督教的传入,第8-10章介绍冰岛的主教。在附录中,冰岛人之书介绍了这些主教的后裔以及伊林格\footnote{瑞典的一支半神话性质的王室族裔,号称是神弗雷(Freyr)的后代。他们中一些人的名字在古英语史诗《贝奥武夫》中也有记载,故在一定程度上可能是真实存在的。伊灵格人起初在瑞典活动,后来也前往挪威,其故事记载于伊林格萨迦。}人和布雷扎湾人\footnote{冰岛西部的一个大型浅水海湾。}的家谱。

本章节选了冰岛人之书的第1-3以及第7章的内容,主要介绍了早期冰岛法律、宗教等内容。部分章节有删节。
\begin{paracol}{2}
    \begin{center}
        \textsc{Frá Íslands byggð}
    \end{center}

    \switchcolumn

    \begin{center}
        \textsc{On the settlement of Iceland}
    \end{center}

    \switchcolumn*

    Ísland byggðist fyrst ór Norvegi á dǫgum Haralds ins hárfagra\footnote{Haraldr inn hárfagri (Harald the Fairhair), 挪威的统一者,第一任挪威国王,公元872年至930年在位。关于其的可靠记载不多,见于几部王室萨迦。}, Hálfdanarsonar ins svarta\footnote{Hálfdanr inn svarti (Halfdan the Black), 哈拉尔德之父。}, í þann tíð, at ætlun ok tǫlu þeira Teits, fóstra míns, þess manns, er ek kunna spakastan, sonar Ísleifs byskups, ok Þorkels, fǫðurbróður míns, Gellissonar, er langt munði fram, ok Þuríðar Snorradóttur goða, er bæði var margspǫk ok óljúgfróð, er Ívarr\footnote{Ívarr Ragnarssonr, 又称Ívarr inn Beinlausi (Ivarr the Boneless), “维京大军”的首领之一,拉格纳之子,率军入侵了英格兰七国。} Ragnarssonr loðbrókar\footnote{Ragnar Loðbrók (Ragar Hairy-breeches), 著名维京英雄,曾在845年3月大举进攻巴黎。} lét drepa Eadmund inn helga Englakonung\footnote{Eadmund inn helga Englakonung, 即“殉道者”埃德蒙(Edmund the Martyr),东盎格利亚王国国王,公元869年抵抗维京人进攻时被杀,但不清楚他具体是死于战场还是被俘后不屈而死。埃德蒙死后被教廷追认为圣徒。}. En þat var átta hundruð ok sjau tigum vetra eftir burð Krists, at því er ritit er í sǫgu hans.
    \switchcolumn

    Iceland was first settled from Norway in the days of Harald the Fairhair, son of Halfdan the Black, at that time when Ivarr, son of Ragnar Hairy-breeches, slew Saint Edmund, king of England, according to the reckoning and telling of Teit, son of Bishop Isleif and my foster father, who I consider the wisest; and of Thorkel, my uncle, son of Gelli, who remembered far back; and of Thurith, daughter of Snorri the Good, who was both very wise and truthful. And that was 874 winters after the birth of Christ, as is written in his [Edmund's] tale.
\end{paracol}

\begin{translation*}{}
    挪威人最初定居冰岛的时候,还是在“美发王”哈拉尔德治下。哈拉尔德是“黑发”哈夫丹的儿子。那时,“毛裤”拉格纳的儿子伊瓦尔杀了英格兰王埃德蒙。我的养父忒特和我说过这个故事,忒特是主教伊斯雷夫的儿子,他是我见过最有学识的人。我的叔叔名叫索科尔,是格里的儿子,他能记得很久之前的旧事;图里斯是“善人”斯诺里的女儿,她不仅十分睿智,所说的也都真实准确。我从索科尔和图里斯那里也听说了这件事,那是耶稣基督降生后的第874个年头。
\end{translation*}
\begin{grammar*}{}
    \begin{enumerate}[leftmargin=*]
        \item í þann tíð ... er Ívarr ...

              注意这里定语从句的先行词tíð距离er非常远,以至于er好像是单独引导了一个时间状语从句。
    \end{enumerate}
\end{grammar*}
\begin{paracol}{2}

    Ingólfr hét maðr nórrænn, er sannliga er sagt, at færi fyrst þaðan til Íslands, þá er Haraldr inn hárfagri var sextán vetra gamall, en í annat sinn fám vetrum síðar. Hann byggði suðr í Reykjarvík. Þar er Ingólfshǫfði kallaðr fyr austan Minþakseyri\footnote{古港口名,应在冰岛南部,具体位置不详。}, sem hann kom fyrst á land, en þar Ingólfsfell fyr vestan ǫlfossá\footnote{河流名,位于冰岛西南部。}, er hann lagði sína eigu á síðan.
    \switchcolumn
    There was a Norwegian named Ingolf, who is said truthfully to first travelled from there to Iceland when Haraldr the Fairhair was sixteen years old, and a second time a few years later. He lived to the south in Reykjavík. The place to the east of Minthakseyri where Ingolf first came ashore, is called Ingolfs-headland, but where he took possession afterward to the west of Olfoss, is called Ingolf-mount.

    \switchcolumn*

    Í þann tíð var Ísland viði vaxit á milli fjalls ok fjǫru. Þá váru hér menn kristnir, þeir er Norðmenn kalla Papa, en þeir fóru síðan á braut, af því at þeir vildu eigi vera hér við heiðna menn, ok létu eftir bækr írskar ok bjǫllur ok bagla\index{bagall!bagla}. Af því mátti skilja, at þeir váru menn írskir.

    \switchcolumn

    At that time, Iceland was covered by forests between the mountains and the shore. Christians were here then, whom the Norsemen call Papar, but then they left because they did not want to be here alongside heathen people. They left Irish books, bells and croziers, from which man can tell that they were Irishmen.

    \switchcolumn*

    En þá varð fǫr manna mikil mjǫk út hingat ór Norvegi, til þess unz konungrinn Haraldr bannaði, af því at honum þótti landauðn nema. Þá sættust þeir á þat, at hverr maðr skyldi gjalda konungi fimm aura, sá er eigi væri frá því skiliðr ok þaðan færi hingat. En svá er sagt, at Haraldr væri sjau tigi vetra konungr ok yrði áttræðr. Þau hafa upphǫf verit at gjaldi því, er nú er kallat landaurar, en þar galzt stundum meira, en stundum minna, unz Óláfr inn digri\footnote{Óláfr inn digri (Olaf the Stout), 即“圣王”奥拉夫二世,挪威国王,公元995年至1000年在位。1030年7月29日的斯蒂克莱斯塔德战役中战死被封圣。} gerði skírt, at hverr maðr skyldi gjalda konungi hálfa mǫrk, sá er færi á milli Norvegs ok Íslands, nema konur eða þeir menn, er hann næmi frá. Svá sagði Þorkell oss Gellissonr.

    \switchcolumn

    And then began a very great migration of people here from Norway, until King Harald forbade it because he thought that his land would be deserted otherwise. Then they settled on it that whoever traveled from Norway to Iceland should pay the king five ounces of silver, and he would not be exempt from this. And it is said that King Harald was king for seventy years, and was turning eighty years old. They have been the origin for the payment which is now called ``land-tax'', and sometimes more was paid for that and sometimes less, until Olaf the Stout made it clear that each person who would travel between Norway and Iceland must pay the king half a mark, except women or men who he took with him. So Thorkel Gellisson told me.
\end{paracol}
\begin{translation*}{}
    据说是个名叫因戈尔夫的挪威人第一次到达了冰岛,那时“美发王”哈拉尔德年方十六。因戈尔夫几年后又成功到达了冰岛。他住在雷克雅未克的南边。因戈尔夫最初登陆的地方在闵撒克堤的东边,现在称为“因戈尔夫岬”,而他后来据为己有的土地则在奥弗斯河西边,今称“因戈尔夫山”。

    那时,冰岛尚未有人烟,山和海岸之间尽是密林。曾有信基督教的爱尔兰僧侣到过岛上,挪威人称他们为“巴帕尔”(Papar),但他们不愿和异教徒为伍,于是离开了。他们留下了爱尔兰的书籍、钟和牧杖,正是这些物件证明了他们爱尔兰人的身份。

    接着,挪威人开始大规模地移民到这里,但后来哈拉尔德国王担心挪威土地会因此荒废,下令禁止了这一行为。之后他们达成了一项约定:凡是从挪威前去冰岛的人就必须向国王缴纳5盎司的银两,任何人都不能免除。据说哈拉尔德当了70年的国王,快满80岁了。这笔钱就是后来所谓“土地税”的前身,土地税时高时低,没有定数,直到奥拉夫二世明确规定每个来往于挪威和冰岛之间的人都要向国王缴纳价值半马克(4盎司)银子的税赋,但与之随行的女人和手下除外。这就是我的叔叔索克尔·格里斯松告诉我的故事。
\end{translation*}
\begin{grammar*}{}
    \begin{enumerate}[leftmargin=*]
        \item hafa upphǫf verit at gjaldi því

              我们已在许多地方见过upphaf接属格的情况,如upphaf þeirar sǫgu, upphaf vers等。upphaf有时也可以接at或á表示相同的意思的。upphǫf at gjaldi `origins of the payment', upphaf at kvæði `beginning of the story'.
        \item sá er eigi væri frá því skiliðr ok þaðan færi hingat

              这里的定语从句和英文的习惯不一致,直译为`who would never be separated from this and travel from here'. 从逻辑上来说,最好应将eigi væri frá því skiliðr放到主句中。这里的frá því指的是税赋。
    \end{enumerate}
\end{grammar*}


\begin{paracol}{2}
    \begin{center}
        \textsc{Frá lagasetning ok alþingissetning\footnotemark}
    \end{center}

    \switchcolumn

    \begin{center}
        \textsc{On the establishment of the law and Althing}
    \end{center}

    \switchcolumn*

    En þá er Ísland var víða byggt orðit, þá hafði maðr austrænn fyrst lǫg út hingat ór Norvegi, sá er Úlfljótr hét, svá sagði Teitr oss, ok váru þá Úlfljótslǫg kǫlluð, en þau váru flest sett at því, sem þá váru Gulaþingslǫg eða ráð Þorleifs ins spaka Hǫrða-Kárasonar váru til, hvar við skyldi auka eða af nema eða annan veg setja. En svá er sagt, at Grímr geitskǫr væri fóstbróðir hans, sá er kannaði Ísland allt at ráði hans, áðr alþingi væri átt.

    \switchcolumn

    And when Iceland had become widely settled, a Norwegian called Ulfjot first brought the laws here from Norway, as Teit told us, and they were called the \textit{Laws of Ulfjot}. But those laws were mostly arranged as \textit{Laws of Gulathing} or the counsel of Thorleif Hortha-Karason the Wise, where something had to be expanded or abolished or arranged in another way. And it is said that Grim the Goat-hair was Ulfjot's foster brother, who explored the whole of Iceland on his advice before the Althing was established.

    \switchcolumn*

    Alþingi var sett at ráði Úlfljóts ok allra landsmanna, þar er nú er, en áðr var þing á Kjalarnesi. En maðr hafði sekr orðit of þræls morð eða leysings, sá er land átti í Bláskógum. Land þat varð síðan allsherjarfé, en þat lǫgðu landsmenn til alþingis neyzlu. Af því er þar almenning at viða til alþingis í skógum ok á heiðum hagi til hrossahafnar.

    \switchcolumn

    Althing was established by the advice of Ulfjot and all the people of the land, in the place where it is now, but before it was on Kjalarnes. A man who owned land in Bláskógar has been outlawed for killing a slave or freedman. That land became general property, and the people of the land set it to the use of the Althing. Therefore it is public land to cut down wood for the Althing in the forests and to graze horses on the heaths.
\end{paracol}

\footnotetext{Alþingi (All-Thing), 即全体议会庭,或音译为阿尔庭。冰岛的议事机构,后作为法律机关。}
\begin{translation*}{}
    忒特跟我们说,当冰岛全境已有人定居后,一位名叫乌尔夫约特的挪威人首先从挪威带来了法律,它们被称为《乌尔夫约特法典》。可这些法律大多和挪威的《古拉庭法典》相仿,又或是照搬了挪威智者索雷夫·霍尔他-卡拉松的意见,因此有些地方需要扩充,删减或重排。乌尔夫约特据说有个养兄弟叫“山羊发”格里姆,他在乌尔夫约特的建议下游历了冰岛,(可能是为了阿尔庭物色地方)。

    阿尔庭是在乌尔夫约特和全体冰岛人的建议下设立的,现在的议会庭沿用其旧址(即今辛格韦德利,Þingvellir)。不过最早阿尔庭则在卡拉尔涅斯地区(今雷克雅未克教区)。布劳斯科加(辛格韦德利旧称)的领主因为杀了一个奴隶还是自由民而被放逐,他的土地就充公用于建设阿尔庭。公民可以在布劳斯科加的森林伐木以供阿尔庭之用,也可在原野上放马。
\end{translation*}
\begin{grammar*}{}
    \begin{enumerate}[leftmargin=*]
        \item var víða byggt orðit

              orðit, verða `turn, become'的过去分词,这里用系动词作为助动词表示完成时,因为verða可被理解为位移和变化的动作。
        \item almenning

              法律术语,指的是公有的土地,通常指允许任何人放牧的草场之类。
    \end{enumerate}
\end{grammar*}

\begin{paracol}{2}
    \begin{center}
        \textsc{Frá því, er kristni kom til Íslands.}
    \end{center}

    \switchcolumn

    \begin{center}
        \textsc{On the coming of Christianity to Iceland}
    \end{center}

    \switchcolumn*

    Óláfr konungr Tryggvasonr\footnote{Óláfr Tryggvasonr, 史称奥拉夫一世。}, Óláfssonar, Haraldssonar ins hárfagra, kom kristni í Norveg ok á Ísland. Hann sendi hingat til lands prest þann, er hét Þangbrandr ok hér kenndi mǫnnum kristni ok skírði þá alla, er við trú tóku. En þeir váru þó fleiri, er í gegn mæltu ok neittu. En þá er hann hafði hér verit einn vetr eða tvá, þá fór hann á braut ok hafði vegit hér tvá menn eða þrjá, þá er hann hǫfðu nítt. En hann sagði konunginum Óláfi, er hann kom austr, allt þat, er hér hafði yfir hann gingit, ok lét ǫrvænt, at hér myndi kristni enn takast. En hann varð við þat reiðr mjǫk ok ætlaði at láta meiða eða drepa ossa\index{óss!ossa} landa fyrir, þá er þar váru austr. En þat sumar it sama kómu útan heðan þeir Gizurr ok Hjalti\footnote{Gizurr, 指Gizurr Teitssonr, 即杀死贡纳尔的冰岛首领。Hjalti, 指Hjalti Skeggiasonr, 亦为冰岛首领。} ok þágu\index{þiggja!þágu} þá undan við konunginn ok hétu honum umbsýslu sinni til á nýjaleik, at hér yrði\index{verða!yrði} enn við kristninni tekit, ok létu sér eigi annars ván en þar myndi hlýða.

    \switchcolumn

    King Olaf, son of Tryggvi, son of Olaf, son of Harald the Fairhair, brought Christianity to Norway and Iceland. He sent hither a priest who was called Thangbrand and preached Christianity to people here and baptized everyone who took the faith. But there were many who spoke against it and refused it. And when he had been there for a year or two, he left after killing two or three men here, who had denied him. And when he came east he told King Olaf everything that had happened to him, and let him believe that Christianity would not prevail yet. And the King became very angry and decided to injure or kill our countrymen who were there in Norway. But in the same summer, Gizur and Hjalti traveled out from here and convinced the king not to and promised him an arrangement of a new attempt, so that Christianity might still be accepted here and that they expected nothing but to succeed.
\end{paracol}
\begin{translation*}{}
    奥拉夫·特里格维松——奥拉夫·哈拉尔德松的孙子,“美发王”哈拉尔德的曾孙——第一次将基督教带到冰岛。他派遣了一名叫作桑格布朗德的传教士去岛上传播福音,又给所有皈依基督的人施洗。可是,岛上有很多人不接受基督教,还出言诋毁它。桑格布朗德在岛上传教两年,终于在杀死两个忤逆他的人后离开了冰岛。他向国王讲述了自己的遭遇,并告诉国王基督教还无法在岛上传开来。国王听后勃然大怒,决定要毁伤或是处死当时在挪威的冰岛人。幸而在那个夏天,齐泽尔和希雅提前去挪威打消了国王的想法,并保证他们会继续在岛上传教,且这一次只会成功,不会失败。
\end{translation*}
\begin{grammar*}{}
    \begin{enumerate}[leftmargin=*]
        \item lét ǫrvænt, at hér myndi kristni enn takast
              lét ǫrvænt, 省略了宾语hann, `make him (the king) disbelieving'. takast, taka的反身式,表示“成功,有效”。
        \item þágu þá undan við konunginn
              þágu, þiggja `accept'的复数过去式,þiggja有时表示“通过请求获得某物”,因此它几乎类似于`obtain, make succeed'. þiggja e-t undan是一个衍生的短语,其含义类似于`get someone/something relieved from/free from', 本句中其宾语是konunginn, 并且额外添加了介词við `against'表示远离的对象(其宾语省略了),表示`get the king relieved from (the thought of injuring or killing ...)'.
    \end{enumerate}
\end{grammar*}
\begin{paracol}{2}
    En it næsta sumar eftir fóru þeir austan ok prestr sá, er Þormóðr hét, ok kómu þá í Vestmannaeyjar\footnote{冰岛西南部岛屿。}, er tíu vikur váru af sumri, ok hafði allt farizt\index{fara!farizt} vel at. Svá kvað Teitr þann segja, er sjálfr var þar.
    \switchcolumn

    Then in the next summer, they and a priest named Thormod came to Vestmannaeyjar when ten weeks of the summer had passed, and they all had a prosperous journey. Thus spoke Teit, who was there himself.

    \switchcolumn*
    En þeir fóru þegar inn til meginlands ok síðan til alþingis ok gátu at Hjalta, at hann var eftir í Laugardali með tólfta mann\footnote{从上下文来看,希雅提一行到达冰岛后应短暂与齐泽尔分开(可能是为了躲避风头),之后再一起前往阿尔庭。}, af því at hann hafði áðr sekr orðit fjǫrbaugsmaðr it næsta sumar á alþingi of goðgá. En þat var til þess haft\index{hafa!haft}, at hann kvað at Lǫgbergi\footnote{法律石,冰岛西南部的一块突出岩石,阿尔庭召开的地点,发言人在石上演说。} kviðling þenna:
    \switchcolumn

    And they traveled from there to the mainland and then to Althing and heard from Hjalti that he was back in Laugardali with twelve men since he had been convicted lesser outlawry for blasphemy the previous summer at the Althing. And the conviction was done because he had recited this at the Law Rock:

    \switchcolumn*

    \begin{quote}
        Vilk eigi goð geyja. \\
        Grey þykki mér Freyja.
    \end{quote}

    \switchcolumn

    \begin{quote}
        I will not decry the God. \\
        I think Freyja is a dog.
    \end{quote}

    \switchcolumn*

    En þeir Gizurr fóru, unz þeir kómu í stað þann í hjá ǫlfossvatni, er kallaðr er Vellankatla\footnote{一处温泉的名字,在阿尔庭附近。}, ok gerðu orð þaðan til þings, at á mót þeim skyldi koma allir fulltingsmenn þeira, af því at þeir hǫfðu spurt, at andskotar þeira vildi verja þeim vígi þingvǫllinn. En fyrr en þeir færi þaðan, þá kom þar ríðandi Hjalti ok þeir, er eftir váru með honum. En síðan riðu þeir á þingit, ok kómu áðr á mót þeim frændr þeira ok vinir, sem þeir hǫfðu æst. En inir heiðnu menn hurfu\index{hverfa!hurfu} saman með alvæpni, ok hafði svá nær, at þeir myndi berjast, at eigi of sá á milli.

    \switchcolumn

    But Gizurr and his companions traveled until they came to the place near Ölfossvatn, which is called Vellankatla, and they sent word from there to the þing that all their supporters should come to meet them, because they had found out that their opponents would keep them away from the Thing-field by force. And before they left there, Hjalti and those who were with him came to them riding. And then they rode to the Thing, and their kinsmen and friends whom they had requested had already come to meet them. But the heathen people gathered together with full arms, and it has become so close to a point that they would fight each other and no one could see a way out.
\end{paracol}


\begin{translation*}{}
    第二年,他们从挪威来到冰岛,还带来了一位名叫索尔莫德的传教士。在夏天过去十周后,他们抵达韦斯特曼纳埃亚尔,一路上诸事顺利。忒特当时就在那里,于是将他亲眼所见告诉了我。

    他们一行随即前往冰岛大陆,然后来到阿尔庭。这时他们得知希雅提与十二个人一起已回到了劳加达尔谷,因为希雅提一年前在阿尔庭上因亵渎神明而被放逐。他的罪行是在法律石前朗诵了这首诗:
    \begin{quote}
        基督圣名我岂敢亵渎,\\
        弗蕾雅却是一条牲畜。
    \end{quote}

    齐泽尔一行随后来到奥弗斯湖边的韦拉卡特拉温泉,从那里放出消息,要求他们的支持者都前来会面,因为他们已经获知对手会用武力阻止他们进入阿尔庭。在齐泽尔等人离开温泉之前,希雅提和他的同伴已骑马赶到加入他们。众人随后一起前往阿尔庭,他们所拜托的亲戚朋友都已在那里迎接他们。而异教徒则全副武装地聚集在一起,双方剑拔弩张,眼看局势就要一发不可收拾。
\end{translation*}
\begin{grammar*}{}
    \begin{enumerate}[leftmargin=*]
        \item fjǫrbaugsmaðr
              fjǫrbaugr `life-money', 是一笔由已决犯付给法庭的钱。一般而言,这类罪犯罪行较轻,故可以付钱“抵罪偿命”。如果罪犯拒绝付钱,他就被终身永久放逐(full outlawry),这种惩罚称为skóggangr `wood-going',终身放逐者则叫作skógarmaðr `wood-man',他们被逐出人类社会,事实上相当于死刑. 反之,愿意付钱的轻罪犯被称为fjǫrbaugsmaðr, 他们只被要求离开国家三年并处没收财产。这种惩罚则叫作“小放逐”(lesser outlawry),古诺尔斯语作fjǫrbaugsgarðr `life-money fence'. 这个词中的garðr `fence, yard'指的是一个限定的区域,由于罪犯已经支付赎金,他在离开国家前待在此区域内是安全的,故而fjǫrbaugsgarðr成为了小放逐的代名词。
        \item goðgá
              由goð和gá组成,gá原意为“犬吠”。

        \item þat var til þess haft
              其中haft是hafa的过去分词,但hafa在这里的含义与hæfa合流,原意是`aim; hit'. vera til þess haft从而有衍生义`to be the reason for sth.' hafa的这个含义亦可参考尼亚尔萨迦中贡纳尔的话`hefir hverr til síns ágætis nǫkkut'.
    \end{enumerate}
\end{grammar*}
\begin{paracol}{2}
    En annan dag eftir gingu þeir Gizurr ok Hjalti til Lǫgbergis ok báru þar upp erendi sín. En svá er sagt, at þat bæri frá, hvé vel þeir mæltu. En þat gerðist af því, at þar nefndi annarr maðr at ǫðrum vátta, ok sǫgðust hvárir ór lǫgum við aðra, inir kristnu menn ok inir heiðnu, ok gingu síðan frá Lǫgbergi. Þá báðu inir kristnu menn Hall á Síðu\footnote{Síðu-Hallr Þorsteinssonr, 冰岛贵族、首领,早期皈依基督教的冰岛首领。}, at hann skyldi lǫg þeira upp segja, þau er kristninni skyldi fylgja. En hann leystist því undan við þá, at hann keypti at Þorgeiri lǫgsǫgumanni\footnote{即Þorgeirr Ljósvetningagoði Þorkelssonr, 冰岛9世纪末至10世纪初的宣法官(Lawspeaker)。宣法官是斯堪的纳维亚的一种特殊法律职业,他们一般由博学之士担任,在庭上引述法律条文,并为人辩护。当时Síðu-Hallr Þorsteinssonr是基督徒的宣法官,而Þorgeirr Ljósvetningagoði则是异教徒的宣法官。}, at hann skyldi upp segja, en hann var enn þá heiðinn. En síðan er menn kómu í búðir, þá lagðist hann niðr Þorgeirr ok breiddi feld sinn á sik ok hvílði þann dag allan ok nóttina eftir ok kvað ekki orð. En of morguninn eftir settist hann upp ok gerði orð, at menn skyldi ganga til Lǫgbergis.
    \switchcolumn

    And the next day Gizur and Hjalti went to the Law Rock and announced their mission there. And it is said that it was extraordinary how well they spoke. And because of that, it turned out that one person named another as a witness, and both the Christian and the heathen people declared others outlawry, and they left the Law Rock. Then the Christian men asked Hall of Síða to proclaim their laws, which the Christians should follow. But he relieved himself from this by making an agreement with Thorgeir the law-speaker, that he should present them, though he was still heathen then. And then when everyone returned to their abodes, Thorgeir laid himself down, spread his cloak over himself, rested there all that day and the following night and did not say a word. But the next morning he got up and made the announcement that everyone should come to the Law Rock.
\end{paracol}

\begin{translation*}{}
    第二天齐泽尔和希雅提前往法律石,在那里宣布了国王的使命。两人口若悬河,引得惊叹连连。然而,人们确因此争吵起来,基督徒和异教徒们各自传唤证人,指责对方不守法律,最后不欢而散。基督徒们于是要求他们的宣法官,西萨的霍尔,为他们宣扬基督徒的律法。但霍尔却和当时还信多神教的宣法官索哥尔达成了协定,转而让他代表基督徒们发言。当众人散去后,索哥尔躺下身来,用斗篷蒙住自己的身体,一人独自沉思了一天一夜。第二天早上,他召集所有人来到法律石集会。
\end{translation*}
\begin{grammar*}{}
    \begin{enumerate}[leftmargin=*]
        \item at þat bæri frá, hvé vel þeir mæltu
              bæri是bera的虚拟式,这里是其无人称结构(þat是几乎形式的宾语)。bera此时有 `it comes to a place'的含义,bæri frá于是衍生出“超过...”的意思。
        \item annarr maðr at ǫðrum
              这里的at表示无间断的接续,类似于英文的`one after another', 其整体是动词短语nefna vátta ·summon witness'的宾语。
        \item keypti at Þorgeiri lǫgsǫgumanni, at ...
              kaupa `buy'在这里也可以表示“达成某种协议”,kaupa e-t at e-m `make a bargain with someone'.

    \end{enumerate}
\end{grammar*}
\begin{paracol}{2}
    En þá hóf\index{hefja!hóf} hann tǫlu sína upp, er menn kómu þar, ok sagði, at honum þótti þá komit hag manna í ónýtt efni, ef menn skyldi eigi hafa allir lǫg ein á landi hér, ok talði fyr mǫnnum á marga vega, at þat skyldi eigi láta verða, ok sagði, at þat myndi at því ósætti verða, er vísaván var, at þær bar\-smíðir gerðust á milli manna, er landit eyddist af. Hann sagði frá því, at konungar ór Norvegi ok ór Danmǫrku hǫfðu haft ófrið ok orrostur á milli sín langa tíð, til þess unz landsmenn gerðu frið á milli þeira, þótt þeir vildu eigi. En þat ráð gerðist svá, at af stundu sendust þeir gersemar á milli, enda hélt friðr sá, meðan þeir lifðu. `En nú þykkir mér þat ráð,' kvað hann, `at vér látim ok eigi þá ráða, er mest vilja í gegn gangast, ok miðlum svá mál á milli þeira, at hvárirtveggju hafi nakkvat síns máls, ok hǫfum allir ein lǫg ok einn sið. Þat mun verða satt, er vér slítum í sundr lǫgin, at vér munum slíta ok friðinn.' En hann lauk\index{lúka!lauk} svá máli sínu, at hvárirtveggju játtu því, at allir skyldi ein lǫg hafa, þau sem hann réði upp at segja.
    \switchcolumn

    And he began his speech when the people arrived there, and said that it seemed to him that man's condition would turn into ill state, if men do not all share one law here in this country, and spoke before the people in various ways, and that it should never be allowed to happen. He said that this would turn into discord, which was sure to happen, that fighting would take place between people and the land would be laid waste. He spoke about how the kings of Norway and Denmark had had disagreements and wars between themselves for a long time, until their countrymen made peace between them, even though they didn't want it. And the settlement is established in the way that they sent each other treasures at times, and then the peace held, while they lived. `And now I have made up my mind that', he said, `we also should not let the way prevail, which leads to the greatest corruption, and let us make a compromise between them, so that both preserve their way to some extent, and all have one law and custom. It shall be true that when we break the law apart, we will also break our peace.' So He closed his speech, that both sides agreed all would have one law, which he suggested aloud.
    \switchcolumn*

    Þá var þat mælt í lǫgum, at allir menn skyldi kristnir vera ok skírn taka, þeir er áðr váru óskírðir á landi hér. En of barnaútburð\footnote{即弃婴,在当时将有残疾或无力抚养的婴儿遗弃一般不受法律处罚。} skyldu standa in fornu lǫg ok of hrossakjǫtsát. Skyldu menn blóta á laun, ef vildu en varða fjǫrbaugsgarðr, er váttum of kæmi við. En síðar fám vetrum var sú heiðni af numin sem ǫnnur.
    \switchcolumn

    Then it is declared in log that all men should be Christian and take baptism, who had not been baptised here in this country. But the old law about child exposure and eating of horse meat will last. Men should sacrifice in secret, if they wished and wanted to avoid outlawry, which (would happen) if witnesses were at hand. And a few years later, that heathen practice was taken away like the others.
    \switchcolumn*

    Þenna atburð sagði Teitr oss at því, er kristni kom á Ísland. En Óláfr Tryggvason fell it sama sumar at sǫgu Sæmundar prests\footnote{即Sæmundr Sigfússon (1056–1133),冰岛牧师、学者,又称Sæmundr inn fróði `Saemund the Learned'.}. Þá barðist hann við Svein Haraldsson\footnote{Sveinn Haraldssonr, 绰号tjúguskegg `Forkbeard', 丹麦国王,986-1014年在位。} Danakonung ok Óláf inn sænska\footnote{即Óláfr Eiríkssonr skautkonungr, 瑞典历史上首位有确切记载的国王“胜利者埃里克”(Eiríkr inn sigrsæli)的儿子,瑞典第二位国王,995–1022在位。}, Eiríksson at Uppsǫlum Svíakonungs, ok Eirík, er síðan var jarl at Norvegi, Hákonarson\footnote{即Eiríkr Hákonarsonr, 挪威贵族,1000-1012年任挪威总督。}. Þat var hundrað ok þremr tigum vetra eftir dráp Eadmundar, en þúsund eftir burð Krists at alþýðu tali.
    \switchcolumn

    Teit told us this circumstance of how Christianity came to Iceland. And Olaf Tryggvason fell the same summer, according to the account of Saemund the priest. He fought then against King Svein Haraldsson of Denmark, Olaf the Swede, who was the son of Eirik and king of the Swedes at Uppsala, and Eirik Hakonarson who was later the jarl of Norway. That was one hundred and thirty years after the death of Edmund, and a thousand after the birth of Christ according to the common tale.
\end{paracol}

\begin{translation*}{}
    人们逐渐到齐后,索哥儿开始了他的演说。他认为,假如这里的人们如果不遵守一套法律,而是各执一词的话,就势必会引起纷争和部落间的战争,而那时岛民的处境就会愈发艰难,土地也将荒废。他引证了挪威和丹麦国王的例子,这两个国王长期征战不休,最后迫不得已才达成和解。然而这种和平不过是靠不时的互赠礼金来维系,等到国王一死,便化为泡影。“我现在已经明白,”他说到:“我们绝不能允许不利于我们前途的事情发生。我们当在两种宗教间达成和解,统一宗教和习俗,但允许双方都保留一定的习惯自由。毫无疑问的是,假如我们分裂了律法,也势必要分裂国家。”索哥尔的陈词让双方都同意要统一律法。

    于是,法律规定当时未受洗的人都应该改信基督并接受洗礼。但是,过去关于弃婴和食用马肉的旧律法还可以延续。如果人们想要祭祀日耳曼的神又不想犯法,那就应该在私下里祭祀。假如有证人证明拜异神的,则要处以小放逐。然而几年后,这些异教的习俗都逐渐被取缔了。

    这就是忒特所述的基督教来到冰岛的过程。据牧师赛蒙德所说,那年是奥拉夫一世驾崩的一年。奥拉夫一世一生征战无数,与丹麦的国王斯文·哈拉尔德松、乌普萨拉的瑞典国王奥拉夫·埃里克松以及后来的挪威伯爵埃里克·哈康松都有交手。这是埃德蒙殉道后的130年,也即大众所说的基督诞生后的一千年。
\end{translation*}
\begin{grammar*}{}
    \begin{enumerate}[leftmargin=*]
        \item ráða upp
              ráða在这里的意思更倾向于“读”,加upp表示`read aloud'.
        \item váttum of kæmi við
              koma e-u við是固定短语,表示`bring about/be able to do sth.'.
    \end{enumerate}
\end{grammar*}
