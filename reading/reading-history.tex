\chapter{冰岛历史选读}
本章对冰岛的历史进行简要的介绍,主要包括冰岛的发现、冰岛的定居情况、基督教的传入以及一些著名的冰岛人的成就——格陵兰岛和“文兰”(Vinland)的发现等。

\section{移民书(Landnámabók)选读}
移民书(Landnámabók, Book of Settlement)是一部冰岛的谱系学著作,它最早由著名冰岛学者Ari Þorgilsson在12世纪早期编写,详细描述了挪威人在9世纪和10世纪挪威人在冰岛的定居情况。

移民书是关于冰岛早期历史的主要来源之一。它共分为5个部分,超过100多章,记录了1400多个定居点和400多个冰岛人的宗谱,包括定居者的名字、家庭、财产和土地所有权。移民书另还包括有关冰岛早期社会和文化的其他信息,例如早期定居者如何建立家庭、养育孩子、种植作物和养殖动物。移民书也涉及冰岛的法律和政治制度。

本章节选了移民书的前言和第一章的部分内容,主要介绍了冰岛的发现和移民

\section{冰岛人之书(Íslendingabók)选读}
冰岛人之书
\begin{paracol}{2}
    \begin{center}
        \textsc{Frá Íslands byggð}
    \end{center}

    \switchcolumn
    \begin{center}
        \textsc{On the settlement of Iceland}
    \end{center}
    \switchcolumn*
    Ísland byggðist fyrst ór Norvegi á dǫgum Haralds ins hárfagra\footnotemark, Hálfdanarsonar ins svarta\footnotemark, í þann tíð, at ætlun ok tǫlu þeira Teits, fóstra míns, þess manns, er ek kunna spakastan, sonar Ísleifs byskups, ok Þorkels, fǫðurbróður míns, Gellissonar, er langt munði fram, ok Þuríðar Snorradóttur goða, er bæði var margspǫk ok óljúgfróð, er Ívarr\footnotemark Ragnarssonr loðbrókar\footnotemark lét drepa Eadmund inn helga Englakonung\footnotemark. En þat var átta hundruð ok sjau tigum vetra eftir burð Krists, at því er ritit er í sǫgu hans.
    \switchcolumn
    Iceland was first settled from Norway in the days of Harald the Fairhair, son of Halfdan the Black, at that time when Ivarr, son of Ragnar Hairy-breeches, slew Saint Edmund, king of England, according to the reckoning and telling of Teit, son of Bishop Isleif and my foster father, who I consider the wisest; and of Thorkel, my uncle, son of Gelli, who remembered far back; and of Thurith, daughter of Snorri the Good, who was both very wise and truthful. And that was 874 winters after the birth of Christ, as is written in his [Edmund's] tale.
\end{paracol}
\footnotetext[1]{Haraldr inn hárfagri (Harald the Fairhair), 挪威的统一者,第一任挪威国王,公元872年至930年在位。关于其的可靠记载不多,见于几部王室萨迦。}
\footnotetext[2]{Hálfdanr inn svarti (Halfdan the Black), 哈拉尔德之父。}
\footnotetext[3]{Ívarr Ragnarsson, 又称Ívarr inn Beinlausi (Ivarr the Boneless), “维京大军”的首领之一,拉格纳之子,率军入侵了英格兰七国。}
\footnotetext[4]{Ragnar Loðbrók (Ragar Hairy-breeches), 著名维京英雄,曾在845年3月大举进攻巴黎。}
\footnotetext[5]{Eadmund inn helga Englakonung, 即“殉道者”埃德蒙(Edmund the Martyr),他抵抗维京人进攻时被杀,但不清楚他具体是死于战场还是被俘后不屈而死。埃德蒙死后被教廷追认为圣徒。}
\begin{translation*}{}
    挪威人最初定居冰岛的时候,还是在“美发王”哈拉尔德治下。哈拉尔德是“黑发”哈夫丹的儿子。那时,“毛裤”拉格纳的儿子伊瓦尔杀了英格兰王埃德蒙。我的养父忒特——主教伊斯雷夫的儿子——和我说过这个故事,他是我见过最有学识的人。我的叔叔名叫索科尔,是格里的儿子,他能记得很久之前的旧事;图里斯是“善人”斯诺里的女儿,她不仅十分睿智,所说的也都真实准确。我从索科尔和图里斯那里也听说了这件事,那是耶稣基督降生后的第874个年头。
\end{translation*}
\begin{grammar*}{}
    \begin{enumerate}
        \item í þann tíð ... er Ívarr ...

              注意这里定语从句的先行词tíð距离er非常远,以至于er好像是单独引导了一个时间状语从句。
    \end{enumerate}
\end{grammar*}
\begin{paracol}{2}

    Ingólfr hét maðr nórrænn, er sannliga er sagt, at færi fyrst þaðan til Íslands, þá er Haraldr inn hárfagri var sextán vetra gamall, en í annat sinn fám vetrum síðar. Hann byggði suðr í Reykjarvík. Þar er Ingólfshǫfði kallaðr fyr austan Minþakseyri, sem hann kom fyrst á land, en þar Ingólfsfell fyr vestan ǫlfossá, er hann lagði sína eigu á síðan.
    \switchcolumn
    There was a Norwegian named Ingolf, who is said truthfully to first travel from there to Iceland when Haraldr the Fairhair was sixteen years old, and a second time a few years later. He lived to the south in Reykjavík. The place to the east of Minþakseyri where Ingolf first came ashore, is called Ingólfshöfði, but where he took possession afterward to the west of Ölfossá, is called Ingólfsfell.
    \switchcolumn*
    Í þann tíð var Ísland viði vaxit á milli fjalls ok fjǫru. Þá váru hér menn kristnir, þeir er Norðmenn kalla Papa, en þeir fóru síðan á braut, af því at þeir vildu eigi vera hér við heiðna menn, ok létu eftir bækr írskar ok bjǫllur ok bagla. Af því mátti skilja, at þeir váru menn írskir.
    \switchcolumn
    At that time, Iceland was covered by forests between the mountains and the shore. Christians were here then, who the Norsemen call Papar, but then they left because they did not want to be here alongside heathen people. They left Irish books, bells and croziers, from which man can tell that they were Irishmen.
    \switchcolumn*
    En þá varð fǫr manna mikil mjǫk út hingat ór Norvegi, til þess unz konungrinn Haraldr bannaði, af því at honum þótti landauðn nema. Þá sættust þeir á þat, at hverr maðr skyldi gjalda konungi fimm aura, sá er eigi væri frá því skiliðr ok þaðan færi hingat. En svá er sagt, at Haraldr væri sjau tigi vetra konungr ok yrði áttræðr. Þau hafa upphǫf verit at gjaldi því, er nú er kallat landaurar, en þar galzt stundum meira, en stundum minna, unz Óláfr inn digri\footnotemark gerði skírt, at hverr maðr skyldi gjalda konungi hálfa mǫrk, sá er færi á milli Norvegs ok Íslands, nema konur eða þeir menn, er hann næmi frá. Svá sagði Þorkell oss Gellissonr.
    \switchcolumn
    And then began a very great migration of people here from Norway, until King Harald forbade it because he thought that his land would be deserted otherwise. Then they settled on it that whoever travels from Norway to Iceland should pay the king five ounces of silver, and he would not be exempt from this. And it is said that King Harald was king for seventy years, and was turning eighty years old. They have been the origin for the payment which is now called ``land-tax'', and sometimes more was paid for that and sometimes less, until Olaf the Stout made it clear that each person who would travel between Norway and Iceland must pay the king half a mark, except women or men who he took with him. So Thorkel Gellisson told me.
\end{paracol}
\footnotetext{Óláfr inn digri (Olaf the Stout), 即“圣王”奥拉夫二世,挪威国王,公元995年至1000年在位。1030年7月29日的斯蒂克莱斯塔德战役中战死被封圣。}
\begin{translation*}{}
    据说是个名叫因戈尔夫的挪威人第一次到达了冰岛,那时“美发王”哈拉尔德年方十六。因戈尔夫几年后又成功到达了冰岛。他住在雷克雅未克的南边。因戈尔夫最初登陆的地方在闵撒克塞里(Minþakseyri)的东边,现在称为“因戈尔夫岬”,而他后来据为己有的土地则在奥弗沙(Ölfossá)西边,今称“因戈尔夫山”。

    那时,冰岛尚未有人烟,山和海岸之间尽是密林。曾有信基督教的爱尔兰僧侣到过岛上,挪威人称他们为“巴帕尔”(Papar),但他们不愿和异教徒为伍,于是离开了。他们留下了爱尔兰的书籍、钟和牧杖,正是这些物件证明了他们爱尔兰人的身份。

    接着,挪威人开始大规模地移民到这里,但后来哈拉尔德国王担心挪威土地会因此荒废,下令禁止了这一行为。之后他们达成了一项约定:凡是从挪威前去冰岛的人就必须向国王缴纳5盎司的银两,任何人都不能免除。据说哈拉尔德当了70年的国王,快满80岁了。这笔钱就是后来所谓“土地税”的前身,土地税时高时低,没有定数,直到奥拉夫二世明确规定每个来往于挪威和冰岛之间的人都要向国王缴纳价值半马克(4盎司)银子的税赋,但与之随行的女人和手下除外。这就是我的叔叔索克尔·格里斯松告诉我的。
\end{translation*}
\begin{grammar*}{}
    \begin{enumerate}
        \item hafa upphǫf verit at gjaldi því

              我们已在许多地方见过upphaf接属格的情况,如upphaf þeirar sǫgu, upphaf vers等。upphaf有时也可以接at或á表示相同的意思的。upphǫf at gjaldi `origins of the payment', upphaf at kvæði `beginning of the story'.
        \item sá er eigi væri frá því skiliðr ok þaðan færi hingat

              这里的定语从句和英文的习惯不一致,直译为`who would never be separated from this and travel from here'. 从逻辑上来说,最好应将eigi væri frá því skiliðr放到主句中。这里的frá því指的是税赋。
    \end{enumerate}
\end{grammar*}


\begin{paracol}{2}
    \begin{center}
        \textsc{Frá lagasetning ok alþingissetning\footnotemark}
    \end{center}
    \switchcolumn
    \begin{center}
        \textsc{On the establishment of the law and Althing}
    \end{center}
    En þá er Ísland var víða byggt orðit, þá hafði maðr austrænn fyrst lǫg út hingat ór Norvegi, sá er Úlfljótr hét, svá sagði Teitr oss, ok váru þá Úlfljótslǫg kǫlluð, en þau váru flest sett at því, sem þá váru Gulaþingslǫg eða ráð Þorleifs ins spaka Hǫrða-Kárasonar váru til, hvar við skyldi auka eða af nema eða annan veg setja. En svá er sagt, at Grímr geitskǫr væri fóstbróðir hans, sá er kannaði Ísland allt at ráði hans, áðr alþingi væri átt.
    \switchcolumn
    And when Iceland had become widely settled, a Norwegian called Ulfjot first brought the laws here from Norway, as Teit told us, and they were called the \textit{Laws of Ulfjot}. But those laws were mostly arranged as \textit{Laws of Gulathing} or the counsel of Thorleif Hortha-Karason the Wise, where something had to be expanded or abolished or arranged in another way. And it is said that Grim the Goat-hair was Ulfjot's foster brother, who explored the whole of Iceland on his advice before the Althing was established.
    \switchcolumn*
    Alþingi var sett at ráði Úlfljóts ok allra landsmanna, þar er nú er, en áðr var þing á Kjalarnesi. En maðr hafði sekr orðit of þræls morð eða leysings, sá er land átti í Bláskógum. Land þat varð síðan allsherjarfé, en þat lǫgðu landsmenn til alþingis neyzlu. Af því er þar almenning at viða til alþingis í skógum ok á heiðum hagi til hrossahafnar.
    \switchcolumn
    Althing was established by the advice of Ulfjot and all the people of the land, in the place where it is now, but before it was on Kjalarnes. A man who owned land in Bláskógar has been outlawed for killing a slave or freedman. That land became general property, and the people of the land set it to the use of the Althing. Therefore it is public land to cut down wood for the Althing in the forests and to graze horses on the heaths.
\end{paracol}

\footnotetext{Alþingi (All-Thing), 即全体议会庭,或音译为阿尔庭。冰岛的议事机构,后作为法律机关。}
\begin{translation*}{}
    忒特跟我们说,当冰岛全境已有人定居后,一位名叫乌尔夫约特的挪威人首先从挪威带来了法律,它们被称为《乌尔夫约特法典》。可这些法律大多和挪威的《古拉庭法典》相仿,又或是照搬了挪威智者索雷夫·霍尔他-卡拉森的意见,因此有些地方需要扩充,删减或重排。乌尔夫约特据说有个养兄弟叫“山羊发”格里姆,他在乌尔夫约特的建议下游历了冰岛,(可能是为了阿尔庭物色地方)。

    阿尔庭是在乌尔夫约特和全体冰岛人的建议下设立的,现在的议会庭沿用其旧址(即今辛格韦德利,Þingvellir)。不过最早阿尔庭则在卡拉尔涅斯地区(今雷克雅未克教区)。布劳斯科加(辛格韦德利旧称)的领主因为杀了一个奴隶还是自由民而被放逐,他的土地就充公用于建设阿尔庭。公民可以在布劳斯科加的森林伐木以供阿尔庭之用,也可在原野上放马。
\end{translation*}
\begin{grammar*}{}
    \begin{enumerate}
        \item var víða byggt orðit

              orðit, verða `turn, become'的过去分词,这里用系动词作为助动词表示完成时,因为verða可被理解为位移和变化的动作。
        \item almenning

              法律术语,指的是公有的土地,通常指允许任何人放牧的草场之类。
    \end{enumerate}
\end{grammar*}

4. Frá misseristali.

Þat var ok þá, er inir spǫkustu menn á landi hér hǫfðu talit í tveim misserum fjóra daga ins fjórða hundraðs, - þat verða vikur tvær ins sétta tegar, en mánuðr tólf þrítǫgnáttar ok dagar fjórir umbfram, - þá merkðu þeir at sólargangi, at sumar munaði aftr til vársins. En þat kunni engi segja þeim, at degi einum var fleira en heilum vikum gegndi í tveim misserum, ok þat olli.
En maðr hét Þorsteinn surtr. Hann var breiðfirzkr, sonr Hallsteins Þórólfssonar Mostrarskeggja landnámamanns ok Óskar Þorsteinsdóttur ins rauða. Hann dreymði þat, at hann hygðist vera at Lǫgbergi, þá er þar var fjǫlmennt, ok vaka, en hann hugði alla menn aðra sofa. En síðan hugðist hann sofna, en hann hugði þá alla menn aðra vakna. Þann draum réð Ósvífr Helgasonr, móðurfaðir Gellis Þorkelssonar, svá, at allir menn myndi þǫgn varða, meðan hann mælti at Lǫgbergi, en síðan, er hann þagnaði, at þá myndi allir þat róma, er hann hefði mælt. En þeir váru báðir spakir menn mjǫk.
En síðan, er menn kómu til þings, þá leitaði hann þess ráðs at Lǫgbergi, at it sjaunda hvert sumar skyldi auka viku ok freista, hvé þá hlýddi. En svá sem Ósvífr réð drauminn, þá vǫknuðu allir menn við þat vel, ok var þá þat þegar í lǫg leitt at ráði Þorkels mána ok annarra spakra manna.
At réttu tali eru í hverju ári fimm dagar ins fjórða hundraðs, ef eigi er hlaupár, en þá einum fleira. En at óru tali verða fjórir. En þá er eykst at óru tali it sjaunda hvert at viku, en engu at hinu, þá verða sjau ár saman jafnlǫng at hvárutveggja. En ef hlaupár verða tvau á milli þeira, er auka skal, þá þarf auka it sétta.


5. Frá fjórðungadeild.

Þingadeild mikil varð á milli þeira Þórðar gellis, sonar Óleifs feilans ór Breiðafirði, ok Odds, þess er kallaðr var Tungu-Oddr. Hann var borgfirzkr. Þorvaldr, sonr hans, var at brennu Þorkels Blund-Ketilssonar með Hæsna-Þóri í ǫrnólfsdali. En Þórðr gellir varð hǫfðingi at sǫkinni, af því at Hersteinn Þorkelssonr, Blund-Ketilssonar, átti Þórunni, systurdóttur hans. Hon var Helgu dóttir ok Gunnars, systir Jófríðar er Þorsteinn átti Egilssonr.
En þeir váru sóttir á þingi því, er var í Borgarfirði í þeim stað, er síðan er kallat Þingnes. Þat váru þá lǫg, at vígsakar skyldi sækja á því þingi, er næst var vettvangi. En þeir bǫrðust þar, ok mátti þingit eigi heyjast at lǫgum. Þar fell Þórólfr refr, bróðir Álfs í Dǫlum, ór liði Þórðar gellis.
En síðan fóru sakarnar til alþingis, ok bǫrðust þeir þar þá enn. Þá fellu menn ór liði Odds, enda varð sekr hann Hæsna-Þórir ok drepinn síðan ok fleiri þeir, er at brennunni váru.
Þá talði Þórðr gellir tǫlu umb at Lǫgbergi, hvé illa mǫnnum gegndi at fara í ókunn þing at sækja of víg eða harma sína, ok talði, hvat honum varð fyrir, áðr hann mætti því máli til laga koma, ok kvað ýmissa vandræði mundu verða, ef eigi réðist bætr á.
Þá var landinu skipt í fjórðunga, svá at þrjú urðu þing í hverjum fjórðungi, ok skyldu þingunautar eiga hvar saksóknir saman, nema í Norðlendingafjórðungi váru fjǫgur, af því at þeir urðu eigi á annat sáttir. Þeir, er fyr norðan váru Eyjafjǫrð, vildu eigi þangat sækja þingit, ok eigi í Skagafjǫrð þeir, er þar váru fyr vestan. En þó skyldi jǫfn dómnefna ok lǫgréttuskipun ór þeira fjórðungi sem ór einum hverjum ǫðrum. En síðan váru sett fjórðungaþing. Svá sagði oss Úlfheðinn Gunnarssonr lǫgsǫgumaðr.
Þorkell máni Þorsteinssonr, Ingólfssonar, tók lǫgsǫgu eftir Þórarin Ragabróður ok hafði fimmtán sumur. Þá hafði Þorgeirr at Ljósavatni Þorkelssonr sautján sumur.


6. Frá Grænlands byggð.

Land þat, er kallat er Grænland, fannst ok byggðist af Íslandi.
Eiríkr inn rauði hét maðr breiðfirzkr, er fór út heðan þangat ok nam þar land, er síðan er kallaðr Eiríksfjǫrðr. Hann gaf nafn landinu ok kallaði Grænland ok kvað menn þat mundu fýsa þangat farar, at landit ætti nafn gott. Þeir fundu þar manna vistir bæði austr ok vestr á landi ok keiplabrot ok steinsmíði þat, er af því má skilja, at þar hafði þess konar þjóð farit, er Vínland hefir byggt ok Grænlendingar kalla Skrælingja. En þat var, er hann tók byggva landit, fjórtán vetrum eða fimmtán fyrr en kristni kæmi hér á Ísland, at því er sá talði fyr Þorkeli Gellissyni á Grænlandi, er sjálfr fylgði Eiríki inum rauða út.


7. Frá því, er kristni kom til Íslands.

Óláfr konungr Tryggvasonr, Óláfssonar, Haraldssonar ins hárfagra, kom kristni í Norveg ok á Ísland.
Hann sendi hingat til lands prest þann, er hét Þangbrandr ok hér kenndi mǫnnum kristni ok skírði þá alla, er við trú tóku. En Hallr á Síðu Þorsteinssonr lét skírast snimhendis ok Hjalti Skeggjasonr ór Þjórsárdali ok Gizurr inn hvíti Teitssonr, Ketilbjarnarsonar frá Mosfelli, ok margir hǫfðingjar aðrir. En þeir váru þó fleiri, er í gegn mæltu ok neittu. En þá er hann hafði hér verit einn vetr eða tvá, þá fór hann á braut ok hafði vegit hér tvá menn eða þrjá, þá er hann hǫfðu nítt. En hann sagði konunginum Óláfi, er hann kom austr, allt þat, er hér hafði yfir hann gingit, ok lét ǫrvænt, at hér myndi kristni enn takast. En hann varð við þat reiðr mjǫk ok ætlaði at láta meiða eða drepa ossa landa fyrir, þá er þar váru austr.
En þat sumar it sama kómu útan heðan þeir Gizurr ok Hjalti ok þágu þá undan við konunginn ok hétu honum umbsýslu sinni til á nýjaleik, at hér yrði enn við kristninni tekit, ok létu sér eigi annars ván en þar myndi hlýða.
En it næsta sumar eftir fóru þeir austan ok prestr sá, er Þormóðr hét, ok kómu þá í Vestmannaeyjar, er tíu vikur váru af sumri, ok hafði allt farizt vel at. Svá kvað Teitr þann segja, er sjálfr var þar.
Þá var þat mælt it næsta sumar áðr í lǫgum, at menn skyldi svá koma til alþingis, er tíu vikur væru af sumri, en þangat til kómu viku fyrr.
En þeir fóru þegar inn til meginlands ok síðan til alþingis ok gátu at Hjalta, at hann var eftir í Laugardali með tólfta mann, af því at hann hafði áðr sekr orðit fjǫrbaugsmaðr it næsta sumar á alþingi of goðgá. En þat var til þess haft, at hann kvað at Lǫgbergi kviðling þenna:

Vilk eigi goð geyja.
Grey þykki mér Freyja.

En þeir Gizurr fóru, unz þeir kómu í stað þann í hjá ǫlfossvatni, er kallaðr er Vellankatla, ok gerðu orð þaðan til þings, at á mót þeim skyldi koma allir fulltingsmenn þeira, af því at þeir hǫfðu spurt, at andskotar þeira vildi verja þeim vígi þingvǫllinn. En fyrr en þeir færi þaðan, þá kom þar ríðandi Hjalti ok þeir, er eftir váru með honum.
En síðan riðu þeir á þingit, ok kómu áðr á mót þeim frændr þeira ok vinir, sem þeir hǫfðu æst. En inir heiðnu menn hurfu saman með alvæpni, ok hafði svá nær, at þeir myndi berjast, at eigi of sá á milli.
En annan dag eftir gingu þeir Gizurr ok Hjalti til Lǫgbergis ok báru þar upp erendi sín. En svá er sagt, at þat bæri frá, hvé vel þeir mæltu. En þat gerðist af því, at þar nefndi annarr maðr at ǫðrum vátta, ok sǫgðust hvárir ór lǫgum við aðra, inir kristnu menn ok inir heiðnu, ok gingu síðan frá Lǫgbergi.
Þá báðu inir kristnu menn Hall á Síðu, at hann skyldi lǫg þeira upp segja, þau er kristninni skyldi fylgja. En hann leystist því undan við þá, at hann keypti at Þorgeiri lǫgsǫgumanni, at hann skyldi upp segja, en hann var enn þá heiðinn.
En síðan er menn kómu í búðir, þá lagðist hann niðr Þorgeirr ok breiddi feld sinn á sik ok hvílði þann dag allan ok nóttina eftir ok kvað ekki orð. En of morguninn eftir settist hann upp ok gerði orð, at menn skyldi ganga til Lǫgbergis.
En þá hóf hann tǫlu sína upp, er menn kómu þar, ok sagði, at honum þótti þá komit hag manna í ónýtt efni, ef menn skyldi eigi hafa allir lǫg ein á landi hér, ok talði fyr mǫnnum á marga vega, at þat skyldi eigi láta verða, ok sagði, at þat myndi at því ósætti verða, er vísaván var, at þær barsmíðir gerðust á milli manna, er landit eyddist af. Hann sagði frá því, at konungar ór Norvegi ok ór Danmǫrku hǫfðu haft ófrið ok orrostur á milli sín langa tíð, til þess unz landsmenn gerðu frið á milli þeira, þótt þeir vildu eigi. En þat ráð gerðist svá, at af stundu sendust þeir gersemar á milli, enda hélt friðr sá, meðan þeir lifðu. "En nú þykkir mér þat ráð," kvað hann, "at vér látim ok eigi þá ráða, er mest vilja í gegn gangast, ok miðlum svá mál á milli þeira, at hvárirtveggju hafi nakkvat síns máls, ok hǫfum allir ein lǫg ok einn sið. Þat mun verða satt, er vér slítum í sundr lǫgin, at vér munum slíta ok friðinn.
En hann lauk svá máli sínu, at hvárirtveggju játtu því, at allir skyldi ein lǫg hafa, þau sem hann réði upp at segja.
Þá var þat mælt í lǫgum, at allir menn skyldi kristnir vera ok skírn taka, þeir er áðr váru óskírðir á landi hér. En of barnaútburð skyldu standa in fornu lǫg ok of hrossakjǫtsát. Skyldu menn blóta á laun, ef vildu en varða fjǫrbaugsgarðr, er váttum of kæmi við. En síðar fám vetrum var sú heiðni af numin sem ǫnnur.
Þenna atburð sagði Teitr oss at því, er kristni kom á Ísland. En Óláfr Tryggvason fell it sama sumar at sǫgu Sæmundar prests. Þá barðist hann við Svein Haraldsson Danakonung ok Óláf inn sænska, Eiríksson at Uppsǫlum Svíakonungs, ok Eirík, er síðan var jarl at Norvegi, Hákonarson. Þat var hundrað ok þremr tigum vetra eftir dráp Eadmundar, en þúsund eftir burð Krists at alþýðu tali.


8. Frá byskupum útlendum.

Þessi eru nǫfn byskupa þeira, er verit hafa á Íslandi útlendir at sǫgu Teits: Friðrekr kom í heiðni hér, en þessir váru síðan: Bjarnharðr inn bókvísi fimm ár, Kolr fá ár, Hróðólfr nítján ár, Jóhan inn írski fá ár, Bjarnharðr nítján ár, Heinrekr tvau ár.
Enn kómu hér aðrir fimm, þeir er byskupar kváðust vera: ǫrnólfr ok Goðiskálkr ok þrír ermskir: Pétrús ok Ábrahám ok Stéphánús.
Grímr at Mosfelli Svertingssonr tók lǫgsǫgu eftir Þorgeir ok hafði tvau sumur, en þá fekk hann lof til þess, at Skafti Þóroddsonr hefði, systursonr hans, af því at hann var hásmæltr sjálfr.
Skafti hafði lǫgsǫgu tuttugu og sjau sumur. Hann setti fimmtardómslǫg ok þat, at engi vegandi skyldi lýsa víg á hendr ǫðrum manni en sér, en áðr váru hér slík lǫg of þat sem í Norvegi. Á hans dǫgum urðu margir hǫfðingjar ok ríkismenn sekir eða landflótta of víg eða barsmíðir af ríkis sǫkum hans ok landstjórn. En hann andaðist á inu sama ári ok Óláfr inn digri fell Haraldssonr, Goðrǫðarsonar, Bjarnarsonar, Haraldssonar ins hárfagra, þremr tigum vetra síðar en Óláfr felli Tryggvasonr.
Þá tók Steinn Þorgestissonr lǫgsǫgu ok hafði þrjú sumur. Þá hafði Þorkell Tjǫrvasonr tuttugu sumur. Þá hafði Gellir Bǫlverkssonr níu sumur.


9. Frá Ísleifi byskupi.

Ísleifr Gizurarsonr ins hvíta var vígðr til byskups á dǫgum Haralds Norvegskonungs Sigurðarsonar, Hálfdanarsonar, Sigurðarsonar hrísa, Haraldssonar ins hárfagra. En er þat sá hǫfðingjar ok góðir menn, at Ísleifr var miklu nýtri en aðrir kennimenn, þeir er á þvísa landi næði, þá seldu honum margir sonu sína til læringar ok létu vígja til presta. Þeir urðu síðan vígðir tveir til byskupa, Kollr, er var í Vík austr, ok Jóan at Hólum.
Ísleifr átti þrjá sonu. Þeir urðu allir hǫfðingjar nýtir, Gizurr byskup ok Teitr prestr, faðir Halls, ok Þorvaldr. Teit fæddi Hallr í Haukadali, sá maðr, er þat var almælt, at mildastr væri ok ágæztr at góðu á landi hér ólærðra manna. Ek kom ok til Halls sjau vetra gamall, vetri eftir þat, er Gellir Þorkelssonr, fǫðurfaðir minn ok fóstri, andaðist, ok vark þar fjórtán vetr.
Gunnar inn spaki hafði tekit lǫgsǫgu, þá er Gellir lét af, ok hafði þrjú sumur. Þá hafði Kolbeinn Flosason sex. Þat sumar, er hann tók lǫgsǫgu, fell Haraldr konungr á Englandi. Þá hafði Gellir í annat sinn þrjú sumur. Þá hafði Gunnarr í annat sinn eitt sumar. Þá hafði Sighvatr Surtssonr, systursonr Kolbeins, átta.
Á þeim dǫgum kom Sæmundr Sigfússonr sunnan af Frakklandi hingat til lands ok lét síðan vígjast til prests.
Ísleifr var vígðr til byskups, þá er hann var fimmtǫgr. Þá var Leó septimus páfi. En hann var inn næsta vetr í Norvegi ok fór síðan út hingat. En hann andaðist í Skálaholti, þá er hann hafði alls verit byskup fjóra vetr ok tuttugu. Svá sagði Teitr oss. Þat var á dróttins degi sex náttum eftir hátíð þeira Pétars ok Páls, átta tigum vetra eftir Óláfs fall Tryggvasonar.
Þar var ek þá með Teiti, fóstra mínum, tólf vetra gamall. En Hallr sagði oss svá, er bæði var minnigr ok ólyginn ok munði sjálfr þat, er hann var skírðr, at Þangbrandr skírði hann þrévetran, en þat var vetri fyrr en kristni væri hér í lǫg tekin. En hann gerði bú þrítǫgr ok bjó sex tigi ok fjóra vetr í Haukadali ok hafði níu tigi ok fjóra vetr, þá er hann andaðist, en þat var of hátíð Martens byskups á inum tíunda vetri eftir andlát Ísleifs byskups.


10. Frá Gizuri byskupi.

Gizurr byskup, sonr Ísleifs, var vígðr til byskups at bæn landsmanna á dǫgum Óláfs konungs Haraldssonar, tveim vetrum eftir þat, er Ísleifr andaðist. Þann var hann annan hér á landi, en annan á Gautlandi. En þá var nafn hans rétt, at hann hét Gisrǫðr. Svá sagði hann oss.
Markús Skeggjasonr hafði lǫgsǫgu næstr Sighvati ok tók þat sumar, er Gizurr byskup hafði einn vetr verit hér á landi, en fór með fjǫgur sumur ok tuttugu. At hans sǫgu er skrifuð ævi allra lǫgsǫgumanna á bók þessi, þeira er váru fyr várt minni, en honum sagði Þórarinn, bróðir hans, ok Skeggi faðir þeira, ok fleiri spakir menn til þeira ævi, er fyr hans minni váru, at því er Bjarni inn spaki hafði sagt, fǫðurfaðir þeira, er munði Þórarin lǫgsǫgumann ok sex aðra síðan.
Gizurr byskup var ástsælli af ǫllum landsmǫnnum en hverr maðr annarra, þeira er vér vitim hér á landi hafa verit. Af ástsælð hans ok tǫlum þeira Sæmundar, með umbráði Markús lǫgsǫgumanns, var þat í lǫg leitt, at allir menn tǫlðu ok virtu allt fé sitt ok sóru, at rétt virt væri, hvárt sem var í lǫndum eða í lausaaurum, ok gerðu tíund af síðan. Þat eru miklar jartegnir, hvat hlýðnir landsmenn váru þeim manni, er hann kom því fram, at fé allt var virt með svardǫgum, þat er á Íslandi var, ok landit sjálft ok tíundir af gervar ok lǫg á lǫgð, at svá skal vera, meðan Ísland er byggt.
Gizurr byskup lét ok lǫg leggja á þat, at stóll byskups þess, er á Íslandi væri, skyldi í Skálaholti vera, en áðr var hvergi, ok lagði hann þar til stólsins Skálaholtsland ok margra kynja auðæfi ǫnnur bæði í lǫndum ok lausum aurum.
En þá er honum þótti sá staðr hafa vel at auðæfum þróazt, þá gaf hann meir en fjórðung byskupsdóms síns til þess, at heldr væru tveir byskupsstólar á landi hér en einn, svá sem Norðlendingar æstu hann til. En hann hafði áðr látit telja búendr á landi hér, ok váru þá í Austfirðingafjórðungi sjau hundruð heil, en í Rangæingafjórðungi tíu, en í Breiðfirðingafjórðungi níu, en í Eyfirðingafjórðungi tólf, en ótalðir váru þeir, er eigi áttu þingfararkaupi at gegna of allt Ísland.
Úlfheðinn Gunnarssonr ins spaka tók lǫgsǫgu eftir Markús ok hafði níu sumur, þá hafði Bergþórr Hrafnssonr sex, en þá hafði Goðmundr Þorgeirssonr tólf sumur.
It fyrsta sumar, er Bergþórr sagði lǫg upp, var nýmæli þat gert, at lǫg ór skyldi skrifa á bók at Hafliða Mássonar of vetrinn eftir at sǫgu ok umbráði þeira Bergþórs ok annarra spakra manna, þeira er til þess váru teknir. Skyldu þeir gerva nýmæli þau ǫll í lǫgum, er þeim litist þau betri en hin fornu lǫg. Skyldi þau segja upp it næsta sumar eftir í lǫgréttu ok þau ǫll halda, er inn meiri hlutr manna mælti þá eigi gegn. En þat varð at framfara, at þá var skrifaðr Vígslóði ok margt annat í lǫgum ok sagt upp í lǫgréttu af kennimǫnnum of sumarit eftir. En þat líkaði ǫllum vel, ok mælti því manngi í gegn.
Þat var ok it fyrsta sumar, er Bergþórr sagði lǫg upp, þá var Gizurr byskup óþingfærr af sótt. Þá sendi hann orð til alþingis vinum sínum ok hǫfðingjum, at biðja skyldi Þorlák Runólfsson, Þorleikssonar, bróður Halls í Haukadali, at hann skyldi láta vígjast til byskups. En þat gerðu allir svá sem orð hans kómu til, ok fekkst þat af því, at Gizurr hafði sjálfr fyrr mjǫk beðit, ok fór hann útan þat sumar, en kom út it næsta eftir ok var þá vígðr til byskups.
Gizurr var vígðr til byskups, þá er hann var fertǫgr. Þá var Grégóríús septimus páfi. En síðan var hann inn næsta vetr í Danmǫrku ok kom of sumarit eftir hingat til lands. En þá er hann hafði verit tuttugu ok fjóra vetr byskup, svá sem faðir hans, þá var Jóan ǫgmundarsonr vígðr til byskups fyrstr til stóls at Hólum. Þá var hann vetri miðr en hálfsextǫgr. En tólf vetrum síðar, þá er Gizurr hafði alls verit byskup þrjá tigi ok sex vetr, þá var Þorlákr vígðr til byskups. Hann lét Gizurr vígja til stóls í Skálaholti at sér lifanda. Þá var Þorlákr tveim vetrum meir en þrítǫgr, en Gizurr byskup andaðist þrjátigi náttum síðar í Skálaholti á inum þriðja degi í viku fimmta kalend. Junii. [Það er 28. mai.]
Á því ári inu sama andaðist Páschalis secundus páfi fyrr en Gizurr byskup ok Baldvini Jórsalakonungr ok Arnaldus patriarcha í Híerúsalem ok Philippús Svíakonungr, en síðar it sama sumar Álexíús Grikkjakonungr. Þá hafði hann þrjátigi ok átta vetr setit at stóli í Miklagarði. En tveim vetrum síðar varð aldamót. Þá hǫfðu þeir Eysteinn ok Sigurðr verit sautján vetr konungar í Norvegi eftir Magnús, fǫður sinn, Óláfsson, Haraldssonar. Þat var hundrað ok tuttugu vetrum eftir fall Óláfs Tryggvasonar, en tvau hundruð ok fimmtigi eftir dráp Eadmundar Englakonungs, en fimm hundruð ok sextán vetrum eftir andlát Grégóríús páfa, þess er kristni kom á England, at því er talit er. En hann andaðist á ǫðru ári konungdóms Fócó keisara, sex hundruð ok fjórum vetrum eftir burð Krists at almannatali. Þat verðr allt saman ellifu hundruð ok tuttugu ár.
Hér lýkst sjá bók.

Bókarauki.

11. Byskupaættir.

Þetta er kyn byskupa Íslendinga ok áttartala:
Ketilbjǫrn landnámsmaðr, sá er byggði suðr at Mosfelli inu efra, var faðir Teits, fǫður Gizurar ins hvíta, fǫður Ísleifs, er fyrstr var byskup í Skálaholti, fǫður Gizurar byskups.    Hrollaugr landnámsmaðr, er byggði austr á Síðu á Breiðabólstað, var faðir ǫzurar, fǫður Þórdísar, móður Halls á Síðu, fǫður Egils, fǫður Þorgerðar, móður Jóans, er fyrstr var byskup at Hólum.
Auðr landnámskona, er byggði vestr í Breiðafirði í Hvammi, var móðir Þorsteins ins rauða, fǫður Óleifs feilans, fǫður Þórðar gellis, fǫður Þórhildar rjúpu, móður Þórðar hesthǫfða, fǫður Karlsefnis, fǫður Snorra, fǫður Hallfríðar, móður Þorláks, er nú er byskup í Skálaholti, næstr Gizuri.
Helgi inn magri landnámamaðr, sá er byggði norðr í Eyjafirði í Kristnesi, var faðir Helgu, móður Einars, fǫður Eyjólfs Valgerðarsonar, fǫður Goðmundar, fǫður Eyjólfs, fǫður Þorsteins, fǫður Ketils, er nú er byskup at Hólum, næstr Jóani.


12. Langfeðgatal.

Þessi eru nǫfn langfeðga Ynglinga og Breiðfirðinga:
i Yngvi Tyrkjakonungr. ii Njǫrðr Svíakonungr. iii Freyr. iiii Fjǫlnir. sá er dó at Friðfróða. v Svegðir. vi Vanlandi. vii Visburr. viii Dómaldr. ix Dómarr. x Dyggvi. xi Dagr. xii Alrekr. xiii Agni. xiiii Yngvi. xv Jǫrundr. xvi Aun inn gamli. xvii Egill Vendilkráka. xviii Óttarr. xix Aðísl at Uppsǫlum. xx Eysteinn. xxi Yngvarr. xxii Braut-ǫnundr. xxiii Ingjaldr inn illráði. xxiiii Óláfr trételgja. xxv Hálfdan hvítbeinn Upplendingakonungr. xxvi Goðrǫðr. xxvii Óláfr. xxviii Helgi. xxix Ingjaldr, dóttursonr Sigurðar, Ragnarssonar loðbrókar. xxx Óleifr inn hvíti. xxxi Þorsteinn inn rauði. xxxii Óleifr feilan, er fyrstr byggði þeira á Íslandi. xxxiii Þórðr gellir. xxxiiii Eyjólfr, er skírðr var í elli sinni, þá er kristni kom á Ísland. xxxv Þorkell. xxxvi Gellir, faðir þeira Þorkels, fǫður Brands, ok Þorgils, fǫður míns, en ek heitik Ari.