\chapter{巴德尔之死}
巴德尔(古诺尔斯语Baldr)是北欧神话中的光明之神,奥丁和弗丽嘉的儿子。关于巴德尔的记载不多,主要是关于他的死亡的。巴德尔的死为什么在北欧神话中如此重要?主要的解读是,由于巴德尔是光明的神,象征着春天与喜悦,巴德尔的死无疑象征着黑暗和寒冬的统治,这在北欧人看来是极为可怖的。从北欧神话的记载来看,巴德尔的死也间接地引发了诸神最后的战争。

巴德尔的死出现在多个故事当中,包括《女巫的预言》(Vǫluspá),《洛基的争吵》(Lokasenna),《巴德尔之梦》(Baldrs draumar)以及《欺骗古洛菲》(Gylfaginning)等,不同的故事中的记载也不尽相同,综合来看,大致的内容是这样的:

巴德尔身形俊美,通身洁白,是诸神中最漂亮的,因此备受母亲弗丽嘉的喜爱。巴德尔后来做噩梦感到死亡的阴影,于是弗丽嘉跑遍整个世界,使万物发誓不会伤害巴德尔。唯独一颗槲寄生没有发誓,因为它年龄太小,弗丽嘉认为它不足以发誓也不至于伤人。诸神得知此事,颇为巴德尔感到高兴,他们试着向巴德尔投掷武器,果然无法伤到他,洛基一向嫉妒巴德尔的容貌和受宠,因此化作妇人从弗丽嘉那里套来了槲寄生没有发誓的消息。洛基让槲寄生变成一把长剑,让巴德尔目盲的兄弟霍德尔(Hǫðr)投向巴德尔,巴德尔果然倒地身亡。巴德尔死后,他的妻子南娜(Nanna)因悲伤过度,心力衰竭而死,诸神将他们二人葬在一艘大船上送入水中。巴德尔死后,弗丽嘉请求赫尔莫德前往死者之国复活巴德尔,死者之神赫尔(Hel)开出条件,倘若一切有生命和无生命的都替巴德尔哭泣,她就把巴德尔还回来。铁石心肠的女巨人Thǫkk(一说是洛基变的)住在地下,不需要光明,因此不愿为巴德尔哭泣,巴德尔于是无法复活。奥丁早在巴德尔死前就已得知他死亡的预言,也知道自己要生下一个儿子为巴德尔复仇,奥丁引诱女巨人琳达尔(Rindr)并诞下一子瓦利(Váli)在诸神的决战时杀死霍德尔。洛基也由于其罪恶被缚在石头上,直到最后的决战时才被释放,洛基也在诸神黄昏时驾驶纳格尔法(Nalgfar,满载反叛的亡灵的船)挑起对阿萨神族的战争。

\section{巴德尔之梦(Baldrs draumar)选读}
《巴德尔之梦》是一首埃达短诗,全文共十四节,这里选择了从第四节开始的内容。前三节的故事是这样的:众神得知巴德尔经常做噩梦,聚在一起讨论这件怪事的成因。奥丁知道死者之国里埋葬着一位颇有智慧的女巫,便前去和她交谈。
\begin{paracol}{2}
  \noindent
  $^4$ Þá reið Óðinn fyrir austan dyrr\\
  \textit{\MakeUppercase Then rode Odin forth eastern door}\\
  \MakeUppercase þar er hann vissi vǫlu leiði\\
  \textit{\MakeUppercase there where he knew volva's grave}\\
  \MakeUppercase nam hann vittugri valgaldr kveða\\
  \textit{\MakeUppercase did he skillful magic speak}\\
  \MakeUppercase unz nauðig reis, nás\index{nár!nás} orð of kvað\\
  \textit{\MakeUppercase Until unwillingly rose, of corpse word about spoke}\\
  \switchcolumn
  \noindent
  Then Odin rode to the eastern door\\
  于是奥丁策马驰向东门\\
  Where he knew Was the volva's grave\\
  他知道有女巫安息此处\\
  Great charm did he speak\\
  奥丁念念有词施展法术\\
  Until she rose bound by spell and in death she spoke\\
  直到她苏醒过来开口说\\
\end{paracol}
\begin{grammar*}{}
  \begin{enumerate}[leftmargin=*]
    \item fyrir austan dyrr

          fyrir + -an型副词+宾格名词的典型用法,表示“沿着某个方向前往...”。注意dyrr `door'是一个特殊的辅音词干阴性强名词,只有复数式。其变形为:主语和宾格dyrr, 与格durum或dyrum, 属格dura或dyra.

    \item nás orð

          nás是nár的属格,字面义为`word of corpse'. 由于奥丁用通灵术复活女巫问卜,因此说女巫说出的是“死尸的话”。
  \end{enumerate}
\end{grammar*}

\begin{paracol}{2}
  \noindent
  $^5$ Hvat er manna þat mér ókunnra\\
  \textit{What is of men that to me unknown}\\
  Er mér hefir aukit erfitt sinni\\
  \textit{Who to me had passed difficult journey}\\
  Var ek snivin\index{sníva(snjóva)!snivin} snævi\index{snær!snævi} \\
  \textit{Was I snowed with snow}\\
  Ok slegin\index{slá!slegin} regni ok drifin\index{drífa!drifin} dǫggu \\
  \textit{And struck with rain and besprent with dew}\\
  Dauð var ek lengi\\
  \textit{Dead was I for long}
  \switchcolumn
  \noindent
  Who is the man that I do not know\\
  是哪个陌生人唤醒我\\
  Who had made me travel a troublesome road\\
  叫我一路奔波不安宁\\
  I was covered by snow\\
  大雪茫茫盖我身\\
  Struck by rain and drenched with dew\\
  风雨飘飘湿我衣\\
  I have long been dead\\
  死者竟要遭罪多\\
\end{paracol}
\begin{grammar*}{}
  \begin{enumerate}[leftmargin=*]
    \item hefir aukit mér erfitt sinni

          auka表示“增加”,可以接双宾语,类似于英文中`add sth. to sb.',这样,auka的意思就和表示“给予”的动词类似。这句话的直译是`give me a difficult journey'.

    \item snivin snævi; slegin regni; drifin dǫggu

          这里的与格表示方式,结构和被动句十分类似。在主动结构中,这里的与格名词都可以做动词的宾格宾语。另注意slegin是稍不规则的动词slá的分词。
  \end{enumerate}
\end{grammar*}

\begin{paracol}{2}
  \noindent
  $^6$ Óðinn kvað: Vegtamr\footnotemark ek heiti\\
  \footnotetext{奥丁的别名Vegtamr意思是“准备好行路的”(way-ready),和其在神话中经常漫游的形象相符合。他父亲的名字也有一样的结构,可能表示“准备好(掌管)死者的”。}
  \textit{Odin spoke: Vegtamr I am called}\\
  Sonr em ek Valtams; segðu\index{segja!segðu} mér ór helju\\
  \textit{Son am I of Valtamr; speak to me of Hel}\\
  Ek mun ór heimi: hveim eru bekkir baugum sánir\index{strá!?sánir}\\
  \textit{I know of world: for whom are benches with rings strewn}\\
  Flet fagrlig flóuð\index{flóa!flóuð} gulli\\
  \textit{Rooms beautiful flooded with gold}\\
  \switchcolumn
  \noindent
  Odin spoke: Vegtamr is my name\\
  奥丁说:我的名字叫作维格坦\\
  I am the son of Valtamr; speak to me about Hel\\
  瓦尔坦是我的父亲;告诉我冥府的故事吧\\
  I know about the world: the benches strewn with rings\\
  人间的事我无一不晓:金碧辉煌的长凳\\
  And room bedecked with gold are prepared for whom  \\
  和流光溢彩的居室是为了留给什么人?\\
\end{paracol}

在这一节中,奥丁首先向女巫发问,目的是试探女巫的智慧。其问题的答案自然是巴德尔本人,因为他备受诸神的宠爱。


\begin{grammar*}{}
  \begin{enumerate}[leftmargin=*]
    \item segðu mér ór helju

          segja ór中的ór表示整体中的一部分,也常和表示言说的动词连用。下面的mun ór也是类似的用法。

    \item baugum sánir

          sánir情况不明,按Cleasby字典中的说法,有可能是stráðir的误写。不过,从语义上来看这个动词无疑应该表示“镶嵌、装饰”之类的含义。

    \item flóuð gulli

          flóuð是动词弱动词flóa `flood'的过去分词的中性复数形式,与同为复数的中性名词flet保持一致。
  \end{enumerate}
\end{grammar*}

\newpage
\begin{paracol}{2}
  \noindent
  $^7$ Vǫlva kvað: Hér stendr Baldri of brugginn mjǫðr\\
  \textit{Volva spoke: Here stands for Balder brewed mead}\\
  Skírar veigar, liggr skjǫldr yfir\\
  \textit{Clean drink, lies shield over}\\
  En ásmegir\index{mǫgr!megir} í ofvæni\\
  \textit{But Aesir-sons in over-expectation}\\
  Nauðug sagðak, nú mun ek þegja\\
  \textit{Unwillingly I spoke, now must I be silent}\\
  \switchcolumn
  \noindent
  Volva spoke: Here stands brewed mead for Balder\\
  女巫说:冥府的密酒献给巴德尔\\
  A clean drink with a shield over it\\
  澄澈清冽的美酒,密封在罐子中\\
  But Aesir will wait in anxious suspense\\
  但神族望眼欲穿等待巴德尔归来\\
  Unwillingly I spoke, now I must cease\\
  我本该守口如瓶,就到此为止吧
\end{paracol}
\begin{grammar*}{}
  \begin{enumerate}[leftmargin=*]
    \item stendr Baldri of

          这里的of的位置比较罕见,一般来说应该在Baldri前面。of表示轻微的因果关系,可以理解为`for'.

    \item ásmegir

          ás-mǫgr `Ase-son'的复数主格形式。
    \item ofvæni

          of-væni `over-weening', of偶尔也用在前缀中,表示`over'.
  \end{enumerate}
\end{grammar*}
\begin{paracol}{2}
  \noindent
  $^8$ Óðinn kvað: Þegjattu, vǫlva, þik vil ek fregna\\
  \textit{Odin spoke: Do not be silent, volva, you want I ask}\\
  Unz alkunna, vil ek enn vita\\
  \textit{Until all-known, want I fain know}\\
  Hverr mun Baldri at bana verða\\
  \textit{Who will Baldr to death turn}\\
  Ok Óðins son aldri ræna\\
  \textit{And Odin's son of age rob}\\
  \switchcolumn
  \noindent
  Odin spoke: Cease not, you volva, I seek from you\\
  奥丁说:女巫啊,切莫缄口不语\\
  I will fain ask from you, until I am all-known\\
  许多事我不曾知晓,愿向你请教\\
  Who shall put Baldr to death\\
  是谁把巴德尔推向死亡?\\
  And steal Odin's son of life\\
  是谁要奥丁爱子的性命?\\
\end{paracol}
\begin{grammar*}{}
  \begin{enumerate}[leftmargin=*]
    \item Baldri at bana verða

          verða接与格宾语的用法,字面意思`turn Baldr to death',另见Hávamál第5节语法。

    \item Óðins son aldri ræna

          ræna `rob'接双宾语结构,用与格表示抢走的东西,用宾格表示从谁手中抢走,类似于英文中`rob sb. of sth.'结构,但与格名词替代了英语的of短语。
  \end{enumerate}
\end{grammar*}

\begin{paracol}{2}
  \noindent
  $^9$ Vǫlva kvað: Hǫðr berr hávan hróðrbaðm þinig\\
  \textit{Volva spoke: Hodr bears high famous-branch from there}\\
  Hann mun Baldri at bana verða\\
  \textit{He shall Baldr to death turn}\\
  Ok Óðins son aldri ræna\\
  \textit{And Odin's son of age rob}\\
  Nauðug sagðak\index{segja!sagðak}, nú mun ek þegja\\
  \textit{Unwillingly I spoke, now must I be silent}\\
  \switchcolumn
  \noindent
  Volva spoke: Hodr will bear the high branch from there\\
  女巫说:霍德尔拿起槲寄生\\
  He shall turn Baldr to death\\
  他把巴德尔推向死亡\\
  And steal Odin's son of life\\
  夺去奥丁爱子的性命\\
  Unwillingly I spoke, now I must cease\\
  我本该守口如瓶,就到此为止吧
\end{paracol}
\begin{grammar*}{}
  \begin{enumerate}[leftmargin=*]
    \item hróðrbaðm

          由hróðr+baðmr构成,字面意思`famous tree/branch',结合其他文献记载,这指的就是夺走巴德尔生命的槲寄生。
  \end{enumerate}
\end{grammar*}

\begin{paracol}{2}
  \noindent
  $^{10}$ Óðinn kvað: Þegjattu, vǫlva, þik vil ek fregna\\
  \textit{Odin spoke: Do not be silent, volva, you want I ask}\\
  Unz alkunna, vil ek enn vita\\
  \textit{Until all-known, want I fain know}\\
  Hverr mun heift Heði hefnt of vinna\\
  \textit{Who shall war to Hodr revenged wage}\\
  Eða Baldrs bana á bál vega\\
  \textit{Or Baldr's bane to flames smite}\\
  \switchcolumn
  \noindent
  Odin spoke: Cease not, you volva, I seek from you\\
  奥丁说:女巫啊,切莫缄口不语\\
  I will fain ask from you, until I am all-known\\
  许多事我不曾知晓,愿向你请教\\
  Who shall wage avenging war against Hodr\\
  谁会为向霍德尔寻求报复\\
  Or cast flames to Baldr's killer\\
  用烈火杀死巴德尔的凶手

\end{paracol}
\begin{grammar*}{}
  \begin{enumerate}[leftmargin=*]
    \item heift Heði hefnt of vinna

          这里的of是小品词。vinna heift+与格宾语表示`wage war against one'. hefnt, hefna `avenge'的分词,这和vinna的另一个用法有关。vinna+宾格宾语+过去分词类似于`make something done', 故也有vinna e-n heift `take revenge'的说法。
    \item á bál

          bál的本意是火葬用的柴堆,即英文中的`pyre', á bál `on pyre'的实际意思就是“用火杀死”。
  \end{enumerate}
\end{grammar*}

\begin{paracol}{2}
  \noindent
  $^{11}$ Vǫlva kvað: Rindr berr Vála í vestrsǫlum\\
  \textit{Volva spoke: Rind bears Vali in west-hall}\\
  Sá mun Óðins sonr einnættr vega\\
  \textit{That shall Odin's son one-night kill}\\
  Hǫnd of þvær\index{þvá!þvær} né hǫfuð kembir\\
  \textit{Hand washes nor hair combs}\\
  Áðr á bál of berr Baldrs andskota\\
  \textit{Before to fire bears Baldr's enemy}\\
  Nauðug sagðak, nú mun ek þegja\\
  \textit{Unwillingly I spoke, now must I be silent}\\
  \switchcolumn
  \noindent
  Volva spoke: Rindr bears Vali in West hall\\
  女巫说:琳达尔在西边厅堂里诞下瓦利\\
  Who will kill Odin's son one night old\\
  他一夜间长大,便手刃奥丁之子\\
  He washes not his hand nor combs his hair\\
  他无暇洗净双手,也不梳理头发\\
  Before he casts flame to Baldr's enemy\\
  直到他用烈火烧死巴德尔的死敌\\
  Unwillingly I spoke, now I must cease\\
  我本该守口如瓶,就到此为止吧\\
\end{paracol}
\begin{grammar*}{}
  \begin{enumerate}[leftmargin=*]
    \item einnættr

          -nættr常构成形容词后缀。这里几乎像副词一样使用,修饰的是vega. 瓦利就是奥丁专门生下为巴德尔复仇的,他唯一的目的就是杀死霍德尔。有说法认为日耳曼人不允许父子相残,因此奥丁必须另生一子替其复仇。
  \end{enumerate}
\end{grammar*}

\begin{paracol}{2}
  \noindent
  $^{12}$ Óðinn kvað: Þegjattu, vǫlva, þik vil ek fregna\\
  \textit{Odin spoke: Do not be silent, volva, you want I ask}\\
  Unz alkunna, vil ek enn vita\\
  \textit{Until all-known, want I fain know}\\
  Hverjar ru þær meyjar er at muni\index{munr!muni} gráta\\
  \textit{Who are those maidens who that in longing weep}\\
  Ok á himin verpa halsa skautum\\
  \textit{And to sky toss neck's sheet}\\
  \switchcolumn
  \noindent
  Odin spoke: Cease not, you volva, I seek from you\\
  奥丁说:女巫啊,切莫缄口不语\\
  I will fain ask from you, until I am all-known\\
  许多事我不曾知晓,愿向你请教\\
  Who are those maidens that shall weep heartily\\
  是哪几个姑娘为巴德尔动容\\
  And toss the ship to the sky\\
  又掀起波浪将帆船抛向天空

\end{paracol}
\begin{grammar*}{}
  \begin{enumerate}[leftmargin=*]
    \item ru

          vera的某些形式,例如eru在早期诗歌中脱去词首的元音,变为ru.

    \item verpa halsa skautum

          verpa `throw'接续与格。halsa-skaut,`neck-sheet'是一个航海的术语,表示风帆的上角。这里大概用来代指船。这里说的姑娘是海神埃吉尔(Ægir)的女儿,她们为巴德尔哭泣,以至于卷起风浪。
  \end{enumerate}
\end{grammar*}

\begin{paracol}{2}
  \noindent
  $^{13}$ Vǫlva kvað: Ertattu Vegtamr sem ek hugða\\
  \textit{Volva spoke: You are not Vegtam, as I yought}\\
  Heldr ertu Óðinn, aldinn gautr\\
  \textit{Rather You are Odin, the old sorcerer}\\
  Óðinn kvað: Ertattu vǫlva né víss kona\\
  \textit{Odin spoke: You are not a volva nor wise woman}\\
  Heldr ertu þriggja þursa móðir\footnotemark\\
  \textit{Rather You are three giants' mother}
  \footnotetext{第13节中同时出现了女巫和奥丁的话,这和前文的习惯不同,可能是由两节合并而来。在本节中,女巫发现了奥丁的身份并戳穿了他。奥丁的回应比较耐人寻味,除了回击女巫外,奥丁还指出了她的身份:三个巨人(þurs)的母亲。巨人一族在古诺尔斯语中有几个说法,包括þurs, risi, j\k{o}tunn, troll等,它们的含义有些许不同。þurs强调他们愚蠢,risi强调他们身形巨大,j\k{o}tunn强调力量过人。本段中的þursa móðir无疑强调女巫蠢笨。女巫的身份并不清楚,三个巨人可能指的是命运三女神(Norn),但三女神的母亲在神话中鲜有提及。也有人认为三个巨人指的是洛基和巨人安格尔博达(Angrboða)生下的三名凶恶的子女:恶狼芬里尔(Fenrir),巨蛇耶梦加得(J\k{o}rmungandr)和司掌冥府的死神赫尔(Hel),考虑到洛基和巴德尔死亡的直接关系,这一说法似乎更有道理。}
  \switchcolumn
  \noindent
  Volva spoke: You are not Vegtam, as I yought\\
  女巫说:正如我所想,你果不是维格坦\\
  Odin You are, the old sorcerer\\
  倒不如说你名叫奥丁,老谋深算的巫师\\
  Odin spoke: You are not a volva nor wise woman\\
  奥丁说:你亦不是女巫,也算不上聪明\\
  You are the mother of three giants\\
  你是三个巨人的母亲
\end{paracol}


\begin{paracol}{2}
  \noindent
  $^{14}$ Vǫlva kvað: Heim ríð þú, Óðinn, ok ver hróðigr\\
  Volva spoke: Home ride you, Odin, and be proud\\
  Svá komir manna meir aftr á vit\\
  \textit{That come men more back towards}\\
  Er lauss Loki líðr ór bǫndum\\
  \textit{When loose Loki wanders out of bonds} \\
  Ok ragnarǫk rjúfendr koma\\
  \textit{And gods-fate destroyers come}

  \switchcolumn
  \noindent
  Volva spoke: Ride home, Odin, and be proud\\
  女巫说:回去吧,奥丁,任你得意忘形\\
  That (no) one shall seek me more\\
  此地无人将再来\\
  Until Loki wanders free from his chains\\
  洛基要从锁链中挣脱\\
  And destroyers come in Ragnarok\\
  众神的末日即将降临
\end{paracol}
\begin{grammar*}{}
  \begin{enumerate}[leftmargin=*]
    \item komir manna meir aftr á vit

          á vit+属格/宾格名词表示“朝着...”,这里与之搭配的是manna meir。koma á vit于是表示“相遇”,注意到komir是第二人称的虚拟式,主语是奥丁。这句话指的是奥丁回去的路上不会遇到任何人了(You would meet no more man),即没有人再来和女巫交谈。手稿中似乎遗漏了表示否定的词。

    \item rjúfendr

          由rjúfa `break, violate'的现在分词rjúfandi派生而来,和ragnarǫk同为复数主格,为同位语。

  \end{enumerate}
\end{grammar*}
\section{欺骗古洛菲(Gylfaginning)选读}
《欺骗古洛菲》(Gylfaginning)是斯诺里·斯图鲁松写的散文埃达的第一篇正文,故事大致是这样的:古洛菲(Gylfi)是古代瑞典的国王,他曾和一位女神订立契约以求得王国风调雨顺,结果却被女神欺骗了。女神是阿萨神族的一员,古洛菲想知道阿萨神族是不是总是依靠欺骗和魔法达成他们的目的。古洛菲于是前往阿斯加德,却在半路又被神欺骗,来到了另一个宫殿。古洛菲自称冈勒里(Gangleri),他在宫殿中遇到了三位神。为了检测古洛菲的智慧,神明们向他问问题,古洛菲也回应他们,这些问题都是关于神话的。其中第49节即是关于巴德尔的。

\begin{paracol}{2}
  En þat er upphaf þeirar sǫgu\index{saga!sǫgu}, at Baldr inn góða dreymði drauma stóra ok hættliga um líf sitt. En er hann sagði ásunum draumana, þá báru þeir saman ráð sín, ok var þat gert\index{gera!gert} at beiða griða Baldri fyrir allskonar háska, ok Frigg tók svardaga til þess, at eira skyldu Baldri eldr ok vatn, járn ok alls konar málmr, steinar, jǫrðin, viðirnir, sóttirnar, dýrin, fuglarnir, eitrit, ormarnir.

  \switchcolumn

  Then the beginning of the story is that Baldr the good dreamt great and terrible dreams about his life. And when he told the Aesir about the dream, they took counsel together, and this was the decision: to seek protection for Baldr against all kinds of harm, and Frigga took oaths to this, that fire and water shall spare Baldr, iron and metal of all kinds, stones, earth, trees, diseases, beasts, birds, venom, worms.
\end{paracol}
\begin{translation*}{}
  故事的开始是这样的:善神巴德尔做了个噩梦,梦到自己就要不久于人世。他把梦境说给神族们听,于是众神们聚在一起商量对策。他们决心要要让巴德尔免于一切的伤害。弗丽嘉起誓说,她要让火和水伤不到巴德尔,铁器和金属也是如此,山石,泥土,树木,疾病,走兽,飞禽,毒液,虫蛇无一能伤巴德尔分毫。
\end{translation*}
\begin{grammar*}{}
  \begin{enumerate}[leftmargin=*]
    \item Baldr inn góða dreymði drauma

          这个短语整体是宾格,góða为弱变格形式。dreymði, dreyma引导的无人称结构,接宾格。

    \item báru þeir saman ráð sín

          bera除了“携带”以外,还有“说”的衍生含义(to bear by word of mouth > to tell),因此可以接续与言说有关的宾语,如kveðju, orð等。

    \item beiða griða Baldri fyrir allskonar háska

          beiða `request'和biðja用法相似,接与格或宾格表示人(Baldri),用属格表示物(griða, grið `truce'的复数属格)。这里的fyrir是接与格的特殊用法,相当于英语`against',类似的表达还有verja land fyrir ... `defend the land against ...'
  \end{enumerate}
\end{grammar*}

\begin{paracol}{2}
  En er þetta var gert\index{gera!gert} ok vitat, þá var þat skemmtun Baldrs ok ásanna, at hann skyldi standa upp á þingum, en allir aðrir skyldu sumir skjóta á hann, sumir hǫggva til, sumir berja grjóti, en hvat sem at var gert\index{gera!gert}, sakaði hann ekki, ok þótti þetta ǫllum mikill frami.

  \switchcolumn

  But when that was done and known, there was an entertainment between Baldr and gods, that he should stand up in a Thing, when all the others would either shoot at him, or hew at him or beat him with stones, but what they had done hurt him not, and they thought these were all things with great respect (to Baldr).
\end{paracol}
\begin{translation*}{}
  等弗丽嘉完成了此事,众神都知晓了这一事实时,他们便想出了一种娱乐:巴德尔站在议事庭的中间,其他人要么拿东西射他,要么用刀砍他,要么用石头砸他,但是这些都伤不了巴德尔。众神认为这些东西都对巴德尔保有大大的敬意。
\end{translation*}
\begin{grammar*}{}
  \begin{enumerate}[leftmargin=*]
    \item allir aðrir

          allir作名词用,被aðrir所修饰,这里表示集体概念,“其他所有人”。
    \item standa upp á þingum

          þing `Thing',常译作“庭”,是日耳曼人社会中的一种政治议会,自由民可在此商讨事宜。这里的þing指的是开展议会的地点,因此standa upp á þingum表示的是物理上的“站在议事庭(的地面)上”。þing和其他介词搭配,则可能有引申的含义。

    \item hvat sem at var gert\index{gera!gert}

          hvat和sem或er连用,表示`whatever',类似于hvatki. gera at这一短语强调做某事对某人产生影响,本句中省略了at的补语Baldr.

    \item þótti þetta ǫllum mikill frami

          þykkja的典型用法,注意这里的主语是中性指示代词þetta,感受者是ǫllum,指代所有的神明。注意这里省略了系动词vera.
  \end{enumerate}
\end{grammar*}

\begin{paracol}{2}
  En er þetta sá\index{sjá!sá} Loki Laufeyjarson, þá líkaði honum illa, er Baldr sakaði ekki. Hann gekk til Fensalar til Friggjar ok brá\index{bregða!brá} sér í konu líki. Þá spyrr Frigg, ef sú kona vissi, hvat æsir hǫfðusk at á þinginu. Hon sagði, at allir skutu\index{skjóta!skutu} at Baldri ok þat, at hann sakaði ekki.

  \switchcolumn

  But when Loki Laufey-son saw this, he liked him ill, that nothing can harm Baldr. He goes to Fensal to Frigg and turned himself into the shape of a woman. Then Frigg asked if that woman knew what the Aesir had done in the Thing. She said that they all shot at Baldr and that no one ever hurt him.
\end{paracol}
\begin{translation*}{}
  可当洛基,劳菲的儿子,看到巴德尔刀枪不入的时候,他心生不满。于是他变成一个妇人的样子,来到弗丽嘉的宫殿芬萨勒,走到她跟前。弗丽嘉就问这妇人知不知道神族们在议事庭里干的事。妇人回答道,他们都向巴德尔投掷武器,但没什么能伤得了他。
\end{translation*}
\begin{grammar*}{}
  \begin{enumerate}[leftmargin=*]
    \item er Baldr sakaði ekki

          区别于上文的hvat ... sakaði hann ekki, saka `hurt'在这个从句中使用了无人称结构,当然这也可以理解为主语被省略。

    \item brá sér í konu líki

          bregða除了“拔出”这个基本含义外,还有“转变形态”的意思。bregða支配与格,bregða e-m í e-s líki表示`turn one into the shape of ...'.
  \end{enumerate}
\end{grammar*}
\begin{paracol}{2}
  Þá mælti Frigg: ``Eigi munu vápn eða viðir granda Baldri. Eiða hefi ek þegit\index{þiggja!þegit} af ǫllum þeim."

  \switchcolumn

  Then said Frigg: ``No weapon or trees shall hurt Baldr, I have received oaths from all of them."

  \switchcolumn*

  Þá spyr konan: ``Hafa allir hlutir eiða unnit\index{vinna!unnit} at eira Baldri?"

  \switchcolumn
  Then asked the woman: ``Have all things taken oaths to spare Baldr?"

  \switchcolumn*
  Þá svarar Frigg: ``Vex viðarteinungr einn fyrir vestan valhǫll. Sá er mistilteinn kallaðr. Sá þótti mér ungr at krefja eiðsins."

  \switchcolumn

  Then answered Frigg: ``A tree-sprout grows alone in the west of Valholl, it is called mistletoe. I thought it was too young to ask it to take an oath."
\end{paracol}
\begin{grammar*}{}
  \begin{enumerate}[leftmargin=*]
    \item þegit

          五类强动词þiggja `receive'的过去分词。
    \item hafa allir hlutir eiða unnit

          unnit, vinna `work, do'的过去分词,注意它不是过去-现在混合动词unna `love'的过去分词,后者应该是unnat. vinna eiða `take oaths'.
    \item ungr at krefja eiðsins

          类似于英语中的`too adj. to do'结构,也是从句作形容词补语的例子。另注意krefja `demand'可支配双宾语,用宾格表人,属格表物。
  \end{enumerate}
\end{grammar*}
\begin{translation*}{}
  于是弗丽嘉说:“任何武器和木头能不能伤害巴德尔,它们都已向我发誓了。”

  妇人接着问道:“所有的东西都发过誓了吗?”

  弗丽嘉回答说:“英灵殿西边的树上有一株嫩芽,名字叫作槲寄生。它年龄太小,我觉得它还不能发誓。”
\end{translation*}

\begin{paracol}{2}
  Því næst hvarf\index{hverfa!hvarf} konan á braut, en Loki tók mistiltein ok sleit\index{slíta!sleit} upp ok gekk til þings. En Hǫðr stóð\index{standa!stóð} útarliga í mann\-hringnum, því at hann var blindr.

  \switchcolumn

  Immediately the woman disappeared and went on her way, but Loki took mistletoe, slit it and went to the Thing. And Hodr stood outside of the ring of men, because he was blind.
\end{paracol}
\begin{translation*}{}
  妇人一眨眼就不见了。洛基却砍下槲寄生,捡起它前往议事庭。此时围观的人站成一圈,霍德尔因为眼睛看不见站在人群外边。
\end{translation*}
\begin{grammar*}{}
  \begin{enumerate}[leftmargin=*]
    \item því næst hvarf konan á braut

          hvarf, hverfa `turn around'的过去式。此外,这句话中有几个短语值得注意。1)því næst表示`immediately after that, then',næst是nær `near'的最高级,nær的补语要用与格,因此有了því næst的说法。2)(á/í) braut(u)/brott `away',braut本是名词,指的是“(树林间,山上开辟的)小路”,但它也可以作副词用,表示“离开,上路”。有时前面也可以加介词á/í. 一些更早的手稿中也写作brott.
  \end{enumerate}
\end{grammar*}
\begin{paracol}{2}
  Þá mælti Loki við hann: ``Hví skýtr þú ekki at Baldri?"

  Hann svarar: ``Því, at ek sé\index{sjá!sé} eigi, hvar Baldr er, ok þat annat, at ek em vápnlauss."

  Þá mælti Loki: ``Gerðu þó í líking annarra manna ok veit Baldri sæmð sem aðrir menn. Ek mun vísa þér til, hvar hann stendr. Skjót at honum vendi þessum."\\

  \switchcolumn

  Then said Loki to him: ``Why don't you shoot at Baldr?"

  He answered: ``Because I can't see where Baldr is and besides, I am weaponless."

  Then said Loki: ``Do as other men and show Baldr honor. I shall direct you to where he stands. Shoot him with this wand."
\end{paracol}

\begin{translation*}{}
  于是洛基对他说:“你怎么不向巴德尔投武器呢?”

  霍德尔回答说:“我看不见巴德尔在哪里,另外,我也没有武器。”

  洛基于是说:“你应像别人一样做,这样才体现巴德尔的光荣。把这个树枝扔过去,我会告诉你他站在哪儿。”
\end{translation*}
\begin{grammar*}{}
  \begin{enumerate}[leftmargin=*]
    \item í líking annarra manna

          í líking e-s表示“模仿某人”,和下面的sem aðrir menn表意是一样的。
  \end{enumerate}
\end{grammar*}
\begin{paracol}{2}
  Hǫðr tók mistiltein ok skaut\index{skjóta!skaut} at Baldri at tilvísan Loka. Flaug\index{flúga!flaug} skotit í gegnum Baldr, ok féll hann dauðr til jarðar, ok hefir þat mest óhapp verit unnit með goðum ok mǫnnum.

  \switchcolumn

  Hodr took the mistletoe and shot at Baldr with the guidance of Loki. The spear flew piercing Baldr, and he fell dead to the ground, and the most hapless mischief has befallen gods and men.
\end{paracol}
\begin{translation*}{}
  霍德尔于是拿起槲寄生,在洛基的指引下把它投向巴德尔。枝干刺穿了巴德尔的身体,巴德尔应声倒地,一命呜呼。人类和神族间最大的不幸就此发生了。
\end{translation*}
\begin{grammar*}{}
  \begin{enumerate}[leftmargin=*]
    \item hefir þat mest óhapp verit unnit

          注意本句中出现的被动态的完成时óhapp hefir verit unnit `mischief has been wrought'.
  \end{enumerate}
\end{grammar*}
\begin{paracol}{2}
  Þá er Baldr var fallinn, þá féllusk ǫllum ásum orðtǫk\index{orðtak!orðtǫk} ok svá hendr\index{hǫnd!hendr} at taka til hans, ok sá hverr til annars, ok váru allir með einum hug til þess, er unnit hafði verkit, en engi mátti hefna. Þar var svá mikill griðastaðr. En þá er æsirnir freistuðu at mæla, þá var hitt þó fyrr, at grátrinn kom upp, svá at engi mátti ǫðrum segja með orðunum frá sínum harmi. En Óðinn bar þeim mun verst þenna skaða sem hann kunni mesta skyn, hversu mikil aftaka ok missa ásunum var í fráfalli Baldrs.
  \switchcolumn
  Then when Baldr was fallen, words failed all the gods and likewise their hands to lay hold of the corpse. Each looked at the other, and all were of one mind as to who had wrought the work, but no one could take vengeance (as they were) in a great sanctuary. But when the Aesir tried to speak, weeping broke out first, so that no one could tell others about their grief with words. But Odin had the greatest misfortune, as he had the most perception about how great a loss Baldr's death was for the gods.
\end{paracol}
\begin{translation*}{}
  巴德尔死后,众神扶着他的遗体,手足无措,面面相觑。他们都想知道是谁害死了巴德尔,可是议事庭里不能动武,没人能为巴德尔寻仇。等到他们终于打算说点什么时,有人开始哭了起来,一时间神族们呻吟呼号,没人能说清自己的悲痛。但奥丁是众神中最痛心的,因为他比任何人都清楚巴德尔的死意味着什么。
\end{translation*}
\begin{grammar*}{}
  \begin{enumerate}[leftmargin=*]
    \item þá féllusk ǫllum ásum orðtǫk

          falla有类似于英语`fail'的含义,除了表示“(使)失效”外,也有和英语类似的引申含义,“使某人不能做某事”。此时常用其反身形式,接与格,如orðtǫk fallask e-m `words fail one'.

    \item ok svá hendr at taka til hans

          ok svá相当于一个副词,`also, likewise'. 本句的动词是上一句中的féllusk, hendr fallask e-m `hands fail one'是古诺尔斯语的一个固定短语,表示人“愣住;手足无措”。at taka til hans是hendr的补语,taka til有几个意思,这里是基础含义,“把手放在某物上;触摸”。

    \item griðastaðr

          字面意思`truce-place', 指的就是þing. 本句比较好的解释是:由于在议事庭中不能动武,因此众神不能为巴德尔报仇。此句与上文有因果关系,但缺少一个表因果的连词。

    \item þá var hitt þó fyrr

          hitt, 中性指示代词作名词用,`but earlier it was ...'.

    \item ǫðrum segja með orðunum frá sínum harmi

          frá和表示言说的动词连用,表示“关于”。于是,segja frá e-u表示`tell about something'. 本句中还有两个与格短语,ǫðrum `others'是讲话的对象,með orðunum `with words'提示方法。注意这里orð是特指形式,表示orð um harmi sínum.

    \item Óðinn bar þeim mun verst þenna skaða

          mun, munr的与格,这个词几乎只用在固定短语中。(þeim) mun+形容词比较级/最高级时强调程度,如muni hægri `much easier', þeim mun fleiri `all the more'. 这里的þeim可加可不加。
  \end{enumerate}
\end{grammar*}
\begin{paracol}{2}
  En er goðin vitkuðusk, þá mælti Frigg ok spurði, hverr sá væri með ásum, er eignask vildi allar ástir hennar ok hylli ok vili hann ríða á helveg ok freista, ef hann fái\index{fá!fái} fundit Baldr, ok bjóða Helju útlausn, ef hon vill láta fara Baldr heim í Ásgarð. En sá er nefndr Hermóðr inn hvati, sveinn Óðins, er til þeirar farar varð. Þá var tekinn Sleipnir, hestr Óðins, ok leiddr fram, ok steig\index{stíga!steig} Hermóðr á þann hest ok hleypði braut.
  \switchcolumn

  Now when the gods came to consciousness, Frigg said and asked, which one of the gods would acquire her favor and goodwill, and he will ride to Hel and try to find Baldr, and offer Hel a ransom, provided that she will let Baldr come back home in Asgard. And the man named Hermod the Quick, son of Odin, took the journey. Then Sleipnir, Odin's horse, was taken and led forward. Hermod mounted the horse and rode on his way.
\end{paracol}
\begin{translation*}{}
  等众神们从悲伤中回过神来,弗丽嘉问到:“谁愿意前去冥府寻找巴德尔?假如赫尔能放他回到阿斯加德,那就给她一笔赎金。”那个人会得到弗丽嘉的信赖和赞赏。赫尔莫德,奥丁的儿子,承担了这个使命。他登上奥丁的坐骑斯莱普尼尔,纵马向冥府骑去。
\end{translation*}
\begin{grammar*}{}
  \begin{enumerate}[leftmargin=*]
    \item vitkuðusk

          vitkask是有特殊意义的反身动词。vitka表示“使迷惑”,vitkask表示“恢复理智;清醒”。

    \item er eignask ... ok vili hann

          注意这里的ok连接的前后两句主语不一致。er eignask ...从句的主语是hverr sá væri með ásum,后一句的主语是Frigg.

    \item hann fái fundit

          fá+过去分词表示“有能力做某事”,相当于`be able to do'.

    \item til þeirar farar varð
          verða til e-s的引申义,`come forth to do something, volunteer'.
  \end{enumerate}
\end{grammar*}
\medskip
【本处删去一段,内容是关于巴德尔的葬礼的:巴德尔的妻子南娜看到巴德尔的尸体,心脏迸裂而死,众神将她和巴德尔一起放到巴德尔的船“灵舡”(Hringhorni)上,灵舡是所有船中最大的,诸神推不动它,于是只好找来女巨人希尔罗金(Hyrrokin)来帮忙。她设法将船送入水中,索尔举起神锤“缪尼尔”(Mjǫllnir)点燃葬礼用的柴堆。一个名叫利特尔(Litr)的矮人在索尔脚边乱跑,被索尔踢进火堆烧死。奥丁的魔戒德罗普尼尔和满载马饰的巴德尔的马也一同献给巴德尔。】
\medskip
\begin{paracol}{2}
  En þat er at segja frá Hermóði, at hann reið níu nætr dǫkkva dala ok djúpa, svá at hann sá ekki, fyrr en hann kom til árinnar\index{á!árinnar} Gjallar\footnotemark ok reið á Gjallarbrúna. Hon var þǫkt\index{þakja!þǫkt} lýsigulli.

  \switchcolumn

  Now the story is to be told about Hermod, that he rode for nine nights in valleys dark and deep so that he saw nothing before he came to the river Gjall and rode to the Gjall-bridge, which was covered with glittering gold.
\end{paracol}
\footnotetext{北欧神话中的冥河。}
\begin{translation*}{}
  话说赫尔莫德在幽暗深邃的低谷里整整骑了九天九夜。外面伸手不见五指,所以在他抵达加尔河之前什么都没看见。一座桥横跨加尔河,上面镶着闪闪发光的黄金,赫尔莫德向桥骑去。
\end{translation*}
\begin{grammar*}{}
  \begin{enumerate}[leftmargin=*]
    \item reið níu nætr dǫkkva dala ok djúpa

          宾格名词níu nætr表示对时间的度量,作为动词的附加语。在古诺尔斯语中ríða可以直接加与道路有关的宾格名词作及物动词用,如reið dǫkkva dala,而英语中则要用介词短语`ride in the dark valley',这时的ride是不及物动词。

    \item þǫkt

          þekja `thatch'的过去分词的阴性形式。
  \end{enumerate}
\end{grammar*}
\begin{paracol}{2}
  Móðguðr er nefnd mær sú, er gætir brúarinnar. Hon spurði hann at nafni eða at ætt ok sagði, at inn fyrra dag riðu um brúna fimm fylki dauðra manna - en eigi dynr brúin minnr undir einum þér, ok eigi hefir þú lit dauðra manna. Hví ríðr þú hér á helveg?
  \switchcolumn

  Modgudr is the name of the maiden, who guards the bridge. She asked him about his name and race and said: ``The day before, five groups of dead men had ridden over the bridge, but the bridge did not make a smaller sound than you are alone, besides you don't have the color of dead men. Why did you ride hither to Hel?"
  \switchcolumn*
  Hann svarar, at ek skal ríða til Heljar at leita Baldrs, eða hvárt hefir þú nakkvat sét Baldr á helvegi?

  \switchcolumn

  He answered: ``I shall ride to Hel to search for Baldr, and if by chance, have you seen Baldr on his way to Hel?"

  \switchcolumn*
  En hon sagði, at Baldr hafði þar riðit um Gjallarbrú, en niðr ok norðr liggr helvegr.

  \switchcolumn
  And she said that Baldr had ridden across Gjall-bridge to there and the way to Hel lies north and beneath.
\end{paracol}
\begin{translation*}{}
  守桥的女巨人叫作莫德古德。她寻问赫尔莫德的姓名和族裔,然后说道:“昨天有五批亡灵从桥上过去,桥发出的声响却和你一个人过去时差不多。另外你身上也没有死者的颜色,你为什么要去冥府呢?”

  他回答说:“我要去那里寻找巴德尔,你可曾见过他?”

  女巨人告诉他,巴德尔已经过了桥到那边去了,去冥府的路还要朝着北面往下走。
\end{translation*}
\begin{grammar*}{}
  \begin{enumerate}[leftmargin=*]
    \item eigi dynr brúin minnr undir einum þér

          dynr, dynja的三单现在时。dynja的本义是指“发出巨大的响声”,但在古诺尔斯语中经常表示液体的“喷涌,爆发”,常和鲜血、大雨这样的名词连用。这里取的反而是其本义。这句话字面意思是说,亡灵过桥时发出的声响不比赫尔莫德小,实际上是说赫尔莫德一个人的重量相当于五批亡灵,因此女巨人已经发现赫尔莫德不是死人。

    \item nakkvat

          nǫkkurr的副词用法,相当于英语的`somewhat, somehow',表示不确定性。
  \end{enumerate}
\end{grammar*}
\begin{paracol}{2}
  Þá reið Hermóðr, þar til er hann kom at helgrindum. Þá sté\index{stíga!sté} hann af hestinum ok gyrði hann fast, steig upp ok keyrði hann sporum, en hestrinn hljóp svá hart ok yfir grindina, at hann kom hvergi nær. Þá reið Hermóðr heim til hallarinnar ok steig\index{stíga!steig} af hesti, gekk inn í hǫllina, sá þar sitja í ǫndugi, Baldr bróður sinn, ok dvalðist Hermóðr þar um nóttina. En at morgni þá beiddist Hermóðr af Helju, at Baldr skyldi ríða heim með honum, ok sagði, hversu mikill grátr var með ásum.

  \switchcolumn

  Then rode Hermod, until he came to the gate of Hel. Then he dismounted from his horse and tied his girth fast, then he mounted the horse and drive it with his spur, and the horse leaped so high over the gate that he came no nearer to it. Then rode Hermod to the hall and came down from his steed, went into the hall, and saw that Baldr his brother was sitting in the high-seat. Hermod stayed there overnight. And in the morning he begged Hel to let Baldr ride home with him and told her how great sorrow was among the Aesir.
\end{paracol}
\begin{translation*}{}
  接着赫尔莫德一路骑到冥府的大门前。他从马上下来,收了收自己的腰带,接着翻身上马,用马刺驱策它。斯莱普尼尔高高地跳起来,一下越过那大门,赫尔莫德终于来到了冥府。他一直骑到赫尔的厅堂门口,下了马,径直走到里面。只见自己的哥哥巴德尔正坐在大厅中间的高椅上。赫尔莫德在那里留宿一夜,第二天早上他请求赫尔放巴德尔和他一道回去,并且告诉她神族们有多么悲伤。
\end{translation*}
\begin{grammar*}{}
  \begin{enumerate}[leftmargin=*]
    \item sté hann af hestinum

          stíga `step'的用法比较有趣,一般来说,这个词表达的是“向上登”的含义。但如果接类似于af, ofan, niðr之类的词,意思就变成“下降”了。这里的sté af hestinum指的就是从马背上下来。

    \item ǫndugi

          或称ǫndvegi,字面意思`opposite-way',指的是两把相对的椅子。维京人的住房多是长条形的,进门后的左右两侧设有面对面的两条长椅(即bekkr),长椅紧挨着墙,向房屋的深处延申。长椅的中间是主座,这两个主座合称为ǫndvegi,多有珠宝装饰,留给地位尊贵的人。坐北朝南的称为ǫndvegi it æðra `the higher seat',一般留给主人。坐南朝北的称为ǫndvegi it úæðra `the lower seat',一般留给贵宾。巴德尔坐在高椅上,指他在冥府被尊为上宾。

    \item beiddist Hermóðr af Helju

          beiddist是beiða的反身式,`request on one's own behalf',总体来说没有改变意思。
  \end{enumerate}
\end{grammar*}
\begin{paracol}{2}
  En Hel sagði, at þat skyldi svá reyna, hvárt Baldr var svá ástsæll, sem sagt er. Ok ef allir hlutir í heiminum, kykvir ok dauðir, gráta hann, þá skal hann fara til ása aftr, en haldask með Helju, ef nakkvarr mælir við eða vill eigi gráta.

  \switchcolumn

  And Hel said that it should be tested whether Baldr was so all-loved as he was said to be and if all the things in the world, quick or dead, will weep for him, that he shall go back to the Aesir, but will stay if anything says otherwise or if some weeps not.
\end{paracol}
\begin{translation*}{}
  于是赫尔说,她要看看巴德尔是否真的像赫尔莫德所说的那样受到所有人的爱戴。假如世界上的一切东西,无论是死是活,都愿意为巴德尔哭泣,那么他就可以回到神族那里,可但凡有人说他不爱巴德尔或是不愿为他哭泣,那么巴德尔就得留在冥府。
\end{translation*}
\begin{grammar*}{}
  \begin{enumerate}[leftmargin=*]
    \item haldask með Helju

          halda `hold'的反身式haldask意思有变化,表示“停留、坚持”等。
  \end{enumerate}
\end{grammar*}
\begin{paracol}{2}
  Þá stóð\index{standa!stóð} Hermóðr upp, en Baldr leiddi hann út ór hǫllinni ok tók hringinn Draupni ok sendi Óðni til minja, en Nanna sendi Frigg ripti ok enn fleiri gjafar\index{gjǫf!gjafar}, Fullu fingrgull. Þá reið Hermóðr aftr leið sína ok kom í Ásgarð ok sagði ǫll tíðendi, þau er hann hafði sét\index{sjá!sét} ok heyrt. Því næst sendu æsir um allan heim erendreka at biðja, at Baldr væri grátinn ór helju, en allir gerðu þat, menninir ok kykvendin ok jǫrðin ok steinarnir ok tré ok allr málmr, svá sem þú munt sét hafa, at þessir hlutir gráta þá, er þeir koma ór frosti ok í hita.

  \switchcolumn

  Then Hermod stood up, but Baldr led him out of the hall and took the ring Draupnir and sent it to Odin as a remembrance, and Nanna sent Frigg a suit as well as some other gifts, and a gold finger-ring to Fulla. Then Hermod rode back on his way and came to Asgard and told everything he had seen and heard. Upon that, the Aesir sent messengers over the whole world to pray that Baldr be wept out of Hel, and all men did that, the living, earth, stones and trees and all metals, all cried as you must have seen when they come out of frost to heat.
\end{paracol}
\begin{translation*}{}
  于是赫尔莫德站起身来,巴德尔却拉着他到外面去,他拿出戒指德罗普尼尔让赫尔莫德带给奥丁留作纪念。巴德尔的妻子南娜给弗丽嘉送去了一件衣服还有一些别的礼物,又给芙拉送去一支金戒指。赫尔莫德骑马回到阿斯加德,告诉了神族们他的所见所闻。听到这些,神族们立刻向全世界送出信使,祈求大家哭泣好让巴德尔离开冥府。所有人都照做了,世上一切的活物,还有土地,山石,树木,金属都为巴德尔落泪,就好像它们离开了寒霜重回温暖时一样。
\end{translation*}
\begin{grammar*}{}
  \begin{enumerate}[leftmargin=*]
    \item sendi Frigg ripti ok enn fleiri gjafar

          senda `send'是典型的接双宾语的动词,送出的物用宾格,赠送的对象用与格。enn, 副词,`yet, more'.
  \end{enumerate}
\end{grammar*}
\begin{paracol}{2}
  Þá er sendimenn fóru heim ok hǫfðu vel rekit\index{reka!rekit} sín erendi, finna þeir í helli nǫkkurum, hvar gýgr sat. Hon nefndist Þǫkk. Þeir biðja hana gráta Baldr ór Helju. Hon segir:

  \switchcolumn

  Then when the messengers went home and had well finished their errands, they found that in a certain cave sat a giantess who called herself Thokk. They ask her to weep Baldr out of Hel. She says:
  \switchcolumn*

  \begin{quote}
    Þǫkk mun gráta\\
    Þurrum tárum\\
    Baldrs bálfarar\\
    Kyks né dauðs\\
    Nautka\index{njóta!nautka} ek Karls sonar\\
    Haldi Hel því er hefir
  \end{quote}

  \switchcolumn

  \begin{quote}
    Thokk will weep\\
    with dry tears\\
    For Baldr's funeral\\
    Neither in life nor death\\
    I liked not Karl's son\\
    Let Hel hold what she had
  \end{quote}

  \switchcolumn*
  En þess geta menn, at þar hafi verit Loki Laufeyjarson, er flest hefir illt gert\index{gera!gert} með ásum.

  \switchcolumn

  And these men believed that the woman who had been there was Loki Laufey-son, who has made the evilest thing among the gods.
\end{paracol}
\begin{translation*}{}
  等到所有的信使都完成了他们的任务回到家时,他们发现在一个山洞里住着一个女巨人,她的名字叫索克。他们请求她为巴德尔哭泣,索克说:
  \begin{quote}
    索克不会留下泪水\\
    为巴德尔之死伤悲\\
    无论他是生还是死\\
    我都不怜悯他一丝\\
    贵族之子已归死神\\
    就让赫尔留他终身
  \end{quote}

  这些信使相信,刚刚的女人就是洛基变的,他已经犯了下神族中最为人不齿的恶行。
\end{translation*}
