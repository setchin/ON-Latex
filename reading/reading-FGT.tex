\chapter{第一文法论(Fyrsta Málfrœðiritgerðin)}

《第一文法论》是冰岛的第一篇语法著作,大约写于12世纪中期,作者不详。这篇文章之所以被称为“第一”文法论,是因为在同一本手稿中还有其他三篇语法著作,这四篇论文分别称为“第一”至“第四”文法论。

《第一文法论》分成三个部分:引言\footnotemark、元音论、辅音论。作者在其中详细论述了当时冰岛语的音系情况,并用音素对立的观点科学地论证了自己的观点。根据冰岛语的语音特点,作者还探讨了正字法的问题,并为冰岛语拟制了一份字母表。
\footnotetext{三个标题均系笔者添加。}
\section{引言}
\begin{paracol}{2}
    Í flestum lǫndum setja menn á bœkr annat tveggja þann fróðleik, er þar innanlands hefir gǫrzk, eða þann annan, er minnisamligstr þykkir, þó at annars sta[ðar hafi h]eldr gǫrzk, eða lǫg sín setja menn á bœkr, hver þjóð á sína tungu. En af því at tungurnar eru [ó]líkar hver annarri, þær þegar er ór einni ok inni sǫmu tungu hafa gengizk eða greinzk, þá þarf ólíka stafi í at hafa, en eigi ina sǫmu alla í ǫllum, sem eigi ríta grikkir látínustǫfum girzkuna ok eigi látinumenn girzkum stǫfum látínu, né enn heldr ebreskir menn ebreskuna hvárki girzkum stǫfum né látínu, heldr rítr sínum stǫfum hver þjóð sína tungu.
    \switchcolumn
    In most places, men write in books either significant events which have occurred within their country, or wonders that seem most memorable even though they happened rather aboard. Men also write in books their laws, each group of people with their own language. However, since languages are different from each other as soon as they have branched off or been separated from the common ancestor, it is required to have different letters in these languages, just as the Greeks do not write Greek in Latin letters, and the Latin men do not use Greek letters to write Latin, and the Hebrews write Hebrew in neither Greek letters nor Latin, rather, different people write their language with their own letters.
\end{paracol}
\begin{translation*}{}
    在世界上的大多数地方,人们都在书中记下他们国家中发生的大事,抑或是那些异域的奇闻轶事,有时人们也记载律法。不同的人群用不用的语言书写他们的故事。然而,当不同的语言从它们共同的祖先中分离开来时,它们就变得形态各异,并且需要采用不同的书写系统了。希腊人不会用拉丁字母写希腊文,罗马人自然也不用希腊字母写拉丁语,同样地,希伯来人既不用拉丁字母也不用希腊字母写希伯来语。不同的人用不同的文字书写语言。
\end{translation*}
\begin{grammar*}{}
    \begin{enumerate}[leftmargin=*]
        \item annat tveggja ... eða ...

              annarr tveggja或annarr hvárr是固定短语,字面意思是`another of two',相当于`either',因此也可以和eða连用。

        \item hefir gǫrzk

              gera的反身式gerask的过去分词,相对罕用。gerask多表示类似于“发生;成为”的含义,相当于`become, arouse'等。

        \item hver þjóð á sína tungu

              本句省略了一个表示书写的动词,如ríta.

        \item þegar er

              相当于副词,`as soon as'.

        \item þær ór einni ok inni sǫmu tungu hafa gengizk eða greinzk

              þær是阴性代词,指的是tungur(阴性)。einni ok inni sǫmu tungu `one and the same tongue',指的就是一种共通语。gengizk, ganga的反身式过去分词,gangask `be altered, changed'. greinzk, greina `discern, divide into branches'的反身式过去分词,greinask `branch out, differ, be separated'.

        \item þá þarf ólíka stafi í at hafa

              þurfa `need'接无人称结构,与宾格搭配。í at hafa,其中的í是í tungum的省略。
    \end{enumerate}
\end{grammar*}
\begin{paracol}{2}
    Hveriga tungu er maðr skal ríta annarar tungu stǫfum, þá verðr sumra stafa vant, af því […] at eigi finnsk þat hljóð í tungunni, sem stafirnir hafa, þeir er af ganga. En þó ríta enskir menn enskuna látínustǫfum, ǫllum þeim er réttræðir verða í enskunni, en þar er þeir vinnask eigi til, þá hafa þeir við aðra stafi, svá marga ok þesskonar sem þarf, en hina taka þeir ór, er eigi eru réttræðir í máli þeira.
    \switchcolumn
    Whatever language a man shall write with letters from another language, there will be some letters missing because it cannot be found in that language the sounds that these missing letters have, which are superfluous. However, Englishmen do write English in Latin letters, they use all the Latin letters that are pronounced correctly in English, but when these no longer suffice, they add other letters, as many and of such a kind as is needed. Also, they take away from the alphabet those that cannot be read correctly in their language.
\end{paracol}
\begin{translation*}{}
    如果有人要用别的语言的文字来书写他自己的语言,那么他就会发现字母不够用了,因为他的语言中的有些发音是那个字母表里所缺失的。然而,英国人确实用拉丁字母写英文,他们先是保留了全部在英文和拉丁文中发音一致的字母,当字母不够用的时候,他们则按需增添字母。当然,他们也剔去了那些发音和英语不一致的字母。
\end{translation*}
\begin{grammar*}{}
    \begin{enumerate}[leftmargin=*]
        \item af ganga

              ganga af有“多余”之意。

        \item þeir vinnask eigi til

              vinnask `last, suffice',一般接til.

        \item hafa við aðra stafi

              hafa við e-u有“与...相适合”的含义,这里指的就是增添字母。
    \end{enumerate}
\end{grammar*}
\begin{paracol}{2}
    Nú eptir þeira dœmum, alls vér erum einnar tungu, þó at gǫrzk hafi mjǫk ǫnnur tveggja eða nǫkkut báðar, til þess at hœgra verði at ríta ok lesa, sem nú tíðisk ok á þessu landi, bæði lǫg ok áttvísi eða þýðingar helgar, eða svá þau in spakligu frœði, er Ari þórgilsson hefir á bœkr sett af skynsamligu viti, þá hefi ek ok ritit oss íslendingum stafróf, bæði látínustǫfum ǫllum þeim er mér þótti gegna til várs máls vel, svá at réttræðir mætti verða, ok þeim ǫðrum, er mér þótti í þurfa at vera, en ór váru teknir þeir, er eigi gegna atkvæðum várrar tungu. Ór eru teknir samhljóðendr nǫkkurir ór látínustafrófi, en nǫkkurir í gǫrvir. Raddarstafir e[ru] engir ór teknir, en í gǫrvir mjǫk margir, því at vár tunga hefir flesta alla hljóðs eða raddar.
    \switchcolumn
    Now, after these examples, as we speak one (common) language, even though one of our languages or two have changed quite a bit, in order to make it easy to read or write, as is now the fashion on this island, both laws and genealogical knowledge or sacred writings, or the wise lore that Ari Thorgilsson has written in his books with discreet wit, I have made an alphabet for all us Icelanders with all the Latin letters which I believe to fit well with our language so that they can be pronounced correctly. I also take the ones that seem needful to me while omitting those that go against the pronunciation of our language. Some consonants of Latin letters are taken away but some are added. No vowels are omitted and quite a lot are supplemented since our language has the most phonemes and vowels.
\end{paracol}
\begin{translation*}{}
    讲了这些例子之后,我们来看看冰岛的情况。在冰岛我们讲的是同一种语言,虽然某几个方言可能因为读写的方便已有了很大改变。现在,在岛上已经形成了一股风气,大家喜欢阅读法律、宗族、祭祀方面的知识,又或者是学习索格尔松在他的书中兢兢业业讲授的知识。那么,为了使读写更加方便,我为我们冰岛人制作了一张字母表,它里面的字母取自于所有适合我们语言的拉丁字母,可以用来正确地拼写我们的语言。我同样加入了那些我认为有必要、但是拉丁字母表里没有的字母;同样地,我删去了一些与我们的语言相龃龉的。有些辅音字母被保留了下来,有些则去掉了。没有任何一个元音字母被删去,反而我增加了不少,因为我们的语言里的元音和音素是最多的。
\end{translation*}
\begin{grammar*}{}
    \begin{enumerate}[leftmargin=*]
        \item vér erum einnar tungu

              字面含义`we are of one tongue', einnar tungu是用于表描述的属格。

        \item tíðisk

              tíða的反身式,tíða又和tíðr `frequent; famous'有关,故表示“成为某种风气”。

        \item þýðingar helgar

              字面意思`interpretations of the sacred',指的是对《圣经》一类的释文。
    \end{enumerate}
\end{grammar*}
\begin{paracol}{2}
    Nú af því, at samhljóðendr megu ekki mál eða atkvæði gøra einir við sik, eigi svá at þeir megi nafn hafa án raddarstafi, en at raddarstafnum einum [ok] sér hverjum má kveða, sem hann heitir, ok at honum kveðr í hverju máli, ok þeir bera svá tígn af samhljóðundum sem almætti af hálfmætti, þá hefi ek af því fyrri setta þá bæði í stafrófi ok í umrœðu hér nú.
    \switchcolumn
    Now that consonants cannot make a sound or sentence alone by themselves, not even able to name themselves without vowels. But anyone can pronounce a vowel just as it is named and as it is spoken in every word, thus vowels bear such supremacy over consonants as almighty does over half-mighty. As such, I have placed vowels in the first place both in the alphabet and the treatise here.
\end{paracol}
\begin{translation*}{}
    辅音本身不能构成声响或者语句,它们甚至没有办法单独构成自己的名字,但任何人都能轻松地按照元音的名字来拼读它,或者任何带有元音的单词。因此,元音比辅音更重要,就像纲比目重要一样。所以,我把元音放在了字母表中的前面,也在这篇论文中先讨论它。
\end{translation*}
\begin{grammar*}{}
    \begin{enumerate}[leftmargin=*]
        \item at raddarstafnum einum ok sér hverjum má kveða

              省略了主语maðr. einn ok sér hverr,固定短语,表示`one and all, every one'. sér hverr或sérhverr本身常用作形容词,相当于`each'. kveða at, `pronounce'.
    \end{enumerate}
\end{grammar*}
\section{元音论}
\begin{paracol}{2}
    Við þá hljóðstafi fimm, er áðr váru í látínustafrófi: a, e, ı, o, u, þar hefi ek við gǫrva þessa stafi fjóra, er hér eru ritnir nú: ǫ, ę, ø, y. Ǫ hefir lykkju af ae, en hringinn af oe, því at hann er af þeira hljóði tveggja saman blandinn, kveðinn minnr opnum munni en a, en meirr en o. Ę er ritinn með lykkju as, en með ǫllum vexti es, sem hann er af þeim tveim samfeldr, minnr opnum munni en a, en meirr en e. Ø, hann er af hljóði es ok os feldr saman, minnr opnum munni kveðinn en [e] ok meirr en o, enda ritinn af því með kvisti e[s] ok með osins hring. Y er af rǫddu ıs ok us gǫrr at einni rǫddu, kveðinn minnr opnum munni en ı ok meirr en u, ok skal af því ina fyrri kvísl af hǫfuðstafs-ue, sem áðr er þeim í stafrófi skipat.
    \switchcolumn
    With regard to the five vowels that were already in the Latin alphabet, that is, \textit{a}, \textit{e}, \textit{ı}, \textit{o}, \textit{u}, I have added these four letters which are now written here, \textit{ǫ}, \textit{ę}, \textit{ø}, \textit{y}. \textit{Ǫ} has its loop from \textit{a} and its circle from \textit{o}, since it comes from the blending sound of these two sounds, pronounced with mouth opening less than \textit{a}, but more than \textit{o}. \textit{Ę} is written with the loop of \textit{a}, but with the full shape of \textit{e}, as it is a combination of these two, with the mouse less open than \textit{a}, but more than \textit{e}. \textit{Ø} is the sound of \textit{e} and \textit{o} combined together, pronounced with mouth less than \textit{e} and more than \textit{o}, hence it is written with the cross of \textit{e} and the circle of \textit{o}. \textit{Y} is the sound of \textit{ı} and \textit{u} made to a single one, pronounced with the mouth less open than \textit{ı} and more than \textit{u}, and it shall have the first branch of capital \textit{u}, as they were placed formerly in the alphabet.
\end{paracol}
\begin{translation*}{}
    对于拉丁字母表里已有的五个元音,也即a, e, ı, o, u,我加上了四个变体,如下所示:ǫ, ę, ø, y。Ǫ的弯钩来自于a,但其环形则来自于o,因为它的发音正好介于这两个字母中间,开度比a大但比o小。Ę 的弯钩来自于a,但它保留了e的整个轮廓,因为它的发音也是这两个字母的结合,开度比a小但比e大。Ø的发音介于e和o之间,因此保留了e的斜杠和o的轮廓。Y的发音是ı和u结合到一起的音色,开度小于ı而大于u,因此其上半部分保留了u的分叉。这些字母按照已有的字母表的顺序排列。
\end{translation*}
\begin{grammar*}{}
    \begin{enumerate}[leftmargin=*]
        \item ae, as; oe, os {\dots}

              这里的-e; -s是词尾,分别代表与格和属格。在当时的正字法中,与格标记-i常写作-e.

        \item sem áðr er þeim í stafrófi skipat

              skipa `arrage'接与格,因此其被动态是无人称的。
    \end{enumerate}
\end{grammar*}
\begin{paracol}{2}
    Nú má verða at því, at nǫkkurr svari svá: ``Ek má fullvel lesa danska tungu, þó at látínustǫfum réttum sé ritit. Má ek þó at líkindum ráða, hvé kveða skal, þó at eigi sé allir stafir réttræðir í því, er ek les. Rœki ek eigi, hvárt þú rítr [ǫ] þítt eða a, [e] eða ę, y ok u." En ek svara svá: ``Eigi er þat rúnanna kostr, þó at þú lesir vel eða ráðir vel at líkindum; þar sem rúnar vísa óskyrt. Heldr er þat þínn kostr; enda er þá eigi ørvænt, at þeygi lesa ek vel eða mínn maki, ef sá finnsk, eða ráða ek vel at líkindum til hvers ins rétta fœra skal, ef fleiri vega má fœra til rétts en einn veg, þat sem á einn veg er þó ritit, ok eigi skýrt á kveðit, ok skal geta til, sem þú lézk þat vel kunna." En þó at allir mætti nǫkkut rétt ór gøra, þá er þó vís ván, at þeygi vili allir til eins fœra, ef máli skiptir allra helzt í lǫgum. Enda tel ek þik þá eigi hafa vel svarat, er þú lætr eigi þurfa í váru máli þessa níu raddarstafi: a, ǫ, e, ę, ı, o, ø, u, y, allra helzt ef ek klýf ór þessum níu sex greinir ins fjórða tegar, þær er sítt mál gøri hverr, ef gløggt eru skildar.
    \switchcolumn
    Now it may happen that someone would reply this wise: ``I can read Danish very well, though it is written in Latin letters. I would probably, though, read what it says, even if not all the letters in what I read are spelled correctly. I don't care whether you write \textit{ǫ} or \textit{a}, \textit{e} or \textit{ę}, \textit{y} or \textit{u}." And to this I say: ``That is not a good quality of a writing system even though you may read well or comprehend, that is where the system vaguely indicates. That is rather your idea, and it is well possible that I or my mate, if there exists, can not read well or decide which correct interpretation is to be given, if there is more than one way to properly interpret it. But since it is written in one way and not expressed clearly, then it has to be guessed, as you claimed that you know well." But even though everyone can make something correct out of it, it may well be expected that everyone will not come to the same consensus if the word changes, especially in law terms. Therefore I can say that you have not answered well when you state that we do not need these nine letters \textit{a}, \textit{ǫ}, \textit{e}, \textit{ę}, \textit{ı}, \textit{o}, \textit{ø}, \textit{u}, and \textit{y} in our language, especially when I can distinguish from these nine letters thirty-six differences, where each produces its own sound, if clearly distinguished.
\end{paracol}
\begin{translation*}{}
    现在有人可能要说:“我的丹麦语讲得很熟练,即便它用拉丁字母写出来也照样能看懂。就算有几个字母写得不正确,应该也不会影响我的理解,我才不管你写的是ǫ还是a,e还是ę,y还是u哩。”对此我的评价是:“就算你看得懂写的是什么,这些稀里糊涂的地方也不是一个良好的书写系统应有的品质,倒不如说是你自己的坏习惯。对于我和与我志同道合的人来说,当一个词有好多种正确的解释时,我们可能就没法搞清楚它到底指的是什么。由于文字上没有区分,虽然你号称能看得懂,我们却只能靠猜了。”退一步来说,就算大家都能或多或少看懂一些,可当一个词的形态发生变化时,还怎么能保证大家还能通过这套书写系统达到同样的认识呢?法律术语中就有很多这样的情况。因此我可以指出你并没有给出充足的理由来证明我们的语言不需要这九个字母a, ǫ, e, ę, ı, o, ø, u, y,相反,我还能用这九个字母举出三十六处区别,每种情况中发音都是不一样的,因此如果不区分字形就会产生混淆。
\end{translation*}
\begin{grammar*}{}
    \begin{enumerate}[leftmargin=*]
        \item ráðir vel at líkindum

              at líkindum, 字面义`at probability', 这个短语整体做副词用,表示“有可能”。

        \item enda er þá eigi ørvænt, at þeygi {\dots}

              ørvænt, 形容词ørvænn `beyond expectation'的中性主格形式。þeygi, þó-eigi的缩略,`yet not'. 本句是多重否定结构, `then it is not beyond expectation that {\dots} not {\dots}', 即`then it is possible that {\dots} not {\dots}'.

        \item ef sá finnsk

              finna的反身式有微弱的被动含义,`if that can be found'. sá指的是mínn maki `my mate'.

        \item ráða ek vel at líkindum til hvers ins rétta fœra skal

              ráða有“阅读,理解”的含义。til hvers ins rétta fœra skal, 字面义`which correct (interpretation) is to be given'. 本句是复杂的疑问句。hvers ins rétta, 比较罕见的对特指形式(inn rétti)提问的情况。til {\dots} fœra skal, fœra til表示类似于`lead to'的含义。另外,这句话省略了主语,可以翻译成英语中的被动句。其完整的陈述句形式类似于þat orð skal fœra til ins rétta þýðing `the word will lead to the right interpretation', 对fœra的宾语提问时,整个介词短语要提到句首。

        \item ef fleiri vega má fœra til rétts en einn veg

              字面义`if more way than one will lead to correct (interpretation)'. 注意fleiri vega中的vega是复数属格。这个短语本身应作主格,但古诺尔斯语中有时可用比较级+复数属格表示同样的意思,此时比较级的变形决定了整个短语的语法范畴。

        \item þú lézk þat vel kunna

              宾格-不定式结构,lézk来自于láta的引申义`express, say'.

        \item þá er þó vís ván, at {\dots}

              字面义`then the hope is certain that {\dots}',表示肯定的语气,`then it is certain that {\dots}'

        \item allra helzt

              固定短语,`especially'.

        \item sex greinir ins fjórða tegar

              注意这里数词表达的方法,sex ins fjórða tegar `six of the fourth tens', 指的是36。因为fjórði tigir指的是30-40. (0-10是第一个tigr, 10-20是第二个,以此类推)
    \end{enumerate}
\end{grammar*}
\begin{paracol}{2}
    Nú mun ek þessa stafi átta, alls engi grein er enn ı[s] gǫr, á meðal inna sǫmu tveggja samhljóða setja sitt sinn hvern, en sýna ok dœmi gefa, hvé sitt mál gøri hverr þeira við inna sǫmu stafa fullting, í inn sama stað settr hverr sem annarr, ok á þann veg svá gefa dœmi of allan þenna bœkling á meða[l] inna líkustu greina, þeira er á stǫfunum verða gǫrvar: sar, sǫr; ser, sęr; sor, sør; sur, syr.
    \switchcolumn
    Now I shall compare these eight letters, as no distinction is made for \textit{ı}. I will in turn put it between the same two consonants, to show and give examples of how each of them makes its meaning when supported by the same letters and placed in the same position. And in this way, I will give examples throughout this little book, of the distinctions made between the most similar words: sar, sǫr; ser, sęr; sor, sør; sur, syr.
\end{paracol}
\begin{translation*}{}
    现在我就比较其中八个字母,因为没有字母和ı形成区分。举例时,我每次把需要区分的元音发在两个相同的辅音之间,由此来展示不同的元音在同一个环境下是如何产生出不同的意思的。在这本小书里,我将使用以下几个长相最相近的单词sar, sǫr; ser, sęr; sor, sør; sur, syr。
\end{translation*}
\begin{grammar*}{}
    \begin{enumerate}[leftmargin=*]
        \item sitt sinn hvern

              sinn hverr是一个固定短语,表示`each',作形容词用,但是这个短语可以拆开使用。这里就是向sinn hverr的中性形式sitt hvern中间插入中性名词sinn `time'. 下面的hvé sitt mál hverr þeira也是类似的结构。请注意不要把sinn的意思搞混。
    \end{enumerate}
\end{grammar*}
\begin{paracol}{2}
    Sar veitti maðr mér eitt, sǫr mǫrg veitta ek honum [...] Sor goðinn ein sørin. Sur eru augu syr, slík duga betr en spryngi ýr.
    \switchcolumn
    A man put a wound (sar) on me, I put many wounds (sǫr) on him. [...] The priest swore (sor) the oath (sœrin). Sow's (syr) eyes are sour (sur), it is more the case than they popped.
\end{paracol}
\begin{translation*}{}
    “伤口”的单数是sar, 但复数是sǫr. 【原文漏掉了ser和sęr的比较】 “发誓”的过去式是sor. “母猪”叫作syr, 但“酸楚”拼作sur.\\

    【注:sar和sǫr今均作sár; ser今应作sér, sęr应作sær, 但由于没有例文,亦可能有别的解读;sor今作sór, 系sverja的过去式,sørin今作sœrin,原形sœri;sur今作súr, 原形súrr, syr今作sýr.】
\end{translation*}

\begin{paracol}{2}
    En nú elr hverr þessa stafa níu annan staf undir sér, ef hann verðr í nef kveðinn. Enda verðr sú grein svá skýr, at hon má ok máli skipta, sem ek sýni hér nu eptir; ok set [ek] punkt fyr ofan þá, er í nef eru kveðnir: har, hȧr; rǫ, rǫ̇; þel, þėl; fęr, fę̇r; ısa, ı̇sa; orar, ȯrar; øra, ø̇ra; þu at, þu̇at; syna, sẏna.
    \switchcolumn
    Now each of these letters has another nine letters (derived) from it, when it is pronounced in the nose. And this distinction is so clear that it may change the meaning, which I will show here, and I put a dot above those that are spoken in the nose: har, hȧr; rǫ, rǫ̇; þel, þėl; fęr, fę̇r; ısa, ı̇ sa; orar, ȯrar; øra, ø̇ra; þu at, þu̇at; syna, sẏna
\end{paracol}
\begin{translation*}{}
    上述的九个字母各自衍生出一个鼻化的元音,这个区别是相当显著的,因为它甚至可以改变词意。我在这里用元音+一个上标点的形式标记鼻化元音,试比较:har, hȧr; rǫ, rǫ̇; þel, þėl; fęr, fę̇r; ısa, ı̇ sa; orar, ȯrar; øra, ø̇ra; þu at, þu̇at; syna, sẏna.
\end{translation*}

\begin{paracol}{2}
    Har vex á kykvendum, en hȧr er fiskr. Rǫ er eitt tré ór seglviðum, en rǫ̇ er hyrning húss. Þel er á hnefa bundnum eða hlutr feldar, en þėl er smíðartól. Annat er þat, er sauðrinn heitir fęr, en annat þat, er han fę̇r lambs. İ sa skýja deild, þá er vér komum í ısa. Orar eru órøkþir ȯrar. Spakt skyldi it ellzta barn, því at it ellra má øra it ø̇ra. þar vart þu at, er fjaðrklæðit þu̇at\footnotemark[1]. Þriggja syna austr mun ek þér sẏna. Nú verðr þetta allt saman raddarstafanna: a, ȧ; ǫ, ǫ̇; e, ė; ę, ę̇; ı, ı̇; o, ȯ; ø, ø̇; u, u̇; y, ẏ.
    \switchcolumn
    Hair (har) grows on living creatures, but shark (hȧr) is a kind of fish. Yard (rǫ) is a pole from a sail, but nook (rǫ̇) is the corner of the house. Nap (þel) is on a fist that is bound up or part of a cloak, but file (þėl) is the tool of a smith. For one thing, a sheep is called fęr, but it is quite another when someone gets (fę̇r) a lamb. We can see through (ı̇ sa) the opening in the cloud when we come on an iceberg (ısa). Madness (Orar) is our (ȯrar) bad reason. The oldest son should be quiet, for the older may madden (øra) the younger (ø̇ra). You were there (þu at) and ???. I will show (sẏna) you bilge water three planks (syna) deep. Now this makes the total of vowels: a, ȧ; ǫ, ǫ̇; e, ė; ę, ę̇; ı, ı̇; o, ȯ; ø, ø̇; u, u̇; y, ẏ.
\end{paracol}
\footnotetext[1]{意义不明。历来的手稿编辑者均不译出。}
\begin{translation*}{}
    “头发”被称为har, 但hȧr却指的是“鲨鱼”;船上的“桅杆”叫rǫ, 但房子的“角落”是rǫ̇;þel指的是织物上的“短绒”,但þėl却是“锉刀”;fęr说的是“羊”,可fę̇r是“得到”的现在时;“冰山”叫作ısa, 但“看穿”叫ı̇ sa; orar是“癫疯”的意思,而ȯrar意思是“我们的”;øra指的是“让人发疯”,但ø̇ra说的是“更年轻”; þu at表示“你在”,但þu̇at是???;syna是“板”的复数属格,但sẏna是“展示”的意思。\\

    【注:har和 h\.{a}r今均作hár; rǫ和r\.{ǫ}今均作rá; þel今仍作þel, þ\.{e}l今作þél; fęr和f\.{ę}r今均作fær; ısa今作ísa, 原形íss, \.{ı}sa今作í sá; orar今作órar, \.{o}rar今作várar或órar; øra今作œra, \.{ø}ra今作œra, 原形œri, 系ungr的比较级的古体形式;þu at今作þú at; syna今作sýna, sýja的复数属格,极罕见;s\.{y}na今作sýna.】
\end{translation*}
\begin{paracol}{2}
    En þó at ek rit eigi fleiri raddarstafi en raddirnar fundusk í váru máli, átján gǫrvar ór fimm látínurǫddum, þá er þó gott at [v]ita þat, at er grein enn á raddarstǫfum bæði þeim, er áðr váru í stafrófi, ok þeim ǫðrum, er nú eru í gǫrvir, grein sú er máli skiptir, hvárt stafr er langr eða skammr, sem grikkir ríta í ǫðru líkneski langan staf, en í ǫðru skamman. Svá ríta þeir e skamman: ε, en svá langan sem sjá stafr er: η; þann veg o skamman: [ο], en þann veg langan: ω. þá grein vil ek enn sýna, því at hon skiptir máli ok, jafnt sem inar fyrri, ok merkja ina lǫngu með stryki frá inum skǫmmum: far, fár; rȧmr, rámr; ǫl, ǫ́l; uǫ̇n, uǫ́n; seþo, séþo; frȧmėr, frȧ mér; uęr, uę́r; uę̇nesc, uę́nesc; uıl, uíl; minna, mínna; goþ, góþ; mȯna, móna; Goþrøþe, góþ rǿþe; mø̇nde, mǿnde; dura, dúra; ru̇nar, rúnar; flytr, flýtr; brẏnna, brýnna.
    \switchcolumn
    Now even though I do not write more vowel letters than the vowels found in our language, that is the eighteen letters made out of five Latin vowels, it is yet good to know that there is a distinction between vowels both in the vowels that were already in the alphabet, but also in those added. The distinction can change the meaning, (according to) whether the letter is long or short, just as Greeks write long letters in a shape different from short ones. Thus, they write short e as ε but the long e like this: η, they write short o as ο, but long o as ω. I also wish to show this distinction for it changes meaning as well, just like the previous one, and I will mark the long one with a stroke from the short one: far, fár; rȧmr, rámr; ǫl, ǫ́l; uǫ̇n, uǫ́n; seþo, séþo; frȧmėr, frȧ mér; uęr, uę́r; uę̇nesc, uę́nesc; uıl, uíl; minna, mínna; goþ, góþ; mȯna, móna; Goþrøþe, góþ rǿþe; mø̇nde, mǿnde; dura, dúra; ru̇nar, rúnar; flytr, flýtr; brẏnna, brýnna.
\end{paracol}
\begin{translation*}{}
    这十八个元音字母由五个拉丁语的元音延伸出来,构成了我们语言中的全部元音。即便我现在不再添加元音符号了,但是读者最好仍要记住,这些原有的元音和新添加的元音还可以构成对立。这种对立能改变词意,它是通过改变元音的音长产生的。希腊语中,长短元音的写法是不一样的,比如把短的e写作ε,长的e写作η; 短的o写作ο,长的o写作ω。我也想按类似上面的方法构造出某种分别,因为这种区分确实会改变意思,因此我在字母上方加一杠来表示长元音,试比较:far, fár; rȧmr, rámr; ǫl, ǫ́l; uǫ̇n, uǫ́n; seþo, séþo; frȧmėr, frȧ mér; uęr, uę́r; uę̇nesc, uę́nesc; uıl, uíl; minna, mínna; goþ, góþ; mȯna, móna; Goþrøþe, góþ rǿþe; mø̇nde, mǿnde; dura, dúra; ru̇nar, rúnar; flytr, flýtr; brẏnna, brýnna.
\end{translation*}
\begin{paracol}{2}
    Far heitir skip, en fár nǫkkurs konar nauð. Rȧmr er sterkr maðr, en rámr inn hási. Ǫl heitir drykkr, en ǫ́l er band. Tungan er málinu uǫ̇n, en at tǫnnunum er bitsins uǫ́n. Seþo, hvé vel þeir séþo er fyr saumfǫrinni réðu. mjǫk eru þeir menn frȧmėr, er eigi skammask at taka mína konu frȧ mér. Svá er mǫrg við uęr sínn uę́r, at varla of sér hon af honum nær. Uę̇nesc eigi góðr maðr því, þó at vándr maðr uę́nesc góðum konum. Dul vættir ok uıl, at lina muni erfiði ok uíl. Huglan mann vil ek minna hugþrá ørenda mínna. Sú kona gǫfgar goþ, er sjálf er góþ. Mȯna mín móna, kveðr barnit, við mik gøra verst hjóna. Vel líkuðu Goþrøþe góþ rǿþe, þat eru góðar árar, sem skáld kvað:

    \switchcolumn

    Vessel (far) is the name of a ship, but danger (fár) is a kind of distress. A strong man is powerful (rȧmr) but a sore throat is hoarse (rámr). Ale (Ǫl) is a drink, but strap (ǫ́l) is a kind of leather band. The tongue is accustomed (u\.{ǫ}n) for speech, but the teeth are expected (u\'{ǫ}n) to bite. Behold (Seþo), how well they sewed (séþo) the seam of the ship's plankings when they are in charge. These men are unscrupulous (frȧmer) as they are not ashamed to take my wife from me (frȧ mér). So many women are fond of (uę́r) her husband (uęr), that she hardly keeps her eyes off him. A good man should not fall into bad habits (uę̇nesc), even though an evil man boasts of (uę́nesc) (getting laid with) a good woman. Proud man hopes and wills (uıl) that hard work and wretchedness (u\'{ı}l) will give away. I will remind (minna) thoughtful man of my (mínna) important errands. The woman who is good (góþ) worships god (goþ). My mammy (móna), says the child will not (mȯna) treat me like the worst of the household. Godred (Goþrøþe) well like góþ rǿþe, that is, good oars, as the skald says:

    \switchcolumn*

    \begin{quote}
        Rétt kann rǿþe slíta\\
        ræsis herr ór verri.\\
    \end{quote}

    \switchcolumn

    \begin{quote}
        Straight can king's men cut\\
        Through the sea with oars \\
    \end{quote}

    \switchcolumn*
    Leka mø̇nde húsit, ef eigi mǿnde smiðrinn. Ef gestrinn kveðr dura, þá skyldi eigi bóndinn dúra. Ru̇nar heita geltir, en rúnar málstafir. Se þú hvé flotinn flýtr, er sækarlinn flytr. Stýrimaðr þarf byrinn brýnna, en sá er nautunum skal brẏnna.

    \switchcolumn
    The house would (mø̇nde) leak, if no craftsman had set a roof (mǿnde). If a guest knocks on the door (dura), then the host shall not doze (dúra). Male pigs are called boars (ru̇nar), but letters are called runes (rúnar). Behold, how raft floats (flýtr) when seaman steers (flytr) it. The helmsman needs a straight (brýnna) wind than who is to water (brẏnna) the cattle.
\end{paracol}
\begin{translation*}{}
    far指的是“帆船”,而fár是“危险”;
    r\.{a}mr形容人“强壮”,但rámr指的是喉咙“嘶哑”;
    ǫl说的是“麦酒”, 而ǫ́l是一种“皮带”;
    u\.{ǫ}n是“习惯”的阴性形式,而uǫ́n是“期待”的意思;
    seþo是“你看”,而séþo是“缝纫”的意思;
    fr\.{a}mer说人“无礼”,而frȧ mér是“从我这里”的意思;
    uęr的意思是“丈夫”,而uę́r是“欢喜”的阴性形式;
    u\.{ę}nesc是“习惯于”的意思,而uę́nesc是说人“自吹自擂”;
    uıl指人的“愿望”,但uíl说的是“苦难”;
    m\.{ı}nna是“提醒”的意思,而mínna是物主代词“我的”;
    goþ是对“神”的称呼,而góþ是“好”的阴性形式;
    mȯna是“不想要”,而móna是“奶妈”的意思;
    Goþrøþe是人名“Godred”,但góþ rǿþe是“好的桨”;
    m\.{ø}nde是“可能”的意思,而mǿnde说的是“修缮屋顶”;
    dura说的是“房门”,而dúra是“打盹”的意思;
    r\.{u}nar是对“公猪”的称呼,而rúnar是一种“字母”;
    操作船叫作flytr,可flýtr是“漂浮”的意思;
    br\.{y}nna是“给牲畜喂水”, 而brýnna是说风“强劲”。\\

    【注:
    far和fár今形式不变;
    r\.{a}mr今作ramr或rammr, rámr今形式不变;
    ǫl今形式不变,ǫ́l今作ál;
    u\.{ǫ}n今作vǫn, 原形vanr, u\'{ǫ}n今作ván; seþo今作sé þú, séþo今作séðu;
    fr\.{a}mer今作framir, 原形framr, frȧ mér今作frá mér;
    uęr今作ver, u\'{ę}r今作vær, 原形værr;
    u\.{ę}nesc今作venisk, 原形venja, u\'{ę}nesc今作vænisk, 原形væna;
    uıl今作vil,uíl今作víl;
    m\.{ı}nna今作minna, mínna今作mínna或minna;
    goþ今作goð, góþ今作góð, 原形góðr;
    mȯna今作mun-a, 由munu+否定后缀构成,móna今形式不变;
    Goþrøþe今作Guðrœði, góþ rǿþe今作góð rœði;
    m\.{ø}nde今作myndi, 原形munu, mǿnde今作mœndi, 原形mœna;
    dura今作dura或dyra, dúra今形式不变,也作dúsa;
    r\.{u}nar今作runar, 原形runi, rúnar今形式不变,原形rún;
    flytr今形式不变,原形flytja, flýtr今形式亦不变,原形fljóta;
    br\.{y}nna今作brynna, brýnna今形式不变,原形brýnn.】
\end{translation*}
\begin{paracol}{2}
    Nú ef nǫkkur þessa greina sex ins fjórða tegar má svá niðr falla, at aldri[gi] þurfi í váru máli, þá skjótumsk ek yfir, sem vís ván er; eða svá, ef fleiri finnask í mannsins rǫddu.
    \switchcolumn
    Now if any of these thirty-six distinctions should be wrong, that they are never needed in our language, then I am mistaken, which is quite possible, or (I may also be mistaken) if there are more to be found in men's speech.
\end{paracol}
\begin{translation*}{}
    如果说这三十六处区别中有任何一处是不对的,在我们的语言中不需要区分它们,那就说明我犯了错误,这是相当可能的事情。同样地,如果我们的语言中还能发现其他区别的话,那也是我的疏忽。
\end{translation*}

\begin{paracol}{2}
    En þat er gott at vita, sem fyrr var getit, er svá kveðr at hverjum raddarstaf í hverju máli, sem hann heitir í stafrófi, nema þá er hann hafnar sínu eðli, ok hann má heldr þá sam\-hljóðandi heita en raddarstafr.\footnotemark[1]
    \switchcolumn
    Now it is good to know that, as I have said before, that man pronounces every vowel in whatever sentences just like what it is named in the alphabet, except when it gives up its own nature and (in this case) it may be called a consonant rather than a vowel.
\end{paracol}
\footnotetext[1]{作者在这里讨论的,实际上就是半元音。}
\begin{translation*}{}
    大家最好要记住我前面所说的话,所有的元音在任何语境下都按它本身的名字来发音,除非它失去了自己的本性。这种情形下,与其说它是元音倒不如说它是辅音。
\end{translation*}

\begin{paracol}{2}
    Þat verðr þá er hann er stafaðr við annan raddarstaf, sem hér eru nǫkkur dœmi nú: austr, earn, eır, eór, eyrer, uín. Nú er eigi ørvænt, at svá svari nǫkkurr maðr: ``þar er orð, at þú rítr þar [e], er flestir menn ríta ı, þá er hann verðr fyr samhljóðanda settr, sem nú er skamt frá því, er þú reitt earn, þar sem ek munda ıarn ríta, eða svá í mǫrgum stǫðum ǫðrum." þá svara ek svá: ``þú hefir þar rétt fundit, ok þó eigi alls getit þess, er þér má ek kynliga þykkja ritit hafa, ok þó hafa ek fyr ǫnnkost svá ritit í flestum stǫðum. Ef ek gerða annat mál, sem þar væri full þǫrf ok œrin efni til, er kœnska væri, of þat, til hverra stafa hver orð hafa eðli, eða á hverja lund hverja stafi skyldi saman stafa, þá væri sú bók ǫnnur ǫll ok miklu meiri, ok má ek af því eigi þat mál nú mæla innan í þessu. En þó mun ek nǫkkurum orðum svara um þetta it eina orð, er þú skoraðir helzt í."
    \switchcolumn
    This happens when it is joined with another vowel, as now the example shows: austr, earn, eır, eór, eyrer, uín. Now it is not unlikely that some man would say: ``There is a word where you write an \textit{e} while most people would write \textit{ı}, when it is used as a consonant, as it is (pronounced) shorter. When you just wrote \textit{earn}, I would write \textit{ıarn}, and so in many other places." And to this I say: ``You have made the right observation, and yet you haven't mentioned everything that may seem strange to you in what I have written, even though I have intentionally written it in most places. If I were to write another book, as there is sufficient necessity and adequate material for it, if only I had the wits to elaborate what letters make up the nature of each word, and in what way each letter should be combined together, and that would be a book totally different and much longer, and therefore, I cannot elaborate the idea here for now. But I would still like to say something about this one word that you have pointed out."
\end{paracol}
\begin{translation*}{}
    当一个元音和另一个元音相邻的时候就会发生上述的情况,譬如这些词austr, earn, eır, eór, eyrer, uín. 当然,肯定有些人会这么说:“这些词中的e大多数人会写作ı,因为它的发音更短些。比如你写成earn,我更愿意写作ıarn,以此类推。”对此我的回应是:“你说的对。我在文中很多地方都特意这么写,不过你还没有把它们全找出来。解释这个问题很有必要,同时也有充足的语料作支撑。如果我有余力来写另一本书解释单词的本性如何以及字母之间要如何组合的话,那就会是一篇截然不同的文章,篇幅也更长。因此,我在这里不能展开解释了,不过还是可以对这个现象简单说一点。”
\end{translation*}
\begin{grammar*}{}
    \begin{enumerate}[leftmargin=*]
        \item er þér má ek kynliga þykkja ritit hafa

              字面的意思是`must seem strange to you that I should have written this word',  hafa是第一人称单数虚拟式,表示与预期相反的情况。注意er在这里充当hafa ritit的宾语。
    \end{enumerate}
\end{grammar*}
\begin{paracol}{2}
    Fyr því at þat hljóð, er samhljóðandinn hefir, eða sá raddarstafr, er í hans stað er settr ok stafaðr við annan raddarstaf, er eigi auðskilit, því at lítit verðr ok við blandit nær eða gróit við raddarstaf þann, er viðr er stafat, þá er þess leitanda, hvar svá finnim vér kveðit it sama orð, at sá raddarstafr sé frá ǫðrum raddarstaf skilinn, ok gøri sína samstǫfun hvarr, er optast er viðr stafaðr, svá at eina samstǫfun gøra báðir.
    \switchcolumn
    Since the sound that the consonant possesses, or of the vowel put in the place of the consonant and combined with another vowel, is not easy to distinguish, as it is short and almost mingled or grown together with the vowels that it is connected to, therefore it must be sought where we would find this same word that it is pronounced (in such a way) that the (first) vowel is separated from the other, and each makes its syllable, when in most cases the two vowels are combined so that they make one syllable together.
\end{paracol}
\begin{translation*}{}
    辅音携带的声响是短小而不容易区分的,而当一个元音被放置在辅音的位置并且与另一个元音结合时,它的声响也如辅音一般与其毗邻的元音混杂在一起。因此我们必须要找到这样的单词,其中的两个元音互相分离,各自构成自己的音节,而一般情况下,两个元音是结合在一起共同构成一个音节的。
\end{translation*}
\begin{grammar*}{}
    \begin{enumerate}[leftmargin=*]
        \item þá er þess leitanda

              现在分词表示义务和必要性的例子, `then it should be sought'.
    \end{enumerate}
\end{grammar*}
\begin{paracol}{2}
    Skáld eru hǫfundar allrar rýnni eða málsgreinar, sem smiðir [smíðar] eða lǫgmenn laga. En þessa lund kvað einn þeira eða þessu líkt:
    \begin{quote}
        Hǫfðu hart of krafðir\\
        Hildr óx við þat skildir\\
        Gang, enn gamlir sprungu\\
        Gunnþings earnhringar\footnotemark[1]
    \end{quote}

    \switchcolumn
    Skalds are the judges in all matters of grammar, just as craftsmen [in crafts] and lawmen in laws. And in this way one of them made a poem like this:
    \begin{quote}
        Shields, hard pressed\\
        Had given away and sprang apart\\
        Iron-rings of battle-meetings [mail-shirt]\\
        Battle increased at that\\
    \end{quote}

\end{paracol}
\footnotetext[1]{这首诗是11世纪挪威王奥拉夫二世(奥拉夫·哈拉尔松,Olaf Haraldsson)的宫廷诗人Óttarr svarti (Óttarr the Black)所作,描述的是奥拉夫率军入侵伦敦的故事。这首诗按散文方法可以写作:\\
    Skildir hart of krafðir, hǫfðu gang, enn gamlir
    gunnþings earnhringar sprungu. Hildr óx við þat. 其中gunnþings earnhringar有被解读为“刀剑”或“盔甲”的。关于本诗的其他解读,参见\cite{townend2012ottarr}.
}
\begin{translation*}{}
    吟游诗人是语法方面的行家,他们精通文法一如匠人精通手艺,律师熟稔律法。有一个诗人写了这样的一首诗:
    \begin{quote}
        盾牌受压难支撑\\
        刀锋相击声不停\\
        盔甲崩裂敌欲走\\
        战火愈烈难平息
    \end{quote}
\end{translation*}

\begin{paracol}{2}
    Nú þó at kveðandin skyldi hann til at slíta eina samstǫfu í sundr ok gøra tvær ór, til þess at kveðandi haldisk í hætti, þá rak hann þó engi nauðr til þess at skipta stǫfunum ok hafa e fyr ı, ef heldr ætti ı at vera en e, þó at mér lítisk eigi at því.\footnotemark En ef nǫkkurr verðr svá einmáll eða hjámáll, at hann mælir á mót svá mǫrgum mǫnnum skynsǫmum, sem bæði létusk sjálfir kveða þetta orð, áðr ek reit þat, ok svá heyra aðra menn kveða, sem nú er ritit, ok þú lætr ı skulu kveða, en eigi e, þó at þat orð sé í tvær samstǫfur deilt, þá vil ek hafa ástráð Cátónis, þat er hann réð syni sínum í versum:

    \switchcolumn
    Now even though the metric would make him split the one syllable into two, so that the line can stand in tune, but yet there is no need for him to change the letter and use \textit{e} for \textit{ı}, if it should have been \textit{ı} instead of \textit{e}, which I disagree. And if some man is so insistent and absurd that he speaks against many sensible men, who have declared that they pronounce this word, before I wrote this, and hear others pronounce this, as what is written now, and you insist that it should be pronounced \textit{ı} but not \textit{e}, even if that word is divided into two syllables, then I will give you Cato's kind advice, with which he advised his son in these verses:

    \switchcolumn*

    \begin{quote}
        Contra verbosos noli contendere verbis\\
        Sermo datur cunctis, animi sapientia paucis
    \end{quote}

    \switchcolumn

    \begin{quote}
        Do not contend with verbose people in words\\ Speech is given to all, the wisdom of mind to few
    \end{quote}

\end{paracol}
\footnotetext[1]{这里作者的解释历来有争议。一些学者认为这个词写作ıarn(现作járn)也能满足格律,另外在原始语中的形式是*īsarną(借自原始凯尔特语*īsarnom),亦能佐证这个元音就是i. 如果认为本文作者的解释是对的,他描述的现象可能反映的是古诺尔斯语元音分割的过程:*e > ea > ia > ja. 其中,从ea向ia变化的过程中,重音从第一个音节移到了第二个音节,使得ia中的i进一步辅音化,形成了半元音j.}
\begin{translation*}{}
    即便格律可能要求他把earn分成两个音节,这样韵律才会整齐。不过他仍然没有必要用e替代ı,如果这个音本来就是拼作ı 而不是 e,因此我不同意这个音拼作ı. 过去的诗人在我写这本书前就记录下了这个词是如何拼读的,如果有人非要固执己见而与这些智者背道而驰,坚称这里应该拼成ı而不是e,那么我只好把卡托的建议送给你,这是他教育自己的儿子的:
    \begin{quote}
        勿与多言者争辩,空耗心力无所获\\
        众人皆能夸海口,真知灼见寥寥数
    \end{quote}

\end{translation*}
\begin{grammar*}{}
    \begin{enumerate}[leftmargin=*]
        \item í hætti

              háttr本是“习惯、方式”之类的意思,í hætti或eptir hætti在这里表示押韵。

        \item rak hann þó engi nauðr

              字面义`no necessity pushed him', nauðr用法特殊,它只用在主语的位置,因此有很多比喻性的说法。这里即表示“没有必要做某事”。
    \end{enumerate}
\end{grammar*}
\begin{paracol}{2}
    þat er svá at skilja: hirð eigi þú at þræta við málrófsmenn; málróf er gefit mǫrgum, en spekin fám. Nu lýk ek hér umrœðu raddarstafanna, en ek leita viðr, ef guð lofar, at rœða nǫkkut um samhljóðendr.
    \switchcolumn
    That is to say: do not quarrel with a glib talker, a big talk is given to many, but wisdom to few. Now I end here the discussion of the vowels, but I will try, if God permits, to say something more about consonants.
\end{paracol}
\begin{translation*}{}
    意思是说:不要与说大话的人争吵,很多人都能夸夸其谈,而有真知灼见的人却很少。对于元音的讨论到此为止,接下来,若是上苍助我,我再讨论一点关于辅音的事。
\end{translation*}

\section{辅音论}
\begin{paracol}{2}
    Í nafni samhljóðanda hvers sem eins er nǫkkurr raddarstafr, því at hvárki nefnir þau nǫfn né ǫnnur engi, ef þeir njóta eigi raddarstafa, sem fyrr var sagt. Nú þó at þat hljóð eða atkvæði, er samhljóðendr hafa, megi varla eitt saman at kveða, enda sé þó nauðr at skilja, hvat þeir stoða í málinu. Enda stoði engi þeira þat allt í málinu, sem nafn hans er til, sem raddarstafirnir gøra, þá mun ek svá haga nafni hvers þeira, er áðr hafði eigi svá nafn til, at þá skal af nafninu skilja hvat hann stoðar í málinu, þó at áðr skili eigi. Skal þat atkvæði hvers þeira í hverju máli vera, sem þá lifir nafnsins eptir, er ór er tekinn raddarstafr ór nafninu.
    \switchcolumn
    In the name of every consonant there is a vowel, because they cannot name themselves or do anything else if they do not have a vowel, as is said before. Even though the sound or syllable of the consonant can hardly be pronounced by itself alone, it would be yet important to point out its significance in sentences. Since none of the consonants is pronounced like its full name, as the vowels do, I shall arrange a (new) name for every one of them, a name that they have never been called, and, shall make others understand their value in the context even though they don't know in advance. The pronunciation of each consonant in whatever context should be what remains in the name when the vowel is taken away.
\end{paracol}
\begin{translation*}{}
    每个辅音的名字里都是有元音的,因此辅音既不能单独拼出自己的名字,也什么都做不了,这在前面已经说过了。不过即使辅音的声响或其构成的音节不能独自成词,讨论辅音在语境中的重要性仍是必要的。由于没有哪一个辅音是根据自己的全名拼读的,我给每个辅音起了一个新名字。这个名字从未有人用过,但能让人一眼就看出在语境中这个辅音的音值,即便他事先并不知道这里是怎么拼的。每个辅音的音值等于其名字去掉元音后剩下来的部分。
\end{translation*}
\begin{grammar*}{}
    \begin{enumerate}[leftmargin=*]
        \item hvat þeir stoða í málinu

              stoða本是“帮助,支持”的意思,在本文的语境中,stoða表示“发...的音”(声音由字母承载或“支持”)。
    \end{enumerate}
\end{grammar*}
\begin{paracol}{2}
    B, [c], d, g, h, p, t, þeir stafir hafa af því mundang mikit eins stafs atkvæði, at aldri má tvá samhljóðendr ins sama hlutar setja í einni samstǫfun fyr raddarstafinn.
    \switchcolumn
    \textit{B}, [\textit{c}], \textit{d}, \textit{g}, \textit{h}, \textit{p}, \textit{t}, these letters have the average length of one letter, and in no case can two identical consonants be put before a vowel in one syllable.
\end{paracol}
\begin{translation*}{}
    b, c, d, g, h, p, t,这些字母的音长平均等于一个字母的长度,而且在任何情况下一个音节中都不会出现两个同样的字母出现在一个元音前的情况。
\end{translation*}
\begin{grammar*}{}
    \begin{enumerate}[leftmargin=*]
        \item tvá samhljóðendr ins sama hlutar

              字面义`two consonants of the same value',指的实际上是双辅音。
    \end{enumerate}
\end{grammar*}
\begin{paracol}{2}
    F, l, m, n, r, s, þeir stafir megu hafa tveggja samhljóðanda atkvæði hverr einn, ef svá mjǫk vil at kveða, svá sem hverr þeira, er eptir raddarstafinn verðr settr, sem þar berr vitni, er vér nefnum þá með svá miklu atkvæði, sem mundim vér, ef svá skyldi ríta nǫfn þeira: eff, ell, emm, enn, err, ess. Má ok minka atkvæði þeira, þó at þeir standi eptir raddarstaf í samstǫfun, ok sé svá nefnd[i]r, sem þessa kostar væri ritin nǫfn þeira: ef, el, em, en, er, es, sem ek læt þá svá heita alla ok aldri hafa meirr en eins stafs atkvæði hvern, hvárt sem þeir standa fyr raddarstaf í samstǫfun eða eptir, nema þar er ek rít samhljóðanda, hverngi er ek rít, með vexti hǫfuðstafsins, enda standi hann eptir raddarstafinn í samstǫfun. Þá læt ek þann einn jarteina jafnmikit, sem þar væri tveir eins konar ok ins sama konar ritnir, til þess at rit verði minna ok skjótara ok bókfell drjúgara.
    \switchcolumn
    \textit{F}, \textit{l}, \textit{m}, \textit{n}, \textit{r}, \textit{s}, these letters can each have the sound of two consonants, if a man wishes to pronounce that long, just like when they are placed after vowels, which is easy to find out if we give them names with long syllables and write them as \textit{eff}, \textit{ell}, \textit{emm}, \textit{enn}, \textit{err}, \textit{ess}. And this can be shortened, even when they stand after the vowel in a syllable, and these would be their names, as they are written in this wise: \textit{ef}, \textit{el}, \textit{em}, \textit{en}, \textit{er}, \textit{es}. So I will name them in this way that they never have more than one letter in each syllable, whether they stand before or after the vowel of the syllable, unless I write any one of the consonants with the shape of a capital letter and it follows the vowel in its syllable. Then I let this one character represent what two of the same kind stand for, so that the writing may be smaller and quicker, and it would save the parchment.
\end{paracol}
\begin{translation*}{}
    f, l, m, n, r, s,只要人们愿意拖长发音,这些字母的音长可以达到两个辅音的长度,比如当它位于元音之后时。如果我们给它们的名字赋一个长音节并分别写作:eff, ell, emm, enn, err, ess,那么就可以清楚地看到这种情况。但它们也可能发短音,而且元音后也允许这种情况。于是它们的名字便写作ef, el, em, en, er, es. 用这种方式给它们命名可以使得在任何一个音节中同一个辅音字母都不会出现两次,无论它处于元音前还是元音后。如果要表示元音后的长辅音,则用一个大写字母来表示两个同质的音,这样可以书写得更快些,也能节省纸张。
\end{translation*}
\begin{grammar*}{}
    \begin{enumerate}[leftmargin=*]
        \item sem þar berr vitni

              bera vitni `bear witness'的无人称用法,不强调bera的主语。

        \item bókfell drjúgara

              drjúgara是drjúgr的比较级,这个形容词本是“坚固”的意思,后来延伸为“持久”,最后延伸出“用量少,节约”的含义。
    \end{enumerate}
\end{grammar*}
\begin{paracol}{2}
    Nú þar er þeir stafir eru, er raddarstaf hefir eptra í nafninu, sem eru: b, c, d, g, p, t, ok af því of eykr eigi atkvæði nafns hvers þeira, þá skipti ek þar hǫfuðstafsins nafni, ok set ek þá raddarstaf fyrr, til þess at aukask megi atkvæði þeira svá í nafninu, sem annars staðar skulu þeir í málinu jarteina. Skal nú hverr samhljóðandi jafnmikit sítt atkvæði leggja til lags við raddarstaf þann, er í nafni hans er, sem hann skal hafa við hverngi annarra, er han verðr [við] stafaðr í hverju máli.
    \switchcolumn
    Now there are those letters that have a vowel at the end of their names, namely \textit{b}, \textit{c}, \textit{d}, \textit{g}, \textit{p}, \textit{t}, and therefore the sound of their names cannot be extended, then I would like to change the names of their capital letters and place the vowel in the front, in order that the syllable in their names can be lengthened according to what it should represent in other places. Now each consonant should properly add an equally long sound to the vowel in its name as it does to any other vowel that it is combined within any context.
\end{paracol}
\begin{translation*}{}
    现在给出的这些字母的名字结尾处有一个元音,因此它们的音值不能延长了。这些字母是b, c, d, g, p, t,对于它们的大写字母,我准备把元音放到它们名字的前头去,这样可使其音值按照情况延长。所有辅音现在都以一个固定的长度附着在其名字中的元音上,对于其他语境中的元音,辅音以相同的方式与之结合。
\end{translation*}
\begin{grammar*}{}
    \begin{enumerate}[leftmargin=*]
        \item annars staðar skulu þeir í málinu jarteina

              annars staðar都是单数属格,整体作副词用,表示“其他地方”。annars本身也可以单独作副词用,相当于`else'.

    \end{enumerate}
\end{grammar*}
\begin{paracol}{2}
    En fyr því nú, at sumir samhljóðendr hafa sín líkneski ok nafn ok jartein, en sumir hafa hǫfuðstafs líkneski ok nafn ok jartein, en sumir hafa hǫfuðstafs líkneski ok skip[t] stǫfum sumra í nafni ok aukit atkvæði bæði nafns ok jarteinar, en sumir halda líkneski sínu, ok er þó minkat atkvæði nafns þeira, ok jartein sú, er þeir skulu hafa í málinu, skal þeiri lík, er í nafninu verð[r], þá skal nú sýna leita bæði líkneski þeira ok svá nǫfn fyr ofan ritin, at yfir þat megi nú allt saman líta, er áðr var sundrlauslega umrœtt:
    \switchcolumn
    And for now, some consonants (b, c, d, g, p, t) have their own shape, name and value but some (\textsc{f, l, m, n, r, s}) have their shape, name and shape from a capital letter. However, others (\textsc{b, c, d, g, p, t}) get the shape from a capital letter but with some letters in its name changed or sound lengthened both in name and value. Some others (f, l, m, n, r, s) maintain its shape but with its name shortened to represent what it should be in speech so that its value can be in accord with its name. Now I shall attempt to show the shape of both the shape of the consonant and its name, which is written above, so that one may altogether find what has been treated separately:
\end{paracol}
\begin{translation*}{}
    有些辅音(即b, c, d, g, p, t)有自己的形状、名字和音值,有些辅音(即\textsc{f, l, m, n, r, s})的形状、名字、音值则是从大写字母里借来的。有些辅音(即\textsc{b, c, d, g, p, t})虽然从大写字母那里借来了外形,但改变了原来的几个字母或是延长了音值。剩下的一些(即f, l, m, n, r, s)则保留了原本的形状,但缩短了音长使其名字的发音与这个辅音在语境中的音值一致。我现在写出这些字母的形状,并在上方注上它们的名字,以便读者直接看出辅音的音与形的关系:
\end{translation*}
\begin{grammar*}{}
    \begin{enumerate}[leftmargin=*]
        \item jartein sú, er þeir skulu hafa í málinu, skal þeiri lík, er í nafninu verðr

              jartein在本文中的意思就是“字母的音值”。skal þeiri lík, 省略了系动词vera. líkr或glíkr `alike, similar'接与格(þeirri er ...)。本句的直译是`the sound value that they have in speech should be similar with that in their name'.

    \end{enumerate}
\end{grammar*}
\begin{center}
    \begin{tabular}{ c c c c c c c c c c c c c c }
        be & ebb        & che & ecc        & de & edd        & ef & eff        & ge & egg        & eng      & ha         & el & ell        \\
        b  & \textsc{b} & c   & \textsc{k} & d  & \textsc{d} & f  & \textsc{f} & g  & \textsc{g} & \textcrg & \textsc{h} & l  & \textsc{l} \\
        em & emm        & en  & enn        & pe & epp        & er & err        & es & ess        & te       & ett        & ex & the        \\
        m  & \textsc{m} & n   & \textsc{n} & p  & \textsc{p} & r  & \textsc{r} & s  & \textsc{s} & t        & \textsc{t} & x  & þ
    \end{tabular}
\end{center}

\begin{paracol}{2}
    Sá stafr, er hér er ritinn c, er látínumenn flestir kalla ce ok hafa fyr tvá stafi: fyr t ok s, þá er þeir stafa hann við e eða ı, þó at þeir stafi hann við a eða o eða u sem k, sem svá stafa skotar þann staf við alla raddarstafi í látínu ok kalla che, hann læt ek ok che heita í váru stafrófi, ok stafa ek svá við alla raddarstafi sem k eða q, en þá tek ek úr stafrófi báða, ok læt þenna einn, c, fyrir hvárn hinna ok svá fyr sjálfan sik, alls þeir hǫfðu áðr allir eitt hljóð í [flestum] stǫðum eða jartein. En fyr því at c hefir inn sama vǫxt, hvárt sem hann er hǫfuðstafr ritinn eða eigi, allra helzt er ek rít[ka] þá hǫfuðstafi stœrri en aðra í riti, er eigi standa í vers upphafi, ok skulu tvá stafi jarteina, ok þá rít ek fyr hans hǫfuðstaf þenna staf: \textsc{k}, fyr því at þá hefr hann sínn vǫxt, þó at nǫkkut lægisk við. Er ok eigi allfjartekit til þess vaxtar honum, alls sá stafr stendr í griksku ok heitir kappa ok jarteinir xx í tǫlu þar. En hér skall hann í máli váru standa fyr cc, sem aðrir inir smæri hǫfuðstafir jarteina tvá stafi í máli. Má hann ok í tǫlu várri jarteina cc tírœð, sem ce tvau í látínu. Áðr hann væri fyr tvá stafi settr ok hann hét che, þá hafði hann eftra e en c í nafni sínu, en nú skipti hann ok hafi e fyrst í nafninu ok heiti ecc, enda siti um svá gǫrt.
    \switchcolumn
    The letter, which is written as \textit{c} and most Latin people call \textit{ce}, has the sound of two letters: \textit{t} and \textit{s} when placed before \textit{e} or \textit{ı}, but \textit{k} before \textit{a}, \textit{o}, and \textit{u}. This is how Scots combine this before all vowels in Latin and named \textit{che}. I shall also call it \textit{che} in our alphabet, and pronounce it before all vowels as \textit{k} or \textit{q}, and then I take both these two letters out of the alphabet, and let this one, \textit{c}, represent both of them as well as itself, since they all had one sound or token in most places. But because \textit{c} has only the same shape whether it is written as a capital one or not, and most importantly, I don't write the capital letter any larger than the others in the text, if it doesn't stand at the beginning of a verse and represent two letters. I will write the following letter \textit{\textsc{k}} for the capital, so that it has its own shape even when it is lowered. Also, the shape is not far-fetched, since it stands in Greek and is called \textit{kappa} and represents \textit{xx} in that language. But here it shall stand for \textit{cc} in our language, representing two letters in the sentence just like other small capitals. It may also stand for two hundred in our language, like double \textit{c} in Latin. Before it was used for two letters and called \textit{che}, it had \textit{e} after \textit{c} in its name, but now it would change the sequence and have \textit{e} in the front of its name, and that is all for this.
\end{paracol}
\begin{translation*}{}
    在这里写作c的字母被大多数拉丁人读作ce。当它出现在e或ı之前时,它的发音是t和s的叠加,但在a o u之前读作k。苏格兰人则一律把元音前的c读作k,并给这个字母取名che。我也给这个字母取名che,并使之在一切元音前表示k或q的音,然后把这两个字母从字母表里去掉,而统一用c表示,因为它们三个的音值在大多数情况下是一致的。由于大小写的c形状一致,而且除了在篇章开头的地方之外,我在书写时并不把大写字母写得比其他更大些,因此我用\textsc{k}表示其大写字母,所以即便是它写成比较小的样子,依旧有自己独立的形状。这个做法也并非牵强附会,因为在希腊语中,这个字母读作kappa并且表示xx这个音,但在我们的语言里就用cc,和其他大写的辅音字母一样用来表示两个字母。这个字母也可用来表示两百,就像罗马数字中用两个C表示两百一样。在用来表示双辅音前,这个字母名字中的e本来是在c后面的,但现在则改变了顺序,把e放到前面,称为ecc了。
\end{translation*}
\begin{grammar*}{}
    \begin{enumerate}[leftmargin=*]
        \item lægisk

              lægja `lower'的中动态,这里表示字母写成小写。

        \item siti um svá gǫrt

              sitja um可以表示“不改变某事”,这句话可直译为`leave (what) was done unchanged',即“结束讨论”的意思。
    \end{enumerate}
\end{grammar*}
\begin{paracol}{2}
    Þat n, er stendr fyr g hit næsta í einni samstǫfu, þat er minnr í nef kveðit en meirr í kverkr en ǫnnur n, af því at þat tekr viðbland nǫkkut af g. Nú gøri ek þeim af því vinveittar samfarar sínar, ok gøri ek einn staf af báðum, þann er ek kalla eng, ok rít ek á þessa lund: \textcrg . Hann læt ek jarteina einn sem hina tvá, svá at allt sé eitt, hvárt þú rítr hrıngr eða hrıgr, nema þat er rit minna, er stafir eru færi.
    \switchcolumn
    The \textit{n} which stands before the next \textit{g} in a syllable is pronounced less in the nose and more in the throat than other \textit{n}, for it is influenced by \textit{g}. Now I shall make them closely united, and make one letter from both, which I call \textit{eng}, and write in this way: \textcrg . I let this letter represent the two sounds so that all these would be the same, whether you write hrıngr or hrı\textcrg r, except that the writing and letters are fewer.
\end{paracol}
\begin{translation*}{}
    在g之前的n发音更偏向于喉音而非鼻音,因为其受到了g的影响。因此我把这两个字母合到一起,用一个字母表示它们两个,写作\textcrg ,而读作eng。由于这一个字母可以等价地表示两个音,所以写hrıngr还是hrı\textcrg r并没有任何区别,只是后者写起来更短些。
\end{translation*}
\begin{grammar*}{}
    \begin{enumerate}[leftmargin=*]
        \item vinveittar samfarar

              字面义`friendly travelling-together',指的是两个字母关系紧密。
    \end{enumerate}
\end{grammar*}
\begin{paracol}{2}
    Hvárki hefi ek brugðit vexti né nafni á h, því at hann má hvárki vaxa né þverra, né á engi veg skapask í sínu atkvæði.
    \switchcolumn
    I have changed neither shape nor name for \textit{h}, since it can neither grow longer nor shorter, nor in any way be changed.
\end{paracol}
\begin{translation*}{}
    h的字形、音值和名字我都没有改,因为在任何情况下其发音都不能延长或缩短,也不会变成其他的样子。
\end{translation*}

\begin{paracol}{2}
    X, ẏ, z, \&, [˜], þeira stafa má þarnask, ef vill, í váru máli, því at engi er einka jartein þeira, alls þeir eru fyr þá eina stafi hafðir, er áðr eru í stafrófi, sumir fyr tvá, sem x ok z, \& eða ˜, er fyr fleiri verðr stundum, en sumir fyr einn sem ẏ eða stundum ˜.
    \switchcolumn
    X, ẏ, z, \&, [˜], these letters can be omitted, if one wishes, from our language, because they do not represent any particular sound, but refer to one letter that is already in the alphabet, and some are for two, like x and z, \& or ˜, which can also stand for more letters while some others can only stand for one, like ẏ and sometimes ˜.
\end{paracol}
\begin{translation*}{}
    x, ẏ, z, \&, ˜ 这些字母其实可以从字母表中略去,因为其不代表任何单独的音,而是代指字母表里已有的字母。x, z, \&或 ˜可以代表两个字母,而ẏ和某些环境下的˜则只能代表一个。
\end{translation*}
\begin{paracol}{2}
    X, hann er samsettr í látínu af c ok s. Hann vil ek hafa svá samsettan í váru máli ok ekki sinn láta hann hǫfuðstaf vera, því at hann verðr aldrigi fyr c tvau né s tvau ok eigi í upphafi vers né orðs né samstǫfunnar.
    \switchcolumn
    \textit{X}, it is a combination of \textit{c} and \textit{s} in Latin. I will have it combined likewise in our language and never make a capital letter from it, since it can never stand for double \textit{c} nor double \textit{s} and is by no means at the beginning of a verse, word or syllable.
\end{paracol}
\begin{translation*}{}
    x在拉丁语中代表的是c和s的叠加。在我们的语言中,我也同样用它表示这两个字母的组合。这个字母没有大写形式,因为它不能代表两个c和两个s,也不会出现在任何段落、单词或者音节的开头。
\end{translation*}

\begin{paracol}{2}
    Y, hann er grikkskr stafr ok heitir þar uı, en látínumenn hafa hann fyr ı, ok í grikkskum orðum at eins þó, ef skynsamliga er ritit. Ok þarf hann af því eigi hér í vára tungu, nema maðr vili setja hann fyr u, þá er hann verðr stafaðr við annan raddarstaf ok hafðr fyr samhljóðanda, er þó láta ek af nú at ríta hann, því at ek sékka u þess meiri þǫrf fulltings en ǫðrum raddarstǫfum, þá er þeir verða fyr samhljóðendr settir.
    \switchcolumn
    \textit{Y}, it is a Greek letter and is called \textit{uı} there, but Latin people use it for \textit{ı}, and it is only in Greek that this letter is written correctly. And hence it is of no need here in our language, except when someone wants to use it for \textit{u}, when it stands next to another vowel and is used as a consonant. And even though I write it here for now, I cannot see that \textit{u} needs such great assistance than any other vowels when they are used for consonants.
\end{paracol}
\begin{translation*}{}
    y是一个希腊字母,希腊人管它叫uı,但是拉丁人则用它表示ı,从而只有希腊人的用法是这个字母的本来面貌。因此,我们的语言中没有必要采用这个字母,除非有人想用它表示元音前被辅音化的u。我姑且把这个字母写在这儿,不过我并不觉得这种情况下的u需要额外的字母来表示。
\end{translation*}
\begin{grammar*}{}
    \begin{enumerate}[leftmargin=*]
        \item at eins

              固定说法,相当于副词`only'.
    \end{enumerate}
\end{grammar*}
\begin{paracol}{2}
    Z, hann er samsettr af deleth, ebreskum staf svá ritnum: \textcjheb{d}, ok settr er fyr d, ok af þeim ǫðrum, er heitir sade, ok er svá ritinn: \textcjheb{.s|}, ok er fyr es í látínu settr. Alls hann sjálfr er ebreskr stafr, er þó sé hann í látínustafrófi ok hafðr, því at ebresk orð vaða opt í látínunni. Honum vísa ek heldr ór váru máli ok stafrófi, því at þó verða fyr nauðsynja sakir fleiri stafir í þar, en elligar vilda ek hafa. Vil ek heldr ríta, þeim inum fám sinnum er þarf, d ok s, alls hann er ofvalt í váru máli af d samsettr ok s, en ekki sinn af [þ] ok s.
    \switchcolumn
    \textit{Z}, is a combination of \textit{deleth}, the Hebrew letter that is written as \textcjheb{d}, which stands for \textit{d}, and of another letter called \textit{sade}, which is written as \textcjheb{.s|} and stands for \textit{es} in Latin. Although it is a Hebrew letter, it could be found in the Latin alphabet, since Hebrew letters often come into Latin. I would rather remove it from our language and alphabet since there is sufficient reason that there have been more letters than I wish to have. I would rather write, only few times when necessary, \textit{d} and \textit{s}, since it is always in our language the combination of \textit{d} and \textit{s}, but never \textit{þ} and \textit{s}.
\end{paracol}
\begin{translation*}{}
    z来自于希伯来语中deleth和sade的组合,前者写作\textcjheb{d},后者写作\textcjheb{.s|},在拉丁语中表示es。尽管这本身是一个希伯来字母,在拉丁字母表里也能找到它的身影,因为许多希伯来字母流入了拉丁字母中。我认为这个字母应该去掉才好,因为有充足的理由说明我们的字母表的体量已经超过了我的预期。在少数必要的情况下,我也宁可用d和s来表示这里的音,因为我们的语言中总是出现d和s的结合,而非þ和s。
\end{translation*}

\begin{paracol}{2}
    \& er heldr samstǫfun en staf[r], ok eru stafaðir saman e ok t í látínu, en e ok þ í váru máli, ef hafa skyldi. En ek hefi hann sem sízt í váru máli ok stafrófi, því at aldrigi verðr sú samstǫfun svá í váru máli ein saman, at eigi standi í þeiri inni sǫmu samstǫfun nǫkkurr samhljóðandi fyr e-it.
    \switchcolumn
    \textit{\&} is a syllable rather than a letter, and is the combination of \textit{e} and \textit{t} in Latin, but \textit{e} and \textit{þ} in our language, if we were to use it. But I have it removed from our language and alphabet, since in no case can this combination occurs alone, so that there is no consonant standing before the \textit{e} in the same syllable.
\end{paracol}
\begin{translation*}{}
    \&与其说是一个字母倒不如说是一个符号,在拉丁语里等同于e加上t,但在我们的语言中则相当于e加上þ,如果非要使用它的话。不过我从我们的语言和字母表里去除了这个字母,因为这个组合在任何情况下不会单独出现,使得e之前没有任何辅音。
\end{translation*}

\begin{paracol}{2}
    Títull hefir enn ekki eðli til stafs, en hann er þó til skyndingar rits ok minkunar settr fyr ýmsa stafi aðra, stundum fyr einn, en stundum fyr fleiri. Set ek hann optast fyr m eða stundum fyr n eða fyr er samstǫfun, þann er [svá] er vaxinn: ˜. Kanka ek til þess meiri ráð en lítil: bindi hverr með títli, sem tilfyndiligt ok auðskilligt þykkir. Títull hefir þó nǫkkura jartein til nafns þess, er hann á, þó at hann megi eigi svá merkja af nafni sem aðra stafi. Títan heitir sól, en þaðan af er minkat þat nafn, er títúlus er á látínu. Títull, kveðum vér, þat er sem lítil sól sé, því at svá sem sól lýsir, þar er áðr var myrkt, þá lýsir svá títull bók, ef fyrir er ritinn, eða orð, ef yfir er settr.
    \switchcolumn
    The Tittle letter has no quality of a letter, but in quick and shortened writings it is used for various other letters, sometimes for a single one and sometimes for more. I mostly use it for \textit{m} and sometimes \textit{n} (¯), or \textit{er}, when written as this: ˜. On this letter I cannot give more advice than this: bind whatever letter to the Tittle letter as long as it is suitable and understandable. Yet the name `Tittle' has somewhat representation, although it cannot be drawn from its name as the other letters. Títan is the name of the sun, and títúlus is diminished from it in Latin. Title, as we say, is like a little sun, for the sun illuminates what was dark, so the `tittle' (title) casts light on the book if it is written in the front of the book, or on the word, when it is written above.
\end{paracol}
\begin{translation*}{}
    上标号(写作¯或˜)不是一个字母,而是一个速写标记,用于表示多个其他字母。有时它可以代表一个单独的字母,有时则表示字母组合。我基本用它代表m,或n(¯),有时也可以表示er,这时候要写成这个形式:˜。对于这个字母的用法我只能提这样一点:只要合情合理,上标号可以解读成任何字母。不过,上标号(Tittle)这个名字却有些深意,虽然这个名字并不像其他字母的那样能给出音值的信息。上标号的拉丁文títúlus是Títan的指小词,而Títan是太阳神的名字。标题(title和tittle是同源的)就像一个太阳一样,太阳照亮了黑暗,而标题则照亮了书籍。标题写在书的前面,等写到单词的前面,也即单词上面时,标题就变成了上标号了。
\end{translation*}

\begin{paracol}{2}
    Staf þann, er flestir menn kalla þorn, þann kalla ek af því heldr the, at þá er þat atkvæði hans í hverju máli, sem eptir lifir nafnsins, er ór er tekinn raddarstafr ór nafni hans, sem alla hefi ek samhljóðendr samða í þat mark nú, sem ek reit snemma í þeira umrœðu. Skal þ standa fyrri í stafrófi en títull, þó at ek hafa síðarr umrœðu um hann, því at hann er síðarst í fundinn. En af því fyrr um títul, at hann var áðr í stafrófi, ok ek lét hann þeim fylgja í umrœðu, er honum líkir þarnask sína jartein. Hǫfuðstaf thesins rít ek hvergi nema í vers upphafi, því at hans atkvæði má eigi œxla, þó at hann standi eptir raddarstaf í samstǫfun.
    \switchcolumn
    The letter, which most men call \textit{þorn}, I would prefer to call it \textit{the}, so that its sound in every context will be what is left of the name when the vowel is removed from it, as I have now arranged all the consonants in this way, as I wrote earlier in the treatise. \textit{þ} shall stand before the Tittle letter in the alphabet, even though I have discussed it later, since it is the last to be brought forward. And because Tittle was already in the alphabet, I let it follow those which as well lack a sound value of their own. I do not write the capital letter of the, except at the beginning of a verse, since its sound cannot be lengthened, even when it stands after the vowel in the syllable.
\end{paracol}
\begin{translation*}{}
    被人们称为þorn的字母(þ),我习惯于称之为the,这样在任何情况下,这个字母的发音就是移去其名字中元音后剩下来的部分,这和我之前命名辅音的方式一致,参考论文的前述部分。þ在字母表里处于上标号的前面,虽然我先讨论了上标号,这是因为þ在字母表的最后,上标号则跟着其他那些没有自己的音值的一起讨论。在开头之外的地方,我不会用到þ的大写形式,因为它的音长不能延长,即便它处于元音之后。
\end{translation*}
\begin{grammar*}{}
    \begin{enumerate}[leftmargin=*]
        \item samða

              semja `arrage'过去分词的异体形式。
    \end{enumerate}
\end{grammar*}
\begin{paracol}{2}
    Nú þó at ek hafa mjǫk skyndiliga mælt um hǫfuðstafanna rit, þeira er fyr tvá skal einn vera, þá kalla ek eigi rangt né illa ritit, þó at hinir tveir sé þar heldr ritnir, er hvárgi er hǫfuðstafr, er þó vilja ek heldr einn staf ríta, þar sem bæði stoðar jafnmikit einn ok [tveir], til þess, sem ek sagða, at rit verði minna ok skjótara ok bókfell drjúgara..
    \switchcolumn
    Now even though I have spoken in haste about the writing of capital letters, which will alone represent two letters, and I do not call it wrongly or badly written even if there are two letters written, which are not capital ones. But I will prefer to write one letter, when both of them have an equal sound value, so that, as I said, the writing can be shorter and quicker and parchment can be saved.
\end{paracol}
\begin{translation*}{}
    即便我已经简要说明了大写字母的用法,也即用来代表两个字母,不过写出两个小写字母也没有问题。当两个字母的音值一样时,我更愿意写一个大写字母,这样书写得更快,也更节省纸张。
\end{translation*}

\begin{paracol}{2}
    En þat veit ek eigi, hvat þá skal at hafa, ef svá illa verðr, at enn høggsk nǫkkurr í ok mælir svá: ``þar sem þú rítr hǫfuðstaf einn" kveðr hann, ef hann ræðr þat, ``eða samhljóðendr tvá eins konar samfelda í einni ok inni sǫmu samstǫfu, segir þú, þar vil ek hvárki ríta samhljóðendr tvá né hǫfuðstaf einn, til þess at auka atkvæðit, né enn heldr þann, sem eigi sé hǫfuðstafr, til þess at minka. Heldr rít ek einn ins sama konar jafnan, ok eigi hǫfuðstaf nema í upphafi orðs ok vers, ok kveð [ek] svá mjǫk eða lítt at hverjum, sem ek ræð síðan, eða eigi rœki ek, at ek kveða jafnmjǫk at ǫllum."
    \switchcolumn
    But I would not know what to do, and it would be unfortunate, if someone breaks in and says so: ``Where you write a capital letter", if he decides to say, ``or two consonants of the same quality in one and the same syllable, as you say, I will write neither two consonants nor one capital letter to lengthen the pronunciation, nor will I use a letter that is not capitalized to shorten its sound. I prefer to write them all the same and not use a capital letter except at the beginning of a word of a verse, and I pronounce each word as long or as short as I wish, or I will not care if I pronounced them all the same."
\end{paracol}
\begin{translation*}{}
    不过,若是有人提出像下面这样的观点的话,那么很遗憾,我对此无话可说了。他如是说:如你所说,你用大写字母表示同一个音节中两个同质的音,不过我既不用两个小写字母,也不用一个大写字母来延长发音,也不用小写字母表示短辅音,我更愿意在所有地方都写成一个样子,只在句首的地方用大写。我无所谓字母发音的长短,就算把它们读成一样长的也无所谓。
\end{translation*}
\begin{grammar*}{}
    \begin{enumerate}[leftmargin=*]
        \item høggsk nǫkkurr í

              hǫggvask í有一种比喻的用法,表示“唐突地打断;突然开始某个话题”,和英语`break in'类似。
    \end{enumerate}
\end{grammar*}
\begin{paracol}{2}
    Hvat þá skal at hafa, kvað ek – hvat þá nema sýna honum svá skýr dœmi þeira greina, er hann skilr engvar áðr vera, at þá þykkisk hann of seinn verða til at mæla sjálfr á mót sér ok verða fyrri at bragði [en] þeir, er ella m[y]nd[i] fífla hann ok kalla, sem væri spakara, ef þegði. Nú eru hér þau dœmi, er bráðafangs fundusk þeir, en síðan nǫkkuru ljósligar til máls fœrð ok skilningar: ú bé, U\textsc{b}e; secr, se\textsc{k}r; hǫ́ dó, hǫ\textsc{d}o; áfarar, a\textsc{f}arar; þagat, þa\textsc{g}at; ǫl, ǫ\textsc{l}; frame, fra\textsc{m}e; uına, uı\textsc{n}a; crapa, cra\textsc{p}a; huer, hue\textsc{r}; fús, fú\textsc{s}; sceót, sceó\textsc{t}.
    \switchcolumn
    What then can be done, I said – only to show him such clear examples of the differences which he never understands before, that then he will find it too late to speak against himself and will hurry to take back his words to get ahead before those who may otherwise mock him and say he would be wiser if he kept silent. Now here are the examples which are found in great haste, and are presented for more clarity: ú bé, U\textsc{b}e; secr, se\textsc{k}r; hǫ́ dó, hǫ\textsc{d}o; áfarar, a\textsc{f}arar; þagat, þa\textsc{g}at; ǫl, ǫ\textsc{l}; frame, fra\textsc{m}e; uına, uı\textsc{n}a; crapa, cra\textsc{p}a; huer, hue\textsc{r}; fús, fú\textsc{s}; sceót, sceó\textsc{t}.
\end{paracol}
\begin{translation*}{}
    对此,我只能给出下列明晰的例子展示长短辅音的区别,这些例子自然是他闻所未闻的,看完之后他必然追悔莫及,想要收回自己的话,不然,他就要受智者的嘲笑,说他还是不说话为好。下面就是这些例子:ú bé, U\textsc{b}e; secr, se\textsc{k}r; hǫ́ dó, hǫ\textsc{d}o; áfarar, a\textsc{f}arar; þagat, þa\textsc{g}at; ǫl, ǫ\textsc{l}; frame, fra\textsc{m}e; uına, uı\textsc{n}a; crapa, cra\textsc{p}a; huer, hue\textsc{r}; fús, fú\textsc{s}; sceót, sceó\textsc{t}.
\end{translation*}
\begin{grammar*}{}
    \begin{enumerate}[leftmargin=*]
        \item þykkisk hann of seinn verða ...

              þykkja的反身式的典型用法,提升的主语和感受者一致,均为hann, `he thinks he is too late ...'.

        \item verða fyrri at bragði en þeir ...

              比喻性的说法,字面意思`become former to draw (back his words) then they ...',即`draw back his words before they ...'

        \item síðan nǫkkuru

              nǫkkurr的属格作副词用,表示“大概”。类似的说法还有svá nǫkkuru等。
    \end{enumerate}
\end{grammar*}
\begin{paracol}{2}
    Ú bé þat eru tvau nǫfn tveggja bókstafa, en U\textsc{b}e þat er eins manns eitt nafn. Secr er skógarmaðr, en se\textsc{k}r er ílát. Hǫ́ dó, þá er Hǫlgatroll\footnotemark dó, en heyrði til hǫ\textsc{d}o, þá er þórr bar hverinn. [...] Betra er hverjum fyrr þagat, en annarr hafi þa\textsc{g}at. Eigi eru ǫl ǫ\textsc{l} at einu. Meiri þykkir stýrimannsins frame, en þess er þiljurnar byggvir fra\textsc{m}e. Sá er mestr guðs uına, er mest vill til uı\textsc{n}a. Vaða opt til kirkju crapa, þó at þar fái leið cra\textsc{p}a. Huer kona ok [hue\textsc{r}] karlmaðr skyldu þess fús, sem guð er fú\textsc{s}. Þá munu þau till góðra verka sceót ok hafa guðs hylli sceó\textsc{r}.
    \switchcolumn
    Ú bé are the names of two letters, but uBe is the name of a man. Convicted (secr) is an outlaw, but sack (se\textsc{k}r) is a bag. The high died (Hǫ́ dó) when Holgatroll died, but you could hear the handle (hǫ\textsc{d}o), when Thor carried the kettle. [...] It is better for every man to be silent (þagat) first than he is silenced (þa\textsc{g}at) by other men. Not all (ǫ\textsc{l}) ale (ǫl) are the same. The steersman's fame (frame) is greater than those who sleep on the forward (fra\textsc{m}e) deck. He is the greatest of god's friends (uına) who will work (uı\textsc{n}a) most for him. People often wade to church through thawed snow (crapa), even though that would produce a hard (cra\textsc{p}a) journey. Every (Huer) woman and every (hue\textsc{r}) man should be willing (fús) for what gods desire (fú\textsc{s}). Then they shall be quick (sceót) to do good things and quickly (sceó\textsc{t}) enjoy the grace of god.
\end{paracol}
\footnotetext{全名Þorgerðr Hǫlgatrolla,在挪威受人尊敬的一个女守护神形象。见于挪威王列传(Heimskringla)。}
\begin{translation*}{}
    ú bé是两个字母的名字,而U\textsc{b}e是男子名;
    secr意思是“被判罚流亡的人”,而se\textsc{k}r是“袋子”的意思;
    hǫ́ dó说的是“高人之死”,而hǫ\textsc{d}o是“挂水壶的钩子”;
    【原文漏掉了áfarar和a\textsc{f}arar比较的例子,这两个词分别是介词á和af和fǫr `journey'构成的合成词。】
    þagat是“保持安静”的意思,而þa\textsc{g}at是“使人安静”的意思;
    ǫl指的是“麦酒”,而ǫ\textsc{l}意思是“全部”;
    frame是人的“名声”,而fra\textsc{m}e是“在前面”的意思;
    uına是“朋友”的复数属格,但uı\textsc{n}a是“工作”的不定式;
    crapa是“融雪”的意思,而cra\textsc{m}a意思是“艰辛”; huer和hue\textsc{r}分别是“所有人”的阴性和阳性形式;
    fús和fú\textsc{s}分别是“愿意”的中性复数和阳性单数形式;sceót是“快速”的形容词,但sceó\textsc{t}却是“快速”的副词。\\

    【注:ú bé今形式不变,U\textsc{b}e今作Ubbi;
    secr今作sekr, se\textsc{k}r今作sekkr;
    h\'{ǫ} dó今作há dó, hǫ\textsc{d}o今作hǫddu, 原形hadda;
    áfarar今形式不变,原形áfǫr, a\textsc{f}arar今作affarar, 原形affǫr;
    þagat今形式不变,原形þegja, þa\textsc{g}at今作þaggat, 原形þeggja;
    ǫl今形式不变, ǫ\textsc{l}今作ǫll, 原形allr;
    frame今作frami, fra\textsc{m}e今作frammi;
    uına今作vina, 原形vinr, uı\textsc{n}a今作vinna; crapa今作krapa, 原形krapi, cra\textsc{p}a今作krappa, 原形krappr;
    huer今作hver, hue\textsc{r}今作hverr;
    fús今形式不变,fú\textsc{s}今作fúss;
    sceót今作skjót, sceó\textsc{t}今作skjótt.】
\end{translation*}
\begin{paracol}{2}
    Nú um þann mann, er ríta vill eða nema at váru máli ritit, annat tveggja helgar þýðingar eða lǫg eð[a] áttvísi eða svá hverigi er maðr vill skynsamliga nytsemi á bók nema eð[a] kenna, enda sé hann svá litillátr í fróðleiksástinni, at hann vili nema lítla skynsemi heldr en engva, þá er á meðal verðr innar meiri, þá lesi hann þetta kápítúlum vandliga, ok bœti, sem í mǫrgum stǫðum mun þurfa, ok meti viðleitni mína en várkynni ókœnsku, hafi stafróf þetta, er hér er áðr ritit, unz hann fær þat, er honum líkar betr:
    \switchcolumn
    Now for any man who wishes to write or learn our written language, either for sacred writings or laws or genealogies or whatever useful knowledge that man will rationally learn or know from books, if he is humble in his love of knowledge so that he will gain a little wisdom than none when he is among more matters, then let him read this chapter carefully, and improve it, as it would need in many places, appreciate my efforts but excuse my ignorance, and use the alphabet which has been written here, until he gets one that he likes better:
\end{paracol}
\begin{translation*}{}
    对于任何一个求知若渴的人来说,若是他想要书写或是阅读我们的书面语,不管是为了阅读宗教典籍或律法,还是为了了解家谱或者其他任何书本上可提供的一切有用的知识,只要他对知识抱有虔敬的爱,想要借此获得一些智慧以面对未来更多的事务,那么就应该请他好好读一读这篇文章,修订其中的错误,尊重我的努力而原谅我的疏忽,在更好的字母表提出之前,使用这套字母吧:
\end{translation*}
\begin{center}
    \begin{tabular}{ c c c c c c c c c c c c }
        a ȧ          & ǫ ǫ̇      & e ė & ı ı̇          & o ȯ          & ø ø̇          & u u̇          & y ẏ          & b \textsc{b} & c \textsc{k} & d \textsc{d} & f \textsc{f} \\
        g \textsc{g} & \textcrg & h   & l \textsc{l} & m \textsc{m} & n \textsc{n} & p \textsc{p} & r \textsc{r} & s \textsc{s} & t \textsc{t} & x            & þ
    \end{tabular}
\end{center}