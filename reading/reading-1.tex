\chapter{诗体埃达选读}
埃达(Edda)是两本古冰岛文学的统称,内容多与北欧神话有关。Edda在古诺尔斯语中是“太祖母”的意思,由此可能延伸出“古老的传统”之意。也有人认为edda一词是冰岛学者斯诺里·斯图鲁松(Snorri Sturluson)从拉丁语的edo(诗歌)仿造而来。

埃达包括两本,按照创作的年代分为“老埃达”和“新埃达”。老埃达的创作年代是维京人活动的时期,大约在9到11世纪之间。从形式上来看,老埃达主要是可以吟诵的诗歌(从某些角度看,十分类似于先秦的《诗经》),由游吟诗人口口相传,形容了一种独特的口头文学形式。因此,老埃达又叫“诗体埃达”。老埃达共包含诗歌三十余篇,今人已不知道他们确切的作者,只能从语言风格等角度进行推断。新埃达与老埃达不同,它是由斯诺里·斯图鲁松于十三世纪写定的散文体神话传奇,因此,也称新埃达为“散文埃达”或“斯诺里埃达”。有趣的是,从创作时间上来看,老埃达基本上比新埃达要早一百余年,但是老埃达成书的时间却可能要比新埃达晚50年,这是因为直到13世纪人们才将传诵的老埃达抄写到手稿上。

本章节节选了两部诗体埃达的名篇:《女巫的预言》和《高人的箴言》。其中的诗篇多是四行诗,每行又分两个半句,每句对音节的轻重又有进一步的要求。古代日耳曼地区的诗歌对韵尾要求不高,但常压头韵(Alliteration),即让一行韵文的好几个词头的第一个字母不断重复,以达到音韵上的和谐。当然,本书的目的主要在于通过诗歌理解语法,而非学习诗歌的格律,因为这部分内容完全可以专门用一本书来讲。不过,这也并不妨碍读者在阅读时感受韵文的节奏感。
\section{女巫的预言(Vǫluspá)选读}
《女巫的语言》(Vǫluspá)出自诗体埃达的第一篇,主要记录了北欧的创世神话以及神族最后的灭亡。故事的内容大致是:奥丁召唤了一个已经死去的女巫(Vǫlva)为他讲述世界的历史和自己的命运。奥丁本身是非常渴求知识的神,而死者在北欧神话里往往有不低的地位,因为人们相信他们具有更加丰富的经验和老道的资历。从下文的对话来看,女巫应当是巨人族中的一员。她依次向奥丁回忆了世界的开辟,时间的流动,人类的创造,命运三女神的到来,矮人的姓名以及两大神族间的战争和诸神最终的命运。在神族最后的战争后,绝大多数神都殒命于此,而世界重新进入了下一个轮回。本书选择的文本都是经过修订的标准古诺尔斯语(关于正字法的说明,请参照交叉引用),并非中世纪手稿中直接记录下来的形式,例如原文中的oc应被改写作ok。第一部分选择了其1-8节,讲述的是世界开辟和阿萨神族生活的故事。
\hspace*{\fill}\\ %空一行用于保持美观
\begin{paracol}{2}
    \noindent
    $^1 $ Hljóðs bið ek allar helgar kindir\\
    \textit{Of silence ask I all sacred kinsmen}\\
    Meiri ok minni, mǫgu Heimdallar\footnotemark\\
    \textit{Bigger or smaller, sons of Heimdall}\\
    Viltu, at ek, Valfǫðr\footnotemark! vel framtelja\\
    \textit{Thou wish, that I, Father of the Slain, well forth-tell}\\
    Forn spjǫll\index{spjall!spjǫll} fíra, þau er fremst um man\\
    \textit{Old tales of men, those which first about I remember}\\
    \switchcolumn

    \noindent
    I ask silence from all sacred kinsmen\\
    列位神明啊!请你们安静下来听我讲\\
    Greater or smaller, sons of Heimdall\\
    无论长幼尊卑,海姆达尔的子孙们啊\\
    Father of the Slain! thou wish, that I well recite\\
    英灵之父奥丁啊!你想让我仔细说说\\
    The old tales of men, which I remember long ago\\
    我所记得的远古往事,如今从头说起\\
\end{paracol}
\footnotetext[1]{海姆达尔是神和人类的守护神。一方面,他在天上观察着想要入侵阿斯加德的敌人,最后在诸神黄昏时,海姆达尔将吹响号角召集诸神和英雄。另一方面,海姆达尔创造了人类的三个阶级:奴隶(Þræll),农民/自由民(Karl)和贵族(Jarl),这三个阶级都认为海姆达尔是守护神。这里所说的海姆达尔的子孙,很可能既包括神也包括人。}
\footnotetext[2]{Valfǫðr (valr + fǫðr),直译为“死者之父”,是奥丁的别名之一。奥丁司掌英灵殿,他每天从战死的人中挑选英勇的战士升入英灵殿。}
\begin{grammar*}{}
    \begin{enumerate}[leftmargin=*]
        \item hljóðs bið

              hljóðs是hljóð的属格形式,动词biðja `ask, bid'接续一个属格宾语和一个宾格或与格宾语,属格宾语表示要求的物,宾格或与格宾语表示向谁提出要求,相当于英语中的ask sb. for sth.结构。许多类似含义的动词也有相似的用法。

        \item helgar

              原型heilagr,多音节形容词变形时常省去非重读的i.

        \item meiri ok minni

              这里的比较级并没有确定的比较的对象,只有强调语义的作用。

        \item viltu

              相当于vilt + þu,人称代词和词尾合写到一起,这种现象在早期诗歌中尤其常见。

        \item spjǫll

              中性a-词干名词,单数主格spjall,相当于英语`spell'. 诗歌中常用其复数形式,表示“消息,故事”。
        \item fíra

              主格fírar,只有复数形式。只在诗歌中出现的古老词汇。
        \item um man

              um的一种古老的用法,有时也写作of,几乎总出现在动词或分词前,它没有明显的含义,几乎是多余的。Fritzner认为早期的古诺尔斯语可能需要这个词来提示动词的及物性(um前的部分常是其接续的动词的宾语)。在诗歌中这种用法并不罕见,下面我们会经常遇到。
    \end{enumerate}
\end{grammar*}
\hspace*{\fill}\\ %空一行用于保持美观
\begin{paracol}{2}
    \noindent
    $^2 $ Ek man\index{muna!man} jǫtna ár um borna\\
    \textit{I remember giants ago over born}\\
    Þá er forðum mik fœdda\index{fœða!fœdda} hǫfðu\\
    \textit{That which ago me fed had}\\
    Níu man ek heima, níu íviði\\
    \textit{Nine remember I worlds, nine trees}\\
    Mjǫtvið mœran fyr mold neðan\\
    \textit{Measure-tree famous before soil neath}\\
    \switchcolumn

    \noindent
    I remember giants that were born long ago\\
    我记得远古时代的巨人们\\
    Who had fed me in years of yore\\
    很久以前他们将我抚养大\\
    I remember nine worlds, nine tress\\
    曾有九个世界与九棵大树\\
    The tree that measures beneath the soil\\
    衡量一切的大树植根地下\\

\end{paracol}

\begin{grammar*}{}
    \begin{enumerate}[leftmargin=*]
        \item níu man ek heima

              ek man níu heima,注意这里形容词被提到句首,其修饰的名词并不总是紧邻着它。

        \item fyr

              fyrir的简写。
    \end{enumerate}
\end{grammar*}
\hspace*{\fill}\\ %空一行用于保持美观

\begin{paracol}{2}
    \noindent
    $^3 $ Ár var alda\index{aldr!alda} þar er Ýmir bygði\index{byggja!bygði}\\
    \textit{Early it was of age when Ymir settled}\\
    Vara sandr né sær né svalar unnir\index{uðr(unnr)!unnir}\\
    \textit{Was no sand nor sea nor cold waves}\\
    Jǫrð fannsk æva né upphiminn\\
    \textit{Earth was found never nor up-heaven}\\
    Gap var ginnunga, enn gras hvergi\\
    \textit{Gap was of void, still grass nowhere}\\
    \switchcolumn

    \noindent
    It was early of age, when Ymir settled\\
    依米尔活着时,时间才刚刚开始\\
    There was no sand no sea no cold waves\\
    没有沙,没有海,没有沁沁波涛\\
    The earth was not found nor heaven above\\
    那时天地尚未分,世界一片混沌\\
    There was a gap of void, yet not any grass \\
    只有鸿沟深不见底,何处寻芳草\\

\end{paracol}

\begin{grammar*}{}
    \begin{enumerate}[leftmargin=*]
        \item vara

              系动词var+否定后缀-a,注意这里的单数形式和sandr保持一致。

        \item æva

              古老的副词,又写作æfa. 这个词表示“永远”或“永不”,古诺尔斯语表示`ever'的副词很多也能表示`never',因为原先的否定词尾-gi脱落了(æva < æva-gi)。类似的例子还有:aldri < aldri-gi; ei < ei-gi等。

        \item fannsk

              finna的反身式,finnask有略微的被动含义,也有一定表示存在性的意味,类似于`to be found, to occur'.

    \end{enumerate}
\end{grammar*}
\hspace*{\fill}\\ %空一行用于保持美观
\begin{paracol}{2}
    \noindent
    $^4 $ Áðr Burs synir\index{sonr!synir} bjǫðum\index{bjǫðr!bjǫðum} um ypðu\index{yppa!ypðu}\\
    \textit{First of Bur sons earth over raised}\\
    Þeir er Miðgarð mœran skópu\index{skapa!skópu}\\
    \textit{Those who Midgard great shaped}\\
    Sól skein\index{skína!skein} sunnan á salar steina\\
    \textit{Sun shone from the south along hall's stones}\\
    Þá var grund gróin\index{gróa!gróin} grœnum lauki\\
    \textit{Then was soil grown green leek}\\
    \switchcolumn

    \noindent
    First the sons of Bur raised the earth\\
    起初布尔的儿子们开天辟地\\
    Who shaped the great Midgard\\
    他们筑起伟岸的米德加尔德\\
    The sun shone from the south on the stone wall\\
    太阳从南升起照亮宫殿石墙\\
    The soil was covered with green plants\\
    大地上生机盎然,郁郁葱葱\\

\end{paracol}

\begin{grammar*}{}
    \begin{enumerate}[leftmargin=*]
        \item sól skein sunnan á salar steina; grund gróin grœnum

              整齐的头韵,古代日耳曼地区的诗歌多有这个特点。

        \item lauki

              laukr的单数属格。laukr(韭葱,英文leek)是一种和大葱形似的绿色植物,欧洲十分常见。这里可以就把它理解为植物。用属格表示方式状语。
    \end{enumerate}
\end{grammar*}
\hspace*{\fill}\\ %空一行用于保持美观

\begin{paracol}{2}
    \noindent
    $^5 $ Sól varp\index{verpa!varp} sunnan, sinni mána\\
    \textit{Sun cast from the south, companion of moon}\\
    Hendi\index{hǫnd!hendi} inni hœgri um himinjódyr\\
    \textit{With hand the more skilful around the rim of heaven}\\
    Sól þat ne vissi hvar hon sali átti\index{eiga!átti}\\
    \textit{Sun that not knew where she hall had}\\
    Máni þat ne vissi hvat hann megins átti\\
    \textit{Moon that not knew what he of strength had}\\
    Stjǫrnur þat ne vissu hvar þær staði áttu\\
    \textit{Stars that not knew where they places had}\\

    \switchcolumn

    \noindent
    The sun, companion of moon, rose from the south\\
    太阳从南升起,月是她的伴侣\\
    Casting the right hand around the rim of heaven  \\
    她伸出右手环绕在天空的银边\\
    The sun knew not where she had her hall\\
    太阳不知道她要进入哪个殿堂\\
    The moon knew not what kind of strength he had\\
    月亮不知道他有什么神力要掌\\
    The stars knew not where they had their places\\
    星星不知道她们的归宿在何方\\

\end{paracol}
\begin{grammar*}{}
    注意本段中的人称代词的用法和汉语的习惯不同。例如太阳一般认为是阳刚的形象,汉语中常用“他”来指称,但由于古诺尔斯语中sól是阴性名词,因此用阴性的人称代词hon. 月亮和星星的情况与之类似。

    其他语法:
    \begin{enumerate}
        \item hendi inni hœgri

              字面意思是“更熟练的手”,hœgri是hœgr `easy, convenient'的比较级,但它的含义是“右”。与之相反的是vinstri,它也按比较级变格,但没有原级和最高级。

        \item hvat hann megins átti

              megins是属格,而eiga接宾格,因此megins在这里和hvat相呼应。hvat+属格或与格表示对种类、性质的提问,相当于`what kind of': hvat megins `what kind of strength'.
    \end{enumerate}
\end{grammar*}
\hspace*{\fill}\\ %空一行用于保持美观

\begin{paracol}{2}
    \noindent
    $^6 $ Þá gengu regin ǫll á rǫkstóla\\
    \textit{Then went powers all to judgement-seats}\\
    Ginnheilug goð, ok um þat gættusk\\
    \textit{Mighty-sacred god, and on that took counsel}\\
    Nátt ok niðjum nǫfn um gáfu\\
    \textit{Night and moon name concerning gave}\\
    Morgin hétu ok miðjan dag\\
    \textit{Morning named and mid day}\\
    Undorn ok aptan, árum at telja\\
    \textit{Afternoon and evening, years to enumerate}\\

    \switchcolumn

    \noindent
    Then all the powers went to the seats of judgemnet\\
    接着所有神明聚起来商议\\
    Holy gods, held their counsel\\
    庄严圣明的神,各抒己见\\
    They named night and moon\\
    他们给夜和月亮命名\\
    They named morning and midday\\
    又给清晨和晌午取名\\
    Afternoon and evening, to count the time by years\\
    还有下午和晚上,时间就能计量\\

\end{paracol}
\begin{grammar*}{}
    \begin{enumerate}[leftmargin=*]

        \item rǫkstóla

              由rǫk和stóll `chair, seat'合成。rǫk的意思比较复杂,它最早的意思可能和道路有关,引申为“(故事等)的展开;推理;(事情的)基础”等等。在神话中,这个词和诸神的末日(ragnarǫk)紧密联系在一起,因此也有和“末日,最终审判”有关。

        \item gættusk

              动词gæta的复数反身式,gæta本是“保卫,关照”之意,反身式中词义演化为“讨论”(表示相互的动作)。

        \item árum at telja

              telja表示“计量”,但注意árum实际上不是telja的宾语,因为telja接续宾格名词。这里árum类似于方式状语,相当于`by years'.
    \end{enumerate}
\end{grammar*}
\hspace*{\fill}\\ %空一行用于保持美观

\begin{paracol}{2}
    \noindent
    $^7 $ Hittusk æsir á Iðavelli\\
    \textit{Met Aesir in Ithavoll}\\
    Þeir er hǫrg ok hof há timbruðu\\
    \textit{They who temple and shrine high timbered}\\
    Afla\index{afl!afla} lǫgðu\index{leggja!lǫgðu}, auð smíðuðu\\
    \textit{Hearths placed, treasure smithed}\\
    Tangir\index{tǫng!tangir} skópu\index{skapa!skópu} ok tól gǫrðu\index{gera!gǫrðu}\\
    \textit{Tongs created and tools made}\\

    \switchcolumn

    \noindent
    The Aesir assemble in Ithavoll\\
    阿萨神族在伊达维儿集会\\
    They timbered high temples and shrine\\
    他们筑起高高的屋宇厅殿\\
    And placed hearths, smithed treasures\\
    他们架起壁炉,抡起锤子\\
    And created tongs as well as tools\\
    做出了珠宝、钳子和工具\\
\end{paracol}

\begin{grammar*}{}
    \begin{enumerate}[leftmargin=*]
        \item Iðavelli

              Iðav\k{o}llr的单数属格。“伊达维儿”似乎只是神明集会的地方,只在本处和第60节提到,iða的意思不明,因此在此处保留了音译。也有译作“伊达平原”的,取了v\k{o}llr表示“地面,平地”的意思。

        \item hittusk

              hitta的反身式,hitta相当于英语的`hit',除了基本含义“打击”外,也有“拜访”的意思。在这里表示相互的动作,“聚会、集会”。
    \end{enumerate}
\end{grammar*}
\hspace*{\fill}\\ %空一行用于保持美观

\begin{paracol}{2}
    \noindent
    $^8 $ Tefldu í túni, teitir váru\\
    \textit{Played at table in courtyard, merry were}\\
    Var þeim vettugis\index{vættki!vettugis} vant ór gulli\\
    \textit{Was to them of anything lack of gold}\\
    Unz þrjár kvámu\index{koma!kvámu} þursa meyjar\\
    \textit{Until three came giants maids}\\
    Ámátkar\index{ámáttigr!ámátkar} mjǫk ór Jǫtunheimum\\
    \textit{Powerful very from Jotunheim}\\

    \switchcolumn

    \noindent
    They played board games in yard at peace\\
    他们坐在院子里下着棋,无忧无虑\\
    They lacked nothing made of gold\\
    他们有大把黄金,什么宝贝都不缺\\
    Until three giant maids came\\
    直到有一天三个女巨人出现在面前\\
    Powerful they were, out of Jotunheim\\
    她们来自巨人之国,各个威风凛凛\\

\end{paracol}

\begin{grammar*}{}
    \begin{enumerate}[leftmargin=*]
        \item telfdu

              不定式tefla,由名词tafl的动词,tafl指的是一种棋类游戏,叫作“板棋”,这是世界上最早的棋类游戏之一。

        \item var þeim vettugis vant ór gulli

              vant是vanr `lack'的中性形式,表示“缺乏”的词常用无人称结构。在这里,vanr接与格表示谁缺少(þeim),接属格表示缺少的东西(vettugis ór gulli)。
        \item ámátkar

              ámáttigr的一种缩略形式,一般都用来形容巨人。
    \end{enumerate}
\end{grammar*}