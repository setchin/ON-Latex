\section{高人的箴言(Hávamál)选读}
《高人的箴言》(Hávamál)是诗体埃达的第二章,区别于第一篇《女巫的预言》,这是一首教谕体诗歌。所谓高人自然指的是奥丁。高人的箴言包含5个部分,第一部分是奥丁的格言与教诲,第二部分是奥丁与格萝德(Gunnlǫð)\footnote[1]{格萝德是一个女巨人,她受父亲的命令在一洞穴里负责看守着象征诗歌与智慧(维京人以能作诗者为智者)的密酒。奥丁为获得智慧哄骗格萝德得到了密酒,还和格萝德生下了诗歌之神布拉基(Bragi)。}邂逅的故事,第三部分是奥丁对洛德法夫尼尔(Loddfáfnir)\footnote[2]{洛德法夫尼尔是仅在Hávamál中出现的人物,其情况不详。}的教导,最后两部分是关于卢恩和卢恩法术的。我们从中选择了一些最典型的、能反映维京人生活哲学的诗节:
\begin{paracol}{2}
    \noindent
    $^1$ Gáttir allar\\
    \textit{Gates all}\\
    Áðr gangi fram\\
    \textit{Before would go forth}\\
    Um skoðask skyli\\
    \textit{Around spy should}\\
    Um skygnask skyli\\
    \textit{Around peer should }\\
    Því at óvíst er at vita \\
    \textit{Since unclear is to know}\\
    Hvar óvinir \\
    \textit{Where enemies}\\
    Sitja á fleti fyrir\\
    \textit{Sit on bench before}\\
    \switchcolumn

    \noindent
    (He who stands at) a strange threshold\\
    举步欲进厅堂的人们啊\\
    Before he goes forth\\
    在你们进去前可要当心\\
    Should peer around himself\\
    仔细看看自己周围\\
    And glance this way and that\\
    绝不错过每个角落\\
    Since it is difficult to know beforehand\\
    因为很难提前知道\\
    Where his foes\\
    你的仇敌身藏哪里\\
    May sit awaiting him\\
    磨刀霍霍蠢蠢欲动\\

\end{paracol}

\begin{grammar*}{}
    \begin{enumerate}[leftmargin=*]
        \item gáttir allar

              均是复数宾格形式,用作地点状语。

        \item \'a\dh r gangi fram, um sko\dh ask skyli

              整句话都采用虚拟式,表示假设的情况

        \item um skoðask/skygnask

              这里的um既可以理解为副词性的用法,表示“周围”,那么这两个短语都类似于英语`look around';也可以认为这是一种小品词的用法,无实义。这里的反身态略有一点表示相互动作的含义。

    \end{enumerate}
\end{grammar*}
\medskip %空一行用于保持美观
\begin{paracol}{2}
    \noindent
    $^5$ Vits er þǫrf\\
    \textit{Of wit is need}\\
    \MakeUppercase þeim er víða ratar\\
    \textit{For him who widely travels}\\
    Dælt er heima hvat\\
    \textit{Easy is at home everything}\\
    At augabragði verðr\\
    \textit{To mockery comes}\\
    Sá er ekki kann \\
    \textit{He who nothing knows}\\
    Ok með snotrum sitr\\
    \textit{And with wise men sits}\\

    \switchcolumn

    \noindent
    There is need of wit\\
    若欲出门闯江湖\\
    For him who travels widely\\
    必须机灵有智慧\\
    Everything is easy at home\\
    蠢人最好坐家中\\
    For mockery happens to\\
    一无所知去赴宴\\
    He who knows nothing\\
    还要坐在智者间\\
    And sits among wise men\\
    岂不是要遭白眼\\
\end{paracol}
\begin{grammar*}{}
    \begin{enumerate}[leftmargin=*]
        \item vits er þǫrf

              这里的er表示存在,相当于英语的`there is',其表语是þǫrf vits `need of wits'. þǫrf可接一个与格表示人,一个属格表示物,表示“某人需要某物”时,可用e-m er þǫrf e-s.

        \item hvat

              中性形容词用作不定代词,相当于`everything'.

        \item at augabragði verðr sá er ...

              verða at接与格宾语表示变化的结果,类似于`come to, turn to',这里是一种引申的用法,表示产生某种结果,可以翻译为“招致”。注意主语sá在verðr后面,并由er引导定语从句对其进一步修饰。
    \end{enumerate}
\end{grammar*}
\medskip %空一行用于保持美观
\begin{paracol}{2}
    \noindent
    $^{15}$ Þagalt ok hugalt skyli þjóðans barn\\
    \textit{Silent and thoughtful should ruler's son }\\
    Ok vígdjarft vera\\
    \textit{And battle-brave be}\\
    Glaðr ok reifr skyli gumna hverr\\
    \textit{Glad and happy should of men every}\\
    Unz sinn bíðr bana\\
    \textit{Until himself suffers death}\\
    \switchcolumn

    \noindent
    The ruler's son should be silent and thoughtful\\
    首领之子当慎言善思\\
    And be brave at battle\\
    征战沙场则身先士卒\\
    Happy and merry should every man be\\
    每个人都应幸福开心\\
    Until he suffers his death\\
    颐养天年待死亡降临\\

\end{paracol}
\begin{grammar*}{}
    \begin{enumerate}[leftmargin=*]
        \item glaðr ok reifr skyli gumna hverr

              省略了系动词vera,因为它容易从情态动词skyli推断出来。

        \item gumna hverr

              gumna是gumi的属格,意味着hverr作名词用,gumna hverr即表示`every one of men'. hverr作名词时常和复数属格搭配使用,如还有manna, seggja, lýða, gumna hverr. 要表达相同的含义,也可以用hverr+主格的结构,这里的hverr是作形容词使用的,例如上面的短语又可以表达为hverr maðr.

        \item sinn bíðr bana

              bíða有两个用法:接属格时表示“等待”,接宾格时表示“遭受,忍耐”,常与不好的事物搭配。这里的宾语是bana sinn,为宾格形式。

    \end{enumerate}
\end{grammar*}
\medskip %空一行用于保持美观
\begin{paracol}{2}
    \noindent
    $^{16}$ Ósnjallr maðr hyggsk munu ey lifa\\
    \textit{Unvaliant man believes will always live}\\
    Ef hann við víg varask\\
    \textit{If he against war guard}\\
    En elli gefr hánum engi frið\\
    \textit{But old-age gives him no peace}\\
    \MakeUppercase þótt hánum geirar gefi\\
    \textit{Though him spears may give}\\
    \switchcolumn

    \noindent
    Cowardly man believes he shall live forever\\
    懦夫自以为能长生不老\\
    If he holds back from war\\
    只要他躲开战事求自保\\
    But old age gives him no pieace\\
    可岁月哪里会给他安宁\\
    Though spears may have spared his life\\
    纵使刀枪饶过他的性命\\

\end{paracol}

\begin{grammar*}{}
    \begin{enumerate}[leftmargin=*]

        \item Ósnjallr maðr hyggsk munu ey lifa

              此句为典型的是宾格-不定式结构,由于从句hann munu ey lifa的主语和ósnjallr maðr一致,动词hyggja使用-sk形式。

        \item við víg varask
              反身动词varask由vara `warn'变来,varask við e-u意为`guard against something'或`take care not to do something',这里就是逃避之意。

        \item þótt hánum geirar gefi

              注意þótt引导的让步状语从句中,动词要用虚拟式。

    \end{enumerate}
\end{grammar*}
\medskip %空一行用于保持美观
\begin{paracol}{2}
    \noindent
    $^{23}$ Ósviðr maðr vakir um allar nætr\\
    \textit{Unwise man wakes through all night}\\
    Ok hyggr at hvívetna\\
    \textit{And thinks to everything}\\
    \MakeUppercase þá er móðr\\
    \textit{Then is tired}\\
    Er at morgni kemr\\
    \textit{When to morning comes}\\
    Allt er víl sem var\\
    \textit{All is trouble as was}\\
    \switchcolumn

    \noindent
    Unwise man is awake all night\\
    蠢夫整夜辗转寝不成寐\\
    And worries about everything\\
    鸡毛蒜皮之事劳神操心\\
    Then he is finally tired\\
    待他终于困欲睡\\
    When it comes to morning \\
    已是鸡鸣破晓时\\
    All the trouble remains as it was \\
    麻烦还是老样子\\

\end{paracol}

\begin{grammar*}{}
    \begin{enumerate}[leftmargin=*]

        \item hyggr at hvívetna

              hyggja {\'a}, at, um都表示“注意,专心”。hvívetna, hvatvetna `everything'的与格。vetna语源不详,似乎是vetta的复数属格(而这个词本身只用在ekki vætta `nothing, naught'这个短语中),只见于hvat-vetna, hvar-vetna这样的复合词中。变格时,只有第一部分变化。

        \item er móðr; at morgni kemr; sem var

              这些句子中都省略了主语,不过不难从上下文中推断出来。
    \end{enumerate}
\end{grammar*}
\medskip %空一行用于保持美观
\begin{paracol}{2}
    \noindent
    $^{77}$ Deyr fé, deyja frændr\\
    \textit{Dies cattle, die kinsmen}\\
    Deyr sjalfr it sama\\
    \textit{Dies self you same}\\
    Ek veit einn, at aldrei deyr\\
    \textit{I know one that never dies}\\
    Dómr um dauðan hvern\\
    \textit{Judgement over dead everyone}\\
    \switchcolumn

    \noindent
    Cattle dies, kinsmen die\\
    牛羊会老死,亲朋终西去\\
    You shall die as well\\
    人生自古谁无死\\
    I know one thing, that never dies\\
    我知一事垂千古\\
    The judgment over dead men\\
    死者美名万世传\\

\end{paracol}

\begin{grammar*}{}
    \begin{enumerate}[leftmargin=*]

        \item deyr sjalfr it sama

              it是双数第二人称þit的古体形式。sjalfr `self'本是形容词,这里做名词用。it sama
              `the same, likewise'作副词。it是定冠词的中性形式,这是一个相对少见的用法,不过和英语中的情况是一致。注意,sjalfr常和人称代词一起使用以加强语义,但如果把it认为是代词的话(第二人称复数),就无法和单数形式的deyr相对应了。

        \item dauðan hvern

              dauðan是形容词dauðr的阳性单数宾格,修饰hvern,这就意味着hverr\textbf{单独}作名词用。注意,这区别于hverr+复数属格的用法,这种情况要说hvern dauða(名词dauði `dead man'的复数属格).

    \end{enumerate}
\end{grammar*}
\medskip %空一行用于保持美观
\begin{paracol}{2}
    \noindent
    $^{139}$ Við hleifi mik sældu né við hornigi\\
    \textit{With bread me blessed nor with mead}\\
    Nýsta ek niðr, nam ek upp rúnar\\
    \textit{Peered I down, picked I up runes}\\
    \MakeUppercase æpandi nam, fell ek aftr þaðan\\
    \textit{Yelling picked, fell I back from there}\\
    \switchcolumn

    \noindent
    No bread was given to me, nor mead\\
    无人喂我食物,无人给我美酒\\
    I peered down, with a loud cry\\
    我从树上往下凝望,欣喜若狂\\
    I picked up runes and fell back from there\\
    我领会了卢恩,却从树上跌落\\

\end{paracol}

\begin{grammar*}{}
    奥丁将自己倒挂在世界之树上九天九夜,既不进食也不饮水,以这种自我牺牲的方式获得智慧。
    \begin{enumerate}[leftmargin=*]

        \item æpandi nam

              分词æpandi作为nam的附加语,表示动作的伴随,和英文中分词的用法一致。

    \end{enumerate}
\end{grammar*}
\medskip %空一行用于保持美观
\begin{paracol}{2}
    \noindent
    $^{146}$ Ljóð ek þau kann\\
    \textit{Lay I those know}\\
    Er kannat þjóðans kona\\
    \textit{Which know-not king's wife}\\
    Ok mannskis mǫgr\\
    \textit{And mankind's sons}\\
    Hjalp heitir eitt\\
    \textit{Help called one}\\
    En þat þér hjalpa mun\\
    \textit{And that you help will}\\
    við sǫkum\index{sǫk!sǫkum} ok sorgum\\
    \textit{Against accusations and worries}\\
    Ok sútum gǫrvǫllum\\
    \textit{And grief quite-all}\\
    \switchcolumn

    \noindent
    Those lays I know\\
    那些歌谣我记得清楚\\
    Cannot be known by king's wife\\
    国王的妻妾却不知道\\
    Nor sons of mankind\\
    凡人之子也无人能知\\
    The first is called Help\\
    第一道符咒叫“佑助”\\
    It will help you\\
    这咒能让你精神焕发\\
    Against accusations and worries\\
    助你摆脱非难和焦虑\\
    As well as all the grief\\
    还有一切的忧愁烦恼\\
\end{paracol}

\begin{grammar*}{}
    从本节开始到文末都是奥丁对于他掌握的歌谣(或卢恩法术)的描述,因此把基数词eitt也翻译为`first'. 奥丁一共讲述了十八首歌谣的内容,这一部分又被称作ljóðatal `lay-talk'.
    \begin{enumerate}[leftmargin=*]
        \item ljóð ek þau kann

              þau既可以是人称代词也可以是指示代词,这里显然是后者,它和ljóð构成指示词短语,但注意这里中心语和修饰语分居名词两侧。

        \item gǫrvǫllum

              由副词gǫrva `quite'和allr复合而成。
    \end{enumerate}
\end{grammar*}
\medskip %空一行用于保持美观
\begin{paracol}{2}
    \noindent
    $^{164}$ Nú eru Háva mál kveðin Háva hǫllu í\\
    \textit{Now are High's words sung High's hall in}\\
    Allþǫrf ýta\index{ýtar!ýta} sonum, óþǫrf jǫtna sonum\\
    \textit{All-need for men's sons, no-need for jotun's sons}\\
    Heill sá er kvað, heill sá er kann\\
    \textit{Hail he who spoke, hail he who knows}\\
    Njóti sá er nam, heilir þeir er hlýddu\\
    \textit{Enjoy he, who took, hail those who heard }\\
    \switchcolumn

    \noindent
    Now the High's words are sung in his hall\\
    高人的箴言已响彻他的厅堂\\
    Needful for sons of men; but not for trolls\\
    人类之子闻之有益,巨人之子听之也无用\\
    Hail to him who spoke and who knows\\
    愿念诵的人受人尊敬,学会的人运用自如\\
    He who took and heard them will rejoice\\
    聆听领教箴言者必受益终身\\

\end{paracol}

\begin{grammar*}{}
    \begin{enumerate}[leftmargin=*]
        \item Háva hǫllu í

              注意这里介词出现在了其补足语的后面,这是少见的用法,主要是出于诗律的考量。

        \item njóti sá er nam

              njóti是njóta `enjoy'的现在虚拟式,在这里表示强烈的愿望。

    \end{enumerate}
\end{grammar*}
\medskip %空一行用于保持美观