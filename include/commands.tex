%gb4e 命令 引入后立刻调用
\counterwithin{exx}{subsection}
\setlength{\glossglue}{5pt plus 2pt minus 1pt}

\setmainfont{FreeSerif}
\renewcommand\th{þ}
\noautomath
% 画图颜色
\definecolor {shadecolor}{rgb}{0.92,0.92,0.92}
\definecolor {WASURENAGUSA}{RGB}{125,185,222}
% 自定义标题
\titleformat{\subsubsection}[block]{\normalsize\normalfont\bfseries}{\Alph{subsubsection}.}{1em}{}[]
\titlespacing*{\subsubsection}{\parindent}{1ex}{1em}
% 修改标题页的橙色带
\definecolor{customcolor}{RGB}{240,248,255}
\colorlet{coverlinecolor}{customcolor}
% 修改quote缩进距离
\patchcmd{\quote}{\rightmargin}{\leftmargin 6em \rightmargin}{}{}
% footnote左对齐
\makeatletter
\renewcommand\@makefntext[1]{%
    \setlength\parindent{1em}%
    \noindent
    \mbox{\textsuperscript{\@thefnmark}\,}{#1}}
\makeatother
% 文字上下标
\makeatletter
\newcommand{\dynscriptsize}{\check@mathfonts\fontsize{\sf@size}{\z@}\selectfont}
\makeatother
\newcommand\textunderset[2]{%
    \leavevmode
    \vtop{\offinterlineskip
        \halign{%
            \hfil##\hfil\cr % center
            \strut#2\cr
            \noalign{\kern-.3ex}
            \dynscriptsize\strut#1\cr
        }%
    }%
}
\newcommand\textoverset[2]{%
    \leavevmode
    \vbox{\offinterlineskip
        \halign{%
            \hfil##\hfil\cr % center
            \dynscriptsize\strut#1\cr
            \noalign{\kern-.3ex}
            \strut#2\cr
        }%
    }%
}
\usepackage{subcaption}
\tcbuselibrary{skins, breakable, theorems}
% 三种盒子
\newtcolorbox{infobox}{
    breakable,enhanced, frame hidden, colback=LightSkyBlue!30!white,parbox = false,
    extras={frame hidden, colback=LightSkyBlue!30!white}}

\newcommand{\noleftmargin}{\setlist[enumerate]{leftmargin=*}}
\newcommand{\keepleftmargin}{\setlist[enumerate]{labelindent=\parindent,leftmargin=*}}
\newenvironment{info}{\noleftmargin \tcbset{left*=\parindent} \begin{infobox}}{\end{infobox} \keepleftmargin}
\keepleftmargin
% set default enumerate indent in info environment
\newtcbtheorem{grammar}{语法}
{enhanced,breakable,
    colback = white, colframe = WASURENAGUSA, colbacktitle = WASURENAGUSA,
    attach boxed title to top left = {yshift = -2mm, xshift = 5mm},
    boxed title style = {sharp corners},
    fonttitle = \sffamily, extras ={enhanced, attach boxed title to top left = {yshift = -2mm, xshift = 5mm},
            boxed title style = {sharp corners}},colback = white, colframe = WASURENAGUSA, colbacktitle = WASURENAGUSA}{gmr}

\newtcbtheorem{translation}{中文意译}
{enhanced, breakable,parbox = false, before upper={\par},
    colback = white, colframe = gray, colbacktitle = gray,
    attach boxed title to top left = {yshift = -2mm, xshift = 5mm},
    boxed title style = {sharp corners},
    fonttitle = \sffamily, extras={enhanced, parbox = false,breakable,
            colback = white, colframe = gray, colbacktitle = gray,
            attach boxed title to top left = {yshift = -2mm, xshift = 5mm},
            boxed title style = {sharp corners}}}{trans}

% 字体
\newfontfamily{\medieval}{Junicode}
[Extension = .ttf,
    BoldFont = *-Bold ,
    ItalicFont = *-Italic ,
    BoldItalicFont = *-BoldItalic]
\newfontfamily{\andron}{Andron_Scriptor}
[Extension = .ttf]

% 中世纪手稿用字符
\newcommand\nwithdescender{{\medieval }}
\newcommand\Eunical{{\medieval }}
\newcommand\Eclosedunical{{\medieval }}
\newcommand\vbart{{\normalfont \textvbaraccent{\medieval ꞇ}}}

% 语法命令
\newcommand{\nom}{\textsc{n}}
\newcommand{\gen}{\textsc{g}}
\newcommand{\dat}{\textsc{d}}
\newcommand{\acc}{\textsc{a}}
\newcommand{\mas}{\textsc{m}}
\newcommand{\neu}{\textsc{neu}}
\newcommand{\fem}{\textsc{f}}
\newcommand{\pl}{\textsc{p}}
\newcommand{\sing}{\textsc{s}}
\newcommand{\sub}{\textsc{sub}}
\newcommand{\DEF}{\textsc{def}}

\makeatletter
\@addtoreset{chapter}{part}
\makeatletter
% part后重置章节号

% paracol 合并footnote
\footnotelayout{m}
