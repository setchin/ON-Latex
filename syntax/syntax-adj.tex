\chapter{形容词短语}
\begin{introduction}
    \item 形容词单独使用
    \item 形容词的限定语
    \item 形容词的补语
    \item 比较级的构成
\end{introduction}

\section{形容词单独使用}
绝大多数情况下,形容词短语中只有形容词本身,即形容词是单独使用的。和属格短语中的属格名词一样,形容词通常放在其修饰的名词短语后面,与它保持格、性、数的一致。如果修饰的名词短语中涉及到不同性的名词,则用中性复数式。
\begin{exe}
    \ex \gll
    Úlfr var búsýslumaðr mikill\\
    Ulf-{\nom}-{\mas} was farmer-{\nom}-{\mas} great-{\nom}-{\mas}\\
    \trans `Ulf was a great farmer'

    \ex \gll
    Gunnhildr ok synir þeira váru farin\\
    Gunnhild-{\nom}-{\fem} and sons-{\nom}-{\mas} their were gone-{\nom}-{\neu}-{\pl}\\
    `Gunnhild and their sons had left'
\end{exe}

形容词作表语时,和主语的格、性、数一致,因此总是用主格形式。
\begin{exe}
    \ex \gll
    Hǫðr, bróðir Baldrs, var blindr\\
    Hod-{\nom}-{\mas}, brother-{\nom}-{\mas} of Baldr, was blind-{\nom}-{\mas}\\
    \trans `Hod, brother of Baldr, was blind'
\end{exe}


形容词有时可当作名词使用,类似于英语中的`the+形容词'结构,如`the rich = rich men'. 古诺尔斯语中也有类似的说法,这时由于没有名词与形容词对应,形容词通常用中性或阳性形式。根据语义的不同,它可能是单数式也可能是复数式。

\begin{exe}
    \ex \gll
    gott var í frændsemi þeirra\\
    good-{\nom}-{\mas}-{\sing} was in kinship their\\
    \trans `They were on good terms'

    \ex \gll
    fleiri vega má fœra til rétts en einn veg\\
    more way may bring to right-{\gen}-{\neu}-{\sing} than one way\\
    \trans `There is more than one way to do the right thing'

    \ex \gll
    váru allir með einum hug til þess\\
    were all-{\nom}-{\mas}-{\pl} with one mind to it\\
    \trans `All men were of one mind about it'
\end{exe}

表示整体性、集合性概念的形容词常用\textbf{单数}阳性或中性形式表示复数概念,如allr `all', margr `many', annarr `other', 和英语中`everything'接单数的原则类似。
\begin{exe}
    \ex \gll
    margr skipask vel við góðan búning\\
    many-{\nom}-{\mas}-{\sing} {moved by} well with good clothing\\
    \trans `People are impressed by good clothing (Clothes make the man)'

    \ex \gll
    eigi er enn þeirra allt\\
    not is still theirs all-{\nom}-{\neu}-{\sing}\\
    \trans `Everything is yet not in their hands (They have not yet won the game)'
\end{exe}

在上两句中,margr相当于margir menn, 而allt相当于allir hlutir. 注意到allt和allir都有表示集体名词的用法。

一些固定说法中也常用到形容词的中性形式,如sannr `true, sooth'就有固定搭配til sanns或at sǫnnu `forsooth, indeed'; gnógr `enough'有固定搭配at gnógu `sufficiently'等。

\section{限定语}
限定语一般出现在形容词的左边,对形容词的性质加以规定。形容词最常见的限定语是各类副词,一些常见且通用的副词包括:
\begin{quote}
    harðla/harla `very': berg harðla hátt `a very high mountain'

    enn `even': enn síðarr, betr, verri `even later, better, worse'

    vel `well': vel borinn, gǫrt `well-bone, well done'

    mjǫk `much; almost': ekkja auðig mjǫk `a very rich widow'

    nǫkkut `somewhat': munnr ljótr nǫkkut `a somewhat ugly mouth'
\end{quote}

副词通常出现在形容词前面,但也可以出现在后面。例如mjǫk就常出现形容词后,这种用法是习惯性的。但是,当形容词带有补语时,副词只能出现在形容词前。

名词短语也可以作为形容词的限定语,但只用在某些固定的表达上。属格短语通常用于表达形容词的限度,尤用于那些表示度量的形容词,如gamall `old', hár `high'等:
\begin{quote}
    tolf vetra gamall `twelve years old'

    ker þriggja alna\index{alin!alna} hátt `a cask three ells tall'
\end{quote}

注意,古诺尔斯语中表达年龄的说法常用x vetra gamall, 字面义`x winters old', 而非英文中的`x years old'.

表示部分的属格短语和形容词的最高级连用,表示程度:
\begin{quote}
    kvenna vænst `the most beautiful of women'

    manna vitrastr `the wisest of men'
\end{quote}
这种说法促生了一些固定的副词,如allra hellzt `most of all, especially'.

在形容词的比较级前,可用形容词的与格表达程度的差异:
\begin{quote}
    fám dǫgum síðarr `few days ago'

    miklu heitari `by far hotter = much hotter'
\end{quote}
例如上句中的miklu几乎成为一个固定的副词用法。类似的还有nǫkkurr的与格nǫkkuru, sýnn `clear'的与格sýnu等。这种副词除了与比较级连用外,有时也和最高级搭配,起到强调程度的作用:
\begin{quote}
    miklu meiri maðr en áðr `far more men then before'

    miklu beztr allra þeirra `the best of them all'

    nǫkkuru meir `somewhat more'

    nǫkkuru vildastr `somehow most agreeable'

    sýnu minna `clearly less = far less'

    sýnu fyrstr `far ahead'
\end{quote}

\section{补语}
形容词可支配名词短语、介词短语或从句作为其补语。
\subsection{名词短语}
属格和与格名词短语都可以作为形容词的补语,这视形容词而定。补语通常出现在形容词后,少数情况下出现在形容词前,这种情况下补语常是代词,试比较:
\begin{quote}
    sonr er glíkr feðr `son is like father = like father, like son'

    því glíkt\footnote{这几乎成为了一种固定的副词用法} `like this, in this manner'
\end{quote}
\subsubsection*{接与格的情况}
形容词接续与格的情况非常多,与格名词大致相当于við短语,类似于英语的`about, with', 宽放地表示形容词所描述的对象。这是与格本质的用法,特别地,有以下两类情况十分常见:

表示心情、态度等的形容词接与格表示其作用的对象,如reiðr `angry', glaðr `glad', sárr `painful', hollr `loyal'等:
\begin{quote}
    hollr konungi `loyal to the king'

    kaup er kunnt þér `the trade is known to you = you are familiar with it'
\end{quote}

表示与某物进行某种对比的形容词接与格表示比较的对象,最常用的两个是jafn `equal'和(g)líkr `like, similar':
\begin{quote}
    jafn Guði `equal to God'

    ǫngum manni líkr `like no man'
\end{quote}
\subsubsection*{接属格的情况}
有些形容词也能接属格表示对象,取代了与格的用法。总体而言,这类形容词数量不多,其中叫常用的列举如下。另外,表示“期待,需要”的形容词也接属格,和同义的动词用法一致:
\begin{quote}
    rúmr inngangs en þrǫngr brottfarar `spacious at the entrance but narrow at the exit'

    skyldr þjónastu `obliged to king's service'

    blindr ins sanna um forlǫgin `blind of the truth about one's fate'

    fúss/þarfi/gjarn liðs `wishing for/needing/eager for help'
\end{quote}
有一种情况必须接属格。此时的属格起到表示部分/数量的作用,典型的形容词是fullr `full', vanr `lacking'和verðr `worth':
\begin{quote}
    fullr mjaðar `full of mead'

    vant gulls `lack of gold' (vanr尤用中性形式接属格)

    mikils verðr `worthy of much = much worth'
\end{quote}
某些副词和形容词的中性形式也有类似的用法。参考下面的例子:
\begin{quote}
    snimma dags `early of the day = early in the morning'

    lengi sumars `long of the summer = a long time in summer'

    fátt manna `few of men = few men'

    fleiri vega `more of ways = more ways'
\end{quote}

\subsection{介词短语}
介词短语基本上是上述的与格/属格名词短语的替代形式,在后期的冰岛语中更加常见。大多数情况下,形容词的补语都指涉对象,类似于英语中的`about, with, with respect to'等含义,这种含义可以用við+宾格或til+属格表示。例如:
\begin{quote}
    eiga við e-t búið `be prepared for sth.'

    frekr til fjár `greedy for money'
\end{quote}

\subsection{从句}
当形容词的补语更加复杂时通常要使用从句。
\begin{quote}
    hann var fúss at fara til Finnlands `he was eager to go to Finland'

    þeir vóru skyldir at hefna fǫður síns `they were obliged to avenge their father'
\end{quote}
如果从句中的动词是及物动词,且其宾语恰好是主句的主语,这时从句的宾语被省略(或可以理解为其提升到了主句主语的位置)。在英语中这种语法现象同样存在,例如:
\begin{quote}
    the problem is easy to solve
\end{quote}
注意到从句中solve的宾语是problem, 与主句的主语一致,因此将其省去了, 也即:
\begin{quote}
    the problem is easy to solve \sout{the problem}
\end{quote}
一些读者可能会使用动词的被动形式,即\dag the problem is easy to be solved, 但这种说法无论是在英语还是古诺尔斯语中都是不正确的,因为从句动词的形式要受从句的主语语义的影响。在这一句中,从句隐含的主语是人,因此和solve构成主动关系。该句的完整形式是:
\begin{quote}
    the problem is easy (for people) to solve
\end{quote}
假如使用被动态,从句的主语就必须是别的名词,这不符合语言习惯。古诺尔斯语中也有许多相似的例子:
\begin{quote}
    fǫgr var sú kveðandi at heyra `the song was pleasant to hear'

    áin var allill at sækja `the river was difficult to cross'

    rýtningar eru fyrirboðnir at bera `the daggers are forbidden to carry'
\end{quote}
\subsection{比较结构}
比较结构(Comparative constructions)
