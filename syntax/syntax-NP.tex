\chapter{名词短语}
\begin{introduction}[章节要点]
    \item 属格短语
    \item 特指形式、定冠词的用法
    \item 名词的限定语
    \item 形容词和名词的搭配
    \item 名词的补语和附加语
\end{introduction}

\section{属格短语}
属格名词与其他名词搭配构成各种不同的语义关系。这一点和英文中的所有格结构类似。除了表达基本的所有关系外,属格名词还能表示度量、修饰、逻辑主语、逻辑宾语等等。参考英文中的说法:
\begin{quote}
    属格名词表示度量: two days' food

    属格名词作另一名词的修饰: man of honour

    属格名词作另一名词的逻辑主语: the king's death

    属格名词作另一名词的逻辑宾语: Lincoln's assassination
\end{quote}

古诺尔斯语中属格的用法和英语的所有格有许多相似之处,但它们的基本语序不同。在英语中,所有格名词一般出现在中心语之前,但古诺尔斯语中属格短语的基本语序是:\textbf{中心语+属格}。

除了名词的属格形式外,物主代词也有与之相同的语义功能。第一人称和第二人称均有专门的物主代词形式,需要变格使之与中心语的格、性、数一致。第三人称物主代词由人称代词的属格充当,没有进一步的变格要求。
\begin{exe}
    \ex \gll
    hinum kærsta syni sínum\\
    the dearest son-{\dat} his\\
    \trans `to his dearest son'

    \ex \gll
    forn spjǫll fíra\\
    old tales men-{\gen}\\
    \trans `old tales of men'
\end{exe}

但作为一种屈折语,古诺尔斯语对词序的要求并不是特别严格。虽然习惯上属格名词放在中心语后面,但像英语一样属格名词在前的形式也是存在的,尤其当属格名词和中心语一起构成一个概念整体时。
\begin{exe}
    \ex \gll
    gáfu Svíar honum \k{O}nundar nafn\\
    gave Swedes him-{\dat} Onund-{\gen} name-{\acc}\\
    \trans `The Swedes gave him the name of Onund'
\end{exe}

在本句中,nafn和\k{O}nundar是一个密不可分的概念,\k{O}nundar和nafn实际上构成了同位语,它们是互相指涉的。类似的用法还有askr Yggdrasils `the ash-tree of Yggdrasil'.

在属格短语中,中心语和属格可以构成多种语义关系,归纳起来,主要由以下四大类:
\begin{enumerate}[itemindent=1em]
    \setlength{\parindent}{2em}
    \item 表示所属

          属格的基本用法。既可以表示具体事物的归属,也适用于抽象名词:
          \begin{exe}
              \ex \gll
              þá þótti Ásbirni vandask um tilfongin búsins\\
              then thought Asbirn difficult about supplies-the-{\acc} household-the-{\gen}\\
              \trans `Then Asbirn thought that there's a problem with the supplies of the household'

              \ex \gll
              Hrani fekk trúnað margra ríkismanna\\
              Hrani got confidence-{\acc} many-{\gen} powerful-men-{\gen}\\
              \trans `Hrani obtained the confidence of many powerful men'
          \end{exe}

    \item 表示描述

          属格名词可以对中心语进行描述、修饰和限定。例如我们之前提到的作同位语的属格的功能就属于这一类。其他例子还包括:
          \begin{quote}
              \begin{quote}
                  mikils háttar maðr `a man of great importance'

                  afreks verk `a deed of heroism'

                  þess konar sending `a message of this sort'

              \end{quote}
          \end{quote}

          它们的用法都和英文中的一致。
    \item 表示部分/数量

          这种属格经常和数词连用,被计数的名词要用属格形式。
          \begin{exe}
              \ex \gll
              hofðu þeir halft annat hundrað skipa\\
              had they half second hundred-{\acc} ships-{\gen}\\
              \trans `They had 180 ships'

              \ex \gll
              sagðir þú þrettán tigu aura silfrs?\\
              said you thirteen tens ounces-{\acc} silver-{\gen}\\
              \trans `Did you say 130 ounces of silver?'
          \end{exe}

          表示整体中的部分时,整体用属格标记:
          \begin{exe}
              \ex \gll
              inn nezti hlutr trésins var rauðr sem blóð\\
              the lowest part tree-the-{\gen} was red as blood\\
              \trans `The lowest part of tree was as red as blood'
          \end{exe}

    \item 表示动作的参与者

          由动词派生出的名词以及表示动作的名词可以和属格名词连用,这个属格短语表达与对应的谓语动词相同的含义。属格名词充当谓语的一个论元,可以是施事(动作的发出者),也可以是受事(动作的承受者),这一般由动词的及物性决定。

          和\textbf{不及物}动词派生出/表意相同的名词连用时,属格名词充当施事的角色,又称主语性属格(Subjective genitive),参见例句:
          \begin{exe}
              \ex
              \begin{xlist}
                  \ex[]{
                  \gll ferð Óláfs af Vinlandi\\
                  journey Olaf-{\gen} from Vinland\\
                  \trans `Olaf's journey from Vinland'
                  }

                  \ex[=]{
                  \gll Óláfr \textit{fór} af Vinlandi\\
                  Olaf-{\nom} went from Vinland\\
                  \trans `Olaf went from Vinland'
                  }
              \end{xlist}
          \end{exe}

          主语性属格也能与\textbf{及物}动词有关的名词连用。许多由-ingi/-ingr词尾派生出的名词总表示及物动词的受事,自然地,可以用属格表示施事:
          \begin{exe}
              \ex \gll
              þá em ek hvers manns níðingr\\
              then am I each-{\gen} man-{\gen} villain\\
              \trans `I will be despised by everyone'

              \ex \gll
              hann mundi vilja vera ræningi þinn\\
              he would want be robbed-one your\\
              \trans `He would want to be your hostage'
          \end{exe}

          níðingr源自PGmc.词根*nīþ-,与“仇恨”等情绪相关。níðingr是对人的最恶劣评价,不守信用者、叛徒、逃兵等就可被称为níðingr,是受人鄙视的对象。ræningi由及物动词ræna `rob'派生而来。

          由-maðr构成的合成词也有类似的表达,例如:
          \begin{exe}
              \ex \gll
              engis manns nauðungarmaðr vil ek vera\\
              no man-{\gen} constraint-man-{\nom} want I be\\
              \trans `I do not want to yield to anyone’
          \end{exe}

          和\textbf{及物}动词派生出/表意相同的名词连用时,属格名词通常充当受事的角色,又称宾语性属格(Objective genitive),参见例如:
          \begin{exe}
              \ex
              \begin{xlist}
                  \ex[]{ \gll
                  Ása var áðr farin á fund f\k{o}ður síns\\
                  Asa was before gone on meeting-{\acc} father-{\gen} her\\
                  \trans `Asa had gone to meet her father' }

                  \ex[=]{
                  \gll Ása var áðr farin at finna f\k{o}ður síns\\
                  Asa was before gone to meet father-{\acc} her\\
                  \trans `Asa had gone to meet her father'
                  }
              \end{xlist}
          \end{exe}

          注意,宾语性属格不仅限于动词的宾格宾语,对于那些支配与格或属格的动词(参见交叉引用),将动宾结构变成属格短语时,原动词的与格或属格宾语同样变成属格:
          \begin{quote}
              \begin{quote}
                  支配与格的动词:valda landinu `rule the land' = vald landsins `dominion of the land'

                  支配属格的动词:hefna Bolla `avenge Bolli' = hefnd Bolla `avengence on Bolli'
              \end{quote}


          \end{quote}

          一些及物动词支配双宾语,一般在属格结构中,双宾语中的直接宾语被转变为属格,而间接宾语通常被省略,必要时,可以用介词短语或是借助其他动词表达出来。例如heita表示“许诺”时,可接两个与格宾语表示承诺的事情(直接宾语)和向谁承诺(简介宾语)在变成属格结构时,只保留承诺的事情:
          \begin{exe}
              \ex
              \begin{xlist}
                  \ex[]{\gll
                  ek hét konungi ýmissum hlutum\\
                  I promised king-{\dat} various-{\dat} things-{\dat}\\
                  \trans `I promised the king many things'
                  }
                  \ex[√]{\gll
                  ek strengda heit ýmissa hluta\\
                  I fastened vow-{\acc} various-{\gen} things-{\gen}\\
                  \trans `I made a vow about various things'
                  }
                  \ex[†]{\gll
                  ek strengda heit konungs\\
                  I fastened vow-{\acc} king-{\gen}\\
                  \trans `I made a vow to the king'
                  }
              \end{xlist}
          \end{exe}

          strengda是strengja的过去式,stengja本意是“系紧”,它和heit `vow'连用时表示“庄重地发誓”。第三句的语法是错误的,如果要表达第三句的意思,必须用konungr的与格形式表示strengja的接受者。

          这个例子也表明,当一个谓语动词可以支配多个论元时,其相应的属格短语结构中只允许一个属格。对于及物动词而言,它至少能支配两个论元,用属格结构既表达主语和宾语是错误的:
          \begin{quote}
              \begin{quote}
                  †Ingólfs byggð Íslands `Ingolf's settlement of Iceland'
              \end{quote}
          \end{quote}

          要表达这个含义,必须用介词短语:
          \begin{quote}
              \begin{quote}
                  Ingólfs byggð á Íslandi `Ingolf's settlement on Iceland'
              \end{quote}

          \end{quote}

\end{enumerate}

\section{限定词}

限定词处于定冠词和名词之前,主要包括冠词、形容词、分词、数量词等。
\subsection{冠词}

冠词的使用和名词的特指形式密切相关。
\subsection{形容词}

形容词是最典型的限定词。许多其他词类在语义和语法上都和形容词相似,例如分词、数词、不定代词,因此,我们把它们归为一类讨论。注意,形容词的原级和最高级和过去分词有强变化和弱变化两种形式。形容词的比较级和现在分词只有弱变化形式。数词、不定代词作形容词用时,只有强变化形式。

形容词搭配的基本规则是弱变化形式与特指的名词短语连用;强变化形式与非特指的名词短语连用。特指的名词短语的标志是定冠词、指示代词、或特指后缀:
\begin{exe}
    \ex \gll
    Frændr \textit{ins} vegna áttu kr\k{o}fu {á hendr} frændum veganda fyrir vígit\\
    Kinsmen the-{\gen} killed-{\DEF}-{\gen} had claim against kinsmen killing for homicide-{the}\\
    \trans `The kinsmen of the killed had a claim on the kinsmen of the killer for the homicide'
\end{exe}

即便没有定冠词,一个名词短语依然可能是特指的,这时的特指性是由语义来决定的。属格短语是一个典型的例子,无论加不加定冠词,被属格修饰的名词在语义上一般都是特指的、限定的:
\begin{exe}
    \ex \gll  Hann bauð ambótt sinni \textit{þrœnzku} at ...\\
    He ordered bondwoman-{\dat} his Throndish-{\DEF}-{\dat} that ...\\
    \trans `He ordered his bondwoman from Trøndelag that ...'
\end{exe}

þrœnzku与ambótt sinni的格、性、数一致,由于ambótt被sinni限定,þrœnzkr要采用弱变格形式。

许多专有名词、人名也默认是限定的:
\begin{exe}
    \ex \gll  Erlingr jarl lét drepa Eindriða unga\\
    Erlingr earl let kill Eindridi-{\acc} young-{\DEF}-{\acc}\\
    \trans `Erlingr earl had Eindridi the young killed'
\end{exe}

在类似于Eindridi the young这样的“人名+绰号/别名”的结构中,这里的人都是特指的,所以形容词要用弱变格(ungi) 。通常情况下,人名后要加定冠词inn,如Óláfr inn helgi `Olaf the saint', Baldr inn góða `Baldr the good'. 在本句中,虽然人名和形容词间没有定冠词,但这并不影响形容词采用弱变化。

\section{补语和附加语}
\subsection{介词短语}
一般而言,介词短语可以作为名词短语的附加语表示位置关系,这种关系一般是静态的,或是表示来源。反之,如果要表示动态的位置变化,则要用动词短语。
\begin{quote}
    spakastan húsbúanda í bœ `the wisest landlord in town'

    fjǫldi manna ór Þrándheimi `many a man from Trondheim'
\end{quote}
一些由动词派生出的名词以及表示动作的名词有时可以接一个介词短语来提示这个动作的宾语,这个介词短语起到的作用和属格短语类似。介词的选择通常和动作本身以及相应的动词有关:
\begin{quote}
    stýrimaðr fyrir skipinu `helmsman of the ship'

    fundr við e-n `meeting with someone'
\end{quote}

\subsection{从句}
非限定性从句和限定性从句均可作名词短语的补足语和附加语,这些从句都有at引导。英语中也有类似的说法:
\begin{quote}
    hendr at taka til hans `the hands to lay hold of him'

    saga at Baldr inn góða dreymði drauma hættliga `the story that Baldr the good dreamt terrible dreams'

\end{quote}

% 当名词短语在句子中充当相对独立的成分时,如主语、表语、状语,通常可以直接在它后面添加从句。但如果整个名词短语充当某个动词的宾语,更常见的说法是在名词后加一个形式上的介词短语til þess, 再添加at引导的从句:
% \begin{quote}
%     hafa vilja ok aðferð til þess at hefna `have the will and aggression to revenge'

%     taka svardaga til þess, at eira skyldu Baldri eldr `take oaths that fire will not hurt Baldr'
% \end{quote}
