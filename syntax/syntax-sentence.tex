\chapter{限定句}
\begin{introduction}[章节要点]
    \item 限定句的结构
    \item 限定动词的时态和语气
    \item 主谓一致规则
    \item 被动句的结构
    \item 无人称结构
    \item 词序
\end{introduction}

\section{限定句的定义和结构}
限定句(Definite sentence),或简称句子(Sentence),是一串能表述完整含义的词。一般来说,句子包括主语和谓语两个部分,以表达“某人做某事”的基本结构。在古诺尔斯语中,这个谓语部分必须由一个限定动词(Definite verb)充当,因此我们又把这种句子称为限定句。

限定动词是屈折语中动词的一种形式,它与句子中主语的人称、数等的一致,并体现时态、体貌、语气和语态的特征。由于这类动词的形态受主语的影响,我们说它被主语“限定”,故称为“限定动词”。限定动词和非限定动词相对,后者包括不定式、分词、动名词等。非限定动词不需要与主语的人称、数等语法范畴一致。在一个限定句中,限定动词有且只有一个,但非限定动词可以有多个,并且出现在各种句子成分之中。参考英语中的两个句子:
\begin{quote}
    1. I wanted to have a cup of coffee.

    2. To see is to believe.
\end{quote}

在句1中,限定动词是`wanted',它不仅和主语`I'在人称和数上一致,还表达出过去时。非限定动词是不定式`to have',它和后面的名词一起构成了非限定从句,作为wanted的宾语。在句2中,限定动词是`is',作为系动词连接主语和表语。在此句中,主语和表语都由不定式(to see \& to believe)承担。

% 限定句由若干短语构成,短语又由单词构成。我们之前介绍的名词、动词、形容词等都可以构成短语,这些短语在限定句中充当不同的句子成分,如主语、谓语、宾语、状语等。各个短语的构成将在(交叉引用)中详述。

古诺尔斯语有一类特殊的句子不需要主语,这称为无人称结构(Impersonal structure)。无人称结构常常由一些没有特定的施事者的动词引起,例如与时间流逝、季节变换有关的动词,但有的时候,无人称结构和被动语态紧密联系在一起,详见(\ref{sec:impersonal})。

\section{限定动词}
限定动词包含时态和语气的形态标记,并根据主语的人称和数发生屈折。限定动词是限定句的必有成分,它的语法范畴对整句话的表意都产生影响。本节主要介绍动词的时态、情态和主谓一致性。
\subsection{时态}
时态是用于描述时间参考的语法范畴,在古诺尔斯语中时态由动词屈折体现。我们在(3.1 交叉引用)中已经指出,动词严格来说只有两个时态:现在时和过去时。现在时是使用最广泛的时态,除了严格需要表达过去的情况,动词都可以使用现在时。因此,现在时是默认的时态(Unmarked tense)。过去时则只能在描述过去事件时使用。现在时有两种相对特殊的用法:
\begin{enumerate}
    \setlength{\parindent}{2em}
    \item 将来的表达

          将来时没有对应的动词屈折,语义上的将来时可以由形态上的现在时表达,例如:
          \begin{exe}
              \ex
              \gll þar	\textit{liggr}	hann	í	bǫndum	til	ragnarøkkrs\\
              there	lies	he	in	bond	until	Ragnarök\\
              \trans `There he will lie in bonds until the end of the world'
          \end{exe}


          这句话描述的是洛基被诸神绑缚在石头上,直到诸神黄昏的那天才能脱逃的故事。主句中的限定动词liggr是liggja的第三人称单数现在时,但实际上表达的是将来概念,其对应的英语句中一般就使用将来时will lie.

          但古诺尔斯语也并非完全没有专门用于表达将来的说法,利用助动词munu或skulu加上动词不定式也可以表达将来,对应英语的will do或shall do, 详见交叉引用。故上句中的liggr也可改成mun liggja或更好的skal liggja.

    \item 历史现在时

          历史现在时(Historical present)指的是在叙述过去的事件时,使用现在时时态的情况。在希腊语和拉丁语中也有这样的情况,修辞学家认为这种写作手法能使表述更加生动。古诺尔斯语的散文中也常有历史现在时,特别是在叙事时经常出现现在时和过去时的交替,相邻的两个小句中的时态就可以不一致,例如:

          \begin{exe}
              \ex
              \gll  Óláfr	\textit{svarar}	fá	ok	\textit{hló}\\
              Olaf	answers	few	and	laughed\\
              \trans `Olaf said little in reply and laughed’
          \end{exe}

          在连词ok引导的两个简单句中,前面一个使用的是现在时,后面使用的是过去时,虽然表达的都是过去发生的事件。

\end{enumerate}
\subsection{语气}
古诺尔斯语的三个语气是:直陈、虚拟、祈使。

直陈是最基本的语气,用于表达说话者所认定的客观事实。直陈语气是陈述句的基本语气,对此不多加赘述。

虚拟语气表达对非现实的场景的描述,如愿望、情感、可能性等。许多动词接续从句作为补足语时常用虚拟语气,这是动词的语气引起的虚拟语气,将在交叉引用中详述。在主句中,使用虚拟语气的动词常常表达愿望,例如:
\begin{exe}
    \ex
    \gll \textit{vænti} ek góðs af yðr\\
    hope-{\sub} I good from you
    \trans `I would like good terms from you'

    \ex
    \gll hverr	er	eyru	hefir,	hann	\textit{heyri}\\
    whoever	who	ears	has	he	hear-\textsc{sub}\\
    \trans `He who has ears ought to hear!’

\end{exe}

祈使语气不在从句中使用,且只有第二人称和复数第一人称形式。最常见的祈使语气用在动词的第二人称上,表示对听者的命令。有时人称词尾附在动词后面:
\begin{exe}
    \ex
    \gll \textit{lát-tu}	mik	sjá	fyrir	honum\\
    let-\textsc{imp}-you	me	see	for	him\\
    \trans `Let me take care of him’
\end{exe}

láttu是动词láta的第二人称单数祈使式和人称词尾þú的合写。
\subsection{主谓一致性}
\label{sec:SVagreement}
限定动词的人称和数与主语的人称和数保持一致,这是主谓一致性的基本原则。有一些特殊情况值得专门说明:
\begin{enumerate}
    \setlength{\parindent}{2em}
    \item 主语是双数的情况

          古诺尔斯语的人称代词中仍保留了双数,没有专门的动词双数形式与之对应,因此总是使用动词复数形式进行对应。
          \begin{exe}
              \ex
              \gll þit	\textit{munuð}	færa	mér	hǫfuð	hans\\
              you-\textsc{du}	shall	bring	me	head	his\\
              \trans `You two shall bring his head to me’
          \end{exe}

    \item 缺少主格主语的情况

          古诺尔斯语中存在一种无人称结构,这类句子中没有显式的主语(以主格标记出来的)。语义上的主语可能以从句承担,但从句作为一个整体而言是没有格标记的;有一些动词也总是不能接施事性论元,详见\ref{sec:impersonal}。此时动词一般采用第三人称单数。
          \begin{enumerate}
              \setlength{\parindent}{2em}
              \item 以从句为主语的情况
                    \begin{exe}
                        \ex
                        \gll óvíst	\textit{er}	at	vita\\
                        unclear	is	to	know\\
                        \trans `It is difficult to understand’
                    \end{exe}
                    这类句子几乎总是翻译为英语中的it is … 结构,it是形式主语,该句的真实主语由后置的从句承担。
              \item 动词的无人称用法
                    \begin{exe}
                        \ex
                        \gll en	er	\textit{haustaði}\\
                        but	when	came-autumn\\
                        \trans `But when autumn came’
                    \end{exe}

                    本句中的动词haustaði的原形是hausta,属二类弱动词。hausta由名词haust `autumn’衍生,作为一个表示季节转变的动词,它没有明显的动作执行者,因此总是用于无人称结构中。
          \end{enumerate}

    \item 并列主语的情况

          并列主语在在多数情况下表达的是复数的概念,虽然有时也可以被理解为一个整体。在古诺尔斯语中,这种情况下并不一定要求用动词的复数式与之对应。动词常与其最邻近的主语保持一致,类似于英语的就近原则,但这也不是严格的规则。总体来说,并列主语对动词数的要求不高。参考下面的例句:
          \begin{enumerate}
              \setlength{\parindent}{2em}
              \item 并列主语用单数
                    \begin{exe}
                        \ex
                        \gll hann	segir	at	korn	ok	malt	\textit{var}\\
                        he	says	that	corn	and	malt	was\\
                        \trans `He says it was corn and malt’
                    \end{exe}

              \item 并列主语用复数
                    \begin{exe}
                        \ex
                        \gll \textit{sat}	konungr	ok	dróttning	í	hásæti ok	\textit{drukku}	bæði	samt\\
                        sat-{\footnotesize 3}\textsc{p}	king	and	queen	in	high-seat and	drank-{\footnotesize 3}\textsc{p}	both	together\\
                        \trans `The king and queen sat in high seat and drank together’
                    \end{exe}
                    本句中的两个动词的数不一样,第一个动词sat是单数式,但drukku是复数式,这进一步说明数的一致性在并列主语的情况下不是特别重要。注意bæði samt一般理解为副词,和allt samt等对应,这样bæði就不作第二个分句的代词主语使用。
          \end{enumerate}


\end{enumerate}
\section{主语及其变形}
主语是在句子中带有主格标记的成分。这种定义方式是纯粹形式化的,有时语义上的主语并不带有主格标记。主语在句子结构中的位置同样很复杂,我们首先引入几个概念。

一个典型的句子由一个名词短语(Noun Phrase,NP)和一个动词短语(Verb Phrase,VP)构成。名词短语作为句子的主语,动词短语作为谓语,对主语进行描述。名词短语和动词短语的内部可以包含各种各样的单词(只要它们能合法地聚合起来),但是名词短语和动词短语本身则有清楚的界限,它们的词性是不同的。包含在动词短语内部的谓语论元称为内部论元(Internal argument),反之则称为外部论元(External argument)。例如在下面的句子中:
\begin{quote}
    John reads the book.
\end{quote}

John是句子中的名词短语。reads the book是句子中的动词短语,它进一步由限定动词reads和名词短语the book构成。在本句中,John是外部论元,the book是内部论元。

一般而言,外部论元是谓语动词的限定词(Specifier),支配其变位。内部论元是动词的补语。
\subsection{主动句—外部论元作主语}
外部论元作主语是最常见也是最基本的情况。绝大多数主动句中的动词都指派外部论元作为主语。根据动词语义的不同,外部论元可能表示动作的发出者,也可能指示性质、状态或发生非自主变化的主体。在下面的两个句子中,前者的动词是表示动作的,后者是表示所有关系或性质的,它们主语的语义角色有所不同。不过,读者应该不会对主语的判断产生困难。
\begin{exe}
    \ex
    \gll    þá	reið	\textit{Óðinn}	fyrir	austan	dyrr\\
    then	rode	Odin-\textsc{n}	forth	eastern	door\\
    \trans `Then Odin rode to the eastern door’
    \ex
    \gll \textit{Óðinn}	átti	tvá	brœðr\\
    Odin-{\footnotesize 3}\textsc{n}	had	two	brothers\\
    \trans `Odin had two brothers’
\end{exe}

\subsection{被动句—内部论元作主语}
\label{sec:passive}
内部论元作主语涉及到将动词的补语提升为句子的主语。一个动词可以接续三个间接格中任意一个作为自己的补语(详见交叉引用),这些都属于内部论元。但是,其中只有宾格宾语能提升为主语。涉及到这一转化的共有三种结构:
\begin{enumerate}
    \setlength{\parindent}{2em}
    \item 被动句

          被动句的主语对应相应的主动句的宾格宾语。被动句的目的是抑制相应的主动句中的主语,把句子的主题转移到谓语及其结果上。因此,被动句的主语是谓语的内部论元。

          在古诺尔斯语中,被动句由助动词vera/verða+动词的过去分词构成,与英语的be done结构相似。过去分词在形态学上和形容词属于同一类,因此理论上需要和主语的格、性、数保持一致。但后来,也可以一律使用单数中性主格。
          \begin{exe}
              \ex
              \gll Óláfr	var	skírðr	þar\\
              Olaf	was	baptized-\textsc{\MakeLowercase{M-SING-N }}	there\\
              \trans `Olaf was baptized there’
          \end{exe}

          一些动词能接一个宾格宾语,一个与格宾语。最典型的动词是gefa `give’, 其与格宾语表示人,宾格宾语表示物,相当于英语中give sb. sth.的结构。在相应的被动句中,宾格宾语变成主格,而与格宾语保持不变:
          \begin{exe}
              \ex
              \gll var	þeim	gefit	ǫl	at	drekka\\
              was	them-\textsc{d}	given-\textsc{\MakeLowercase{NEU-N}}	ale-\textsc{n}	to	drink\\
              \trans `Ale was given to them to drink’
          \end{exe}

          在本句中,ǫl从宾语变成了主语,而表示承受者的þeim保持不变,其对应的主动句可以为:
          \begin{exe}
              \ex
              \gll Konungr	gaf	þeim	ǫl	at	drekka\\
              King	gave	them-\textsc{d}	ale-\textsc{a}	to	drink\\
              \trans `The king gave them ale to drink’
          \end{exe}

    \item vera+现在分词

          vera+现在分词有一种特殊的被动含义,意为该动作是合适的、可能的、必要的意味,相当于英文中should be done结构。因此,这种结构中的主语也是由相应动词的宾语变化而来。现在分词在变形上也和形容词同类,与主语的性、数、格保持一致。但考虑到现在分词只有弱变格,其要么以-i结尾,要么以-a结尾。
          \begin{exe}
              \ex
              \gll at	kveldi	er	dagr	lofandi\\
              at	knight	is	day-\textsc{n}	prasing-\textsc{\MakeLowercase{M-SING-N}}\\
              \trans `At evening the day should be praised’
          \end{exe}

    \item vera + at-不定式

          vera+at-不定式有类似于vera+现在分词的作用。
          \begin{exe}
              \ex
              \gll eru	slíkar	mínar	at	segja	frá	honum\\
              are	such-\textsc{n}	mine-\textsc{n}	to	say	about	him\\
              \trans`This is all I have to say about him’
          \end{exe}

          其中slíkar mínar本是segja的宾语,现在移动到了主格位置。
\end{enumerate}

\subsection{主语提升}
主语提升(Subject raising)指的是一个低一级分句中的主语提升到高一级的分句中作主语,原分句中的谓语动词改为不定式作主动词的补语。参见英文中含义相同的两个句子:
\begin{quote}
    1. It seems that he has left.\\
    2. He seems to have left.
\end{quote}

在句1中,没有出现主语提升的现象,he是从句he has left的主语,这个从句比主句It seems that … 要低一级。句2是句1发生主语抬升后的现象,低一级的从句的主语he“提升”到了主句当中,成为了整个句子的主语,同时原先的谓语has left变成了不定式to have left作为主动词seems的补足语。

并不是所有谓语动词都支持这样的提升,在古诺尔斯语中,最常见的提升动词是þykkja `seem’,这是一个不规则动词,参考交叉引用. 这个动词最常见的用法如下所示:
\begin{exe}
    \ex
    \gll þá	þótti	mér	þeir	sœkja	at	ǫllum	megin\\
    then	seemed	me-\textsc{d}	they-\textsc{n}	attack	at	all	sides\\
    \trans (a) `Then it seemed to me that they attacked on all sides’\\
    (b) `Then I thought they attacked on all sides’
\end{exe}

这句话中的主语是þeir,通常þykkja还接续一个体验者,即产生感受的主体,用与格标出,即本句中的mér. 主语的动作用不定式(本句中的sœkja)标记,这个不定式不需要at引导 ,类似于宾格-不定式结构(参见交叉引用)。þykkja可以和句中的主语保持主谓一致,也可以用第三人称单数式,一般是þykkir或þótti, 有时当体验者为第一或第二人称时,还可以用þykki. 故þykkja引起的句式一般为:
\begin{info}
    þykkir/þótti + 体验者-\textsc{d} + [主语-\textsc{n}\footnotemark + 不定式 + 其他成分]
\end{info}
\footnotetext{少数时候,主语不以主格形式呈现。例如当非限定句是一个被动语态,且动词不接宾格宾语时,主语就可能以间接格呈现。}
其中以[]标出的部分为一个非限定性的从句,它和正常的限定句的区别仅在于限定动词变成了不定式。

如果这个从句的谓语部分是由vera引导的主语补足语,如vera+名词短语或形容词短语。则主语补足语的格、性、数要与与整句的主语一致,例如:
\begin{exe}
    \ex
    \gll torsóttr	þótta	ek	yðr	næstum	vera\\
    difficult-\textsc{n}	seemed-{\footnotesize 1}\textsc{s}	I-\textsc{n}	you-\textsc{d}	last	be\\
    \trans `You thought I was difficult last time’

\end{exe}

注意本句中þótta和主语ek一致,而非采用第三人称形式。这句话的语序和我们前面给出的标准结构略有区别,这是因为作者有意强调形容词torsóttr, 如果转变成我们熟悉的语序,本句为þótta yðr ek vera torsóttr næstum. 但无论如何,torsóttr都以主格形式与ek相对应。

这种结构中的vera有时可以省略:
\begin{exe}
    \ex
    \gll ǫll	þín	orðrœða	þykki	mér	góð\\
    all	your	talk-\textsc{n}	seems	me-\textsc{d}	good-\textsc{n}\\
    \trans `All your talk sounds good to me’
\end{exe}

如果句子的主语和体验者恰好是同一个,þykkja用其反身形式þykkjask:
\begin{exe}
    \ex
    \gll þykkjask	þeir	þar	kenna	Lúsa-Odda\\
    seem-{\footnotesize 3}\textsc{p}-\textsc{rfl}	they-\textsc{n}	there	know	Lusa-Oddi-\textsc{a}\\
    \trans `They think they recognize Lusa-Oddi there’
\end{exe}

除þykkja以外,一些动词的反身形式也有相同的意思,如sýnask `appear’, virðask `deem’,它们的用法和þykkja一样,但是这里的-sk并不代表句子的主语和体验者是同一个,相反这些动词只是固定下来的形式,它们依旧可以添加其他与格宾语作为体验者。

\subsection{从句作主语}
作主语的从句是名词性的,这类从句总以引导词at开头(参考交叉引用),相当于英文that. 从句可以是限定性的,也可以是非限定性的。从句作主语时,谓语用第三人称单数形式(另见\ref{sec:SVagreement})。
\begin{exe}
    \ex
    \gll hǫrmuligt	er	slíkt	at	vita\\
    sad-\textsc{neu}-\textsc{n}	is	such-\textsc{a}	to	know\\
    \trans `It is sad to know that’

    \ex
    \gll oss	sýnisk	úmakligt,	at	…\\
    us-\textsc{d}	seems	unproper-\textsc{neu}-\textsc{n}	that	…\\
    \trans `We consider it unproper that …’
\end{exe}

在上面的两句中,前者是一个非限定性从句,后者是一个没有写完的限定性从句,at后面的部分可以是任意的限定句。

这种把从句直接指派为论元的用法实际上并不多。更多的情况下,是用代词þat作为主语,把从句作为þat的补足语,它表达的意思与从句主语相同,不过这种用法更符合一般的陈述句的结构:
\begin{exe}
    \ex
    \gll þat	er	upphaf	þeirar	sǫgu,	at	…\\
    that-\textsc{n}	is	beginning-\textsc{n}	that-\textsc{g}	story-\textsc{g}	that	…\\
    \trans `The beginning of the story is that …’
\end{exe}

\subsection{无人称结构}
\label{sec:impersonal}
无人称结构,或称无主句,指的是句子中没有主格主语的情况。虽然从句作为主语时严格意义上也是没有主格主语的,但我们在这里不讨论这种情况。句中没有主语有三种可能的情况,一是谓语不指派外部论元;二是某些动词的被动语态;三是主语被省略或作者认为不必要表达。前两种情况是词法上是所要求的,第三种情况则主要受语义或上下文的驱动。下面逐一介绍这三类情况:
\begin{enumerate}
    \setlength{\parindent}{2em}
    \item 谓语不指派外部论元

          有几类动词总是不能添加主语:
          \begin{enumerate}
              \setlength{\parindent}{2em}
              \item 表示时间流逝、季节转换以及自然发生的事件的动词

                    这类动词缺乏明确的主体,也没有承受者,因此既不能接主语,也不能接宾语,总是单独使用,以第三人称单数出现。包括hausta `become autumn’, snjófa `snow’, fjara `ebb’, flæða `flood’, dimma `get dark’, birta `get bright, dawn’, reka `drift’等:
                    \begin{exe}
                        \ex
                        \gll en	áðr	hafði	snjófat	nǫkkut\\
                        but	before	had	snowed	little\\
                        \trans `But earlier it had snowed a bit’
                        \ex \gll
                        fjarar	nú	undan	skipinu\\
                        ebbs	now	under	ship-the\\
                        \trans `The tide now recedes from under the ship’
                    \end{exe}

              \item 表示身体状态、感觉或思维过程的动词

                    这类动词一般接续一个间接格表示感受者,有的动词还能进一步接一个间接格表示感觉的内容。
                    下列的常见动词只接一个间接格表示感受者:
                    \begin{table}[H]
                        \centering
                        \begin{tabular}{ll}
                            \toprule
                            接宾格                   & 接与格                  \\ \midrule
                            þyrsta `be thirsty’      & bregða `startle’        \\
                            svengja `be hungry’      & bjóða `find disgusting’ \\
                            svimra/svima `be dizzy’  & líka `like’             \\
                            syfja `be sleepy’        & blæða `bleed’           \\
                            verkja `be painful’      & hitna `feel hot’        \\
                            langa `desire, long for’ &                         \\
                            \bottomrule
                        \end{tabular}
                    \end{table}

                    其基础用法都与下句类似:
                    \begin{exe}
                        \ex \gll
                        hana	þyrsti	mjók\\
                        her-\textsc{a}	thirsts	muck	\\
                        \trans `She feels very thirsty’
                    \end{exe}

                    有时,这些动词也可以接续介词短语表示原因或感受的对象等,例如:
                    \begin{exe}
                        \ex \gll
                        brá	þeim	mjök	við	þessi	tíðindi\\
                        startled	them-\textsc{d}	much	with	these-\textsc{a}	news-\textsc{a}\\
                        \trans `The news startled them a lot’
                    \end{exe}

                    langa和líka常接til+属格表示表示期待、喜欢的对象:
                    \begin{exe}
                        \ex \gll
                        mik	langar	ekki	til	þess\\
                        me-\textsc{d}	longs	not	towards	this\\
                        \trans `I do not long for this’
                    \end{exe}

                    由于介词也只接续间接格,句中仍没有主格主语。

                    以下的常见动词可以接两个简介格表示感受者和感受的内容:
                    \begin{table}[H]
                        \centering
                        \begin{tabular}{llll}
                            \toprule
                            动词                 & 感受者 & 感受的内容 & 表意                         \\
                            \midrule
                            vanta `want’         & 宾格   & 宾格       & sb. want sth.                \\
                            dreyma `dream’       & 宾格   & 宾格       & sb. dream about sth.         \\
                            gruna `suspect’      & 宾格   & 宾格/属格  & sb. be suspicious about sth. \\
                            minna `remember’     & 宾格   & 属格       & sb. remember sth.            \\
                            semja `agree on’     & 与格   & 宾格       & sb. agree on sth.            \\
                            bresta `lack’        & 宾格   & 宾格       & sb. lack sth.                \\
                            skorta `be short of’ & 宾格   & 宾格       & sb. lack sth.                \\
                            \bottomrule
                        \end{tabular}
                    \end{table}

                    其用法形如:
                    \begin{exe}
                        \ex \gll
                        minnir	mik	hinnar	konunnar\\
                        remembers	me-\textsc{a}	that-\textsc{g}	woman-\textsc{g}\\
                        \trans `I remember that woman’
                        \ex \gll
                        skortir	þik	eigi	metnað\\
                        lacks	you-\textsc{a}	no	pride-\textsc{a}\\
                        \trans `You have no lack of pride‌’
                    \end{exe}

                    表示“缺乏”的动词在语义上和主观的感情或思维过程略有区别,它们反映的是客观事实。不过它们的用法十分相似,因此把它们都归到一类。

                    上述的这些动词也并非都只有无人称的用法,有时这些动词也能像正常的动词一般使用,如minna也可以指派一个主格主语,这时候它的含义是“提醒”,并有形如英语remind sb. of sth.的结构,即提醒的对象用与格(区分于无人称结构中感受者用宾格),提醒的内容用属格。

                    与事件的“发生、结束”有关的动词可以接两个宾格,一个宾格表示事件发生的对象,另一个宾格表示事件本身。这类动词包括henda `be caught in, happen’, þrjóta `come to end, fail’, takask `succeed’等。有时它们也可以只接一个表示事件的宾格,这时不关注事件发生的对象。参考下面的两句,前者接两个宾格,后者只接表示事件的宾格:
                    \begin{exe}
                        \ex \gll
                        mik	hefir	hent	mart	til	afgerða	við	Guð\\
                        me-{{\acc}}	has	happened	many-{\acc} towards	offence	against	God\\
                        \trans `I have happened to commit many sins against God’
                        \ex \gll
                        en	er	veizluna	þrýtr\\
                        but	when	banquet-the-\textsc{a}	ends\\
                        \trans `But when it comes to the end of the banquet’
                    \end{exe}

                    一些形容词有和上述动词相同的表意,它们构成的系表结构也是无人称的:
                    \begin{exe}
                        \ex \gll
                        þá	var	myrkt	mjǫk\\
                        then	was	dark	muck\\
                        \trans `The it became very dark’
                        \ex \gll
                        mér	er	kalt\\
                        me-\textsc{d}	is	cold\\
                        \trans `I feel cold’
                    \end{exe}
          \end{enumerate}

    \item 无人称被动

          我们在\ref{sec:passive}中提到,被动句的主语来自于相应的主动句的谓语的内部论元,且只有宾格论元可提升为被动句的主格主语。按照这种原则,有两类动词的被动句缺少主语。
          \begin{enumerate}
              \setlength{\parindent}{2em}
              \item 不及物动词

                    不及物动词只指派一个外部论元作为主语,因此根本不存在内部论元可供提升。这类句子在英语中是不能变成被动句的:
                    \begin{quote}
                        主动句:He danced.\\
                        被动句:†is danced.
                    \end{quote}

                    但在古诺尔斯语中,不及物动词也能变成被动句,这种被动句是无人称的,且其表意一般是对存在性的判断。要构成不及物动词的被动句,只需把相应主动句的动词改成vera+过去分词。和其他无人称结构一样,vera一般用第三人称单数式。过去分词用中性形式。试比较下面两句:

                    主动句:
                    \begin{exe}
                        \ex \gll
                        gekk	hann	inn	nǫkkut	fyrir	lýsing\\
                        went	he-\textsc{b}	in	shortly	before	dawn\\
                        \trans `He went in shortly before dawn’
                    \end{exe}

                    被动句:
                    \begin{exe}
                        \ex \gll
                        var	gengit	inn	nǫkkut	fyrir	lýsing\\
                        was	gone-\textsc{\MakeLowercase{NEU-N}}	in	shortly	before	dawn\\
                        \trans `Someone went in shortly before dawn’
                    \end{exe}

                    被动句同样不强调动作的主体,其表意侧重“某事发生了”。
              \item 不接宾格宾语的动词

                    许多动词不接续宾格宾语,如bjarga `save’, lesa `read’, fagna `welcome’接与格,leita `seek’, bíða `wait for’接属格等等(详见交叉引用)。这些动词构成被动句时不能把其非宾格宾语提升为主格主语,因此这类动词也引起无人称结构。要构成被动句,只需要把主动句动词改为vera+过去分词,原先的非宾格宾语保持原先的格不变。试比较:

                    主动句:
                    \begin{exe}
                        \ex \gll
                        þar	fǫgnuðu	þeir	Þorsteini	vel\\
                        there	welcomed	they-{\nom}	Thorstein-{\dat}	well\\
                        \trans `They welcomed Thorstein warmly’
                    \end{exe}

                    被动句:
                    \begin{exe}
                        \ex \gll
                        Þorsteini	var	þar	vel	fagnat\\
                        Thorstein-{\dat}	was	there	well	welcomed-{\neu}-{\nom}\\
                        \trans `Thorstein was well received there’
                    \end{exe}

                    我们在\ref{sec:passive}中还提到了与vera+过去分词类似的被动结构,即vera+现在分词或at-不定式。这类结构有一个特点,即不严格要求把宾语提升为主格主语,下面的句子也是可行的:
                    \begin{exe}
                        \ex \gll
                        nú	er	at	verja	\textit{sik}\\
                        now	is	to	defend	oneself-{\acc}\\
                        \trans `Now it is about time to defend oneself’
                    \end{exe}

                    本句话保持宾格sik不变的另一个原因在于反身代词没有主格形式。其他一些例子还包括:
                    \begin{exe}
                        \ex \gll
                        eigi	er	virðandi	\textit{ásjónir}	manna	í	dómum\\
                        not	is	considering	appearance-{\acc}	men-{\gen}	i	judgement\\
                        \trans `Judge not by one’s appearance’
                        \ex \gll
                        \textit{þess}	er	fyrst	leitanda\\
                        that-{\gen}	is	first	examining\\
                        \trans `It should be examined first’
                    \end{exe}

                    和其他无主语句一样,动词使用第三人称单数,现在分词使用-i/-a词尾均可。
          \end{enumerate}

    \item 主语省略或隐含

          上述表示时间流逝、季节转换以及自然发生的事件的动词有一些近义的表达,这些表达里并不涉及典型的无人称动词,但是它们也不需要添加主语:
          \begin{exe}
              \ex \gll
              gerði	nú	myrkt	af	nótt\\
              made	now	dark	at	night\\
              \trans `It got dark at night’
          \end{exe}

          如果句子的主语是泛指且谓语的主体并非表意的重心时,在不引起歧义的情况下主语可以省略。当句中的谓语是由情态动词构成的短语时,这种情况尤其常见:
          \begin{exe}
              \ex \gll
              má	þar	ekki	stórskipum	fara\\
              can	there	not	big-ships-{\dat}	travel\\
              \trans `One cannot travel there with big ships’
              \ex \label{standi} \gll
              standi	menn	upp	ok	taki	hann, ok	skal	hann	drepa	\\
              stand-\sub -{\footnotesize 3} \pl	men-{\nom}	up	and	seize-\sub -{\footnotesize 3} \pl	him-{\acc} and	shall--{\footnotesize 3} {\sing}	him-{\acc}	kill		\\
              \trans `Let men stand up and seize him, and he is to be killed’
          \end{exe}

          第\ref{standi}句比较复杂,主要是因为hann既是主格又是宾格导致的。注意menn是maðr的复数主格形式,它是动词standi和taki的主语,动词虚拟式表示要求。第一句中taki hann和第二句中skal hann的hann是同指的,且都是宾格,这是因为drepa不能作不及物动词使用。第二句中的skal暗示被省略的主语是单数形式,从上下文来看是前一句中menn之中的某一人。

          主语省略还可能发生在复合句中,当后一句的主语在第一句中已经提到,并可以不产生歧义地被推断出来时,后一句的主语可以省略。在英语中,只有后一句的主语也是前一句的主语时,后一句的主语才能省略:
          \begin{quote}
              He rushed out of the building, ran across the road and disappeared in the crowd.\\
              †He is reading the book, is about the history of Iceland.
          \end{quote}

          但在古诺尔斯语中,后一句中的主语可以在前一句中充当任何成分:

          主格主语:
          \begin{exe}
              \ex \gll Egill	kastar	horninu,	en	greip	sverðit	ok	Brá\\
              Egil	casts	horn-the,	but	grasped	sword-the	and	drew\\

              \trans `Egil cast the horn as he grasped and drew his sword’
          \end{exe}

          直接宾语:
          \begin{exe}
              \ex \gll þá	skar	Rǫgnvaldr	hár	hans,	en	áðr	hafði verit	úskorit	tíu	vetr\\
              then	cut	Rognvald	hair-{\acc}	his	but	before	had been	uncut	ten	winter	\\		 \trans`Then Rognvald cut his hair which had been uncut for ten years’
          \end{exe}

          无人称结构中的与格对象:
          \begin{exe}
              \ex \gll þat	líkaði	henni	vel	ok	þakkaði	honum\\
              it	pleased	her-{\dat}	well	and	thanked	him\\
              \trans `She liked it very much and thanked him’
          \end{exe}

          其他情况下,一般不省略主语。

\end{enumerate}

\section{词序}
像很多屈折语一样,古诺尔斯语丰富的形态曲折使得句子成分的判断几乎不依赖于词序(主要是主语和宾语的判断),因而古诺尔斯语的词序往往被当作是“自由”的。不过,有一些规律可供总结。

\subsection{限定动词的位置}
限定动词是句子的必有成分,因而动词的位置可以作为整句词序的锚点。一般来说,动词位于句中的第二位,这称之为动词第二顺位(V2 Word order)。许多日耳曼语中都有这样的结构,例如英语中的neither do I. 这里的第二顺位指的并不一定是句子中的第二个词,而是第二个句子成分。一个句子成分可以由多个词组成,例如下面的英语句也符合动词第二顺位:
\begin{quote}
    Never in my life have I seen such a thing.
\end{quote}
古诺尔斯语中,动词也符合上述的规则。
\subsection{动词前后的位置}
动词前的位置,即句子的第一个成分,在古诺尔斯语中是任意的,这和英语这样强调词序的语言有着根本的区别。句子的首位可以是主语,也可以是宾语,还可能是状语等等。一般来说,移动到句子开头的成分有被强调的作用,这个位置也被称为主题(Topic)。根据主题的不同,整个句子表现出不同的词序:

\textbf{副词}

许多句子副词被提到句首位置,最典型的是þá `then’, nú `now’, síðan `since’, svá `thus’等. 这时,典型的句型是:
\begin{center}
    \textbf{Adv + V + S + (O)}
\end{center}

句子的主语一般紧随动词之后,之后可接宾语以及其他状语等,例如:
\begin{exe}
    \ex \gll
    þá	reið	Óðinn	fyrir	austan	dyrr.\\
    Then	rode	Odin-{\nom}	forth	eastern	door\\
    \trans `Then Odin rode to the eastern door’
\end{exe}

否定副词也可以移动到句子首位:
\begin{exe}
    \ex \gll
    eigi	munu	vápn	eða	viðir	granda	Baldri\\
    not	shall	weapons-N	or	trees-{\nom}	hurt	Baldr-{\dat}\\
    \trans `No weapon or tree shall hurt Baldr’
\end{exe}

\textbf{主语}

主语出现在句首后,动词后的位置常常被句子副词占据,因此结构类似于:
\begin{center}
    \textbf{S + V + (Adv) + (O)}
\end{center}
\begin{exe}
    \ex \gll
    ljós	brann	í	skálanum	um	nóttina\\
    light-{\nom}	burnt	in	hall-the	throughout	night-the\\
    \trans `A light burnt in the hall throughout the night’
\end{exe}

\textbf{宾语}

宾语出现在句首的情况比较罕见,此时动词后一般立刻接主语:
\begin{center}
    \textbf{O+ V + S + (Adv)}
\end{center}
\begin{exe}
    \ex \gll
    þik	vil	ek	enn	fregna\\
    you-{\acc}	will	I	more	ask\\
    \trans `I wish to ask you more’
\end{exe}

\textbf{短语的中心语}

中心语是短语中被修饰语修饰、限制的中心成分。一般来说,一个短语会作为一个整体出现在句子中的某个位置。但在古诺尔斯语中,偶尔可以把短语拆开,并将其中心语前置:
\begin{exe}
    \ex \gll
    styrks	eiga	ván	af	Skota-konungi\\
    support-{\gen}	have	hope	from	Scots-king\\
    \trans `Have faith in the support of the king of the Scots’
\end{exe}

本句中styrks af Skota-konungi这个短语整体作为ván的补足语,但为了强调styrks,把这个短语拆开了。

指示代词前移也比较常见:
\begin{exe}
    \ex \gll
    þau	skal	segja	orð	mín	maðr	manni\\
    those-{\acc}	shall	say	words-{\acc}	my	man-{\nom}	man-{\dat}\\
    \trans `Those words of mine shall be told from man to man’
\end{exe}

\textbf{短语的修饰语}

和中心语一样,修饰语也可以移动至句子首位。一般来说,最常见的情况是把名词短语中的形容词前置以构成强调:
\begin{exe}
    \ex \gll
    góðan	eignum	vér	konung\\
    good-{\acc}	have	we	king-{\acc}\\
    \trans `We have a good king’

\end{exe}

\textbf{空主题}

空主题指的是动词前没有其他的句子成分,即句子本身就以动词开头。按照前述的分类方法,这是一种特殊的情况。造成这种现象的原因主要是句子中缺少某类成分。\ref{sec:impersonal}中介绍的无人称结构就容易引起空主题:
\begin{exe}
    \ex \gll
    skal	hana	engan	hlut	skorta\\
    shall	her-{\acc}	no-{\acc}	things-A	lack\\
    \trans `She shall lack nothing’

    \ex \gll
    brá	þeim	mjök	við	þessi	tíðindi\\
    startled	them-{\dat}	much	with	these-{\acc}	news-{\acc}\\
    \trans `The news startled them a lot’
\end{exe}

祈使句有时也可以略去主语,将动词放到句首:
\begin{exe}
    \ex \gll
    trúið	á	goð	várt\\
    believe-\textsc{imp}-{\footnotesize 2}{\pl}	on	god	our\\
    \trans `Trust our god!’
\end{exe}

一种特殊的情况值得注意。用ok `and‌’作连词连接两句时,它后面总是紧跟着第二个句子的动词:
\begin{exe}
    \ex \gll
    hann	lætr	Hǫtt	fara	með	sér,	ok	\textit{gørir}	hann þat	nauðugr	ok	\textit{kallaði}	hann	…	\\	he	lets	Hott	go	with	himself,	and	did	he
    that	unwillingly	and	said	he	…\\
    \trans `He commanded Hott to go with him; and he did so unwillingly, and said he …‌’
\end{exe}






