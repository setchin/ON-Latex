\chapter{动词与变位法}

\begin{introduction}[章节要点]
    \item 古诺尔斯语的动词系统
    \item 动词的范畴
    \item 动词词干的分类
    \item 各类动词的变位法
    \item 不规则动词
\end{introduction}

\section{动词的概述}\label{sec:verb-intro}
和我们已经介绍过的名词一样,动词也有“强动词”和“弱动词”两种类型(或称“强变化”和“弱变化”动词)。不过需要注意,动词的强弱和名词的强弱指的不是同一个东西。简单来说,强动词可以理解为以元音交替(Ablaut)为特征的“不规则动词”(这里的不规则是借用英语的概念来说的,读者通过下文不难发现,这些变化实际上遵守一定的规律),这些动词的词根元音随时态和人称的变化而改变;对应地,弱动词可以类比为英语的规则动词,这些动词的词根元音在变形过程中保持不变,除非其受到人称词尾中的i-/u-变异的影响。和名词类似,动词的强弱也只有类型学上的意义,它们完全可以用“I型动词”和“Ⅱ型动词”这样的术语来代替。

我们在这里简单介绍一下元音交替的来历。元音交替是原始印欧语的重要特征,它可以用于同根词的进一步变形。根据历史比较语言学的发现,原始印欧语的词根结构通常是*CeC,即两个辅音之间包含一个元音*e. 从这个词根出发,将*e变成其他元音(如*o, *ē, *ō等)甚至使其完全脱落,以此改变词意的过程就是元音交替。更一般地来说,元音交替不仅发生在词根中,在许多后缀里也有相似的现象,因此,这个过程对印欧语的形态学具有很大的意义:在名词中,元音交替可以区分变格;在动词中它则可以区分变位。参考古希腊语中πατήρ `father'的元音交替:

\begin{longtable}{llll}
    \toprule
    \multicolumn{2}{c}{古希腊语} & 格                & 元音              \\
    \midrule
    \endhead
    \bottomrule
    \endfoot
    πα-τ\textbf{έ}ρ-α            & pa-t\textbf{é}r-a & 单数宾格 & 短e    \\
    πα-τ\textbf{ή}ρ              & pa-t\textbf{ḗ}r   & 单数主格 & 长e    \\
    πα-τρ-ός                     & pa-tr-ós          & 单数属格 & 无元音 \\
\end{longtable}

在动词中,元音交替的现象完好地保留在梵语等更早的语言(梵语成文时期比古诺尔斯语早十余个世纪)中。在英语中,能体现元音交替是一些所谓的不规则动词:

\begin{quote}
    r\textbf{i}de r\textbf{o}de r\textbf{i}dden

    s\textbf{i}ng s\textbf{a}ng s\textbf{u}ng

    fl\textbf{y} fl\textbf{ew} fl\textbf{o}wn
\end{quote}

但是,元音交替作为一种非常古老的构词方法,在原始日耳曼语从原始印欧语分离出来的时候,已经逐渐被舍弃了,因此原始日耳曼语采用了新的方法来衍生动词,这就是“弱变化”规则(英文对应-ed式规则动词)。继承于原始印欧语的更古老的动词仍然保留了强变化规则,不过随着时间的推移,强变化愈发被人遗忘,许多历史上的强动词也归入弱动词之中。

\subsubsection*{语法范畴}
古诺尔斯语动词根据人称、时态、语态、语气发生屈折。这些与词形变化相关的语法意义的概括就是语法范畴。古诺尔斯语的动词系统具有如下的范畴:

三种\textbf{人称}(Person):第一、第二和第三人称。在每个人称中,还区分单数和复数。更古老的日耳曼语中出现的双数形式也在古诺尔斯语中消失。

两种\textbf{时态}(Tense):现在(Present)和过去(Preterite)。读者最好把二者理解为过去和非过去,因为简单来说,现在时既支配现在的动作,也支配未来的动作。相同地,古诺尔斯语的过去式同样可以和英语的若干与过去有关的时态有关。有英语基础的读者会经常联想到英语中复杂的时态系统,但事实上,这些表达严格来说叫作``体''或``体貌''(Aspect),例如英语中的完成时实际上是一种体,而非时态。时态用来区分动词在时间尺度上的位置;体貌则描述关于该动作的开始、持续、完成或重复等方面的情况,但不涉及该动作发生的时间。当然,许多动词的表达既涉及时态又涉及体貌,例如英语中的现在完成时描述了完成体,但隐含的意思是一个现在的状态。因此在许多印欧语中,这个时态(暂且仍粗略地使用这个不太准确的术语)采取动词现在时的词干并加上一些派生后缀。在古诺尔斯语中,表达体态的方法是添加助动词,而非词形屈折。

两种\textbf{语态}(Voice):主动(Active)和中动(Middle/Mediopassive)。中动态在古诺尔斯语中一方面有一定被动的含义,另一方面还表达一些反身的动作,详见(待完成)

三种\textbf{语气}(Mood):直陈(Indicative),虚拟(Subjunctive),祈使(Imperative)。直陈语气表示一般地陈述;虚拟语气主要表示可能发生但尚未发生的动作或愿望;祈使语气表示命令。

\section{强动词的变位法}
强动词的特征是元音交替。如\ref{sec:verb-intro}所述,元音的交替仍然保留在英语中,例如sing-sang-sung-song. 词干部分s-ng 加上i得现在时,加a得过去式,加u得过去分词,加o得衍生的名词。但不是每个动词的元音交替模式都是一致的,比如hang-hung-hung,它不仅采取不同的元音添加,也没有对应的衍生名词。

在古诺尔斯语中,根据元音交替的模式(例如上面的i-a-u和a-u-u)可以将强动词分为七类。前六类动词比较规则,第七类动词则是一些历史上不太规则的动词的残留。接下来,为了描述这些强动词元音交替的模式,我们需要选择动词的一些形式作为基础。在上面英语的例子中,我们使用现在时、过去时和过去分词(英语动词的人称范畴基本退化了,因此我们没有指明人称)的词形就可以描述出i-a-u这套元音交替模式。在古诺尔斯语中,也有类似的情况。

在进一步了解词形的变化,必须首先明确古诺尔斯语动词的最基本形式:\textbf{不定式}。不定式在句法上属于非限定动词的一种,即这种动词还没有人称、时态等的“限定”,但我们在这一章从形态学角度谈到的不定式只是动词词形的一种形式。不定式通常很简单,未经过变形,并能引导我们推断出动词的其他形式。有时情况下,这种基本的形式也被称为词典形(Dictionary form),即词典上提供的基本形态。古诺尔斯语的不定式可以和以下几种语言的``不定式''\footnote{词典形是形态学的概念,而不定式是句法的概念。有许多语言的词典形就是属于非限定动词的不定式,但还有一些语言的词典形是限定动词,例如古希腊语常用第一人称现在时单数式作为词典形,梵语常用第三人称现在时单数式。只要形式简单基本,就可以作为词典形。}类比:拉丁语port\textcolor{cyan}{are},德语lib\textcolor{cyan}{en},日语考\textcolor{cyan}{える}。动词变化丰富的语言基本都有不定式的标记,用蓝色字体标出,蓝色之外的部分可以认为是词干。就古诺尔斯语而言,不定式的标记是-a,-a前面的部分是词干。词干有时以半元音-j-或者-v-结尾,半元音的出现(特别是-j-)有时会引起进一步的音变(西弗斯定律的作用,见\ref{a/ja/wa-词干})。
% to be done
% 进一步来讲,古诺尔斯语的不定式也和英语一样有两种,第一种是单独的以-a结尾的词典形,第二种是以at+词典形引导的at不定式。这两种不同的不定式都属于非限定动词,它们的语法功能和英文中带不带to的不定式类似,都可以作某些动词的补足语等(参见)。

下面,我们可以从前六类动词的各个形态中初步发现元音交替的模式。参考bíta `bite', skjóta `shoot', bresta `burst', bera `bear, carry', reka `drive‌', fara `go/fare‌'的变位:
\begin{longtable}{lllllll}
    \toprule
    类  & 不定式 & 三单现在时 & 三单过去时 & 三复过去时 & 三单过去虚拟式 & 过去分词 \\
    \midrule
    \endhead
    \bottomrule
    \endfoot
    I   & bíta   & bítr       & beit       & bitu       & biti           & bitinn   \\
    II  & skjóta & skýtr      & skaut      & skutu      & skyti          & skotinn  \\
    III & bresta & brestr     & brast      & brustu     & brysti         & brostinn \\
    IV  & bera   & berr       & bar        & báru       & bæri           & borinn   \\
    V   & reka   & rekr       & rak        & ráku       & ræki           & rekinn   \\
    VI  & fara   & ferr       & fór        & fóru       & fœri           & farinn   \\
\end{longtable}

通过上表可以发现:
\begin{info}
    \begin{enumerate}
        \item  单数现在时的元音要么和不定式一致,要么由它发生i-变异得到。
        \item  单数过去式的元音是独立的。
        \item  复数过去式的元音是独立的。
        \item  过去虚拟式的元音由直陈复数过去式的元音i-变异得到。
        \item  过去分词的元音是独立的。
    \end{enumerate}
\end{info}

因此,古诺尔斯语的强动词系统中有四种元音的交替。那么,最少用四个动词形式即可推断出整个变位表的形式,这四个形式称为四个基本元(Principal parts)。这四个基本元在词典上一般选用:
\begin{quote}
    第一基本元:不定式,或词典形;

    第二基本元:第三人称单数过去式;

    第三基本元:第三人称复数过去式;

    第四基本元:过去分词。
\end{quote}

有时词典中也额外标记第三人称单数现在时。例如bera在Cleasby \& Vigfússon的An Icelandic-English Dictionary上就记为:

\begin{quote}
    \textit{BERA, bar, báru, borit, pres. berr}
\end{quote}

\subsection{强动词的主动词尾}\label{强动词的主动词尾}
古诺尔斯语的动词分为主动词尾和中动词尾,主动态动词添加主动词尾,其含义和英文的主动态没有区别。强动词的主动词尾如下所示:
\begin{longtable}{llll}
    \toprule
    强动词     & 直陈 & 虚拟 & 祈使 \\
    \midrule
    \endhead
    \bottomrule
    \endfoot
    单数现在时 &      &      &      \\
    1          & -ø   & -a   &      \\
    2          & -r   & -ir  & -ø   \\
    3          & -r   & -i   &      \\
    复数现在时 &      &      &      \\
    1          & -um  & -im  & -um  \\
    2          & -ið  & -ið  & -ið  \\
    3          & -a   & -i   &      \\
    单数过去时 &      &      &      \\
    1          & -ø   & -a   &      \\
    2          & -t   & -ir  &      \\
    3          & -ø   & -i   &      \\
    复数过去时 &      &      &      \\
    1          & -um  & -im  &      \\
    2          & -uð  & -ið  &      \\
    3          & -u   & -i   &      \\
\end{longtable}

-ø表示无须添加词尾,表格中的空白指这种形式不存在。这完全体现在动词的祈使式上,只有单复数的现在时第二人称、复数现在时第一人称有祈使式。即祈使式可以用在下面两种情况下:

\begin{enumerate}
    \item
          要求你、你们做,参考英语 `do it!';
    \item
          提议我们做,参考英语 `let's do it!'.
\end{enumerate}

在不同时态、数的词尾前,要选取不同的基本元,其变化方式如下:

\begin{info}
    单数现在直陈式:取不定式词干,如果词干有后元音,则施加i-变异,加词尾; \\
    复数现在直陈式/一切现在虚拟式:取不定式词干,加词尾;          \\
    单数过去直陈式:取第二基本元词干,加词尾;                \\
    复数过去直陈式:取第三基本元词干,加词尾;                \\
    一切过去虚拟式:取第三基本元词干,如果词干有后元音,则施加i-变异,加词尾;                                   \\
    一切祈使式:取不定式词干,加词尾;                    \\
    现在分词:取不定式词干,加词尾;                     \\
    过去分词:取第四基本元词干,加词尾。
\end{info}

注意,某些情况下,非圆唇元音受-um的影响有可能会发生u-变异,这是规则音变的结果。用(1), (2), (3), (4)分别标记四个基本元,变形方式如下:

\begin{longtable}{llll}
    \toprule
    强动词     & 直陈              & 虚拟               & 祈使              \\
    \midrule
    \endhead
    \bottomrule
    \endfoot
    单数现在时 &                   &                    &                   \\
    1          & (1) + (i-变异) -ø & (1) +-a            &                   \\
    2          & (1) + (i-变异) -r & (1) +-ir           & (1) +-ø           \\
    3          & (1) + (i-变异) -r & (1) +-i            &                   \\
    复数现在时 &                   &                    &                   \\
    1          & (1) +(u-变异) -um & (1) +-im           & (1) +(u-变异) -um \\
    2          & (1) +-ið          & (1) +-ið           & (1) +-ið          \\
    3          & (1) +-a           & (1) +-i            &                   \\
    单数过去时 &                   &                    &                   \\
    1          & (2) +-ø           & (3) + (i-变异) -a  &                   \\
    2          & (2) +-t           & (3) + (i-变异) -ir &                   \\
    3          & (2) +-ø           & (3) + (i-变异) -i  &                   \\
    复数过去时 &                   &                    &                   \\
    1          & (3) +-um          & (3) + (i-变异) -im &                   \\
    2          & (3) +-uð          & (3) + (i-变异) -ið &                   \\
    3          & (3) +-u           & (3) + (i-变异) -i  &                   \\
    不定式     & (1) + -a          &                    &                   \\
    现在分词   & (1) + -andi       &                    &                   \\
    过去分词   & (4) + -inn        &                    &                   \\
\end{longtable}
说明:

\begin{enumerate}
    \item
          现在时的单数直陈式中普遍地出现了i-变异,但没有任何i的痕迹。在其他西日耳曼语中,只有第二人称和第三人称出现了i-变异,例如古英语的单数现在时bēode-bīetst-bīetst,这表明古诺尔斯语中第一人称的i-变异是类推的影响。从中动态(见\ref{强动词的中动词尾})的词尾来看,第一人称的词尾本是*u.
          第二人称的-r由-ir演变而来,第三人称的-r也由第二人称类推得到,在卢恩铭文中记载到了早期的-iþ形式。
    \item
          绝大多数情况下(除了过去虚拟式)词尾-ið不造成i-变异。在原始日耳曼语中,这个词尾已经是*-id了,其不能造成i-变异应当是受同时态的其他词形的影响。
    \item
          \phantomsection\label{_Ref116919964}{}虚拟式的词尾都是一样的,但现在虚拟式和过去虚拟式由词干的i-变异所区分。在原始日耳曼语中,现在时的词尾包含一个双元音*ai(相当于比过去式的词尾多了*-a-),双元音发生缩略变为*e,后来抬升为i.
          这样,例如第二人称现在时词尾*-aiz > *-ez > -ir就没有i-变异条件。在过去式中则不存在这样的双元音,i-变异正常发生。无疑,类推作用使得过去式中某些本不能造成i-变异的词尾也引起了元音变异,这样才能与现在时区分开来。
    \item
          -um词尾比较规则地引起u-变异,这和名词中的情况类似。复数过去直陈式的词尾都含有-u,理论上都能造成u-变异,但是古诺尔斯语中保留下来的复数过去式词干没有可以发生u-变异的元音。
\end{enumerate}

\subsection{强动词的中动词尾}\label{强动词的中动词尾}

动词的语态的概念涉及给定陈述中施事和受事发挥作用的方式。施事是一个动作的逻辑执行者;受事是动作的逻辑接受者,或其对象。施事可能是但也可能不是其分句的语法主语,同样地,受事也可能是也可能不是其分句的直接宾语。当述语处于主动态时,施事也是语法主语;在这种情况下,如果存在受事,它就是直接宾语。例如``狗咬人''一句中,狗既是行动的逻辑执行者,也即施事,同时也是动词``咬''的语法主语。因此,这个表述是主动的;受事``人'',是直接宾语。当述语处于被动态时,受事成为语法主语。同样的句子可以改写为``人被狗咬''。动作的逻辑执行者``狗''仍然是施事。但它不再是语法主语了;而受事``人'',是动词``被咬''的语法主语,使之成为被动陈述。

中动态,正如其名字暗示的那样,处于主动和被动之间,是一种很难准确定义的语态。中动态表达的动作是对施事产生某种影响的,比如简单的反身动作(I washed myself‌),影响个人利益的动作(I had a sacrifice performed),或是内在变化(I called to mind what he said)还有一些其它差异不大的表达。在古诺尔斯语中,中动态也有被动态的含义。

中动态的表达最初是由主动态增加人称代词。第一人称单数添加mik `me',
其他人称都添加反身代词sik,参见\ref{人称代词}。后来这个表达固定成了词尾,其形式一般就是在主动态后加-sk,第一人称单数则是加-mk,同时进行一些语音变化。比较明显的有:

\begin{enumerate}
    \item
          -r + -sk > *-ssk > -sk
    \item
          -t + -sk > -zk
    \item
          -ð + -sk > -zk
\end{enumerate}

\begin{longtable}{llll}
    \toprule
    强动词     & 直陈  & 虚拟  & 祈使  \\
    \midrule
    \endhead
    \bottomrule
    \endfoot
    单数现在时 &       &       &       \\
    1          & -umk  & -umk  &       \\
    2          & -sk   & -isk  & -sk   \\
    3          & -sk   & -isk  &       \\
    复数现在时 &       &       &       \\
    1          & -umsk & -imsk & -umsk \\
    2          & -izk  & -izk  & -izk  \\
    3          & -ask  & -isk  &       \\
    单数过去时 &       &       &       \\
    1          & -umk  & -umk  &       \\
    2          & -zk   & -isk  &       \\
    3          & -sk   & -isk  &       \\
    复数过去时 &       &       &       \\
    1          & -umsk & -imsk &       \\
    2          & -uzk  & -izk  &       \\
    3          & -usk  & -isk  &       \\
\end{longtable}

说明:

\begin{enumerate}
    \item
          单数第一人称的词尾-umk反应了原始语直陈式的元音词尾*-u,而在主动态中,-u已经脱落。虚拟式的-umk可能是从直陈式借来的。
    \item
          *-rsk > -sk的变化实则有一点可疑,因为s在r之后本不会发生同化音变,例如sumar `summer'的属格sumars中r和s不同化。
\end{enumerate}

-umk可能会导致前面的前元音发生u-变异,加上词干后,其变化表如下:

\begin{longtable}{llll}
    \toprule
    强动词     & 直陈                                  & 虚拟                                  & 祈使                  \\
    \midrule
    \endhead
    \bottomrule
    \endfoot
    单数现在时 &                                       &                                       &                       \\
    1          & (1) + (u-变异) -umk                   & (1) + (u-变异) -umk                   &                       \\
    2          & (1) + (i-变异) -sk                    & (1) +-isk                             & (1) +-sk              \\
    3          & (1) + (i-变异) -sk                    & (1) +-isk                             &                       \\
    复数现在时 &                                       &                                       &                       \\
    1          & (1) +(u-变异) + -umsk                 & (1) +-imsk                            & (1) +(u-变异) + -umsk \\
    2          & (1) +-izk                             & (1) +-izk                             & (1) +-izk             \\
    3          & (1) +-ask                             & (1) +-isk                             &                       \\
    单数过去时 &                                       &                                       &                       \\
    1          & \textcolor{cyan}{(3)} + (u-变异) -umk & (3) + \textcolor{cyan}{(i-变异)} -umk &                       \\
    2          & (2) +-zk                              & (3) + (i-变异) -isk                   &                       \\
    3          & (2) +-sk                              & (3) + (i-变异) -isk                   &                       \\
    复数过去时 &                                       &                                       &                       \\
    1          & (3) + (u-变异) + -um                  & (3) + (i-变异) -imsk                  &                       \\
    2          & (3) +-uzk                             & (3) + (i-变异) -izk                   &                       \\
    3          & (3) +-usk                             & (3) + (i-变异) -isk                   &                       \\
    不定式     & (1) + -ask                            &                                       &                       \\
    现在分词   & (1) + -andisk                         &                                       &                       \\
    过去分词   & (4) + -izk                            &                                       &                       \\
\end{longtable}

说明:

\begin{enumerate}
    \item
          \phantomsection\label{_Ref117719619}{}第一人称单数直陈过去式用了复数词干。造成这个现象的原因是对词尾成分的误读和重解。由于第一人称单数在主动直陈式中没有词尾,在主动虚拟式中只有词尾-a,中动态词尾-umk可能被重新理解为-um + -k,而-um是复数式第一人称的词尾(无论是现在时还是过去时),因此所有的单数第一人称的词干全部采用了复数形式。在现在时中,单复数都采用了同一种词干,因此无法看出任何异常,而在过去时中,单数词干和复数词干不一致,就出现了单数中两种词干的混用。
    \item
          第一人称单数虚拟过去式的-umk没有造成u-变异,反而造成了i-变异。读者应该不难想到,这个i-变异是从其他过去时类推的结果,用以和现在时进行区分。否则,第一人称的现在时和过去时将无法区分。
\end{enumerate}

\subsection{第一强变位法}\label{第一强变位法}

古诺尔斯语的每一种强变位法都有自己的特征元音交替模式,第一强变位法的动词最基本的元音交替特征如下:

\begin{center}
    \textbf{í --- ei --- i --- i}
\end{center}


少数不定式词干以-g结尾的一类强动词hníga `fall' , míga `piss', síga
`sink', stíga `step'的第一或第三人称单数过去式还有另一种形式。这时,元音í变为长元音é而不是ei,同时词干结尾的g脱去\footnote{造成词尾-g脱落的原因是浊塞音g在原始诺尔斯语的词尾变成其音位变体h,但后来又系统地发生了词尾-h的脱落,如古诺尔斯语þó(比较Go. þau\textbf{h}; OE þēa\textbf{h};现代英语thou\textbf{gh}),因此例如sté这样的形式是更早期的词形的反映。ei > é的音变是一种元音缩合,但主要只出现在这类动词变形中。类似地,原始诺尔斯语还有大量的ai > á的缩合(ON. sár vs Go. sair `wound'),因此这个音变并不奇怪。}:

\begin{quote}
    stíga > steig, steigt, steig或sté, stétt, sté
\end{quote}

注意这里变化得到的长元音加词尾-t时会触发\ref{辅音的音变}中的辅音延长音变,因此第二人称单数直陈过去式为stétt而不是†stét.

我们以一类强动词bíta
`bite'为例,展示其完整的变位。词典中会给出其四个基本元bíta --- beit
--- bitu -- bitinn.

主动态:

\begin{longtable}{llll}
    \toprule
    一类强动词 & 直陈    & 虚拟  & 祈使  \\
    \midrule
    \endhead
    \bottomrule
    \endfoot
    单数现在时 &         &       &       \\
    1          & bít     & bíta  &       \\
    2          & bítr    & bítir & bít   \\
    3          & bítr    & bíti  &       \\
    复数现在时 &         &       &       \\
    1          & bítum   & bítim & bítum \\
    2          & bítið   & bítið & bítið \\
    3          & bíta    & bíti  &       \\
    单数过去时 &         &       &       \\
    1          & beit    & bita  &       \\
    2          & \textcolor{cyan}{beizt}   & bitir &       \\
    3          & beit    & biti  &       \\
    复数过去时 &         &       &       \\
    1          & bitum   & bitim &       \\
    2          & bituð   & bitið &       \\
    3          & bitu    & biti  &       \\
    不定式     & bíta    &       &       \\
    现在分词   & bítandi &       &       \\
    过去分词   & bitinn  &       &       \\
\end{longtable}

中动态:

\begin{longtable}{llll}
    \toprule
    一类强动词 & 直陈      & 虚拟    & 祈使    \\
    \midrule
    \endhead
    \bottomrule
    \endfoot
    单数现在时 &           &         &         \\
    1          & bítumk    & bítumk  &         \\
    2          & bízk      & bítisk  & bízk    \\
    3          & bízk      & bítisk  &         \\
    复数现在时 &           &         &         \\
    1          & bítumsk   & bítimsk & bítumsk \\
    2          & bítizk    & bítizk  & bítizk  \\
    3          & bítask    & bítisk  &         \\
    单数过去时 &           &         &         \\
    1          & bitumk    & bitumk  &         \\
    2          & \textcolor{cyan}{beizk}     & bitisk  &         \\
    3          & \textcolor{cyan}{beizk}     & bitisk  &         \\
    复数过去时 &           &         &         \\
    1          & bitumsk   & bitimsk &         \\
    2          & bituzk    & bitizk  &         \\
    3          & bitusk    & bitisk  &         \\
    不定式     & bítask    &         &         \\
    现在分词   & bítandisk &         &         \\
    过去分词   & bitizk    &         &         \\
\end{longtable}

说明:

\begin{enumerate}
    \item
          以-t为词干末尾的动词接续-t词尾时,发生*-tt > *-tst
          >
          -zt,这是古老的日耳曼语擦音定律的残留。中动态-zk发生了类似的音变。这种现象十分常见,在后续出现时,不做过多说明。
\end{enumerate}

\subsection{第二强变位法}\label{第二强变位法}

第二变位法的不定式词干的特征元音是双元音jú,其元音交替模式如下:
\begin{center}
    \textbf{jú, jó(ý) --- au --- u(y) --- o}
\end{center}


其中,用圆括号标记的是i-变异下的词根元音音变后的形式。

当跟随词根元音的辅音是齿音(ð, d, s, t)时,jú分割成jó,例如下列动词brjóta `break‌', ljósta `smite‌', skjóta `shoot‌', bjóða `offer‌', þrjóta `come to an end‌', kjósa `choose‌',由于双元音jú, jó都是后元音,再加某些词尾前需要将其i-变异。jú首先变成*jý,进一步变成ý,这是由于j在y前规则地脱去,我们已经在\ref{半元音的保持性}提到了。

在一类强动词中提到的词尾-g脱落现象在二类强动词中仍然存在,这种情况下,对应的元音变成ó。例如动词fljúga
`fly‌'的词干fljúg-在单数过去式中有fló-和flaug-两种形式,类似地,smjúg- `creep‌'有smó-和smaug-两种形式,如\ref{第一强变位法}所讲的一样,这里的长元音也会触发词尾的辅音延长音变。

二类动词中有以下一些不太规则的动词:

\begin{enumerate}
    \item
          以咝音s结尾的动词frjósa `freeze‌'和kjósa `choose‌'的过去式可以按照弱变位法(见\ref{第二弱变位法})变位,词干分别是frør-和kór-,出现在东部的冰岛方言中。例如单数过去式frøra, frørir, frøri,这里没有塞音的痕迹。
    \item
          三个动词lúka `finish‌', súpa `sip‌', lúta `bow‌'词干中没有j.\footnote{这些动词来源古老,仅在印度语支和日耳曼语中有所保留。简单来说,它们的现在时词干和正常的动词有所区别(因此在古诺尔斯语中反映为元音的不同),以表达某些一次性事件。介绍这些动词的元音交替需要大量PIE知识,读者可自查所谓的tudáti-type(命名来自梵语).}
\end{enumerate}

我们选用规则的动词skjóta来展示二类动词的强变位法。

主动态:

\begin{longtable}{llll}
    \toprule
    二类强动词 & 直陈      & 虚拟    & 祈使    \\
    \midrule
    \endhead
    \bottomrule
    \endfoot
    单数现在时 &           &         &         \\
    1          & skýt      & skjóta  &         \\
    2          & skýtr     & skjótir & skjót   \\
    3          & skýtr     & skjóti  &         \\
    复数现在时 &           &         &         \\
    1          & skjótum   & skjótim & skjótum \\
    2          & skjótið   & skjótið & skjótið \\
    3          & skjóta    & skjóti  &         \\
    单数过去时 &           &         &         \\
    1          & skaut     & skyta   &         \\
    2          & skauzt    & skytir  &         \\
    3          & skaut     & skyti   &         \\
    复数过去时 &           &         &         \\
    1          & skutum    & skytim  &         \\
    2          & skutuð    & skytið  &         \\
    3          & skutu     & skyti   &         \\
    不定式     & skjóta    &         &         \\
    现在分词   & skjótandi &         &         \\
    过去分词   & skotinn   &         &         \\
\end{longtable}

中动态:

\begin{longtable}{llll}
    \toprule
    二类强动词 & 直陈        & 虚拟      & 祈使   \\
    \midrule
    \endhead
    \bottomrule
    \endfoot
    单数现在时 &             &           &        \\
    1          & skýtumk     & skjótumk  &        \\
    2          & skýzk       & skjótisk  & skjózk \\
    3          & skýzk       & skjótisk  &        \\
    复数现在时 &             &           &        \\
    1          & skjótumsk   & skjótimsk & skjót  \\
    2          & skjótizk    & skjótizk  & skjót  \\
    3          & skjótask    & skjótisk  &        \\
    单数过去时 &             &           &        \\
    1          & skutumk     & skytumk   &        \\
    2          & skauzk      & skytisk   &        \\
    3          & skauzk      & skytisk   &        \\
    复数过去时 &             &           &        \\
    1          & skutumsk    & skytim    &        \\
    2          & skutuzk     & skytizk   &        \\
    3          & skutusk     & skytisk   &        \\
    不定式     & skjótask    &           &        \\
    现在分词   & skjótandisk &           &        \\
    过去分词   & skotizk     &           &        \\
\end{longtable}

\subsection{第三强变位法}\label{第三强变位法}

第三强变位法动词不定式的词干总是以一个双辅音结尾,其词根的元音是e或i,其元音交替模式如下:

\begin{center}
    \textbf{e, i --- a --- u(y) --- u, o}
\end{center}


三类强动词的基础元音是e,但如果元音后面紧随着n,这个元音抬升为i.
同样重要的元音音变发生在过去分词中,一般来说过去分词词根的元音都是u,但当它后面紧随着l或r时,u变为o.
第三强变位法涉及一些重要的语音规则,最显著的几个列举如下:

\begin{enumerate}
    \item
          单数过去式中词干结尾的-nd > -tt;-ng > -kk:

          \begin{quote}
              binda `bind‌' > batt, batzt, batt

              springa `spring, jump‌' > sprakk, sprakkt, sprakk

              注意词尾和词干相接触时 *-tt > -zt的变化
          \end{quote}
    \item
          单数过去式中词干结尾的-ld >- lt:

          \begin{quote}
              gjalda `pay‌' > galt, galzt, galt

              复数形式guldum, gulduð, guldu不变
          \end{quote}

    \item
          在即以l或r开头的辅音簇前,e分割为ja.
          这条规则对单数直陈现在时以外的词形都有效,因此可以理解为这类动词的单数直陈现在时是特殊的:

          \begin{quote}
              不定式gjalda > 单数现在时geld, geldr, geldr

              复数现在时gjǫldum, gjaldið, gjalda ǫ是u-变异后的结果

              不定式bjarga `rescue‌' > bergr vs. bjargið

              一个特殊的动词是skjálfa `shiver‌'
              这里e分割为长元音já,例如第一人称复数skjálfum, 而其他形式如skelf, skalf,
              skulfum, skolfinn等没有长元音。
          \end{quote}
    \item
          v在o或u前脱去。这条规则在古诺尔斯语中是通用的,但在第三强变格法中值得特别提及。例如verða
          `become‌' 的四个基本元的词干分别是verð-, varð-, urð-, orð-.
    \item
          一部分动词的不定式标记-a前有半元音-v-, 这导致了这些动词中保留了一些古老的u-变异:a > ǫ, i > y, e > ø. 因此søkkva
          `sink‌'的基本元是sekk-v-, sǫkk-v-, sukk-;sangv- > sǫng.
          注意,这里的-ng不会变成-kk,因为词干是包括-v-的。
\end{enumerate}




上述规则也有一些例外:

\begin{enumerate}
    \item
          brenna `burn‌' 和renna `run‌'的现在时词干中的e并没有变成i.
    \item
          \phantomsection\label{_Ref116921872}{}常见动词finna
          `find‌'的词干本是*finþ-, 但原始诺尔斯语晚期发生了*nþ >
          nn的音变。在第三第四基本元中,按照维尔纳定律发生了*nþ >
          nd的浊化,因此在古诺尔斯语中有不规则的词干finn-, fann-,
          \textbf{fund-}, \textbf{fund-}.
    \item
          bregða `hasten‌'的第二基本元可以像第二变位法一样先脱去词尾辅音再变成长元音,得到不规则的brá-,且有brá, brátt, brá的不规则变形。
\end{enumerate}

以规则动词springa为例,展示第三强变位法。

主动态如下:

\begin{longtable}{llll}
    \toprule
    三类强动词 & 直陈       & 虚拟     & 祈使     \\
    \midrule
    \endhead
    \bottomrule
    \endfoot
    单数现在时 &            &          &          \\
    1          & spring     & springa  &          \\
    2          & springr    & springir & spring   \\
    3          & springr    & springi  &          \\
    复数现在时 &            &          &          \\
    1          & springum   & springim & springum \\
    2          & springið   & springið & springið \\
    3          & springa    & springi  &          \\
    单数过去时 &            &          &          \\
    1          & sprakk     & sprynga  &          \\
    2          & sprakkt    & spryngir &          \\
    3          & sprakk     & spryngi  &          \\
    复数过去时 &            &          &          \\
    1          & sprungum   & spryngim &          \\
    2          & sprunguð   & spryngið &          \\
    3          & sprungu    & spryngi  &          \\
    不定式     & springa    &          &          \\
    现在分词   & springandi &          &          \\
    过去分词   & sprunginn  &          &          \\
\end{longtable}

中动态如下:

\begin{longtable}{llll}
    \toprule
    三类强动词 & 直陈         & 虚拟       & 祈使       \\
    \midrule
    \endhead
    \bottomrule
    \endfoot
    单数现在时 &              &            &            \\
    1          & springumk    & springumk  &            \\
    2          & springsk     & springisk  & springsk   \\
    3          & springsk     & springisk  &            \\
    复数现在时 &              &            &            \\
    1          & springumsk   & springimsk & springumsk \\
    2          & springizk    & springizk  & springizk  \\
    3          & springask    & springisk  &            \\
    单数过去时 &              &            &            \\
    1          & sprungumk    & spryngumk  &            \\
    2          & sprakkzk     & spryngisk  &            \\
    3          & sprakksk     & spryngisk  &            \\
    复数过去时 &              &            &            \\
    1          & sprungumsk   & spryngimsk &            \\
    2          & sprunguzk    & spryngizk  &            \\
    3          & sprungusk    & spryngisk  &            \\
    不定式     & springask    &            &            \\
    现在分词   & springandisk &            &            \\
    过去分词   & sprungizk    &            &            \\
\end{longtable}

\subsection{第四强变位法}\label{第四强变位法}

第四类强动词数量不多,它们的不定式词干以响音l, m, r, n结尾,词根的元音是e,其元音交替模式是:

\begin{center}
    \textbf{e --- a --- á(æ) --- o}
\end{center}


一般来说,过去分词词根的元音是o,极少数情况下是u\footnote{PGmc.过去分词的元音曾经是*u,但按照音变规律*u受到后面词尾中*a的作用应该规则地下降为o(这个音变叫作a-Umlaut). nema在一些古挪威语的文献中也有过去分词nominn的形式,但在冰岛语中似乎恢复了古老的*u. 除此之外唯一常见的过去分词元音是u的四类强动词是svima `swim',其中词干的i是有e下降所得。}。比较下列的两个动词:

\begin{quote}
    bera `carry‌',bera --- bar --- báru --- borinn ------ 分词元音是o

    nema `take‌',nema --- nam --- námu --- numinn ------ 分词元音是u
\end{quote}

四类强动词中有一个常见动词略不规则:koma `come' < PGmc.
*kwemaną, 它的直陈式变位如下:

\begin{longtable}{lll}
    \toprule
    直陈式 & 现在       & 过去          \\
    \midrule
    \endhead
    \bottomrule
    \endfoot
    1单    & køm, kem   & kom, kvam     \\
    2单    & kømr, kemr & komt, kvamt   \\
    3单    & kømr, kemr & kom, kvam     \\
    1复    & komum      & kómum, kvámum \\
    2复    & komið      & kómuð, kvámuð \\
    3复    & koma       & kómu, kvámu   \\
\end{longtable}

也就是说,koma的四个基本元可以是:(1)kom- (2)kom-/kvam- (3) kóm-/kvám (4)kom-. 现在时中ø和e的交替是古诺尔斯语,特别是古冰岛语的常见现象,没有完全的规则能解释其发生的条件。

以规则动词bera为例,变位如下:

主动态:

\begin{longtable}{llll}
    \toprule
    四类强动词 & 直陈    & 虚拟  & 祈使  \\
    \midrule
    \endhead
    \bottomrule
    \endfoot
    单数现在时 &         &       &       \\
    1          & ber     & bera  &       \\
    2          & berr    & berir & ber   \\
    3          & berr    & beri  &       \\
    复数现在时 &         &       &       \\
    1          & berum   & berim & berum \\
    2          & berið   & berið & berið \\
    3          & bera    & beri  &       \\
    单数过去时 &         &       &       \\
    1          & bar     & bæra  &       \\
    2          & bart    & bærir &       \\
    3          & bar     & bæri  &       \\
    复数过去时 &         &       &       \\
    1          & bárum   & bærim &       \\
    2          & báruð   & bærið &       \\
    3          & báru    & bæri  &       \\
    不定式     & bera    &       &       \\
    现在分词   & berandi &       &       \\
    过去分词   & borinn  &       &       \\
\end{longtable}

中动态:

\begin{longtable}{llll}
    \toprule
    四类强动词 & 直陈      & 虚拟    & 祈使    \\
    \midrule
    \endhead
    \bottomrule
    \endfoot
    单数现在时 &           &         &         \\
    1          & berumk    & berumk  &         \\
    2          & bersk     & berisk  & bersk   \\
    3          & bersk     & berisk  &         \\
    复数现在时 &           &         &         \\
    1          & berumsk   & berimsk & berumsk \\
    2          & berizk    & berizk  & berizk  \\
    3          & berask    & berisk  &         \\
    单数过去时 &           &         &         \\
    1          & bárumk    & bærumk  &         \\
    2          & barzk     & bærisk  &         \\
    3          & barsk     & bærisk  &         \\
    复数过去时 &           &         &         \\
    1          & bárumsk   & bærimsk &         \\
    2          & báruzk    & bærizk  &         \\
    3          & bárusk    & bærisk  &         \\
    不定式     & berask    &         &         \\
    现在分词   & berandisk &         &         \\
    过去分词   & borizk    &         &         \\
\end{longtable}

\subsection{第五强变位法}\label{第五强变位法}

第五强变位法的元音交替模式和第四强变位法基本相同,它们的区别在于词干末尾的辅音,对于除l, r, m, n以外的单辅音结尾的动词要按第五变位法变位。另外第五强变位法过去分词词根的元音一般是e.

\begin{center}
    \textbf{e --- a --- á(æ) --- e}
\end{center}


五类强动词中包括许多稍不规则的情况:

\subsubsection{-j-词干动词}

原始日耳曼语中的一部分派生动词在词根后插入*-ye-构成现在时,这个后缀在古诺尔斯语中就表现为词干末尾的-j-. 通常,这些派生动词都是第一类弱动词(参见\ref{第一弱变位法}),但有极少数是强动词。\textbf{-j-只在现在时中出现},这使得这些强动词的现在时变位和弱动词一致,因此它们被称为“-j-词干强动词”(Strong verb with \textit{j}-present)或是“弱现在时的强动词”(Strong verb with weak present)。在我们已经介绍过的强动词类别中,只有第一和第二类偶尔包含-j-词干动词,但这些动词绝大多数时候都按弱动词变位。

第五类强动词中有许多常用的-j-词干动词,在现在时变位中,-j-使得元音e > i. 过去时中,-j-脱落,因此它们的过去时都是规则的。常见的例子是biðja `bid, ask‌', 词干bið-j-;sitja `sit‌', 词干sit-j-,它们的直陈式变形类似于:

\begin{longtable}{lll}
    \toprule
    直陈式 & 现在   & 过去  \\
    \midrule
    \endhead
    \bottomrule
    \endfoot
    1单    & sit    & sat   \\
    2单    & sitr   & sazt  \\
    3单    & sitr   & sat   \\
    1复    & sitjum & sátum \\
    2复    & sitið  & sátuð \\
    3复    & sitja  & sátu  \\
\end{longtable}

上表展示了词干末尾的-j-接续辅音和不同类型的元音时的情况,事实上,这里完全符合\ref{半元音的保持性}的规则,读者不难推断出虚拟式以及中动态的情况。

liggja `lie, recline‌'和þiggja `accept, receive‌'这两个动词的词干含-j-, 且辅音是双写辅音,它们的第一基本元的词干是ligg-j-, þigg-j-;单数过去式的词干末尾的双辅音缩短为单辅音,在古诺尔斯语中脱落,同时词干中的元音变为长音(第二基本元lá-, þá-)。第三、第四基本元保留单辅音,如lág-, þág-; leg-, þeg-.\footnote{这些词本来的形式就如*ligja,-g在i前发生延长为双辅音发生在原始诺尔斯语晚期,一些i-词干名词中也有这样的现象,如bekkr. 由于-j-只在现在时中出现,辅音延长只影响了现在时,可以认为其过去时基本是规则的。另外,这个音变没有发生完全,在一些挪威方言中也有不双写g的现在时形式存在。}
þiggja的直陈式变位如下:

\begin{longtable}{lll}
    \toprule
    直陈式 & 现在    & 过去  \\
    \midrule
    \endhead
    \bottomrule
    \endfoot
    1单    & þigg    & þá    \\
    2单    & þiggr   & þátt  \\
    3单    & þiggr   & þá    \\
    1复    & þiggjum & þágum \\
    2复    & þiggið  & þáguð \\
    3复    & þiggja  & þágu  \\
\end{longtable}

\subsubsection{以-g结尾的动词}

常见动词vega `weigh‌' (词干veg-)的单数过去式脱去g,并将a变为长元音,如vá, vátt,这类似于第一、第二强变位法中的情况,以及上述的liggja/þiggja的单数过去式。
\begin{longtable}{lll}
    \toprule
    直陈式 & 现在  & 过去  \\
    \midrule
    \endhead
    \bottomrule
    \endfoot
    1单    & veg   & vá    \\
    2单    & vegr  & vátt  \\
    3单    & vegr  & vá    \\
    1复    & vegum & vágum \\
    2复    & vegið & váguð \\
    3复    & vega  & vágu  \\
\end{longtable}

vega有两个意思,作为“称重”时按第五变位法变位(\textless PGmc.
*weganą),另一个意思“杀;战斗”继承于PGmc. *wiganą
`fight',这本来是一个第一变位法动词。两个原始日耳曼动词在古诺尔斯语中合并了,但哥特语中仍保留了两个分别的形式(weihan---wigan)。vega在古诺尔斯语中只按第五变位法变位,但其形式可能受到了第一变位法的影响。

原始日耳曼语中以-g结尾的五类强动词只有两个,另一个动词*treganą `suffer; grieve'在古诺尔斯语中变成了弱动词,因此缺少对以-g结尾动词变位规律的确认。读者可以认为vega是一个不规则动词。


\subsubsection{不规则动词}

这些动词形式比较不规则。它们有时被归类到第四类强动词中,但在历史上是第五类强动词。

\begin{enumerate}
    \item
          常见动词sofa `sleep‌' < PGmc. swefaną的变位类似于koma:

          \begin{longtable}{lll}
              \toprule
              直陈式 & 现在       & 过去   \\
              \midrule
              \endhead
              \bottomrule
              \endfoot
              1单    & søf, sef   & svaf   \\
              2单    & søfr, sefr & svaft  \\
              3单    & søfr, sefr & svaf   \\
              1复    & sofum      & sváfum \\
              2复    & sofið      & sváfuð \\
              3复    & sofa       & sváfu  \\
          \end{longtable}

    \item
          vefa `weave' < PGmc. *webaną的过去式变位也和koma类似,但也可以按规则方法变形:

          \begin{longtable}{lll}
              \toprule
              直陈式 & 现在  & 过去        \\
              \midrule
              \endhead
              \bottomrule
              \endfoot
              1单    & vef   & óf, vaf     \\
              2单    & vefr  & óft, vaft   \\
              3单    & vefr  & óf, vaf     \\
              1复    & vefum & ófum, váfum \\
              2复    & vefið & ófuð, váfuð \\
              3复    & vefa  & ófu, váfu   \\
          \end{longtable}
    \item
          troða
          `tread‌'的不定式和过去分词的词干都是o而不是e,但它的过去式变位是规则的:

          \begin{longtable}{lll}
              \toprule
              直陈式 & 现在   & 过去   \\
              \midrule
              \endhead
              \bottomrule
              \endfoot
              1单    & trøð   & trað   \\
              2单    & trøðr  & tratt  \\
              3单    & trøðr  & trað   \\
              1复    & troðum & tráðum \\
              2复    & troðið & tráðuð \\
              3复    & troða  & tráðu  \\
          \end{longtable}

    \item
          fregna `ask'和以-g结尾的动词类似,-n-是一个遗留的现在时中后缀,以至于单数现在时的-r词尾消失了:
          %   to be done: check it
          \begin{longtable}{lll}
              \toprule
              直陈式 & 现在    & 过去   \\
              \midrule
              \endhead
              \bottomrule
              \endfoot
              1单    & fregn   & frá    \\
              2单    & fregn   & frátt  \\
              3单    & fregn   & frá    \\
              1复    & fregnum & frágum \\
              2复    & fregnið & fráguð \\
              3复    & fregna  & frágu  \\
          \end{longtable}

    \item
          唯一一个动词sjá `see‌'高度不规则,现在时由sé-, sjá-构成,过去时由sá-构成,过去分词的词干为sé-,完整的变位如下:

          \begin{longtable}{llll}
              \toprule
              sjá        & 直陈       & 虚拟            & 祈使       \\
              \midrule
              \endhead
              \bottomrule
              \endfoot
              单数现在时 &            &                 &            \\
              1          & sé         & sé              &            \\
              2          & sér        & sér             & sé         \\
              3          & sér        & sé, sjái, sjáir &            \\
              复数现在时 &            &                 &            \\
              1          & sjám       & sém             & sjám       \\
              2          & séð, sjáið & séð             & séð, sjáið \\
              3          & sjá        & sé              &            \\
              单数过去时 &            &                 &            \\
              1          & sá         & sæa             &            \\
              2          & sátt       & sæir            &            \\
              3          & sá         & sæi             &            \\
              复数过去时 &            &                 &            \\
              1          & sám        & sæim            &            \\
              2          & sáuð       & sæið            &            \\
              3          & sá, sáu    & sæi             &            \\
              不定式     & sjá        &                 &            \\
              现在分词   & sjándi     &                 &            \\
              过去分词   & sénn       &                 &            \\
          \end{longtable}
\end{enumerate}

动词reka `drive‌'代表了典型的五类强动词变位,其主动态如下:

\begin{longtable}{llll}
    \toprule
    五类强动词 & 直陈    & 虚拟  & 祈使  \\
    \midrule
    \endhead
    \bottomrule
    \endfoot
    单数现在时 &         &       &       \\
    1          & rek     & reka  &       \\
    2          & rekr    & rekir & rek   \\
    3          & rekr    & reki  &       \\
    复数现在时 &         &       &       \\
    1          & rekum   & rekim & rekum \\
    2          & rekið   & rekið & rekið \\
    3          & reka    & reki  &       \\
    单数过去时 &         &       &       \\
    1          & rak     & ræka  &       \\
    2          & rakt    & rækir &       \\
    3          & rak     & ræki  &       \\
    复数过去时 &         &       &       \\
    1          & rákum   & rækim &       \\
    2          & rákuð   & rækið &       \\
    3          & ráku    & ræki  &       \\
    不定式     & reka    &       &       \\
    现在分词   & rekandi &       &       \\
    过去分词   & rekinn  &       &       \\
\end{longtable}

中动态如下:

\begin{longtable}{llll}
    \toprule
    五类强动词 & 直陈      & 虚拟    & 祈使    \\
    \midrule
    \endhead
    \bottomrule
    \endfoot
    单数现在时 &           &         &         \\
    1          & rekumk    & rekumk  &         \\
    2          & reksk     & rekisk  & reksk   \\
    3          & reksk     & rekisk  &         \\
    复数现在时 &           &         &         \\
    1          & rekumsk   & rekimsk & rekumsk \\
    2          & rekizk    & rekizk  & rekizk  \\
    3          & rekask    & rekisk  &         \\
    单数过去时 &           &         &         \\
    1          & rákumk    & rækumk  &         \\
    2          & rakzk     & rækisk  &         \\
    3          & raksk     & rækisk  &         \\
    复数过去时 &           &         &         \\
    1          & rákumsk   & rækimsk &         \\
    2          & rákuzk    & rækizk  &         \\
    3          & rákusk    & rækisk  &         \\
    不定式     & rekask    &         &         \\
    现在分词   & rekandisk &         &         \\
    过去分词   & rekizk    &         &         \\
\end{longtable}

\subsection{第六强变位法}\label{第六强变位法}

第六类强动词词根元音为a,元音交替模式如下:
\begin{center}
    \textbf{a(e) --- ó --- ó(œ) -- a, e}
\end{center}


过去分词的元音一般是a,如果词尾的辅音是软腭音k或g,a会被前移至e。单数现在时中,a总发生i-变异变为e,因此fara `go‌'要变为fer, ferr。

由于直陈过去式中都是圆唇元音,因此总是导致v的脱落。例如vaxa `grow, wax‌'的整个过去式词干都是óx- , 同理有vaða `wade‌'> óð-.

六类强动词中也有一些特殊情况:


\subsubsection{以-g结尾的动词}

常见动词draga `drag‌'的单数过去式脱去了-g,由于元音变成了ó ,如dró,
drótt,因此看起来和二类动词的变形很相像。

还有一些动词历史上曾有词尾-g(来自于-h),但是后来在现在时中脱落了,这类动词包括klá `scratch', slá `slay', flá `flay', þvá `wash'和hlæja
`laugh‌',其中带有长音á的动词都模仿slá. slá和hlæja的直陈式变位如下:

\begin{longtable}{llllll}
    \toprule
    直陈式 & 现在  & 过去   &  & 现在   & 过去   \\
    \midrule
    \endhead
    \bottomrule
    \endfoot
    1单    & slæ   & sló    &  & hlæ    & hló    \\
    2单    & slær  & slótt  &  & hlær   & hlótt  \\
    3单    & slær  & sló    &  & hlær   & hló    \\
    1复    & slám  & slógum &  & hlæjum & hlógum \\
    2复    & sláið & slóguð &  & hlæið  & hlóguð \\
    3复    & slá   & slógu  &  & hlæja  & hlógu  \\
\end{longtable}

它们的过去分词都有-g,且受到-g的影响变成sleginn; hleginn.

注意长音动词接续元音开头的词尾时的元音缩合现象。


\subsubsection{-j-词干动词}

-j-词干动词中-j-的出现导致整个现在时变位全部受i-音变影响。这些动词包括sverja `swear', hefja `heave; raise', skepja `shape'以及kefja
`sink'\footnote{kvefja通常是一类弱动词,但也可按照六类强动词变位。}等。动词deyja `die'也属于-j-词干动词,但它的复数过去式与同类略有不同,因为它的词干尾没有辅音,使得ó直接与词尾接触,触发元音缩合,故其复数过去式为dóm < †dóum, dóð, dó. 以相同方式变位的还是一个不太常见的动词geyja `bark'. 参考sverja和deyja的变位(注意sverja词干中的-v-在过去式中脱落):

\begin{longtable}{llllll}
    \toprule
    直陈式 & 现在    & 过去  &  & 现在   & 过去 \\
    \midrule
    \endhead
    \bottomrule
    \endfoot
    1单    & sver    & sór   &  & dey    & dó   \\
    2单    & sverr   & sórt  &  & deyr   & dótt \\
    3单    & sverr   & sór   &  & deyr   & dó   \\
    1复    & sverjum & sórum &  & deyjum & dóm  \\
    2复    & sverið  & sóruð &  & deyið  & dóð  \\
    3复    & sverja  & sóru  &  & deyja  & dó   \\
\end{longtable}

\subsubsection{不规则动词standa}

非常常见的动词standa `stand‌'在除了现在时以外的地方脱去-n-,d以异构ð保留,其四个基本元为stand-, stóð-, stóð-, stað-.


规则动词fara的完整变格如下:

主动态:


\begin{longtable}{llll}
    \toprule
    六类强动词 & 直陈    & 虚拟  & 祈使  \\
    \midrule
    \endhead
    \bottomrule
    \endfoot
    单数现在时 &         &       &       \\
    1          & fer     & fara  &       \\
    2          & ferr    & farir & far   \\
    3          & ferr    & fari  &       \\
    复数现在时 &         &       &       \\
    1          & fǫrum   & farim & fǫrum \\
    2          & farið   & farið & farið \\
    3          & fara    & fari  &       \\
    单数过去时 &         &       &       \\
    1          & fór     & fœra  &       \\
    2          & fórt    & fœrir &       \\
    3          & fór     & fœri  &       \\
    复数过去时 &         &       &       \\
    1          & fórum   & fœrim &       \\
    2          & fóruð   & fœrið &       \\
    3          & fóru    & fœri  &       \\
    不定式     & fara    &       &       \\
    现在分词   & farandi &       &       \\
    过去分词   & farinn  &       &       \\
\end{longtable}

中动态如下所示:

\begin{longtable}{llll}
    \toprule
    六类强动词 & 直陈      & 虚拟    & 祈使    \\
    \midrule
    \endhead
    \bottomrule
    \endfoot
    单数现在时 &           &         &         \\
    1          & fǫrumk    & fǫrumk  &         \\
    2          & fersk     & farisk  & farsk   \\
    3          & fersk     & farisk  &         \\
    复数现在时 &           &         &         \\
    1          & fǫrumsk   & farimsk & fǫrumsk \\
    2          & farizk    & farizk  & farizk  \\
    3          & farask    & farisk  &         \\
    单数过去时 &           &         &         \\
    1          & fórumk    & fœrumk  &         \\
    2          & fórzk     & fœrisk  &         \\
    3          & fórsk     & fœrisk  &         \\
    复数过去时 &           &         &         \\
    1          & fórumsk   & fœrimsk &         \\
    2          & fóruzk    & fœrizk  &         \\
    3          & fórusk    & fœrisk  &         \\
    不定式     & farask    &         &         \\
    现在分词   & farandisk &         &         \\
    过去分词   & farizk    &         &         \\
\end{longtable}

\subsection{第七强变位法}\label{第七强变位法}

第七类强动词的定义是元音交替不符合上述六类的动词,在古诺尔斯语中形成了下列五大元音交替模式,它们和对应的五类变位法的元音交替有一定的联系:

\begin{longtable}{lllll}
    \toprule
    类别                 & 不定式 & 单数过去 & 复数过去 & 过去分词 \\
    \midrule
    \endhead
    \bottomrule
    \endfoot
    1. heita `be called‌' & heita  & hét      & hétum    & heitinn  \\
    2a. auka `increase‌'  & auka   & jók      & jókum    & aukinn   \\
    2b. búa `inhabit‌'    & búa    & bjó      & bjuggum  & búinn    \\
    3. falla `fall‌'      & falla  & fell     & fellum   & fallinn  \\
    4. láta `let‌'        & láta   & lét      & létum    & látinn   \\
    5. blóta `offer‌'     & blóta  & blét     & blétum   & blótinn  \\
\end{longtable}

七类强动词的基本特点是第一和第四基本元的词根元音一致,第二和第三也一致(2b除外)。

\subsubsection{第1类动词}

第1类动词的不定式词根元音是ei,一共有三个:heita `be called', leika
`play', sveipa `sweep'

\begin{longtable}{llll}
    \toprule
    不定式 & 单数过去 & 复数过去 & 过去分词 \\
    \midrule
    \endhead
    \bottomrule
    \endfoot
    heita  & hét      & hétum    & heitinn  \\
    leika  & lék      & lékum    & leikinn  \\
    sveipa & sveip    & svipum   & sveipinn \\
\end{longtable}

这三个动词中最典型的是leika,它只按照第七强变位法变位;

heita的\textbf{单数现在时}词干可以按强动词或弱动词变位,这取决它的意思。当heita表示“A名叫B”的时候,要用弱动词词尾,即按照三类弱动词变形(见\ref{第三弱变位法}):heiti, heitir, heitir. 取其它意思时,都按强动词变形:heit, heitr, heitr;

sveipa经常按弱动词变位,其残存的强动词变化形式和第一类强动词发生了混淆。

\subsubsection{第2类动词}

第2类动词的过去式词干中有双元音jó为特征,其中的子类a的不定式元音是au,其他归为b类,这类动词包括:auka `increase', ausa `sprinkle', hlaupa `jump', búa `inhabit', hǫggva `hew‌'等。

\begin{longtable}{lllll}
    \toprule
    不定式 & 单数现在   & 单数过去 & 复数过去                 & 过去分词 \\
    \midrule
    \endhead
    \bottomrule
    \endfoot
    auka   & eyk        & jók      & jókum, jukum             & aukinn   \\
    ausa   & eys        & jós      & jósum, jusum             & ausinn   \\
    hlaupa & hleyp      & hljóp    & hljópum, hljupum         & hlaupinn \\
    búa    & bý         & bjó      & bjoggum, bjuggum, buggum & búinn    \\
    hǫggva & høgg, hegg & hjó      & hjoggum, hjuggum         & hǫggvinn \\
\end{longtable}

复数过去式中,ju和jó经常交替,ju是比较后期的形式。在búa中甚至出现了ju > u的形式,其复数过去式中的g可能是从一个同根弱动词byggva
`reside'中借来的。

hǫggva的单数现在时值得注意,va-不定式触发了u-变异,因此其现在时词干应该是hagg-v-,但在单数现在时中i-变异的形式是ø,而非预期的e,不过它又在后来的形式中规则化了。

\subsubsection{第3类动词}

第3类动词相对比较规则,且都非常常见,这主要包括blanda `mix', ganga
`walk', hanga `hang', falla `fall', halda `hold', falda `fold', fá `get',它们的变形都发生了类似三类强动词的音变(回顾\ref{第三强变位法}的音变规则1,2):

\begin{longtable}{lllll}
    \toprule
    不定式 & 单数现在 & 单数过去 & 复数过去       & 过去分词         \\
    \midrule
    \endhead
    \bottomrule
    \endfoot
    blanda & blend    & blett    & blendum        & blandinn         \\
    ganga  & geng     & gekk     & gengum, gingum & genginn, gingum  \\
    hanga  & heng     & hekk     & hengum         & hanginn          \\
    falla  & fell     & fell     & fell           & fallin           \\
    halda  & held     & helt     & heldum         & haldinn          \\
    falda  & feld     & felt     & feldum         & faldinn          \\
    fá     & fæ       & fekk     & fengum, fingum & fenginn, finginn \\
\end{longtable}

fá的变位和ganga非常一致,它曾经的形式是*fanhaną,词尾的-g也发生了脱落,但在复数过去式中恢复。注意这两个动词复数过去式、过去分词的词根元音发生了抬升,一些很古老的文献中记录了i,但后来变成了e.

\subsubsection{第4、5类动词}

这两类动词形式规则,它们的区别仅在于不定式的元音是á还是ó,常见动词包括:blása `blow', gráta `weep', láta `let', ráða `advise', blóta `offer‌'.

以láta为例,展示规则的七类强动词的完整变位。其主动态为:

\begin{longtable}{llll}
    \toprule
    七类强动词 & 直陈    & 虚拟  & 祈使  \\
    \midrule
    \endhead
    \bottomrule
    \endfoot
    单数现在时 &         &       &       \\
    1          & læt     & láta  &       \\
    2          & lætr    & látir & lát   \\
    3          & lætr    & láti  &       \\
    复数现在时 &         &       &       \\
    1          & látum   & látim & látum \\
    2          & látið   & látið & látið \\
    3          & láta    & láti  &       \\
    单数过去时 &         &       &       \\
    1          & lét     & léta  &       \\
    2          & lézt    & létir &       \\
    3          & lét     & léti  &       \\
    复数过去时 &         &       &       \\
    1          & létum   & létim &       \\
    2          & létuð   & létið &       \\
    3          & létu    & léti  &       \\
    不定式     & láta    &       &       \\
    现在分词   & látandi &       &       \\
    过去分词   & látinn  &       &       \\
\end{longtable}

中动态为:

\begin{longtable}{llll}
    \toprule
    七类强动词 & 直陈      & 虚拟    & 祈使    \\
    \midrule
    \endhead
    \bottomrule
    \endfoot
    单数现在时 &           &         &         \\
    1          & látumk    & látumkm &         \\
    2          & Læzk      & Látisk  & lázk    \\
    3          & læzk      & látisk  &         \\
    复数现在时 &           &         &         \\
    1          & látumsk   & látimsk & látumsk \\
    2          & Látizk    & látizk  & látizk  \\
    3          & Látask    & látisk  &         \\
    单数过去时 &           &         &         \\
    1          & Létumk    & létumk  &         \\
    2          & Lézk      & létisk  &         \\
    3          & Lézk      & létisk  &         \\
    复数过去时 &           &         &         \\
    1          & létumsk   & létimsk &         \\
    2          & létuzk    & létizk  &         \\
    3          & létusk    & létisk  &         \\
    不定式     & látask    &         &         \\
    现在分词   & látandisk &         &         \\
    过去分词   & látizk    &         &         \\
\end{longtable}

\subsubsection{复音动词}

这类动词继承了原始印欧语的特点。日耳曼语的过去式主要是由原始印欧语的完成时演变过来的,完成时的构词法是在词根前添加一个重复成分(词根的辅音+元音),在一些古典语言中有清楚的痕迹(连字符是为了更清晰地展示重复成分):

\begin{quote}
    梵 语:现在时bharati `bear, carry' → 完成时 \textbf{ba}-bhāra

    希腊语:现在时leípō `leave' → 完成时\textbf{lé}-loipā
\end{quote}

在日耳曼语中,一些动词的过去式也有这样的特点,称为复音动词(Reduplicated verb),但复音动词在不同的语言中可能被简化,也可能发生进一步的音变。比较下面几个比较明显的复音动词的基本元:

\begin{quote}
    哥特语:现在时hláupan 单数过去\textbf{haí}-hláup

    古英语:现在时hātan 单数过去hēt < \textbf{he}-ht
\end{quote}

但是,古诺尔斯语中这两个动词的同源词hlaupa和heita都失去了重复成分,只有三个动词róa `row‌', sá `sow‌'和snúa `turn‌'还保有复音动词的特征,但它们都已经发生了比较明显的音变。这些动词的过去式构成方法是用-ø-/-e-替换词根中的元音,并添加-r构成词干。其单数过去式的人称词尾是-a, -ir, -i词尾,复数过去式的人称词尾是-um, -uð, -u词尾(整个过去式实则添加的是弱动词词尾)。这些动词因其过去式的形态有时被称为ra-动词(现代冰岛语中叫作ri-动词),它们的变位如下所示:

\begin{longtable}{llll}
    \toprule
    不定式     & róa          & sá           & snúa           \\
    \midrule
    \endhead
    \bottomrule
    \endfoot
    单数现在时 &              &              &                \\
    1          & rœ           & sæ           & sný            \\
    2          & rœr          & sær          & snýr           \\
    3          & rœr          & sær          & snýr           \\
    复数现在时 &              &              &                \\
    1          & ró(u)m       & sám          & snúm           \\
    2          & róið         & sáið         & snúið          \\
    3          & róa          & sá           & snúa           \\
    单数过去时 &              &              &                \\
    1          & røra, rera   & søra, sera   & snøra, snera   \\
    2          & rørir, rerir & sørir, serir & snørir, snerir \\
    3          & røri, reri   & søri, seri   & snøri, sneri   \\
    复数过去时 &              &              &                \\
    1          & rørum, rerum & sørum, serum & snørum, snerum \\
    2          & røruð, reruð & søruð, seruð & snøruð, sneruð \\
    3          & røru, reru   & søru, seru   & snøru, sneru   \\
    过去分词   & róinn        & sáinn        & snúinn         \\
\end{longtable}

形似的动词 gróa `grow‌'和gnúa `rub‌'本来不是复音动词,但它们的变形受类比的影响与róa和snúa一致。

\section{弱动词的变位法}\label{弱动词的变位法}

区别于强动词用元音变换指示过去式,弱动词的过去时标记是添加在词干和人称词尾间的-ði-,例如:

\begin{longtable}{llll}
    \toprule
    词干      & 过去时标记 & 人称标记 & 完整变形              \\
    \midrule
    \endhead
    \bottomrule
    \endfoot
    sigl-i/j- & -ði-       & -a       & siglða `I sailed‌'     \\
    kall-a-   & -ði-       & -r       & kallaðir `you called‌' \\
    lif-i-    & -ði-       & -um      & lifðum `we lived‌'     \\
\end{longtable}

弱动词的词干和强动词稍有不同,我们在后面会具体的介绍。在此之前,我们先熟悉一下变位中常常触发的一些重要的音变:

\begin{enumerate}
    \item
          i-的删去。根据音变规律\ref{元音的音变},后缀ði中的i在元音开头的词尾前前脱落,同时词干元音i或-i/j-在-ði-前脱落:

          \begin{quote}
              kall-a- + -ði- + -a > kallaða

              lif-i- + -ði- + -u > lifðu
          \end{quote}
    \item
          u-变异。在u之前,非重读的a变成u,重读的a变成ǫ,在许多弱动词中会产生连锁反应:

          \begin{quote}
              kall-a- + -ði- + um > *kalluðum > kǫlluðum

              tal-i/j- + -ði- + -uð > *talðuð > tǫlðuð
          \end{quote}

    \item
          -ð + ð- > -dd-:

          \begin{quote}
              beiða `ask' > beidda

              eyða `waste' > eydda
          \end{quote}

    \item
          -t, s + ð- > -tt, st:

          \begin{quote}
              flytja `move‌' > flutta

              sæta `undergo‌' > sætta
          \end{quote}

    \item
          -p, k, f, l + ð- > -pt, kt, ft,
          lt,但这个变化不总是发生:

          \begin{quote}
              þurfa `need‌' > þurfta

              但hafa `have‌' > hafða不变

              mæla `speak‌' > mælta

              但也有 vilja `want‌' > vilda
          \end{quote}

    \item
          辅音简化:-Cdd > -Cd以及-Ctt > -Ct. 这表明,当上述的规则导致dd/tt出现时,如果它们前面还有一个辅音,则双辅音简化为单辅音:

          \begin{quote}
              senda `send‌' > *sendda > senda

              skipta `shift‌' > *skiptta > skipta
          \end{quote}
\end{enumerate}

弱动词共可分为三类,这是根据词干元音的特征来分的。下面的三个动词krefja
`demand‌', kalla `call‌'和 vaka `wake‌'分属三类弱动词:

\begin{longtable}{lllllll}
    \toprule
    类别 & 不定式          & 三单现在时 & 三单过去时 & 三复过去时 & 三单过去虚拟式 & 过去分词 \\
    \midrule
    \endhead
    \bottomrule
    \endfoot
    I    & krefja `demand‌' & krefr      & krafði     & krǫfðu     & krefði         & krafðr   \\
    II   & kalla `call‌'    & kallar     & kallaði    & kǫlluðu    & kallaði        & kallaðr  \\
    III  & vaka `wake‌'     & vakir      & vakþi      & vǫkþu      & vekþi          & vakaðr   \\
\end{longtable}

弱动词的分类是一个值得探讨的问题。从共时层面看,第三人称现在时的形态(-r, -ar, -ir)可以用于决定弱动词的种类。但有时这和历史演变的情况是矛盾的,不同的学者可能根据不同的标准划分词类。譬如heyra `hear‌'在过去是第一类弱动词,但古诺尔斯语中它的第三人称现在时是heyrir,应当算作第三类弱动词。因此heyra既可以归为第一类,也可归为第三类弱动词,这视分类者的习惯而定。为了便于理解弱动词的来历,本书仍采用历史语言学的划分方式。

读者可以从此表中发现,弱动词的形式间不存在元音交替的现象,只有元音变异。不同类别的动词发生元音变异的形态不完全相同,见下文详述。

弱动词的变位可以由3个基本元确定,即不定式,单数过去式和过去分词。与强动词不同,这三个基本元在很大程度上是可以根据弱动词的类型互相推断的。

\subsection{弱动词的主动词尾}\label{弱动词的主动词尾}

弱动词的主动词尾如下所示:

\begin{longtable}{llll}
    \toprule
    弱动词     & 直陈          & 虚拟 & 祈使 \\
    \midrule
    \endhead
    \bottomrule
    \endfoot
    单数现在时 &               &      &      \\
    1          & -ø / -a / -i  & -a   &      \\
    2          & -r / -ar/ -ir & -ir  & -    \\
    3          & -r / -ar/ -ir & -i   &      \\
    复数现在时 &               &      &      \\
    1          & -um           & -im  & -um  \\
    2          & -ið           & -ið  & -ið  \\
    3          & -a            & -i   &      \\
    单数过去时 &               &      &      \\
    1          & -ða           & -ða  &      \\
    2          & -ðir          & -ðir &      \\
    3          & -ði           & -ði  &      \\
    复数过去时 &               &      &      \\
    1          & -ðum          & -ðim &      \\
    2          & -ðuð          & -ðið &      \\
    3          & -ðu           & -ði  &      \\
\end{longtable}

这张表中的词尾单数现在时的词尾事实上考虑了词干元音,它们事实上都是-ø, -r, -r的变体。弱动词和强动词最大的区别在于单数过去式的词尾变成了-a, -ir, -i.

类似于强动词,弱动词的变形的过程如下所示:

\begin{info}

    所有的现在时形式:取不定式词干,加词尾 (需用动词类别区分词尾 -r / -ar/-ir);

    过去直陈式: 取第二基本元词干,必要时将词干中的元音u-音变,加词尾;

    现在虚拟式:取第二基本元词干,加词尾;过去虚拟式: 取第二基本元词干,将后元音i-变异,加词尾;

    现在分词: 取不定式词干,加词尾;

    过去分词: 取第三基本元,加词尾。
\end{info}

将三个基本元标记为 (1), (2), (3), 变形方式如下:

\begin{longtable}{llll}
    \toprule
    弱动词     & 直陈                  & 虚拟                  & 祈使                 \\
    \midrule
    \endhead
    \bottomrule
    \endfoot
    单数现在时 &                       &                       &                      \\
    1          & (1) + - / -a / -i     & (1) + -a              &                      \\
    2          & (1) + -r / -ar/ -ir   & (1) + -ir             & (1) + -              \\
    3          & (1) + -r / -ar/ -ir   & (1) + -i              &                      \\
    复数现在时 &                       &                       &                      \\
    1          & (1) + (u-变异) + -um  & (1) + -im             & (1) + (u-变异) + -um \\
    2          & (1) + -ið             & (1) + -ið             & (1) + -ið            \\
    3          & (1) + -a              & (1) + -i              &                      \\
    单数过去时 &                       &                       &                      \\
    1          & (2) + -ða             & (2) + (i-变异) + -ða  &                      \\
    2          & (2) + -ðir            & (2) + (i-变异) + -ðir &                      \\
    3          & (2) + -ði             & (2) + (i-变异) + -ði  &                      \\
    复数过去时 &                       &                       &                      \\
    1          & (2) + (u-变异) + -ðum & (2) + (i-变异) + -ðim &                      \\
    2          & (2) + (u-变异) + -ðuð & (2) + (i-变异) + -ðið &                      \\
    3          & (2) + (u-变异) + -ðu  & (2) + (i-变异) + -ði  &                      \\
    不定式     & (1) + -a              &                       &                      \\
    现在分词   & (1) + -andi           &                       &                      \\
    过去分词   & (3) + -ðr             &                       &                      \\
\end{longtable}

\subsection{弱动词的中动词尾}\label{弱动词的中动词尾}

弱动词的中动态的构成和强动词一致,也是在词尾上添加-sk。我们在强动词部分已经介绍了-sk词尾导致的音变,这里不再赘述。

\begin{longtable}{llll}
    \toprule
    弱动词     & 直陈              & 虚拟   & 祈使       \\
    \midrule
    \endhead
    \bottomrule
    \endfoot
    单数现在时 &                   &        &            \\
    1          & -umk              & -umk   &            \\
    2          & -sk / -ask / -isk & -isk   & -sk / -ask \\
    3          & -sk / -ask / -isk & -isk   &            \\
    复数现在时 &                   &        &            \\
    1          & -umsk             & -imsk  & -umsk      \\
    2          & -izk              & -izk   & -izk       \\
    3          & -ask              & -isk   &            \\
    单数过去时 &                   &        &            \\
    1          & -ðumk             & -ðumk  &            \\
    2          & -ðisk             & -ðisk  &            \\
    3          & -ðisk             & -ðisk  &            \\
    复数过去时 &                   &        &            \\
    1          & -ðumsk            & -ðimsk &            \\
    2          & -ðuzk             & -ðizk  &            \\
    3          & -ðusk             & -ðisk  &            \\
\end{longtable}

中动态的变化模式归纳如下:

\begin{longtable}{llll}
    \toprule
    弱动词     & 直陈                    & 虚拟                    & 祈使                   \\
    \midrule
    \endhead
    \bottomrule
    \endfoot
    单数现在时 &                         &                         &                        \\
    1          & (1) + (u-变异) + -umk   & (1) + (u-变异) + -umk   &                        \\
    2          & (1) + -sk / -ask/ -isk  & (1) + -isk              & (1) + -sk / -ask       \\
    3          & (1) + -sk / -ask/ -isk  & (1) + -isk              &                        \\
    复数现在时 &                         &                         &                        \\
    1          & (1) + (u-变异) + -umsk  & (1) + -imsk             & (1) + (u-变异) + -umsk \\
    2          & (1) + -izk              & (1) + -izk              & (1) + -ið              \\
    3          & (1) + -ask              & (1) + -isk              &                        \\
    单数过去时 &                         &                         &                        \\
    1          & (2) + (u-变异) + -ðumk  & (2) + (u-变异) + -ðumk  &                        \\
    2          & (2) + -ðisk             & (2) + (i-变异) + -ðisk  &                        \\
    3          & (2) + -ðisk             & (2) + (i-变异) + -ðisk  &                        \\
    复数过去时 &                         &                         &                        \\
    1          & (2) + (u-变异) + -ðumsk & (2) + (i-变异) + -ðimsk &                        \\
    2          & (2) + (u-变异) + -ðuzk  & (2) + (i-变异) + -ðizk  &                        \\
    3          & (2) + (u-变异) + -ðusk  & (2) + (i-变异) + -ðisk  &                        \\
    不定式     & (1) + -ask              &                         &                        \\
    现在分词   & (1) + -andisk           &                         &                        \\
    过去分词   & (3) + -zk               &                         &                        \\
\end{longtable}

注意,弱动词的过去式词干是一样的,因此没有强动词中第一人称的不规则现象。

\subsection{第一弱变位法}\label{第一弱变位法}

第一类弱动词的词干元音是-i/j-,和ja-词干名词一样,这个元音性质的确定由西弗斯定律支配,即:

\begin{info}
    短词干:词干音节只有一个单辅音+不超过一个辅音/一个双元音或长元音

    长词干:词干音节为单辅音+辅音簇/双元音或长元音+任意数量的辅音

    短词干后,词干元音是j;长词干后,词干元音是i.
\end{info}

正如长短ja-词干名词的变格不同,词干元音完全决定了第一类弱动词的变位方法,因此有些语法也把第一弱变位法进一步分成两类。

试比较下面不同类型的长短词干动词的不定式:

\begin{longtable}{llll}
    \toprule
    词干               & 词干类型             & 词干元音 & 不定式                                      \\
    \midrule
    \endhead
    \bottomrule
    \endfoot
    var-i/j- `defend‌'  & 单元音+单辅音 短词干 & -j-      & verja                                       \\
    sigl-i/j- `sail‌'   & 单元音+辅音簇 长词干 & -i-      & sigla                                       \\
    sát-i/j- `undergo‌' & 长元音+单辅音 长词干 & -i-      & sæta                                        \\
    knú-i/j- `knock'   & 单个长元音 短词干    & -j-      & knýja                                       \\
    þrá-i/j- `desire'  & 单个双元音 短词干    & -j-      & þreyja\footnote{弱动词中有一部分动词发生了á
    > ey的i-变异。但从词源上来说,长元音á的确由au缩合得到,这符合元音变异的性质。}                     \\
\end{longtable}

词干元音-i/j-都触发后元音的i-变异,由于动词的不定式是弱动词的第一基本元,现在时都在第一基本元的基础上添加词尾,因此无论词干长短,弱动词的现在时中一律出现i-变异。

弱动词的第二基本元构成过去式词干,其词干构成和第一基本元(不定式)是密切相关的。具体来说:

\begin{info}
    短词干的过去时词干不发生i-变异;长词干的过去式词干发生i-变异(和现在时一致)。
\end{info}

读者也可以把这个规律理解为短词干动词现在时的词干元音在过去式中脱落(导致i-变异无法发生),长词干动词的词干元音则在过去时中也保留,如下所示:\footnote{这种理解方法只能帮助读者了解共时问题,但它与历史情况恰恰相反。古诺尔斯语的第一类弱动词过去时词干中的i-变异问题,请参考疑难问题。}

\begin{longtable}{lll}
    \toprule
    基础词干  & 现在时词干 & 过去时词干 \\
    \midrule
    \endhead
    \bottomrule
    \endfoot
    var-i/j-  & ver-j-     & var-       \\
    sigl-i/j- & sigl-i-    & sigl-i-    \\
    sát-i/j-  & sæt-i-     & sæt-i-     \\
    knú-i/j-  & kný-j-     & knú-       \\
    þrá-i/j-  & þrey-j-    & þrá-       \\
\end{longtable}

但根据元音省略规则,过去时词干的词干元音总在-ði-前脱落。因此现在时和过去时的词干差别实际上仅在于i-变异是否发生。

一些值得注意的例外是:

\begin{enumerate}
    \item
          如果长音节词干以-k或-g结尾,词干元音-i-在a/u前变为j,这个规律在ja-词干名词中也有体现:

          \begin{quote}
              \begin{tabular}{lrl}
                  lág-i/j- `lower‌' & 现在时词干 & læg-i- + -r > lægir \\
                                   & 但有       & læg-i- + -a > lægja
              \end{tabular}

          \end{quote}
    \item
          某些以-g结尾(但一般不是-k)的短词干动词在现在时词干末尾双写-g:

          \begin{quote}lag-i/j- `lay‌' 现在时词干legg-j- + -r > legg\end{quote}

          比较下列以-g结尾的动词:

          \begin{longtable}{lll}
              \toprule
                         & 短音节   & 长音节    \\
              \midrule
              \endhead
              \bottomrule
              \endfoot
              词干       & lag-i/j- & talg-i/j- \\
              单数现在时 &          &           \\
              1          & legg     & telgi     \\
              2          & leggr    & telgir    \\
              3          & leggr    & telgir    \\
              复数现在时 &          &           \\
              1          & leggjum  & telgjum   \\
              2          & leggið   & telgið    \\
              3          & leggja   & telgja    \\
              单数过去时 &          &           \\
              1          & lagða    & telgða    \\
              2          & lagðir   & telgðir   \\
              3          & lagði    & telgði    \\
              复数过去时 &          &           \\
              1          & lǫgðum   & telgðum   \\
              2          & lǫgðuð   & telgðuð   \\
              3          & lǫgðu    & telgðu    \\
          \end{longtable}

    \item
          不规则动词。部分长词干动词中的i-变异不规则,这主要是受词根元音后面的辅音的历史音变的影响造成的,我们在此不展开讨论。这类动词只有4个,但都不算罕见,其基本元如下所示:

          \begin{longtable}{llll}
              \toprule
              不定式             & 三单现在时  & 三单过去时 & 过去分词 \\
              \midrule
              \endhead
              \bottomrule
              \endfoot
              sœkja/sækja `seek' & sœkir/sækir & sótti      & sóttr    \\
              yrkja `work'       & yrkir       & orti       & ort      \\
              þekkja `know'      & þekkir      & þátti      & þektr    \\
              þykkja `seem'      & þykkir      & þótti      & þóttr    \\
          \end{longtable}
\end{enumerate}

弱动词的第三基本元只用在过去分词中。长短词干动词的过去分词也有i-变异的区分,结果和过去时词干一致,即长词干动词发生i-变异,短词干动词不发生i-变异。短词干动词也可以在词干尾加可选的-i,如var\textbf{i}ðr,长词干则没有额外的-i.

动词 verja, varði, variðr/varðr `defend‌', 词干var-i/j-, 是标准的短词干一类弱动词,主动态如下:

\begin{longtable}{llll}
    \toprule
    一类弱动词 & 直陈     & 虚拟   & 祈使   \\
    \midrule
    \endhead
    \bottomrule
    \endfoot
    (var-i/j-) &          &        &        \\
    单数现在时 &          &        &        \\
    1          & ver      & verja  &        \\
    2          & verr     & verir  & ver    \\
    3          & verr     & veri   &        \\
    复数现在时 &          &        &        \\
    1          & verjum   & verim  & verjum \\
    2          & verið    & verið  & verið  \\
    3          & verja    & veri   &        \\
    单数过去时 &          &        &        \\
    1          & varða    & verða  &        \\
    2          & varðir   & verðir &        \\
    3          & varði    & verði  &        \\
    复数过去时 &          &        &        \\
    1          & vǫrðum   & verðim &        \\
    2          & vǫrðuð   & verðið &        \\
    3          & vǫrðu    & verði  &        \\
    不定式     & verja    &        &        \\
    现在分词   & verjandi &        &        \\
    过去分词   & variðr   &        &        \\
\end{longtable}

中动态如下:

\begin{longtable}{llll}
    \toprule
    一类弱动词 & 直陈       & 虚拟     & 祈使     \\
    \midrule
    \endhead
    \bottomrule
    \endfoot
    (var-i/j-) &            &          &          \\
    单数现在时 &            &          &          \\
    1          & verjumk    & verjumk  &          \\
    2          & versk      & verisk   & versk    \\
    3          & versk      & verisk   &          \\
    复数现在时 &            &          &          \\
    1          & verjumsk   & verimsk  & verjumsk \\
    2          & verizk     & verizk   & verizk   \\
    3          & verjask    & verisk   &          \\
    单数过去时 &            &          &          \\
    1          & vǫrðumk    & verðumk  &          \\
    2          & varðisk    & verðisk  &          \\
    3          & varðisk    & verðisk  &          \\
    复数过去时 &            &          &          \\
    1          & vǫrðumsk   & verðimsk &          \\
    2          & vǫrðuzk    & verðizk  &          \\
    3          & vǫrðusk    & verðisk  &          \\
    不定式     & verjask    &          &          \\
    现在分词   & verjandisk &          &          \\
    过去分词   & varizk     &          &          \\
\end{longtable}

动词fella, felldi, felldr `fell (vt.)‌', 词干 fall-i/j-,
是典型的长音节弱动词,主动态如下:

\begin{longtable}{llll}
    \toprule
    一类弱动词  & 直陈     & 虚拟    & 祈使   \\
    \midrule
    \endhead
    \bottomrule
    \endfoot
    (fall-i/j-) &          &         &        \\
    单数现在时  &          &         &        \\
    1           & felli    & fella   &        \\
    2           & fellir   & fellir  & fell   \\
    3           & fellir   & felli   &        \\
    复数现在时  &          &         &        \\
    1           & fellum   & fellim  & fellum \\
    2           & fellið   & fellið  & fellið \\
    3           & fella    & felli   &        \\
    单数过去时  &          &         &        \\
    1           & fellda   & fellda  &        \\
    2           & felldir  & felldir &        \\
    3           & felldi   & felldi  &        \\
    复数过去时  &          &         &        \\
    1           & felldum  & felldim &        \\
    2           & fellduð  & felldið &        \\
    3           & felldu   & felldi  &        \\
    不定式      & fella    &         &        \\
    现在分词    & fellandi &         &        \\
    过去分词    & felldr   &         &        \\
\end{longtable}

中动态如下:

\begin{longtable}{llll}
    \toprule
    一类弱动词  & 直陈       & 虚拟      & 祈使     \\
    \midrule
    \endhead
    \bottomrule
    \endfoot
    (fall-i/j-) &            &           &          \\
    单数现在时  &            &           &          \\
    1           & fellumk    & fellumk   &          \\
    2           & fellisk    & fellisk   & fellsk   \\
    3           & fellisk    & fellisk   &          \\
    复数现在时  &            &           &          \\
    1           & fellumsk   & fellimsk  & fellumsk \\
    2           & fellizk    & fellizk   & fellizk  \\
    3           & fellask    & fellisk   &          \\
    单数过去时  &            &           &          \\
    1           & felldumk   & felldumk  &          \\
    2           & felldisk   & felldisk  &          \\
    3           & felldisk   & felldisk  &          \\
    复数过去时  &            &           &          \\
    1           & felldumsk  & felldimsk &          \\
    2           & fellduzk   & felldizk  &          \\
    3           & felldusk   & felldisk  &          \\
    不定式      & fellask    &           &          \\
    现在分词    & fellandisk &           &          \\
    过去分词    & fellzk     &           &          \\
\end{longtable}

\subsection{第二弱变位法}\label{第二弱变位法}

第二弱变位法动词的特征是词干元音-a-,例如kall-a- `call‌', kast-a- `cast‌'.
词干元音-a-存在于各个时态中,但在紧随的元音前脱去。词干元音-a不是重读元音,在含有-u-的词尾前变成u,并且进一步引起词根中重读元音的u-变异。

第二弱变位法动词数量很多,但基本上是最规则的词类。它的三个基本元间没有i-变异的现象,词根元音完全一致,只有少数形式中有明显的u-变异。注意:即便是过去虚拟式当中也不发生i-变异,这是和其他所有强弱动词的重大区别。其原因正是词干元音-a-和-ið-结合,使得过去虚拟式的词尾中出现了不能引起i-变异的双元音(参见\ref{强动词的主动词尾}中对i-变异由来的解释)。

二类弱动词中有两类容易引起问题或混淆的:


\subsubsection{-j-词干动词}


有相当一部分动词中词干元音前有-j-,如herja `wage war; harry', bryja
`begin',这些动词仅凭不定式非常无法与短词干的一类弱动词区分开来。这些动词是由名词派生而来,其中的-j-实际上是名词的词干元音(ja-/jō-词干),不过古诺尔斯语中名词的词干元音-j-只在部分形式中出现了,比较herja的原始日耳曼语形式:

\begin{longtable}{llllll}
    \toprule
    ON    & herja    & < & herr `army' & + & -a    \\
    \midrule
    \endhead
    \bottomrule
    \endfoot
    PGmc. & *harjōną & < & *harjaz     & + & *-ōną \\
\end{longtable}

这些动词的-j-是词干的一部分,与一类弱动词有根本区别,参考herja的直陈式变位:

\begin{longtable}{lll}
    \toprule
    直陈式 & 现在   & 过去     \\
    \midrule
    \endhead
    \bottomrule
    \endfoot
    1单    & herja  & herjaða  \\
    2单    & herjar & herjaðir \\
    3单    & herjar & herjaði  \\
    1复    & herjum & herjuðum \\
    2复    & herið  & herjuðuð \\
    3复    & herja  & herjuðu  \\
\end{longtable}


\subsubsection{以-á结尾的动词}


另有一小类动词以长元音-á结尾,如spá `foretell', þrá `desire', fá `draw'\footnote{注意区分按照七类强动词变形的fá `fetch',二者词源不同,弱动词来源于*faihijaną,长元音由ai缩合得到;强动词来源于*fanhaną,长元音来源于-n和-h的脱落。}等。这些动词除了形态略和一般动词不同外,变形实际上没有不规则的情况,参考þrá的直陈式变位:

\begin{longtable}{lll}
    \toprule
    直陈式 & 现在  & 过去   \\
    \midrule
    \endhead
    \bottomrule
    \endfoot
    1单    & þrá   & þráða  \\
    2单    & þrár  & þráðir \\
    3单    & þrár  & þráði  \\
    1复    & þrám  & þráðum \\
    2复    & þráið & þráðuð \\
    3复    & þrá   & þráðu  \\
\end{longtable}

动词kalla, kallaði, kallaðr `call‌', 词干kall-a-,
是标准的第二弱变位法动词,主动态如下:

\begin{longtable}{llll}
    \toprule
    二类弱动词 & 直陈     & 虚拟     & 祈使   \\
    \midrule
    \endhead
    \bottomrule
    \endfoot
    (kall-a-)  &          &          &        \\
    单数现在时 &          &          &        \\
    1          & kalla    & kalla    &        \\
    2          & kallar   & kallir   & kalla  \\
    3          & kallar   & kalli    &        \\
    复数现在时 &          &          &        \\
    1          & kǫllum   & kallim   & kǫllum \\
    2          & kallið   & kallið   & kallið \\
    3          & kalla    & kalli    &        \\
    单数过去时 &          &          &        \\
    1          & kallaða  & kallaða  &        \\
    2          & kallaðir & kallaðir &        \\
    3          & kallaði  & kallaði  &        \\
    复数过去时 &          &          &        \\
    1          & kǫlluðum & kallaðim &        \\
    2          & kǫlluðuð & kallaðið &        \\
    3          & kǫlluðu  & kallaði  &        \\
    不定式     & kalla    &          &        \\
    现在分词   & kallandi &          &        \\
    过去分词   & kallaðr  &          &        \\
\end{longtable}

中动态如下:

\begin{longtable}{llll}
    \toprule
    二类弱动词 & 直陈       & 虚拟       & 祈使     \\
    \midrule
    \endhead
    \bottomrule
    \endfoot
    (kall-a-)  &            &            &          \\
    单数现在时 &            &            &          \\
    1          & kǫllumk    & kǫllumk    &          \\
    2          & kallask    & kallisk    & kallask  \\
    3          & kallask    & kallisk    &          \\
    复数现在时 &            &            &          \\
    1          & kǫllumsk   & kallimsk   & kǫllumsk \\
    2          & kallizk    & kallizk    & kallizk  \\
    3          & kallask    & kallisk    &          \\
    单数过去时 &            &            &          \\
    1          & kǫlluðumk  & kǫlluðumk  &          \\
    2          & kallaðisk  & kallaðisk  &          \\
    3          & kallaðisk  & kallaðisk  &          \\
    复数过去时 &            &            &          \\
    1          & kǫlluðumsk & kallaðimsk &          \\
    2          & kǫlluðuzk  & kallaðizk  &          \\
    3          & kǫlluðusk  & kallaðisk  &          \\
    不定式     & kallask    &            &          \\
    现在分词   & kallandisk &            &          \\
    过去分词   & kallazk    &            &          \\
\end{longtable}

\subsection{第三弱变位法}\label{第三弱变位法}

一小类动词属于第三弱变位法,其特征是词干元音-i-,例如 lif-i- `live‌', þor-i- `dare‌', vak-i- `wake‌'.
-i-出现在所有现在时中,但在元音前脱去,过去时中没有这个元音。区别于一类弱动词,这里的-i-不会导致i-变异,因为它由PGmc. *-ai-演变得来(*-ai- > *-e- > -i-)。

第三弱变位法包括以下一些形态略不规则的动词:


\subsubsection{以-á结尾的动词}


这类动词主要有两个ná `reach', gá
`heed',它们的变形和第二弱变位法基本一致(除了单数现在时),参考ná的变位:

\begin{longtable}{lll}
    \toprule
    直陈式 & 现在 & 过去  \\
    \midrule
    \endhead
    \bottomrule
    \endfoot
    1单    & nái  & náða  \\
    2单    & náir & náðir \\
    3单    & náir & náði  \\
    1复    & nám  & náðum \\
    2复    & náið & náðuð \\
    3复    & ná   & náðu  \\
\end{longtable}


\subsubsection{包含i-变异的动词}


古诺尔斯语只有三个第三类弱动词的现在时词干中有i-变异的痕迹,这是因为在原始日耳曼语中这些动词的词干元音是*-ja-而不是*-ai-.
这三个动词是segja `say', þegja `be silent'和hafa `have'.

segja, þegja的变形一致,它们的整个现在时词干都发生了i-变异,和第一类弱动词非常相似(但单数现在时的词尾不能用短词干一类弱动词的词尾来解释),例如segja的直陈式为:

\begin{longtable}{lll}
    \toprule
    直陈式 & 现在   & 过去   \\
    \midrule
    \endhead
    \bottomrule
    \endfoot
    1单    & segi   & sagða  \\
    2单    & segir  & sagðir \\
    3单    & segir  & sagði  \\
    1复    & segjum & sǫgðum \\
    2复    & segið  & sǫgðuð \\
    3复    & segja  & sǫgðu  \\
\end{longtable}

hafa的变形和上面的两个动词略有不同,它的复数现在时中没有i-变异:

\begin{longtable}{lll}
    \toprule
    直陈式 & 现在  & 过去   \\
    \midrule
    \endhead
    \bottomrule
    \endfoot
    1单    & hefi  & hafða  \\
    2单    & hefir & hafðir \\
    3单    & hefir & hafði  \\
    1复    & hǫfum & hǫfðum \\
    2复    & hafið & hǫfðuð \\
    3复    & hafa  & hǫfðu  \\
\end{longtable}

在单数现在时中,形如seg, segr; hef, hefr的古老形式也有记载。\footnote{这三个动词在原始语中的形态和在古诺尔斯语的发展颇有争议。现在时中没有-j-的形式可能是由第一类弱动词类比得到。}

我们以动词vaka, vakði `wake‌', 词干vak-i-, 展示标准的第三弱变位法的规则,主动态如下:

\begin{longtable}{llll}
    \toprule
    三类弱动词 & 直陈                                                                                                      & 虚拟   & 祈使  \\
    \midrule
    \endhead
    \bottomrule
    \endfoot
    (vak-i-)   &                                                                                                           &        &       \\
    单数现在时 &                                                                                                           &        &       \\
    1          & vaki                                                                                                      & vaka   &       \\
    2          & vakir                                                                                                     & vakir  & vaki  \\
    3          & vakir                                                                                                     & vaki   &       \\
    复数现在时 &                                                                                                           &        &       \\
    1          & vǫkum                                                                                                     & vakim  & vǫkum \\
    2          & vakið                                                                                                     & vakið  & vakið \\
    3          & vaka                                                                                                      & vaki   &       \\
    单数过去时 &                                                                                                           &        &       \\
    1          & vakða                                                                                                     & vekða  &       \\
    2          & vakðir                                                                                                    & vekðir &       \\
    3          & vakði                                                                                                     & vekði  &       \\
    复数过去时 &                                                                                                           &        &       \\
    1          & vǫkðum                                                                                                    & vekðim &       \\
    2          & vakðuð                                                                                                    & vekðið &       \\
    3          & vakðu                                                                                                     & vekði  &       \\
    不定式     & vaka                                                                                                      &        &       \\
    现在分词   & vakandi                                                                                                   &        &       \\
    过去分词   & vakaðr \footnote{这个动词的过去分词实际上只记录到中性形式vakat。关于过去分词的变形,将在(交叉)中详述。}
               &                                                                                                           &                \\
\end{longtable}

中动态如下:

\begin{longtable}{llll}
    \toprule
    三类弱动词 & 直陈      & 虚拟     & 祈使    \\
    \midrule
    \endhead
    \bottomrule
    \endfoot
    (vak-i-)   &           &          &         \\
    单数现在时 &           &          &         \\
    1          & vǫkumk    & vǫkumk   &         \\
    2          & vakisk    & vakisk   & vakisk  \\
    3          & vakisk    & vakisk   &         \\
    复数现在时 &           &          &         \\
    1          & vǫkumsk   & vakimsk  & vǫkumsk \\
    2          & vakizk    & vakizk   & vakizk  \\
    3          & vakask    & vakisk   &         \\
    单数过去时 &           &          &         \\
    1          & vǫkðumk   & vekðumk  &         \\
    2          & vakðisk   & vekðisk  &         \\
    3          & vakðisk   & vekðisk  &         \\
    复数过去时 &           &          &         \\
    1          & vǫkðumsk  & vekðimsk &         \\
    2          & vakðuzk   & vekðizk  &         \\
    3          & vakðusk   & vekðisk  &         \\
    不定式     & vakask    &          &         \\
    现在分词   & vakandisk &          &         \\
    过去分词   & vakazk    &          &         \\
\end{longtable}

\section{过去-现在混合动词}\label{过去-现在混合动词}

过去-现在混合动词(Preterite-present)指的是该动词的现在时使用强动词的过去词尾,按强动词过去时变位;过去时采取弱动词过去时词尾,按弱动词的过去时变位。造成这个现象的原因是:原始日耳曼语的过去时继承于原始印欧语的完成态,但在这类动词中,完成态演化成了现在时\footnote{造成这种现象的原因还没有共识。可能的解释是,这些动词都是表示状态而非动作的,天生带有完成的含义。},因此原始日耳曼语中过去-现在混合动词的过去时就空缺出来了。对此,这类动词只好采用日耳曼语独立的弱变化结构来构成过去时。

这种新的过去时构成法在这类动词中甚至应用到了不定式上。动词munu `will‌', skulu `shall‌', 以及 *knega `know, be able‌'(这个词的不定式没有被直接记录到)有用齿音构成的过去不定式:mundu, skyldu, knáttu.
这些过去不定式也使用于非限定性的结构,但一般用于主句是过去时的情况。读者可以类比英文中ask him \textbf{to do}和ask\textbf{ed} him \textbf{to do},在英语中只有一个不定式to do,但在古诺尔斯语中,后一句可以用过去不定式。

由于过去-现在混合动词的过去时按照强变位法变形,其元音变换也有一定的规律,可以大致地和前面的六大规则的强变位法对等起来,例如下面的的第一类动词的现在时中用到了ei和i的交替,和第一强变位法的元音交替有关(不完全一致)。但在古诺尔斯语的十个过去-现在混合动词中,没有一个和第二或第六强变位法的元音交替模式对应。另外,这些动词的含义一般决定了它们没有中动态。

\subsubsection{第一类}

有两个动词属于这一类,vita `know‌', eiga `have, own‌':

\begin{longtable}{lllll}
    \toprule
    第一类混合动词  & \multicolumn{2}{c}{现在} & \multicolumn{2}{c}{过去}                  \\
    \midrule
    \endhead
    \bottomrule
    \endfoot
    \textbf{直陈式} & ~                        & ~                        & ~      & ~     \\
    1单             & veit                     & á                        & vissa  & átta  \\
    2单             & veizt                    & átt                      & vissir & áttir \\
    3单             & veit                     & á                        & vissi  & átti  \\
    1复             & vitum                    & eigum                    & vissum & áttum \\
    2复             & vituð                    & eiguð                    & vissuð & áttuð \\
    3复             & vitu                     & eigu                     & vissu  & áttu  \\
    \textbf{虚拟式} & ~                        & ~                        & ~      & ~     \\
    1单             & vita                     & eiga                     & vissa  & ætta  \\
    2单             & vitir                    & eigir                    & vissir & ættir \\
    3单             & viti                     & eigi                     & vissi  & ætti  \\
    1复             & vitim                    & eigim                    & vissim & ættim \\
    2复             & vitið                    & eigið                    & vissið & ættið \\
    3复             & viti                     & eigi                     & vissi  & ætti  \\
    \textbf{祈使式} & ~                        & ~                        & ~      & ~     \\
    2单             & vit                      & eig                      & ~      & ~     \\
    1复             & vitum                    & eigum                    & ~      & ~     \\
    2复             & vituð                    & eiguð                    & ~      & ~     \\
    \textbf{不定式} & vita                     & eiga                     & ~      & ~     \\
    \textbf{分词}   & vitandi                  & eigandi                  & vitaðr & áttr  \\
\end{longtable}

说明:

\begin{enumerate}
    \item
          vita的过去式vissa中的-ss-实际上是由-tt-(< -t +
          -ð)变化得到,这依旧是日耳曼擦音定律的残留。
    \item
          eiga的变形相比vita不规则一些。其单数现在时á来源于*aig,*aig首先脱去词尾的-g,再发生ai
          > á的缩合。
\end{enumerate}


\subsubsection{第三类}

这类动词有三个,其元音交替规律比较规则:
\begin{center}
    \textbf{不定式u --- 单数现在时a --- 复数现在时u --- 过去分词u}
\end{center}


这三个动词是unna `love‌', kunna `know, be able‌', þurfa `need‌':


\begin{longtable}{lllllll}

    \toprule
    第三类混合动词  & \multicolumn{3}{c}{现在} & \multicolumn{3}{c}{过去}                                         \\
    \midrule
    \endhead
    \bottomrule
    \endfoot
    \textbf{直陈式} & ~                        & ~                        & ~        & ~       & ~      & ~       \\
    1单             & ann                      & kann                     & þarf     & unna    & kunna  & þurfta  \\
    2单             & annt                     & kannt                    & þarft    & unnir   & kunnir & þurftir \\
    3单             & ann                      & kann                     & þarf     & unni    & kunni  & þurfti  \\
    1复             & unnum                    & kunnum                   & þurfum   & unnum   & kunnum & þurftum \\
    2复             & unnuð                    & kunnuð                   & þurfuð   & unnuð   & kunnuð & þurftuð \\
    3复             & unnu                     & kunnu                    & þurfu    & unnu    & kunnu  & þurftu  \\
    \textbf{虚拟式} & ~                        & ~                        & ~        & ~       & ~      & ~       \\
    1单             & unna                     & kunna                    & þurfa    & ynna    & kynna  & þyrfta  \\
    2单             & unnir                    & kunnir                   & þurfir   & ynnir   & kynnir & þyrftir \\
    3单             & unni                     & kunni                    & þurfi    & ynni    & kynni  & þyrfti  \\
    1复             & unnim                    & kunnim                   & þurfim   & ynnim   & kynnim & þyrftim \\
    2复             & unnið                    & kunnið                   & þurfið   & ynnið   & kynnið & þyrftið \\
    3复             & unni                     & kunni                    & þurfi    & ynni    & kynni  & þyrfti  \\
    \textbf{祈使式} & ~                        & ~                        & ~        & ~       & ~      & ~       \\
    2单             & unn                      & kunn                     & -        & ~       & ~      & ~       \\
    1复             & unnum                    & kunnum                   & -        & ~       & ~      & ~       \\
    2复             & unnuð                    & kunnuð                   & -        & ~       & ~      & ~       \\
    \textbf{不定式} & unna                     & kunna                    & þurfa    & ~       & ~      & ~       \\
    \textbf{分词}   & unnandi                  & kunnandi                 & þurfandi & unn(a)t & kunnat &
    þurft                                                                                                         \\
\end{longtable}


说明:


\begin{enumerate}

    \item
          动词unna和kunna的过去式略不规则,其中没有塞音-ð-的痕迹,但在过去分词中仍有体现。实际上,过去式的-nn-是由*-nþ-变化得到的(参考哥特语过去式kunþa,这个音变还发生在动词finna中,见\ref{第三强变位法}),只是在共时层面上很难发现。
    \item
          þurfa的祈使式没有被记录到。在后面的表格中,一律用``-''表示未被记录到的形式。
\end{enumerate}

\subsubsection{第四类}

正如三类、四类强动词一样,第三类、第四类的过去-现在混合动词的区别仅在于词干尾的辅音,其元音交替模式是完全一样的。这类动词包括3个,muna
`remember‌', munu `will‌', skulu `shall‌' :

\begin{longtable}{lllllll}
    \toprule
    第四类混合动词  & \multicolumn{3}{c}{现在} & \multicolumn{3}{c}{过去}                                        \\
    \midrule
    \endhead
    \bottomrule
    \endfoot
    \textbf{直陈式} & ~                        & ~                        & ~        & ~      & ~      & ~       \\
    1单             & man                      & mun                      & skal     & munda  & munda  & skylda  \\
    2单             & mant                     & munt                     & skalt    & mundir & mundir & skyldir \\
    3单             & man                      & mun                      & skal     & mundi  & mundi  & skyldi  \\
    1复             & munum                    & munum                    & skulum   & mundum & mundum & skyldum \\
    2复             & munið                    & munuð                    & skuluð   & munduð & munduð & skylduð \\
    3复             & muna                     & munu                     & skulu    & mundu  & mundu  & skyldu  \\
    \textbf{虚拟式} & ~                        & ~                        & ~        & ~      & ~      & ~       \\
    1单             & muna                     & myna                     & skyla    & mynda  & mynda  & skylda  \\
    2单             & munir                    & mynir                    & skylir   & myndir & myndir & skyldir \\
    3单             & muni                     & myni                     & skyli    & myndi  & myndi  & skyldi  \\
    1复             & munim                    & mynim                    & skylim   & myndim & myndim & skyldim \\
    2复             & munið                    & mynið                    & skylið   & myndið & myndið & skyldið \\
    3复             & muni                     & myni                     & skyli    & myndi  & myndi  & skyldi  \\
    \textbf{祈使式} & ~                        & ~                        & ~        & ~      & ~      & ~       \\
    2单             & mun                      & -                        & -        & ~      & ~      & ~       \\
    1复             & munum                    & -                        & -        & ~      & ~      & ~       \\
    2复             & munuð                    & -                        & -        & ~      & ~      & ~       \\
    \textbf{不定式} & muna                     & munu                     & skulu    & -      & mundu  & skyldu  \\
    \textbf{分词}   & munandi                  & -                        & skulandi & munaðr & -      & skyldr  \\
\end{longtable}

说明:

\begin{enumerate}
    \item
          muna和munu本身是同源的,因此在有些形式上必须做出区分。例如muna的过去式munið,
          muna区别于munu所对应的munuð,
          muni;现在虚拟式中前者发生i-变异,后者不发生i-变异。但在约14世纪后,这些形式开始混杂,以至于muna和munu的许多变位在中世纪手稿中混用。例如munu的单数现在时也可以是man,
          mant;虚拟式中也可以不发生i-变异。
    \item
          skulu的直陈过去式和整个虚拟式中都发生了i-变异,但是在现在虚拟式中,i-变异也可以不发生,即skulu的变位中只有过去时全部要发生i-变异。
\end{enumerate}

\subsubsection{第五类}

这类动词的元音交替也比较规则:

\begin{center}
    \textbf{不定式e --- 单数现在时á --- 复数现在时e --- 过去分词e}
\end{center}

这类动词包括两个,mega `be able‌', kná (*kenga), `know, be able‌':

\begin{longtable}{lllll}
    \toprule
    第五类混合动词  & \multicolumn{2}{c}{现在} & \multicolumn{2}{c}{过去}                    \\
    \midrule
    \endhead
    \bottomrule
    \endfoot
    \textbf{直陈式} & ~                        & ~                        & ~      & ~       \\
    1单             & má                       & kná                      & mátta  & knátta  \\
    2单             & mátt                     & knátt                    & máttir & knáttir \\
    3单             & má                       & kná                      & mátti  & knátti  \\
    1复             & megum                    & knegum                   & máttum & knáttum \\
    2复             & meguð                    & kneguð                   & máttuð & knáttuð \\
    3复             & megu                     & knegu                    & máttu  & knáttu  \\
    \textbf{虚拟式} & ~                        & ~                        & ~      & ~       \\
    1单             & mega                     & knega                    & mætta  & knætta  \\
    2单             & megir                    & knegir                   & mættir & knættir \\
    3单             & megi                     & knegi                    & mætti  & knætti  \\
    1复             & megim                    & knegim                   & mættim & knættim \\
    2复             & megið                    & knegið                   & mættið & knættið \\
    3复             & megi                     & knegi                    & mætti  & knætti  \\
    \textbf{祈使式} & ~                        & ~                        & ~      & ~       \\
    2单             & -                        & -                        & ~      & ~       \\
    1复             & -                        & -                        & ~      & ~       \\
    2复             & -                        & -                        & ~      & ~       \\
    \textbf{不定式} & mega                     & *knega                   & -      & knáttu  \\
    \textbf{分词}   & megandi                  & -                        & mátt   & -       \\
\end{longtable}

说明:

\begin{enumerate}
    \item
          不定式*kenga并没有被记录到,但其过去不定时knáttu并不少见。
\end{enumerate}

\section{不规则动词}\label{不规则动词}

在前面介绍强弱变位法时,我们已经比较详细地介绍了每一类中形态不规则的动词的变形及其来历。这些动词或多或少还可以归类到七类强动词和三类弱动词中,具体来说是valda,vilja和vera三个异态动词。

\begin{enumerate}
    \item
          valda `cause; dominate‌'的现在时词干是vald-,过去分词词干valdin-,现在时系统的变位是规则的,但过去时有明显的不规则。较最早的文献中,单数过去时词干是oll-,复数是ull-,接着加上弱动词的词尾。\footnote{这个动词虽然有强动词的元音交替特征,但它的过去式实际上按弱动词变位。其中,-ll是*-lþ同化得到,因此词干中已经有弱动词的塞音标记。}但后来一般整个过去时词干都变成oll-,最后这个词甚至可以按规则的弱动词变位,且词首的v不脱落,文献中记载到了oldi或者voldi的形式。
    \item
          vilja `want‌'的变位和一类弱动词十分相似,但在现在时中词尾-r和-l同化为-ll,即vil + r > vill,(注意,其他一类弱动词没有这样的音变,如skilja
          > skil, skilr)第二人称单数式有时也写作vilt,类似于过去-现在混合动词。\footnote{vilja的直陈现在时实际上是古诺尔斯语中增补出来的,因此出现了一些不规则现象。在原始语中,现在时中反常地没有直陈式,而只有虚拟式。哥特语中这个动词也没有直陈现在时。}过去时部分规则地按照弱动词变位,齿音以-d-形式实现。它的过去分词是viljat,缺少过去时的标记。另外,这个词也有过去不定式vildu.
    \item
          最不规则的动词是vera `be‌', 和大多数语言的be动词一样,这是一个异干互补(Suppletion)动词,意味着其词形变化中涉及多个词干:现在时中是*(e)s-
          , 过去时中是*wes-. vera中的r是由早期的s变化过来的,因此后期的vera的变形:vera, ert, er, var, vart分别对应早期的vesa, est, es, vas, vast\footnote{但是,复数式中并没有记录到含有s的形式。},这一变化大致在1100年左右完成,前者成为主流的用法。vera的变形经常和前后的词合写,这时的vera类似于一种后缀,例如:

          \begin{quote}
              nús < nú es `now is‌',

              þaz < þat es `that is‌',

              vér(r)óm < vér erum `we are‌',

              þeir(r)ó < þeir ero `they are‌',

              emk < em ek `I am'

              sják < sjá ek `I may be'
          \end{quote}
\end{enumerate}

上述的三个动词的主动态如下,除valda以外的动词没有中动态:

\begin{longtable}{lllllllll}
    \toprule
    不规则动词      & 现在     & 过去               &  & 现在       & 过去   &  & 现在     & 过去       \\
    \midrule
    \endhead
    \bottomrule
    \endfoot
    \textbf{直陈式} &          &                    &  &            &        &  &          &            \\
    1单             & veld     & olla               &  & vil        & vilda  &  & em       & var, vas   \\
    2单             & veldr    & ollir              &  & vill, vilt & vildir &  & ert, est & vart, vast \\
    3单             & veldr    & olli (oldi, voldi) &  & vill       & vildi  &  & er, es   & var,
    vas                                                                                                 \\
    1复             & vǫldum   & ullum, ollum       &  & viljum     & vildum &  & erum     & várum      \\
    2复             & valdið   & ulluð, olluð       &  & vilið      & vilduð &  & eruð     & váruð      \\
    3复             & valda    & ullu, ollu         &  & vilja      & vildu  &  & eru      & váru       \\
    \textbf{虚拟式} &          &                    &  &            &        &  &          &            \\
    1单             & valda    & ylla               &  & vilja      & vilda  &  & sjá, sé  & væra       \\
    2单             & valdir   & yllir              &  & vilir      & vildir &  & sér      & værir      \\
    3单             & valdi    & ylli (vyldi)       &  & vili       & vildi  &  & sé       & væri       \\
    1复             & valdim   & yllim              &  & vilim      & vildim &  & sém      & værim      \\
    2复             & valdið   & yllið              &  & vilið      & vildið &  & séð, sét & værið      \\
    3复             & valdi    & ylli               &  & vili       & vildi  &  & sé       & væri       \\
    \textbf{祈使式} &          &                    &  &            &        &  &          &            \\
    2单             &          &                    &  &            &        &  & ver      &            \\
    2 复            &          &                    &  &            &        &  & verið    &            \\
    \textbf{不定式} & valda    &                    &  & vilja      & vildu  &  & vera     &            \\
    \textbf{分词}   & valdandi & valdit             &  & viljandi   & viljat &  &          &
    verit                                                                                               \\
\end{longtable}

注意vera的现在虚拟式中的许多形式与sjá `see‌'一致,必须按照句意判断。
