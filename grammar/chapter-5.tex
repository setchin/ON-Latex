\chapter{代词}\label{代词}

\begin{introduction}[章节要点]
\item 人称代词及其变格
\item 物主代词及其变格
\item 指示代词及其变格
\item 疑问代词hverr
\item 常见的不定代词
\end{itemize}

\section{人称代词}\label{人称代词}

古诺尔斯语的人称代词系统部分保留了双数。双数仅出现在第一和第二人称代词中,第三人称代词只有单复数之分,但区分性别。人称代词中的变形涉及多个词干,具体如下:

\begin{longtable}{lll}
    \toprule
    人称代词 & 第一人称 & 第二人称 \\
    \midrule
    \endhead
    \bottomrule
    \endfoot
    单数     &          &          \\
    N        & ek       & þú       \\
    A        & mik      & þik      \\
    G        & mín      & þín      \\
    D        & mér      & þér      \\
    双数     &          &          \\
    N        & vit      & it, þit  \\
    A        & okkr     & ykkr     \\
    G        & okkar    & ykkar    \\
    D        & okkr     & ykkr     \\
    复数     &          &          \\
    N        & vér      & ér, þér  \\
    A        & oss      & yðr      \\
    G        & vár      & yðar     \\
    D        & oss      & yðr      \\
\end{longtable}

古诺尔斯语中动词已经没有双数形式,双数人称代词同样支配动词的复数式。同时,双数代词使用的并不多,并且有和复数合并的趋势。在近现代冰岛语中,双数第一人称逐渐承担了复数的作用。

人称代词作主语时,主格有时以后缀形式粘着在动词后,这在早期的诗歌中尤为常见。第一人称ek失去元音e,以-k的形式添加在动词后,例如:

\begin{quote}
    mælik `I speak‌' < mæli + ek

    mákat `I cannot‌' < má + ek + at
\end{quote}

第二人称þú中的þ有时和前面的辅音发生同化现象,例如:

\begin{quote}
    heyrðu `you hear' < heyr þú

    skaltu `you shall' < skalt þú

    seldu `you sell' < sel þú
\end{quote}

这种合写在解读了造成了一些偏差,例如skuluðér(来自skuluð ér)可以被理解为skuluð þér.
正是由于这个元音,第二人称双数和复数才会出现异体形式þit和þér,它们本来的形式就是it和ér.

第一和第二人称下,间接格同时也可以当作反身代词使用,反身代词本身也不存在主格,因为习惯上只有I hurt myself, 而不可能有†myself hurt I. 在古老的诗歌中,反身代词也经常以后缀形式黏附在动词后,宾格mik变成-mk,但是与格也常常以同样的形式添加在动词后。即-mk是一种近乎通用的表示宾语(无论与格还是宾格)的办法:

\begin{quote}
    þóttumk < þótti mér `it seemed to me‌'

    gáfumk íþrótt < gáf mér íþrótt `gave me skill'
\end{quote}

比较反常的情况是,当这种后缀添加在动词的单数式上时,动词反而要采用对应人称的复数式,这和强动词第三人称单数的中动词尾相呼应(参见\ref{强动词的中动词尾})。

第三人称人称代词虽然没有双数,但区分阴阳中三性。

\begin{longtable}{llll}
    \toprule
    性   & 阳性     & 阴性     & 中性     \\
    \midrule
    \endhead
    \bottomrule
    \endfoot
    单数 &          &          &          \\
    N    & hann     & hon      & þat      \\
    A    & hann     & hana     & þat      \\
    G    & hans     & hennar   & þess     \\
    D    & honum    & henni    & því, þí  \\
    复数 &          &          &          \\
    N    & þeir     & þær      & þau      \\
    A    & þá       & þær      & þau      \\
    G    & þeir(r)a & þeir(r)a & þeir(r)a \\
    D    & þeim     & þeim     & þeim     \\
\end{longtable}

第三人称不能像上面一样用间接格表示反身代词,相反另有一个专门的反身代词sik。如上面所说,sik没有主格,它既不区分性也不区分数,只按照格变化。

\begin{longtable}{ll}
    \toprule
      & 反身代词 \\
    \midrule
    \endhead
    \bottomrule
    \endfoot
    N & -        \\
    A & sik      \\
    G & sín      \\
    D & sér      \\
\end{longtable}

sik也添加在动词后面,演变为中动态的标记-sk.

\section{物主代词}\label{物主代词}

古诺尔斯语的物主代词一般是形容词性的,但少数情况下也可以作名词用。

物主代词由人称代词的单数属格衍生出来。由于第一、第二人称的人称代词和第三人称人称代词的构成有区别之处,它们对应的物主代词的构词法也有不同。

物主代词的形容词性要求它必须能按性屈折,这样才能与其修饰的名词保持一致。对于第一、第二人称代词而言,人称代词本来不区分性,因此把它的属格原型作为物主代词的词干,接着按照强变格法添加各个性、数、格的词尾,例如ek的属格mín `of me‌', 构成形容词词干mín-,加单数阳性词尾有*mín- + -r > minn,注意词干中的长元音在双辅音前缩短。以minn `my‌'为例,它的完整变格如下:

\begin{longtable}{llll}
    \toprule
    性   & 阳性  & 阴性   & 中性  \\
    \midrule
    \endhead
    \bottomrule
    \endfoot
    单数 &       &        &       \\
    N    & minn  & mín    & mitt  \\
    A    & minn  & mína   & mitt  \\
    G    & míns  & minnar & míns  \\
    D    & mínum & minni  & mínu  \\
    复数 &       &        &       \\
    N    & mínir & mínar  & mín   \\
    A    & mína  & mínar  & mín   \\
    G    & minna & minna  & minna \\
    D    & mínum & mínum  & mínum \\
\end{longtable}

注意:minn的变格有些许不规则之处,它类似于强动词过去分词的变形。单数阳性宾格以及单数中性主格/宾格都值得注意。

类似地,其余第一、第二人称的物主代词以及反身代词的变化都模仿上述过程,下表给出了它们各个性的单数主格形式:

\begin{longtable}{lllll}
    \toprule
         & 词干     & 阳性     & 阴性  & 中性     \\
    \midrule
    \endhead
    \bottomrule
    \endfoot
    单数 &          &          &       &          \\
    1单  & mín-     & minn     & mín   & mitt     \\
    2单  & þín-     & þinn     & þín   & þitt     \\
    双数 &          &          &       &          \\
    1双  & okkar-   & okkarr   & okkur & okkart   \\
    2双  & ykkar-   & ykkarr   & ykkur & ykkart   \\
    复数 &          &          &       &          \\
    1复  & vár-     & várr     & vár   & várt     \\
    2复  & yð(v)ar- & yð(v)arr & yður  & yð(v)art \\
         &          &          &       &          \\
    反身 & sín-     & sinn     & sín   & sitt     \\
\end{longtable}

说明:

\begin{enumerate}
    \item
          第二人称复数词干yð(v)ar-有两写,如果选择不包含v的词干yðar-,它的变形就和其他双音节形容词一致,弱读元音a在元音开头的词尾前脱落,如yðar-+-um
          \textgreater{} yðrum.
          但是,如果选择包含v的词干yðvar-,则不会有这种现象。虽然yðvar-也有两个音节,但如果脱去了a,半元音v就出现在两个辅音之间,这是古诺尔斯语所不允许的。
    \item
          词干ykkar-和okkar-属于双音节词干,它们规则地适用于双音节形容词的变格法。
\end{enumerate}

第三人称人称代词没有对应的物主代词,只沿用其属格形式就足以表达物主代词的含义,其单数为hans, hennar, þess复数都是þeira.

\section{指示代词}\label{指示代词}

指示代词一般即可以作名词用也可以作形容词用,它也采取和第三人称代词类似的异干互补系统。古诺尔斯语中的指示代词的用法和英语类似,可以用于空间上、时间上或逻辑上的近指或远指。

\subsubsection{远指代词sá}

sá有两层含义,它既可以指示较远的事物,相当于英语的`that‌',也可以用作类似于定冠词含义的`the‌',
用于指示已经提到过的事物。对应地,在作名词时,它也可以表示远处的东西或是已经提到过的事物。sá的变格不规则,但还是有明显的形容词词尾的痕迹,如下所示:

\begin{longtable}{llll}
    \toprule
         & 阳性     & 阴性      & 中性     \\
    \midrule
    \endhead
    \bottomrule
    \endfoot
    单数 &          &           &          \\
    N    & sá       & sú        & þat      \\
    A    & þann     & þá        & þat      \\
    G    & þess     & þeir(r)ar & þess     \\
    D    & þeim     & þeir(r)i  & því, þí  \\
    复数 &          &           &          \\
    N    & þeir     & þær       & þau      \\
    A    & þá       & þær       & þau      \\
    G    & þeir(r)a & þeir(r)a  & þeir(r)a \\
    D    & þeim     & þeim      & þeim     \\
\end{longtable}

注意:sá的所有复数形式以及中性的单数形式都和第三人称代词一致,所以在这种情况下,同一个词可能有多种含义(虽然很多情况下,表意是类似的)。

\subsubsection{近指代词sjá}

sjá是sá的反义词,它指示相对较近的事物,但一般不提示上下文中出现过的事物。sjá的用法和sá类似,也兼具形容词和代词的功能。其变格如下:

\begin{longtable}{llll}
    \toprule
         & 阳性       & 阴性       & 中性   \\
    \midrule
    \endhead
    \bottomrule
    \endfoot
    单数 &            &            &        \\
    N    & sjá, þessi & sjá, þessi & þetta  \\
    A    & þenna      & þessa      & þetta  \\
    G    & þessa      & þessar     & þessa  \\
    D    & þessum     & þessi      & þessu  \\
    复数 &            &            &        \\
    N    & þessir     & þessar     & þessi  \\
    A    & þessa      & þessar     & þessi  \\
    G    & þessa      & þessa      & þessa  \\
    D    & þessum     & þessum     & þessum \\
\end{longtable}

\subsubsection{代词hinn}

hinn一般不用作近指或远指代词,它一般表示``另一个'',与前文提到的名词形成对比。hinn也有类似于冠词的用法,这种情况下,它和名词的特指后缀-inn(或独立形式inn)表意完全一致。除了中性单数主格/宾格的词尾-tt外,它的变形和inn一致:

\begin{longtable}{llll}
    \toprule
         & 阳性  & 阴性   & 中性  \\
    \midrule
    \endhead
    \bottomrule
    \endfoot
    单数 &       &        &       \\
    N    & hinn  & hin    & hitt  \\
    A    & hinn  & hina   & hitt  \\
    G    & hins  & hinnar & hins  \\
    D    & hinum & hinni  & hinu  \\
    复数 &       &        &       \\
    N    & hinir & hinar  & hin   \\
    A    & hina  & hinar  & hin   \\
    G    & hinna & hinna  & hinna \\
    D    & hinum & hinum  & hinum \\
\end{longtable}

\section{关系代词}\label{关系代词}

古诺尔斯语只有一个关系代词er, 其早期形式为es, 后期也用sem表示和er相同的意思。准确来说,这个er应当表达为关系小品词,因为所有的定语从句都可以用er引导(包括英语中需要使用关系副词的情况)。er不可变格,没有人称、数或者性的概念。有时在er的前面加上sá的变格来指示对应的格、性、数,例如sú er指示阴性单数主格。但这种说法也常常会产生分歧,因为指示代词sá完全可以理解为修饰先行词的形容词,这样sá就是主句的一部分,从而与er在从句中的格、性、数没有任何关系了。

在下面的句型中:
\begin{quote}
    ...sverð þat er...,
\end{quote}
根据省略号处填补的内容,可以有若干种解读:
\begin{enumerate}
    \item (that is) the sword, \textbf{which} (bears his name)‌

          þat er整体作引导词,在从句中充当主格

    \item (that is) the sword, \textbf{by which} (he was killed)‌

          þat修饰sverð, er在从句中充当与格

    \item (that is) the sword, \textbf{of which} (the legacy is well-known)‌

          þat修饰sverð, er在从句中充当属格
\end{enumerate}

另一个常见的用法是将er与一些副词合用,构成具体的关系副词,例如:

\begin{quote}
    þar `there‌' < þar er `where‌'

    þá `then‌' < þá er `when‌'
\end{quote}

关于er引导的从句,参见(交叉引用)

\section{疑问代词}\label{疑问代词}

古诺尔斯语中基本的疑问代词词干是hverj- `who, which, what‌',它既是名词性的又是形容词性的。作形容词时,按照强变格法变格,如下所示:

\begin{longtable}{llll}
    \toprule
         & 阳性           & 阴性     & 中性     \\
    \midrule
    \endhead
    \bottomrule
    \endfoot
    单数 &                &          &          \\
    N    & hverr          & hver     & hvert    \\
    A    & hverjan, hvern & hverja   & hvert    \\
    G    & hvers          & hverrar  & hvers    \\
    D    & hverjum, hveim & hverri   & hverju   \\
    复数 &                &          &          \\
    N    & hverir         & hverjar  & hver     \\
    A    & hverja         & hverjar  & hver     \\
    G    & hverra         & hverra   & hverra   \\
    D    & hverjum        & hverrjum & hverrjum \\
\end{longtable}

作名词时形式和形容词一致,但有一些非常常见的两写:

\begin{longtable}{lll}
    \toprule
         & 阳性/阴性      & 中性           \\
    \midrule
    \endhead
    \bottomrule
    \endfoot
    单数 &                &                \\
    N    & hverr          & hvat           \\
    A    & hverjan, hvern & hvat           \\
    G    & hvers, hves(s) & hvers, hves(s) \\
    D    & hverjum, hveim & hví            \\
\end{longtable}

更准确地来说,hverr是对大于三个的物体中的哪一个提问,如果要对两个中的哪一个提问,要用另一个词hvárr `which of two‌'.
这两个疑问代词反映了古诺尔斯语中残留的双数和复数的区别。hvárr的变形和hverr完全一致,词干为hvár-,见下表:

\begin{longtable}{llll}
    \toprule
         & 阳性   & 阴性    & 中性   \\
    \midrule
    \endhead
    \bottomrule
    \endfoot
    单数 &        &         &        \\
    N    & hvárr  & hvár    & hvárt  \\
    A    & hvárn  & hvára   & hvárt  \\
    G    & hvárs  & hvárrar & hvárs  \\
    D    & hvárum & hvárri  & hváru  \\
    复数 &        &         &        \\
    N    & hvárir & hvárar  & hvár   \\
    A    & hvára  & hvárar  & hvár   \\
    G    & hvárra & hvárra  & hvárra \\
    D    & hvárum & hvárum  & hvárum \\
\end{longtable}

某些疑问代词的形式(或变体)固定下来成为副词形式:

\begin{longtable}{ll}
    \toprule
    疑问副词 & 含义                          \\
    \midrule
    \endhead
    \bottomrule
    \endfoot
    hvaðan   & 从哪里 `whence‌'               \\
    hvar     & 哪里 `where‌'                  \\
    hvert    & 到哪里 `whither‌'              \\
    hvárt    & 是否 `whether (or not)‌'       \\
    hvé      & 怎样 `how‌'                    \\
    hvenær   & 何时 `when‌'                   \\
    hví      & 为何 `why‌'                    \\
    hversu   & 到什么程度 `how‌' (degree)     \\
    hvernig  & 以什么方法 `how, in what way‌' \\
\end{longtable}

这些疑问代词(副词)也可以引导间接疑问句。

\section{不定代词}\label{不定代词}

总体来说,不定代词的变格和强形容词一致,它们大多数本身也是形容词。最常见的不定代词如下所示,它们的词干首先给出:

\subsubsection{all- `all‌'}
其构词如规则的形容词,它一般都用作强形容词,见下表:

\begin{longtable}{llll}
    \toprule
         & 阳性  & 阴性   & 中性  \\
    \midrule
    \endhead
    \bottomrule
    \endfoot
    单数 &       &        &       \\
    N    & allr  & ǫll    & allt  \\
    A    & allan & alla   & allt  \\
    G    & alls  & allrar & alls  \\
    D    & ǫllum & allri  & ǫllu  \\
    复数 &       &        &       \\
    N    & allir & allar  & ǫll   \\
    A    & alla  & allar  & ǫll   \\
    G    & allra & allra  & allra \\
    D    & ǫllum & ǫllum  & ǫllum \\
\end{longtable}

其主要用法是:

\begin{enumerate}
    \item
          所有形式都可当作形容词或名词用,表示``所有'';
    \item
          单数形式在许多短语中几乎按副词用,表示``完全'':allr í sundr `all
          asunder';
    \item
          单数中性尤常作不定代词用,类似于英语'everything';
    \item
          allt可当作一个宽泛的副词,表示``完全地;直接地;在所有地方;基本上''等;
    \item
          复数allir单独使用,表示``所有人;一起''。
\end{enumerate}

\subsubsection{sum- `some, certain‌'}

其构词如规则的强形容词,可以作形容词和代词用。

\subsubsection{ein- `one‌'}
作不定代词或形容词时区分于ein作数词(交叉引用)的情况。其变格大部分是规则的,只有中性的单数主格和宾格是eitt.

\begin{longtable}{llll}
    \toprule
         & 阳性        & 阴性   & 中性  \\
    \midrule
    \endhead
    \bottomrule
    \endfoot
    单数 &             &        &       \\
    N    & einn        & ein    & eitt  \\
    A    & einn, einan & eina   & eitt  \\
    G    & eins        & einnar & eins  \\
    D    & einum       & einni  & einu  \\
    复数 &             &        &       \\
    N    & einir       & einar  & ein   \\
    A    & eina        & einar  & ein   \\
    G    & einna       & einna  & einna \\
    D    & einum       & einum  & einum \\
\end{longtable}

其主要用法是:

\begin{enumerate}
    \item
          用作单数时,表示不定代词,表示不特指的某一个;
    \item
          单数或复数都可以表示``单独的'',用法类似于副词:láta einan `let
          alone';
    \item
          einna和其他名词连用,表示强调含义:einna manna bezt `best of all
          single man';
    \item
          eins作副词用,表示``以同一方式'',但常和其他词连用,如eins ok `as if',
          at eins `only';
    \item
          和其他代词、名词连用,如einn hverr `each; some‌'(见下), einn saman
          `together', hverr ok einn, `each and one', né einn `none', fáir einir
          `few'.
\end{enumerate}

\subsubsection{annar- `other‌, another'}

其变格参见\ref{不规则形容词}。annat尤其常作为名词用。

\subsubsection{nǫkkur- `any, some; a certain‌'}

这个词也按强形容词规则变化,但词形非常复杂\footnote{这个词最早的形式是nekkverr,来自于né hverr。因此,其最早的变格和hverr相同,词干为nekkverj-;后来这个词干脱去-j,按nekkver-变格,同时,元音发生了变化,有nekkvar-, nakkver-, nakkvar-等形式,受v的影响,a又变为ǫ;v有时也从词干上脱落。最后,这个词在现代冰岛语中还能像双音节形容词一样发生省略。因此这个词记录到的形态非常复杂,有nøkkurr, nakkurr, nekkverr, nakkvarr, nǫkkverr, nǫkkvarr, nǫkkr, nǫkkurr等。}。就nǫkkur-这个词干而言,有阳性单数宾格nǫkkurn;中性单数主格/宾格nǫkkut.
有时也用词干nakkvar-,变形如下:

\begin{longtable}{lllllll}
    \toprule
         & \multicolumn{2}{c}{阳性} & \multicolumn{2}{c}{阴性} & \multicolumn{2}{c}{中性}                                   \\
    \midrule
    \endhead
    \bottomrule
    \endfoot
    单数 &                          &                          &                          &          &           &          \\
    N    & nakkvarr                 & nǫkkurr                  &                          & nǫkkur   & nakkvat   & nǫkkut   \\
    A    & nakkvarn                 & nǫkkurn                  & nakkvara                 & nǫkkura  & nakkvat   & nǫkkut   \\
    G    & nakkvars                 & nǫkkurs                  & nakkvarar                & nǫkkurar & nakkvars  & nǫkkurs  \\
    D    &                          & nǫkkurum                 & nakkvarri                & nǫkkurri &           & nǫkkuru  \\
    复数 &                          &                          &                          &          &           &          \\
    N    & nakkvarir                & nǫkkurir                 & nakkvarar                & nǫkkurar &           & nǫkkur   \\
    A    & nakkvara                 & nǫkkura                  & nakkvarar                & nǫkkurar &           & nǫkkur   \\
    G    & nakkvarra                & nǫkkurra                 & nakkvarra                & nǫkkurra & nakkvarra &
    nǫkkurra                                                                                                                \\
    D    &                          & nǫkkurum                 &                          & nǫkkurum &           & nǫkkurum \\
\end{longtable}

表格中的空白处表示没有记录到以词干nakkvar-构成的形式。

这个词没有特别费解的用法,但常用阳性nǫkkurr指代“任何人”,相当于anyone,中性nǫkkurt指代“任何事”,相当于anything。 nǫkkurr也可以和数词连用,表示“大约”。

\subsubsection{hverj- `each, every‌'}

它按强变格变化,参见\ref{疑问代词}。hverr作不定代词时表示``每一个'',它修饰的名词总要用属格,如gumna hverr `each man (=every one of men)'.

\subsubsection{ein- + hverj- `each; some‌'}

einn和hverr连在一起构成不定代词。第二个词干hverj-总是要变格,ein-可能保持ein-不变,也可变格使之与第二个词干一致。例如单数阳性属格可以是einshvers或einhvers. 这个词有两个含义:

\begin{enumerate}
    \item
          类似于hverr,表示``每一个'',但语义更强。
    \item
          类似于einn,表示``某一个'',如eina hverja nótt `some night'.
\end{enumerate}

\subsubsection{báð- `both‌'}
这个词只以复数形式出现,注意中性式的不规则之处。特别地,bæði常作为副词用,构成bæði ... ok ... `both ... and ...'结构。

\begin{longtable}{llll}
    \toprule
         & 阳性   & 阴性   & 中性   \\
    \midrule
    \endhead
    \bottomrule
    \endfoot
    复数 &        &        &        \\
    N    & báðir  & báðar  & bæði   \\
    A    & báða   & báðar  & bæði   \\
    G    & beggja & beggja & beggja \\
    D    & báðum  & báðum  & báðum  \\
\end{longtable}

\subsubsection{nein- `none, not any‌'}

这个词由né + einn得到,其变格参照einn. neinn虽然是一个表示否定的代词,但是它单独不能表示否定,必须和其他否定副词连用,最常见的是ekki:ekki neitt `nothing'

\subsubsection{engi `no, none‌'}

它由einn和否定后缀-gi结合得到,其部分形式是不规则的:

\begin{longtable}{llll}
    \toprule
         & 阳性           & 阴性   & 中性           \\
    \midrule
    \endhead
    \bottomrule
    \endfoot
    单数 &                &        &                \\
    N    & engi           & engi   & ekki           \\
    A    & engan, engi    & enga   & ekki           \\
    G    & einskis, engis & engrar & einskis, engis \\
    D    & engum          & engri  & engu           \\
    复数 &                &        &                \\
    N    & engir          & engar  & engi           \\
    A    & enga           & engar  & engi           \\
    G    & engra          & engra  & engra          \\
    D    & engum          & engum  & engum          \\
\end{longtable}

这个词的词性也像nǫkkurr一样多变,它过去常用eing-或øng-词干,有时词干上还有额外的-v,出现-a或-ir前,如øngvar, engvar; øngvir, engvir. 另外,有时还在主格上添加-nn/-n后缀,有单数阳性主格enginn,单数阴性主格eigin,复数中性主格enginn,其他形式都不添加这个后缀。

engi可以单数使用,相当于英语`none'.

\subsubsection{hvárgi `neither'},这个词的变形也比较多样,参见下表:

\begin{longtable}{llll}
    \toprule
         & 阳性                              & 阴性     & 中性              \\
    \midrule
    \endhead
    \bottomrule
    \endfoot
    单数 &                                   &          &                   \\
    N    & hvárgi, hvárigr, hvárrgi, hvárugr & hvárgi   & hvárki, hvártki   \\
    A    & hvárgan, hvárngan, hvárngi        & hvárga   & hvárki, hvártki   \\
    G    & hvárkis, hvárskis                 & hvárgrar & hvárkis, hvárskis \\
    D    & hvárungi, hvárgum                 & hvárgri  & hvárugi, hvárgu   \\
    复数 &                                   &          &                   \\
    N    & hvárgir, hvárigir                 & hvárgar  & hvárgi            \\
    A    & hvárga                            & hvárgar  & hvárgi            \\
    G    & hvárgra                           & hvárgra  & hvárgra           \\
    D    & hvárgum                           & hvárgum  & hvárgum           \\
\end{longtable}

hvárgi的中性形式hvárki常构成固定搭配hvárki ... né `neither ... nor', 例如hvárki til laga né til úlaga `be neither friendly nor hostile'