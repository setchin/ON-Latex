\chapter{拼写与语音}
\label{chp:morph-phonology}

\begin{introduction}[章节要点]
    \item 古诺尔斯语的两种书写系统
    \item 古诺尔斯语字母的音值及单词拼读
    \item 音节划分规则
    \item 古诺尔斯语的音变
\end{introduction}

\section{书写系统和读音}
\label{sec:writing_system}

古诺尔斯语主要使用两种字母书写。其一是较早期的卢恩字母(Rune),后来则采用拉丁字母。最早发现的卢恩文字可追溯到公元2世纪。此时的古诺尔斯语尚处在非常原始的时期,故称为原始诺尔斯语。卢恩一词在日耳曼语中的意思是“秘密”,据神话记载,奥丁曾将自身作为祭品倒挂在世界之树上,在历经九夜的折磨后终于拾起了卢恩文字。这个神话的象征是奥丁通过苦行获得了智慧和奥义,因而卢恩的含义远不止一种书写系统那么简单。维京人认为卢恩可以用于占卜,到了中世纪晚期,北欧的文化已经受到了严重的基督教影响,其文字大量被拉丁化,卢恩字母丧失了日常沟通的功能,反而更加往神秘学的方向发展。

卢恩文字最初有24个,这套字母表称之为Elder Futhark,futhark是前六个字母的读音,和alphabet的含义(希腊字母表的前两个字母)类似。后来卢恩字母也发展出了16个字母的版本,称为Younger Futhark.
\begin{table}[H]
    \centering
    \caption*{\textbf{Elder Futhark} }
    \begin{tabular}{@{}llllllllllllllllllllllll@{}}
        \toprule
        \textarc{f} & \textarc{u} & \textarc{\th} & \textarc{a} & \textarc{r} & \textarc{k} & \textarc{g} & \textarc{w} & \textarc{h} & \textarc{n} & \textarc{i} & \textarc{j} & \textarc{p} & \textarc{I} & \textarc{R} & \textarc{s} & \textarc{t} & \textarc{b} & \textarc{e} & \textarc{m} & \textarc{l} & \textarc{\ng} & \textarc{d} & \textarc{o} \\ \midrule
        f           & u           & þ             & a           & r           & k           & g           & w           & h           & n           & i           & j           & p           & ï           & z           & s           & t           & b           & e           & m           & l           & ŋ             & d           & o           \\ \bottomrule
    \end{tabular}
\end{table}
\begin{table}[H]
    \centering
    \caption*{\textbf{Younger Futhark} }
    \begin{tabular}{@{}llllllllllllllll@{}}
        \toprule
        \textarm{f} & \textarm{u}                                              & \textarm{\th} & \textarn{\A} & \textarm{r} & \textarm{G} & \textarm{h} & \textarm{n}/\textarm{N} & \textarm{i} & \textarm{a} & \textarn{R} & \textarm{s}/\textarm{c} & \textarn{t} & \textarn{b} & \textarn{m} & \textarn{l} \\ \midrule
        f/v         & \begin{tabular}[c]{@{}l@{}}u/v/w,\\ y, o, ø\end{tabular} & þ, ð          & ą, o, æ      & r           & k, g, ŋ     & h           & n                       & e           & a, æ, e     & R           & s                       & t, d        & b, p        & m           & l           \\ \bottomrule
    \end{tabular}
\end{table}
卢恩文字或许有非常隐秘的作用,但这不在本书的讨论范围内。对于大部分中世纪的手稿而言,古诺尔斯语已经用上了我们熟悉的拉丁字母。

除了有最常见的26个字母外,古诺尔斯语的字母表中还包括几个特殊的辅音字母、元音字母和长音字母。这里我们只谈标准正字法下的字母,关于原始手稿中更复杂的情况,将在读本中进一步探索。

\begin{table}[H]
    \centering
    \begin{tabular}{@{}llll@{}}
        \toprule
        小写字母 & 大写字母 & 发音(国际音标) & 环境                                                         \\ \midrule
        á        & Á        & ɔː               &                                                              \\
        a        & A        & ɑ                &                                                              \\
        b        & B        & b                &                                                              \\
        c        & C        & k                &                                                              \\
        d        & D        & d                &                                                              \\
        ð        & Ð        & ð                &                                                              \\
        é        & É        & eː               &                                                              \\
        e        & E        & e                &                                                              \\
        f        & F        & (1) f            & 词首                                                         \\
                 &          & (2) v            & 除词首外的其他位置                                           \\
        g        & G        & (1) g            & 词首,双写时,或在\textless{}gn\textgreater{}中              \\
                 &          & (2) x            & 在\textless{}gs\textgreater{}或\textless{}gt\textgreater{}中 \\
                 &          & (3) ɣ            & 在\textless{}gh\textgreater{}中                              \\
        h        & H        & h                &                                                              \\
        í        & Í        & iː               &                                                              \\
        i        & I        & i                &                                                              \\
        j        & J        & j                &                                                              \\
        k        & K        & (1)k             & 除了下面的情况                                               \\
                 &          & (2)x             & 在\textless{}ks\textgreater{}或\textless{}kt\textgreater{}中 \\
        l        & L        & l                &                                                              \\
        m        & M        & m                &                                                              \\
        n        & N        & n                &                                                              \\
        ó        & Ó        & oː               &                                                              \\
        o        & O        & o                &                                                              \\
        p        & P        & (1) p            & 除了下面的情况                                               \\
                 &          & (2) f            & \textless{}ps\textgreater{}或\textless{}pt\textgreater{}中   \\
        q        & Q        & k                & 总和u一起出现,qu和kv是一样的                                \\
        r        & R        & r                &                                                              \\
        s        & S        & s                &                                                              \\
        t        & T        & t                &                                                              \\
        ú        & Ú        & uː               &                                                              \\
        u        & U        & u                &                                                              \\
        v        & V        & v                &                                                              \\
        w        & W        & w                &                                                              \\
        x        & X        & xs               &                                                              \\
        ý        & Ý        & yː               &                                                              \\
        y        & Y        & y                &                                                              \\
        z        & Z        & ts               & 极少出现,主要是-t/-d/-ð和-s的合写                           \\
        þ        & Þ        & θ                &                                                              \\
        æ        & Æ        & ɛː               &                                                              \\
        ǫ́        & Ǫ́        & ɔː               &                                                              \\
        ǫ        & Ǫ        & ɔ                &                                                              \\
        ø        & Ø        & ø                &                                                              \\
        œ        & Œ        & øː               &                                                              \\ \bottomrule
    \end{tabular}
\end{table}

总体来说,古诺尔斯语有9个对立的基本元音音素,每个元音都有一个对应地长音。要构成长音,只需要在短音字母上添加锐音符`ˊ'。但有3个例外:
\begin{info}
    \begin{enumerate}
        \item a的长音á,在12世纪的古诺尔斯语已经与ǫ́合流,所以当时的音系中并没有一个长的a /ɑ:/。
        \item æ总是长元音,没有短元音与之对应 \footnotemark
              %!!!以下条目有字母缺失,须审阅!!!%
        \item ø的长元音是œ,一般不写†\'{ø}这种字母(但手稿中也有记载)。
    \end{enumerate}
\end{info}

\footnotetext{更早期的古诺尔斯语实际上有短的/ɛ/,这是a发生i-变异(参见1.3)的结果。}

这些音素,以及对应的字母在下表中标出,后面用尖括号<>标出的是这个元音的写法。关于前元音、后元音等术语,不熟悉的读者可以参照1.3节中关于元音性质的描述。

\begin{table}[H]
    \centering
    \begin{tabular}{@{}ccccccccc@{}}
        \toprule
               & \multicolumn{4}{c}{\textbf{前元音}} & \multicolumn{4}{c}{\textbf{后元音}}                                                                                                                                                                                 \\ \cmidrule(l){2-9}
               & \multicolumn{2}{c}{非圆唇}          & \multicolumn{2}{c}{圆唇}            & \multicolumn{2}{c}{非圆唇}   & \multicolumn{2}{c}{圆唇}                                                                                                                       \\ \cmidrule(l){2-9}
        高元音 & i \textless{}i\textgreater{}        & iː \textless{}í\textgreater{}       & y \textless{}y\textgreater{} & yː \textless{}ý\textgreater{} &                              &                  & u \textless{}u\textgreater{} & uː \textless{}ú\textgreater{} \\
        中元音 & e \textless{}e\textgreater{}        & eː \textless{}é\textgreater{}       & ø \textless{}ø\textgreater{} & øː \textless{}œ\textgreater{} &                              &                  & o \textless{}o\textgreater{} & oː \textless{}ó\textgreater{} \\
        低元音 & ɛ                                   & ɛː \textless{}æ\textgreater{}       &                              &                               & a \textless{}a\textgreater{} & \textbackslash{} & ɔ \textless{}ǫ\textgreater{} & ɔː \textless{}ǫ́\textgreater{} \\ \bottomrule
    \end{tabular}
\end{table}

古诺尔斯语还有3个双元音/ɛi/, /ɔu/, /øy / 拼作 <ei>, <au>, <ey>.

低元音/ɛ/只出现在上述的双元音中。

古诺尔斯语的辅音比较规则,少数的例外一般表现为:

\begin{info}
    s/t之前的塞音会变成对应发音部位的清擦音。\footnotemark
\end{info}
\footnotetext{这个规律是原始语发生的日耳曼语擦音定律(Germanic spirant law)的残留。擦音定律和格林定律、维尔纳定律密切相关,涉及较为复杂的历史音变,请有兴趣的读者自查。}

古诺尔斯语的辅音也成对出现,双辅音与单辅音的区别仅在于前者的音值更长一些。j和v是半元音,它们的性质分别与i和u相似,在古诺尔斯语中经常发生音变(见1.4.3)。

\section{音节和重音}
\label{sec:accent}

音节是构成语音序列的单位,也是语音中最自然的语音结构单位。英文中的water就分为wa-和-ter两个音节。以英语为母语的人在拼读这个词的时候可能在t前停顿,但不可能在t后停顿,把water读成wat'er这样的形式。不同语言常有不同的音节划分规则,音节的类型也影响语言的韵律甚至是形态。

包括古诺尔斯语在内的大多数早期日耳曼语有复杂的音节划分模式,目前尚无一个统一的理论能够解释这些语言中的所有现象。语言学家主要从两个方面推测古代语言的音节划分,一是根据诗歌的规则;二是通过观察手稿中单词的记法。特别地,当一个单词出现在一行的末尾而恰好写不下时,这个词如何被拆分能够很好地反映它的音节情况。

就语法而言,古诺尔斯语的音节类型会影响部分动词(见3.3.3)和名词(见2.2.2 ja-词干名词)的变形,因此掌握音节的划分十分重要。有两种音节划分的方法可供参考,一种称为传统式,另一种称为格律式。如果读者之前没有接触过音韵学的知识,用传统式的划分已经可以很好地解决古诺尔斯语的问题(但通常这种方式和其他语言的音节划分不一样)。格律式划分和希腊语、拉丁语等划分一致,也更偏向于现代语音学的划分方法,因此介绍起来相对复杂些。

\subsection*{传统式}

古诺尔斯语中有许多单音节词,例如á, til, at, rann. 单音节词只有一个元音,可以是长元音也可以是短元音。在多音节词中,如果这个词不是合成词,那么根据元音的位置划分音节(即非词首音节总以元音开头),例如far-a, kall-a, gǫrð-um, gam-all-a, hundr-að-a。在合成词中,根据组词的语素划分音节,例如vápn-lauss (< vápn + lauss, weapon-less), vík-ing-a-hǫfð-ing-i (< víkinga + hǫfðingi, Viking's chieftain). 由于元音和辅音有长短之分,音节也被分成以下四类:

\begin{table}[H]
    \centering
    \begin{tabular}{@{}cccc@{}}
        \toprule
        \multicolumn{2}{c}{种类} & 描述 & 举例                                  \\ \midrule
        1                        & 短   & 短元音+短辅音        & bað            \\
        2                        & 长   & 短元音+辅音簇        & rann, ǫnd      \\
        3                        & 长   & 长元音+短辅音/零辅音 & hús, fé, gnúa  \\
        4                        & 加长 & 长元音+辅音簇        & nótt,   blástr \\ \bottomrule
    \end{tabular}
\end{table}
辅音簇(Consonant cluster)的意思是多个辅音的集合。

\subsection*{格律式}

格律式的划分方法与大多数音系学的理论一致,一个音节一般包括以下3个结构:
\begin{enumerate}
    \item 音节首(Onset)

          音节首总是由辅音充当。音节首可以是单辅音,也可以是多个辅音(辅音簇)。古诺尔斯语有些单音节词以元音开头,这时没有音节首。
    \item 音节核(Nucleus)

          音节核是一个响音,即可以是元音或者成音节的辅音。这是大多数语言的必有成分。
    \item 音节尾(Coda)

          由辅音充当,没有音节尾的音节是开音节,反之是闭音节。

\end{enumerate}

