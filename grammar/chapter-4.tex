\chapter{动词与变位法}

\begin{introduction}[章节要点]
    \item 古诺尔斯语的动词系统
    \item 动词的范畴
    \item 动词词干的分类
    \item 各类动词的变位法
    \item 不规则动词
\end{introduction}

\section{动词的概述}\label{sec:verb-intro}
和我们已经介绍过的名词一样,动词也有“强动词”和“弱动词”两种类型(或称“强变化”和“弱变化”动词)。不过需要注意,动词的强弱和名词的强弱指的不是同一个东西。简单来说,强动词可以理解为以元音交替(Ablaut)为特征的“不规则动词”(这里的不规则是借用英语的概念来说的,读者通过下文不难发现,这些变化实际上遵守一定的规律),这些动词的词根元音随时态和人称的变化而改变;对应地,弱动词可以类比为英语的规则动词,这些动词的词根元音在变形过程中保持不变,除非其受到人称词尾中的i-/u-变异的影响。和名词类似,动词的强弱也只有类型学上的意义,它们完全可以用“I型动词”和“Ⅱ型动词”这样的术语来代替。

我们在这里简单介绍一下元音交替的来历。元音交替是原始印欧语的重要特征,它可以用于同根词的进一步变形。根据历史比较语言学的发现,原始印欧语的词根结构通常是*CeC,即两个辅音之间包含一个元音*e. 从这个词根出发,将*e变成其他元音(如*o, *ē, *ō等)甚至使其完全脱落,以此改变词意的过程就是元音交替。更一般地来说,元音交替不仅发生在词根中,在许多后缀里也有相似的现象,因此,这个过程对印欧语的形态学具有很大的意义:在名词中,元音交替可以区分变格;在动词中它则可以区分变位。参考古希腊语中πατήρ `father'的元音交替:

\begin{longtable}{llll}
    \toprule
    \multicolumn{2}{c}{古希腊语} & 格                 & 元音         \\
    \midrule
    \endhead
    \bottomrule
    \endfoot
    πα-τ\textbf{έ}ρ-α        & pa-t\textbf{é}r-a & 单数宾格 & 短e  \\
    πα-τ\textbf{ή}ρ          & pa-t\textbf{ḗ}r   & 单数主格 & 长e  \\
    πα-τρ-ός                 & pa-tr-ós          & 单数属格 & 无元音 \\
\end{longtable}

在动词中,元音交替的现象完好地保留在梵语等更早的语言(梵语成文时期比古诺尔斯语早十余个世纪)中。在英语中,能体现元音交替是一些所谓的不规则动词:

\begin{quote}
    r\textbf{i}de r\textbf{o}de r\textbf{i}dden

    s\textbf{i}ng s\textbf{a}ng s\textbf{u}ng

    fl\textbf{y} fl\textbf{ew} fl\textbf{o}wn
\end{quote}

但是,元音交替作为一种非常古老的构词方法,在原始日耳曼语从原始印欧语分离出来的时候,已经逐渐被舍弃了,因此原始日耳曼语采用了新的方法来衍生动词,这就是“弱变化”规则(英文对应-ed式规则动词)。继承于原始印欧语的更古老的动词仍然保留了强变化规则,不过随着时间的推移,强变化愈发被人遗忘,许多历史上的强动词也归入弱动词之中。

\subsubsection*{语法范畴}
古诺尔斯语动词根据人称、时态、语态、语气发生屈折。这些与词形变化相关的语法意义的概括就是语法范畴。古诺尔斯语的动词系统具有如下的范畴:

三种\textbf{人称}(Person):第一、第二和第三人称。在每个人称中,还区分单数和复数。更古老的日耳曼语中出现的双数形式也在古诺尔斯语中消失。

两种\textbf{时态}(Tense):现在(Present)和过去(Preterite)。读者最好把二者理解为过去和非过去,因为简单来说,现在时既支配现在的动作,也支配未来的动作。相同地,古诺尔斯语的过去式同样可以和英语的若干与过去有关的时态有关。有英语基础的读者会经常联想到英语中复杂的时态系统,但事实上,这些表达严格来说叫作``体''或``体貌''(Aspect),例如英语中的完成时实际上是一种体,而非时态。时态用来区分动词在时间尺度上的位置;体貌则描述关于该动作的开始、持续、完成或重复等方面的情况,但不涉及该动作发生的时间。当然,许多动词的表达既涉及时态又涉及体貌,例如英语中的现在完成时描述了完成体,但隐含的意思是一个现在的状态。因此在许多印欧语中,这个时态(暂且仍粗略地使用这个不太准确的术语)采取动词现在时的词干并加上一些派生后缀。在古诺尔斯语中,表达体态的方法是添加助动词,而非词形屈折。

两种\textbf{语态}(Voice):主动(Active)和中动(Middle/Mediopassive)。中动态在古诺尔斯语中一方面有一定被动的含义,另一方面还表达一些反身的动作,详见(待完成)

三种\textbf{语气}(Mood):直陈(Indicative),虚拟(Subjunctive),祈使(Imperative)。直陈语气表示一般地陈述;虚拟语气主要表示可能发生但尚未发生的动作或愿望;祈使语气表示命令。

\section{强动词的变位法}
强动词的特征是元音交替。如\ref{sec:verb-intro}所述,元音的交替仍然保留在英语中,例如sing-sang-sung-song. 词干部分s-ng 加上i得现在时,加a得过去式,加u得过去分词,加o得衍生的名词。但不是每个动词的元音交替模式都是一致的,比如hang-hung-hung,它不仅采取不同的元音添加,也没有对应的衍生名词。

要了解词形的变化,必须首先搞清楚古诺尔斯语动词的最基本形式:不定式。不定式在语法上属于非限定动词的一种,即这种动词还没有人称、时态等的“限定”,但我们在这一章谈到的不定式只是动词词形的一种形式,它最简单,未经过变形,根据动词不定式能推导出动词的其他形式。有时情况下,也把这种形式称为词典形(Dictionary form),即词典上提供的基本形态。古诺尔斯语的不定式可以和以下几种语言的``不定式''\footnote{有许多语言的词典形就是属于非限定动词的不定式,但还有一些语言的词典形是限定动词,例如古希腊语常用第一人称现在时单数式作为词典形,梵语常用第三人称现在时单数式。}类比:英语eat,拉丁语portare,德语liben,日语考える。动词变化丰富的语言基本都有不定式的标记,用蓝色字体标出,蓝色之外的部分可以认为是词干。就古诺尔斯语而言,不定式的标记是-a,也有时候-a前紧跟着-j-或者-v-,称之为ja/va-不定式\footnote{j/v事实上都是词干的一部分,即不定式标记总是-a,-a之前的部分为词干。}。和ja-/jō-词干一样,j的出现有时会引起进一步的音变(西弗斯定律的作用,见\hyperref[ajawa-ux8bcdux5e72]{2.2.2}),但va-不定式一般比较规则。进一步来讲,古诺尔斯语的不定式也和英语一样有两种,第一种是单独的以-a结尾的词典形,第二种是以at+词典形引导的at不定式。这两种不同的不定式都属于非限定动词,它们的语法功能和英文中带不带to的不定式类似,都可以作某些动词的补足语等。
