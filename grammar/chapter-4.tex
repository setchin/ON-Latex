\chapter{动词与变位法}

\begin{introduction}[章节要点]
    \item 古诺尔斯语的动词系统
    \item 动词的范畴
    \item 动词词干的分类
    \item 各类动词的变位法
    \item 不规则动词
\end{introduction}

\section{动词的概述}\label{sec:verb-intro}
和我们已经介绍过的名词一样,动词也有“强动词”和“弱动词”两种类型(或称“强变化”和“弱变化”动词)。不过需要注意,动词的强弱和名词的强弱指的不是同一个东西。简单来说,强动词可以理解为以元音交替(Ablaut)为特征的“不规则动词”(这里的不规则是借用英语的概念来说的,读者通过下文不难发现,这些变化实际上遵守一定的规律),这些动词的词根元音随时态和人称的变化而改变;对应地,弱动词可以类比为英语的规则动词,这些动词的词根元音在变形过程中保持不变,除非其受到人称词尾中的i-/u-变异的影响。和名词类似,动词的强弱也只有类型学上的意义,它们完全可以用“I型动词”和“Ⅱ型动词”这样的术语来代替。

我们在这里简单介绍一下元音交替的来历。元音交替是原始印欧语的重要特征,它可以用于同根词的进一步变形。根据历史比较语言学的发现,原始印欧语的词根结构通常是*CeC,即两个辅音之间包含一个元音*e. 从这个词根出发,将*e变成其他元音(如*o, *ē, *ō等)甚至使其完全脱落,以此改变词意的过程就是元音交替。更一般地来说,元音交替不仅发生在词根中,在许多后缀里也有相似的现象,因此,这个过程对印欧语的形态学具有很大的意义:在名词中,元音交替可以区分变格;在动词中它则可以区分变位。参考古希腊语中πατήρ `father'的元音交替:

\begin{longtable}{llll}
    \toprule
    \multicolumn{2}{c}{古希腊语} & 格                 & 元音         \\
    \midrule
    \endhead
    \bottomrule
    \endfoot
    πα-τ\textbf{έ}ρ-α        & pa-t\textbf{é}r-a & 单数宾格 & 短e  \\
    πα-τ\textbf{ή}ρ          & pa-t\textbf{ḗ}r   & 单数主格 & 长e  \\
    πα-τρ-ός                 & pa-tr-ós          & 单数属格 & 无元音 \\
\end{longtable}

在动词中,元音交替的现象完好地保留在梵语等更早的语言(梵语成文时期比古诺尔斯语早十余个世纪)中。在英语中,能体现元音交替是一些所谓的不规则动词:

\begin{quote}
    r\textbf{i}de r\textbf{o}de r\textbf{i}dden

    s\textbf{i}ng s\textbf{a}ng s\textbf{u}ng

    fl\textbf{y} fl\textbf{ew} fl\textbf{o}wn
\end{quote}

但是,元音交替作为一种非常古老的构词方法,在原始日耳曼语从原始印欧语分离出来的时候,已经逐渐被舍弃了,因此原始日耳曼语采用了新的方法来衍生动词,这就是“弱变化”规则(英文对应-ed式规则动词)。继承于原始印欧语的更古老的动词仍然保留了强变化规则,不过随着时间的推移,强变化愈发被人遗忘,许多历史上的强动词也归入弱动词之中。

\subsubsection*{语法范畴}
古诺尔斯语动词根据人称、时态、语态、语气发生屈折。这些与词形变化相关的语法意义的概括就是语法范畴。古诺尔斯语的动词系统具有如下的范畴:

三种\textbf{人称}(Person):第一、第二和第三人称。在每个人称中,还区分单数和复数。更古老的日耳曼语中出现的双数形式也在古诺尔斯语中消失。

两种\textbf{时态}(Tense):现在(Present)和过去(Preterite)。读者最好把二者理解为过去和非过去,因为简单来说,现在时既支配现在的动作,也支配未来的动作。相同地,古诺尔斯语的过去式同样可以和英语的若干与过去有关的时态有关。有英语基础的读者会经常联想到英语中复杂的时态系统,但事实上,这些表达严格来说叫作``体''或``体貌''(Aspect),例如英语中的完成时实际上是一种体,而非时态。时态用来区分动词在时间尺度上的位置;体貌则描述关于该动作的开始、持续、完成或重复等方面的情况,但不涉及该动作发生的时间。当然,许多动词的表达既涉及时态又涉及体貌,例如英语中的现在完成时描述了完成体,但隐含的意思是一个现在的状态。因此在许多印欧语中,这个时态(暂且仍粗略地使用这个不太准确的术语)采取动词现在时的词干并加上一些派生后缀。在古诺尔斯语中,表达体态的方法是添加助动词,而非词形屈折。

两种\textbf{语态}(Voice):主动(Active)和中动(Middle/Mediopassive)。中动态在古诺尔斯语中一方面有一定被动的含义,另一方面还表达一些反身的动作,详见(待完成)

三种\textbf{语气}(Mood):直陈(Indicative),虚拟(Subjunctive),祈使(Imperative)。直陈语气表示一般地陈述;虚拟语气主要表示可能发生但尚未发生的动作或愿望;祈使语气表示命令。

\section{强动词的变位法}
强动词的特征是元音交替。如\ref{sec:verb-intro}所述,元音的交替仍然保留在英语中,例如sing-sang-sung-song. 词干部分s-ng 加上i得现在时,加a得过去式,加u得过去分词,加o得衍生的名词。但不是每个动词的元音交替模式都是一致的,比如hang-hung-hung,它不仅采取不同的元音添加,也没有对应的衍生名词。

在古诺尔斯语中,根据元音交替的模式(例如上面的i-a-u和a-u-u)可以将强动词分为七类。前六类动词比较规则,第七类动词则是一些历史上不太规则的动词的残留。接下来,为了描述这些强动词元音交替的模式,我们需要选择动词的一些形式作为基础。在上面英语的例子中,我们使用现在时、过去时和过去分词(英语动词的人称范畴基本退化了,因此我们没有指明人称)的词形就可以描述出i-a-u这套元音交替模式。在古诺尔斯语中,也有类似的情况。

在进一步了解词形的变化,必须首先明确古诺尔斯语动词的最基本形式:\textbf{不定式}。不定式在句法上属于非限定动词的一种,即这种动词还没有人称、时态等的“限定”,但我们在这一章从形态学角度谈到的不定式只是动词词形的一种形式。不定式通常很简单,未经过变形,并能引导我们推断出动词的其他形式。有时情况下,这种基本的形式也被称为词典形(Dictionary form),即词典上提供的基本形态。古诺尔斯语的不定式可以和以下几种语言的``不定式''\footnote{词典形是形态学的概念,而不定式是句法的概念。有许多语言的词典形就是属于非限定动词的不定式,但还有一些语言的词典形是限定动词,例如古希腊语常用第一人称现在时单数式作为词典形,梵语常用第三人称现在时单数式。只要形式简单基本,就可以作为词典形。}类比:拉丁语port\textcolor{cyan}{are},德语lib\textcolor{cyan}{en},日语考\textcolor{cyan}{える}。动词变化丰富的语言基本都有不定式的标记,用蓝色字体标出,蓝色之外的部分可以认为是词干。就古诺尔斯语而言,不定式的标记是-a,-a前面的部分是词干。词干有时以半元音-j-或者-v-结尾,半元音的出现(特别是-j-)有时会引起进一步的音变(西弗斯定律的作用,见\hyperref[ajawa-ux8bcdux5e72]{2.2.2})。
% to be done
% 进一步来讲,古诺尔斯语的不定式也和英语一样有两种,第一种是单独的以-a结尾的词典形,第二种是以at+词典形引导的at不定式。这两种不同的不定式都属于非限定动词,它们的语法功能和英文中带不带to的不定式类似,都可以作某些动词的补足语等(参见)。

下面,我们可以从前六类动词的各个形态中初步发现元音交替的模式。参考bíta `bite', skjóta `shoot', bresta `burst', bera `bear, carry', reka `drive‌', fara `go/fare‌'的变位:
\begin{longtable}{lllllll}
    \toprule
    类   & 不定式    & 三单现在时  & 三单过去时 & 三复过去时  & 三单过去虚拟式 & 过去分词     \\
    \midrule
    \endhead
    \bottomrule
    \endfoot
    I   & bíta   & bítr   & beit  & bitu   & biti    & bitinn   \\
    II  & skjóta & skýtr  & skaut & skutu  & skyti   & skotinn  \\
    III & bresta & brestr & brast & brustu & brysti  & brostinn \\
    IV  & bera   & berr   & bar   & báru   & bæri    & borinn   \\
    V   & reka   & rekr   & rak   & ráku   & ræki    & rekinn   \\
    VI  & fara   & ferr   & fór   & fóru   & fœri    & farinn   \\
\end{longtable}

通过上表可以发现:
\begin{info}
    \begin{enumerate}
        \item  单数现在时的元音要么和不定式一致,要么由它发生i-变异得到。
        \item  单数过去式的元音是独立的。
        \item  复数过去式的元音是独立的。
        \item  过去虚拟式的元音由直陈复数过去式的元音i-变异得到。
        \item  过去分词的元音是独立的。
    \end{enumerate}
\end{info}

因此,古诺尔斯语的强动词系统中有四种元音的交替。那么,最少用四个动词形式即可推断出整个变位表的形式,这四个形式称为四个基本元(Principal parts)。这四个基本元在词典上一般选用:
\begin{quote}
    第一基本元:不定式,或词典形;

    第二基本元:第三人称单数过去式;

    第三基本元:第三人称复数过去式;

    第四基本元:过去分词。
\end{quote}

有时词典中也额外标记第三人称单数现在时。例如bera在Cleasby \& Vigfússon的An Icelandic-English Dictionary上就记为:

\begin{quote}
    \textit{BERA, bar, báru, borit, pres. berr}
\end{quote}

\subsection{强动词的主动词尾}\label{强动词}
古诺尔斯语的动词分为主动词尾和中动词尾,主动态动词添加主动词尾,其含义和英文的主动态没有区别。强动词的主动词尾如下所示:
\begin{longtable}{llll}
    \toprule
    强动词   & 直陈  & 虚拟  & 祈使  \\
    \midrule
    \endhead
    \bottomrule
    \endfoot
    单数现在时 &     &     &     \\
    1     & -ø  & -a  &     \\
    2     & -r  & -ir & -ø  \\
    3     & -r  & -i  &     \\
    复数现在时 &     &     &     \\
    1     & -um & -im & -um \\
    2     & -ið & -ið & -ið \\
    3     & -a  & -i  &     \\
    单数过去时 &     &     &     \\
    1     & -ø  & -a  &     \\
    2     & -t  & -ir &     \\
    3     & -ø  & -i  &     \\
    复数过去时 &     &     &     \\
    1     & -um & -im &     \\
    2     & -uð & -ið &     \\
    3     & -u  & -i  &     \\
\end{longtable}

-ø表示无须添加词尾,表格中的空白指这种形式不存在。这完全体现在动词的祈使式上,只有单复数的现在时第二人称、复数现在时第一人称有祈使式。即祈使式可以用在下面两种情况下:

\begin{enumerate}
    \item
          要求你、你们做,参考英语 `do it!';
    \item
          提议我们做,参考英语 `let's do it!'.
\end{enumerate}

在不同时态、数的词尾前,要选取不同的基本元,其变化方式如下:

\begin{info}
    单数现在直陈式:取不定式词干,如果词干有后元音,则施加i-变异, 加词尾; \\
    复数现在直陈式/一切现在虚拟式:取不定式词干, 加词尾;          \\
    单数过去直陈式:取第二基本元词干, 加词尾;                \\
    复数过去直陈式:取第三基本元词干, 加词尾;                \\
    一切过去虚拟式:取第三基本元词干, 如果词干有后元音,则施加i-变异,
    加词尾;                                   \\
    一切祈使式:取不定式词干, 加词尾;                    \\
    现在分词:取不定式词干, 加词尾;                     \\
    过去分词:取第四基本元词干,加词尾。
\end{info}

注意,某些情况下,非圆唇元音受-um的影响有可能会发生u-变异,这是规则音变的结果。用(1), (2), (3), (4)分别标记四个基本元,变形方式如下:

\begin{longtable}{llll}
    \toprule
    强动词   & 直陈              & 虚拟               & 祈使              \\
    \midrule
    \endhead
    \bottomrule
    \endfoot
    单数现在时 &                 &                  &                 \\
    1     & (1) + (i-变异) -ø & (1) +-a          &                 \\
    2     & (1) + (i-变异) -r & (1) +-ir         & (1) +-ø         \\
    3     & (1) + (i-变异) -r & (1) +-i          &                 \\
    复数现在时 &                 &                  &                 \\
    1     & (1) +(u-变异) -um & (1) +-im         & (1) +(u-变异) -um \\
    2     & (1) +-ið        & (1) +-ið         & (1) +-ið        \\
    3     & (1) +-a         & (1) +-i          &                 \\
    单数过去时 &                 &                  &                 \\
    1     & (2) +-ø         & (3) + (i-变异) -a  &                 \\
    2     & (2) +-t         & (3) + (i-变异) -ir &                 \\
    3     & (2) +-ø         & (3) + (i-变异) -i  &                 \\
    复数过去时 &                 &                  &                 \\
    1     & (3) +-um        & (3) + (i-变异) -im &                 \\
    2     & (3) +-uð        & (3) + (i-变异) -ið &                 \\
    3     & (3) +-u         & (3) + (i-变异) -i  &                 \\
    不定式   & (1) + -a        &                  &                 \\
    现在分词  & (1) + -andi     &                  &                 \\
    过去分词  & (4) + -inn      &                  &                 \\
\end{longtable}
说明:

\begin{enumerate}
    \item
          现在时的单数直陈式中普遍地出现了i-变异,但没有任何i的痕迹。在其他西日耳曼语中,只有第二人称和第三人称出现了i-变异,例如古英语的单数现在时bēode-bīetst-
          bīetst,这表明古诺尔斯语中第一人称的i-变异是类推的影响。从中动态(见\hyperref[ux5f3aux52a8ux8bcdux7684ux4e2dux52a8ux8bcdux5c3e]{3.2.2})的词尾来看,第一人称的词尾本是*u.
          第二人称的-r由-ir演变而来,第三人称的-r也由第二人称类推得到,在卢恩铭文中记载到了早期的-iþ形式。
    \item
          绝大多数情况下(除了过去虚拟式)词尾-ið不造成i-变异。在原始日耳曼语中,这个词尾已经是*-id了,其不能造成i-变异应当是受同时态的其他词形的影响。
    \item
          \phantomsection\label{_Ref116919964}{}虚拟式的词尾都是一样的,但现在虚拟式和过去虚拟式由词干的i-变异所区分。在原始日耳曼语中,现在时的词尾包含一个双元音*ai(相当于比过去式的词尾多了*-a-),双元音发生缩略变为*e,后来抬升为i.
          这样,例如第二人称现在时词尾*-aiz \textgreater{} *-ez \textgreater{}
          -ir就没有i-变异条件。在过去式中则不存在这样的双元音,i-变异正常发生。无疑,类推作用使得过去式中某些本不能造成i-变异的词尾也引起了元音变异,这样才能与现在时区分开来。
    \item
          -um词尾比较规则地引起u-变异,这和名词中的情况类似。复数过去直陈式的词尾都含有-u,理论上都能造成u-变异,但是古诺尔斯语中保留下来的复数过去式词干没有可以发生u-变异的元音。
\end{enumerate}