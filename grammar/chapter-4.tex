\chapter{形容词与变格法}\label{形容词与变格法}

\begin{introduction}[章节要点]
    \item 形容词性词尾
    \item 形容词的强弱变格法及其含义
    \item 形容词的级
    \item 副词的构成与变格法
    \item 分词的构成与变格法
\end{introduction}

\section{形容词的概述}\label{形容词的概述}

形容词最常见的用法有两种:作定语或作表语,但无论是哪一种,形容词的存在都与一个名词紧密关联(作定语时,与其所修饰的名词;作表语时,与作主语的名词)。名词的四个基本范畴是:格、性、数、特指性,因而形容词的各项属性要和名词保持一一对应。

形容词可以按格、性、数进行变化,这一点与大多数印欧语都是类似的,但在古诺尔斯语中,每个形容词还有强变格和弱变格两种形式。一般来说,当形容词和非特指名词发生关系时,要用其强变格形式;反之,如果句中出现的是特指名词,对应地就要使用弱变格的形容词。因此,形容词的强弱实际上是与名词的特指性相呼应的语法范畴。类似于名词,形容词的强变格和弱变格分别对应了一套依据格、性、数变化的词尾,但与名词不同的是,\textbf{名词的强弱是其固有属性},每个名词只能接续一套词尾;而\textbf{形容词的强弱是一种可变的属性},它在一定程度上反映的是语义的区别,每个形容词都既可以按强变格法变形,也可以按弱变格法变形。

古诺尔斯语的形容词也有“级”的概念,分别是原级、比较级、最高级。前文所述的变形方式是针对形容词的基础形态——原级的,比较级和最高级的变形则略有不同。它们先在词干后添加一个后缀-ar-(比较级)/-ast-(最高级),然后再添加词尾。对于比较级而言,其变格有一套专门的词尾,形式上和原级的弱变格词尾非常相似。所有形容词的比较级无论是否修饰了特指名词一律都按弱变格词尾变形,即\textbf{形容词的比较级不区分强弱}。形容词的最高级没有专门的词尾,它和原级一样可以按照强变格和弱变格进行变化。

和许多古代印欧语类似,形容词的词尾和名词的词尾也有很大的相似性。读者可能会联想到,形容词是否也像名词一样区分词干的类型呢?在原始日耳曼语中,词干的类型的确影响形容词的屈折,但在古诺尔斯语中,形容词的变形已经大大规则化了。因此,没有必要介绍形容词原来的词干形式。

除形容词本身外,还有一些词类也添加形容词的词尾,例如动词的现在分词和过去分词、代词(交叉引用)等,这些词在语义和语法上并非严格的形容词。因此,我们把形容词所添加的词尾称为“形容词性词尾”,这些词尾以后将经常用到。

% to do: check whether to keep it in grammar or syntax
形容词作表语时,要用其强变格形式。这时候的形容词就是形容词性的,但古诺尔斯语也可以把形容词作名词用,此时的强弱就要根据句义中名词是否特指来决定了。在诗歌中,作定语的形容词几乎都是强的,即便其所修饰的是特指名词,如:
fyrr vil ek kyssa konung ólifðan, en þú blóðogri bryniu kastir `sooner
will I kiss the lifeless king, than you cast off the bloody byrnie‌'.
这种用法并不局限于诗歌,事实上在散文中也不罕见: með fǫður sinn gamlan
`with his old
father‌'.这里反身代词sinn使fǫður成为特指词,但形容词依然使用了强形式。有时在绰号之类中也用强形容词,当然弱形容词更常用一些,比如:
Eiríkr rauðr `Erik the Red‌'. 散文中,在表示呼唤的情况下也常用强形容词:
forða þér, vesall maðr! `save yourself, unfortunate man!‌'

古诺尔斯语中的一些形容词承担了英语中需要使用短语的情况,这些形容词是强的,即便用上了特指冠词/后缀也不例外,如:í
miðjum hauginum `in the middle of the hill‌'; um þueran skálann `straight
across the hall‌'; ǫndorðan vetr `the first part of the winter‌'.
形容词甚至可以修饰代词,如þeir margir `many of them‌'.

诗歌和某些小众风格的文章经常把强形容词作名词用,如: blindr er betri, en
brendr sé `a blind (man) is better off than a burned (man) would be‌';
hvat muntu, ríkr, vinna `what would you tell, powerful (man)‌'; rétt
`Right (vs.
Wrong)‌'。作名词用时,弱形容词和定冠词一道使用是很少见的(我们已经说过,冠词常常省略),在诗歌中也很难见到。散文中这种用法局限于表达专有名词,如:inir
ensku `the English‌'. 这种用法在名词序列中也可以使用,如:inn yngri, inn
ellri `the younger, the older‌'.

在散文中,也可以在人名后加上表褒贬的修饰形容词,这时不加定冠词使用,
如:Hákon góði `Hakon the Good‌'. 类似的用法还有: fyrra sumar `last
summer‌'; á vinstri hlið `on the left side‌'; í næsta hús `in the next
house‌'; við þriðja mann `with the third man‌'.

\section{形容词的强变格法}\label{形容词的强变格法}

形容词的强变格词尾和强名词的词尾有一定相似之处,它们列举在下表。其中,一切以u开头的词尾都会导致前方的元音发生u-变异,某些词尾的u可能已经脱去,但u-变异仍旧存在。这种情况下,另标在表中:

\begin{longtable}{llll}
    \toprule
    性   & 阳性 & 阴性              & 中性              \\
    \midrule
    \endhead
    \bottomrule
    \endfoot
    单数 &      &                   &                   \\
    N    & -r   & (词干u-变异) + -ø & -t                \\
    A    & -an  & -a                & -t                \\
    G    & -s   & -rar              & -s                \\
    D    & -um  & -ri               & -u                \\
    复数 &      &                   &                   \\
    N    & -ir  & -ar               & (词干u-变异) + -ø \\
    A    & -a   & -ar               & (词干u-变异) + -ø \\
    G    & -ra  & -ra               & -ra               \\
    D    & -um  & -um               & -um               \\
\end{longtable}

一些规律可供参考:

\begin{info}
    \begin{enumerate}
        \item -r是阳性单数主格的标记,这也作为形容词的基本形式。
        \item 复数属格都是-ra,复数与格都是-um.
        \item 单数属格的词尾和名词类似,但阴性词尾是-rar.
        \item 中性名词不论单数复数,主格和宾格总一样。
    \end{enumerate}
\end{info}

以sterkr `strong‌', vænn `handsome‌', gamall `old‌', nýr `new‌', frægr
`famous‌', rǫskr `brave‌', fagr `beautiful, fair'
为例,它们的阳性形式如下:

\begin{longtable}{llllllll}
    \toprule
    词干 & sterk-  & væn-  & gamal-  & fagr-  & nýj-    & frægj-  & rǫskv-  \\
    \midrule
    \endhead
    \bottomrule
    \endfoot
    单数 &         &       &         &        &         &         &         \\
    N    & sterkr  & vænn  & gamall  & fagr   & nýr     & frægr   & rǫskr   \\
    A    & sterkan & vænan & gamlan  & fagran & nýjan   & frægjan & rǫskvan \\
    G    & sterks  & væns  & gamals  & fagrs  & nýs     & frægs   & rǫsks   \\
    D    & sterkum & vænum & gǫmlum  & fǫgrum & nýjum   & frægjum & rǫskum  \\
    复数 &         &       &         &        &         &         &         \\
    N    & sterkir & vænir & gamlir  & fagrir & nýir    & frægir  & rǫskvir \\
    A    & sterka  & væna  & gamla   & fagra  & nýja    & frægja  & rǫskva  \\
    G    & sterkra & vænna & gamalla & fagra  & nýr(r)a & frægra  & rǫskra  \\
    D    & sterkum & vænum & gǫmlum  & fǫgrum & nýjum   & frægjum & rǫskum  \\
\end{longtable}

其阴性形式如下:

\begin{longtable}{llllllll}
    \toprule
    词干 & sterk-   & væn-   & gamal-   & fagr-  & nýj-     & frægj-  & rǫskv-  \\
    \midrule
    \endhead
    \bottomrule
    \endfoot
    单数 &          &        &          &        &          &         &         \\
    N    & sterk    & væn    & gǫmul    & fǫgr   & ný       & fræg    & rǫsk    \\
    A    & sterka   & væna   & gamla    & fagra  & nýja     & frægja  & rǫskva  \\
    G    & sterkrar & vænnar & gamallar & fagrar & nýr(r)ar & frægrar & rǫskrar \\
    D    & sterkri  & vænni  & gamalli  & fagri  & nýrri    & frægri  & rǫskri  \\
    复数 &          &        &          &        &          &         &         \\
    N    & sterkar  & vænar  & gamlar   & fagrar & nýjar    & frægjar & rǫskvar \\
    A    & sterkar  & vænar  & gamlar   & fagrar & nýjar    & frægjar & rǫskvar \\
    G    & sterkra  & vænna  & gamalla  & fagra  & nýr(r)a  & frægra  & rǫskra  \\
    D    & sterkum  & vænum  & gǫmlum   & fǫgrum & nýjum    & frægjum & rǫskum  \\
\end{longtable}

其中性形式如下:

\begin{longtable}{llllllll}
    \toprule
    词干 & sterk-  & væn-  & gamal-  & fagr-  & nýj-    & frægj-  & rǫskv- \\
    \midrule
    \endhead
    \bottomrule
    \endfoot
    单数 &         &       &         &        &         &         &        \\
    N    & sterkt  & vænt  & gamalt  & fagrt  & nýtt    & frægt   & rǫskt  \\
    A    & sterkt  & vænt  & gamalt  & fagrt  & nýtt    & frægt   & rǫskt  \\
    G    & sterks  & væns  & gamals  & fagrs  & nýs     & frægs   & rǫsks  \\
    D    & sterku  & vænu  & gǫmlu   & fǫgru  & nýju    & frægju  & rǫsku  \\
    复数 &         &       &         &        &         &         &        \\
    N    & sterk   & væn   & gǫmul   & fǫgr   & ný      & fræg    & rǫsk   \\
    A    & sterk   & væn   & gǫmul   & fǫgr   & ný      & fræg    & rǫsk   \\
    G    & sterkra & vænna & gamalla & fagra  & nýr(r)a & frægra  & rǫskra \\
    D    & sterkum & vænum & gǫmlum  & fǫgrum & nýjum   & frægjum & rǫskum \\
\end{longtable}

说明:

\begin{enumerate}
    \item
          vænn和gamall中出现了大量的同化现象,对于所有以-r起首的词尾,都触发\ref{辅音的音变}的辅音同化。
    \item
          gamall代表了典型的多音节形容词的变格。其第二个音节的弱读元音a在以元音开头的词尾前省略,但在辅音前完全保留。古诺尔斯语中多音节的非派生形容词本身很少,除了gamall之外,常见的还有heilagr `holy'等。但是,绝大多数由名词、动词等派生出的形容词都是多音节的,这是因为它们包含了派生词缀。这些形容词并不都符合元音省略的规律。例如由-ligr, -aðr后缀派生的形容词总是不发生省略,但是由-ull, -igr, -ugr构成的派生词一般都发生省略。双写辅音后的元音也不发生省略,试比较:

          %   to do: heilagr is not derivative?
          \begin{quote}
              samligr `friendly': sam-lig- + -an > samligan

              auðigr `rich': auð-ig- + -um > auðgum

              minnigr `mindful': minn-ig- + -a > minniga
          \end{quote}

    \item
          部分形容词的词干上有增音-j-或-v-,它们的出现条件完全符合\ref{半元音的保持性}中的规律。另外,-j-和-v-分别触发了整个词干中
          的i-变异或u-变异,这可以提示我们,rǫskr的词干是包括-v-的。
    \item
          nýr展示了长音节形容词的变格,添加辅音词尾时,触发了\ref{辅音的音变}的辅音延长。一些早期的文本中nýra也是可行的变格,但后来逐渐都变为更规则的nýrra,但是nýtt中的中性词尾-t始终双写。
    \item
          某些形容词词干本身以-r结尾,如fagr,这时再添加辅音词尾时触发\ref{辅音的音变}的辅音简化。因此其阳性单数主格形式不能区分词干与词尾。
\end{enumerate}

\section{形容词的弱变格法}\label{形容词的弱变格法}

形容词的弱变格法词尾比较简单,它们的单数形式和名词的弱变格词尾完全一致,复数形式虽然与名词有区别,但更加简单且各个性都完全一样:

\begin{longtable}{llll}
    \toprule
    性   & 阳性 & 阴性 & 中性 \\
    \midrule
    \endhead
    \bottomrule
    \endfoot
    单数 &      &      &      \\
    N    & -i   & -a   & -a   \\
    A    & -a   & -u   & -a   \\
    G    & -a   & -u   & -a   \\
    D    & -a   & -u   & -a   \\
    复数 &      &      &      \\
    N    & -u   & -u   & -u   \\
    A    & -u   & -u   & -u   \\
    G    & -u   & -u   & -u   \\
    D    & -um  & -um  & -um  \\
\end{longtable}

我们仍然使用上面已经提到的几个十分常见的形容词作为例子,比较它们的弱变格阳性形式:

\begin{longtable}{llllllll}
    \toprule
    词干 & sterk-  & væn-  & gamal- & fagr-  & nýj-  & frægj-  & rǫskv- \\
    \midrule
    \endhead
    \bottomrule
    \endfoot
    单数 &         &       &        &        &       &         &        \\
    N    & sterki  & væni  & gamli  & fagri  & ný    & frægi   & rǫskvi \\
    A    & sterka  & væna  & gamla  & fagra  & nýja  & frægja  & rǫskva \\
    G    & sterka  & væna  & gamla  & fagra  & nýja  & frægja  & rǫskva \\
    D    & sterka  & væna  & gamla  & fagra  & nýja  & frægja  & rǫskva \\
    复数 &         &       &        &        &       &         &        \\
    N    & sterku  & vænu  & gǫmlu  & fǫgru  & nýju  & frægju  & rǫsku  \\
    A    & sterku  & vænu  & gǫmlu  & fǫgru  & nýju  & frægju  & rǫsku  \\
    G    & sterku  & vænu  & gǫmlu  & fǫgru  & nýju  & frægju  & rǫsku  \\
    D    & sterkum & vænum & gǫmlum & fǫgrum & nýjum & frægjum & rǫskum \\
\end{longtable}

其阴性形式如下:

\begin{longtable}{llllllll}
    \toprule
    词干 & sterk-  & væn-  & gamal- & fagr-  & nýj-  & frægj-  & rǫskv- \\
    \midrule
    \endhead
    \bottomrule
    \endfoot
    单数 &         &       &        &        &       &         &        \\
    N    & sterka  & væna  & gamla  & fagra  & nýja  & frægja  & rǫskva \\
    A    & sterku  & vænu  & gǫmlu  & fǫgru  & nýju  & frægju  & rǫsku  \\
    G    & sterku  & vænu  & gǫmlu  & fǫgru  & nýju  & frægju  & rǫsku  \\
    D    & sterku  & vænu  & gǫmlu  & fǫgru  & nýju  & frægju  & rǫsku  \\
    复数 &         &       &        &        &       &         &        \\
    N    & sterku  & vænu  & gǫmlu  & fǫgru  & nýju  & frægju  & rǫsku  \\
    A    & sterku  & vænu  & gǫmlu  & fǫgru  & nýju  & frægju  & rǫsku  \\
    G    & sterku  & vænu  & gǫmlu  & fǫgru  & nýju  & frægju  & rǫsku  \\
    D    & sterkum & vænum & gǫmlum & fǫgrum & nýjum & frægjum & rǫskum \\
\end{longtable}

其中性形式如下:

\begin{longtable}{llllllll}
    \toprule
    词干 & sterk-  & væn-  & gamal- & fagr-  & nýj-  & frægj-  & rǫskv- \\
    \midrule
    \endhead
    \bottomrule
    \endfoot
    单数 &         &       &        &        &       &         &        \\
    N    & sterka  & væna  & gamla  & fagra  & nýja  & frægja  & rǫskva \\
    A    & sterka  & væna  & gamla  & fagra  & nýja  & frægja  & rǫskva \\
    G    & sterka  & væna  & gamla  & fagra  & nýja  & frægja  & rǫskva \\
    D    & sterka  & væna  & gamla  & fagra  & nýja  & frægja  & rǫskva \\
    复数 &         &       &        &        &       &         &        \\
    N    & sterku  & vænu  & gǫmlu  & fǫgru  & nýju  & frægju  & rǫsku  \\
    A    & sterku  & vænu  & gǫmlu  & fǫgru  & nýju  & frægju  & rǫsku  \\
    G    & sterku  & vænu  & gǫmlu  & fǫgru  & nýju  & frægju  & rǫsku  \\
    D    & sterkum & vænum & gǫmlum & fǫgrum & nýjum & frægjum & rǫskum \\
\end{longtable}

\section{形容词的比较级和最高级}\label{形容词的比较级和最高级}

绝大多数形容词通过在词干后添加-ar-,进一步添加词尾来形成比较级。最高级的构成与之类似,将-ar-换成-ast-,添加对应的词尾的即可。

最高级添加的词尾与原级相同,它可以是强的,也可以是弱的;比较级只添加弱词尾,且与原级略有区别。它们的阴性单数以及整个复数(除与格外)都与原级不同。比较级的词尾如下所示:

\begin{longtable}{llll}
    \toprule
    性   & 阳性 & 阴性 & 中性 \\
    \midrule
    \endhead
    \bottomrule
    \endfoot
    单数 &      &      &      \\
    N    & -i   & -i   & -a   \\
    A    & -a   & -i   & -a   \\
    G    & -a   & -i   & -a   \\
    D    & -a   & -i   & -a   \\
    复数 &      &      &      \\
    N    & -i   & -i   & -i   \\
    A    & -i   & -i   & -i   \\
    G    & -i   & -i   & -i   \\
    D    & -um  & -um  & -um  \\
\end{longtable}

在形态上,形容词比较级只添加弱词尾,但并不表示在语义上所有的形容词比较级都只修饰特指名词。修饰非特指名词一样可以使用形容词比较级,只不过不使用额外的词尾。

形容词hvass `sharp‌'的比较级完整变形如下:

\begin{longtable}{llll}
    \toprule
    性   & 阳性      & 阴性      & 中性      \\
    \midrule
    \endhead
    \bottomrule
    \endfoot
    单数 &           &           &           \\
    N    & hvassari  & hvassari  & hvassara  \\
    A    & hvassara  & hvassari  & hvassara  \\
    G    & hvassara  & hvassari  & hvassara  \\
    D    & hvassara  & hvassari  & hvassara  \\
    复数 &           &           &           \\
    N    & hvassari  & hvassari  & hvassari  \\
    A    & hvassari  & hvassari  & hvassari  \\
    G    & hvassari  & hvassari  & hvassari  \\
    D    & hvǫssurum & hvǫssurum & hvǫssurum \\
\end{longtable}
多音节形容词的比较级常常引起元音省略,使得词缀-ar-都变成了变成-r,例如:
\begin{quote}
    auðig- + -ar- + -i > auðigri
\end{quote}

形容词的最高级遵循规则的变格法,即在插入-ast-后根据形容词的强弱添加强变格或弱变格词尾。值得注意的是,有时受u-变异影响,-ast-会变成-ust-, 后者进一步触发词根元音的u-变异,类似于二类弱动词中的连续音变。例如,hvass的最高级为:

强变格形式:

\begin{longtable}{llll}
    \toprule
    性   & 阳性       & 阴性        & 中性       \\
    \midrule
    \endhead
    \bottomrule
    \endfoot
    单数 &            &             &            \\
    N    & hvassastr  & hvǫssust    & hvassast   \\
    A    & hvassastan & hvassasta   & hvassast   \\
    G    & hvassasts  & hvassastrar & hvassasts  \\
    D    & hvǫssustum & hvassastri  & hvǫssustu  \\
    复数 &            &             &            \\
    N    & hvassastir & hvassastar  & hvǫssust   \\
    A    & hvassasta  & hvassastar  & hvǫssust   \\
    G    & hvassastra & hvassastra  & hvassastra \\
    D    & hvǫssustum & hvǫssustum  & hvǫssustum \\
\end{longtable}

弱变格形式:

\begin{longtable}{llll}
    \toprule
    性   & 阳性       & 阴性       & 中性       \\
    \midrule
    \endhead
    \bottomrule
    \endfoot
    单数 &            &            &            \\
    N    & hvassasti  & hvassasta  & hvassasta  \\
    A    & hvassasta  & hvǫssustu  & hvassasta  \\
    G    & hvassasta  & hvǫssustu  & hvassasta  \\
    D    & hvassasta  & hvǫssustu  & hvassasta  \\
    复数 &            &            &            \\
    N    & hvǫssustu  & hvǫssustu  & hvǫssustu  \\
    A    & hvǫssustu  & hvǫssustu  & hvǫssustu  \\
    G    & hvǫssustu  & hvǫssustu  & hvǫssustu  \\
    D    & hvǫssustum & hvǫssustum & hvǫssustum \\
\end{longtable}

多音节形容词的最高级很少缩略-ast-,但有时缩略词干的弱读元音,例如:

\begin{quote}
    auðig- + -ast- + -r > auðgastr
\end{quote}

当然,不发生省略的形式也经常出现,特别是在近现代冰岛语中,省略的现象大大减少了。

另有一部分形容词通过插入-r-或-st-来分别构成比较级和最高级,同时词根元音发生i-变异\footnote{造成i-变异的原因是词尾-r-/-st-来自于更早的*-iz-/*-ist-,而一般的-ar-/-ast-则来自于*-ōz-/*-ōst-,这两套词尾接续的词干不同。前者可以适用于各类词干,但后者只用于最常见的a-/ō-词干形容词(试比较古诺尔斯语a-词干/ō-词干名词,它们的词干元音都合并为-a,见\ref{ō/jō/wō-词干}),由于a-/ō-词干形容词的广泛性,*-iz-/*-ist-逐渐被弃用,因此只留下了非常少的痕迹。}。这些形容词的词尾和一般的比较级和最高级一致。许多常用的形容词都属于这一类,如下所示:

\begin{longtable}{llll}
    \toprule
    词干           & 原级   & 比较级  & 最高级 (弱,强)     \\
    \midrule
    \endhead
    \bottomrule
    \endfoot
    fá- `few‌'      & fár    & færi    & fæsti, fæstr       \\
    fagr- `fair‌'   & fagr   & fegri   & fegrsti, fegrstr   \\
    há- `high‌'     & hár    & hæri    & hæsti, hæstr       \\
    lág- `low‌'     & lágr   & lægri   & lægsti, lægstr     \\
    sein- `late‌'   & seinn  & seinni  & seinsti, seinstr   \\
    skamm- `short‌' & skammr & skemmri & skemmsti, skemmstr \\
    smá- `small‌'   & smár   & smæri   & smæsti, smæstr     \\
    stór- `big‌'    & stórr  & stœri   & stœrsti, stœrstr   \\
    lang- `long‌'   & langr  & lengri  & lengsti, lengstr   \\
    ung- `young‌'   & ungr   & yngri   & yngsti, yngstr     \\
\end{longtable}

注意:fár, hár, smár的单数阳性主格词尾-r没有被延长,但是其他以r开头的词尾都被延长,例如fárri, fárrar等。

有些形容词将两种方式混合使用,这意味着这类词的比较级和最高级都有两写。例如:

\begin{quote}
    \begin{tabular}{lrl}
        djúpr ‘deep‌’   & >  & djúpari,	djúpasti,	djúpastr    \\
                       & 或 & dýpri,	dýpsti,	dýpstr          \\
        frægr ‘famous‌’ & >  & frægjari,	frægjasti,	frægjastr \\
                       & 或 & frægri,	frægsti,	frægstr
    \end{tabular}
\end{quote}

\section{不规则形容词}\label{不规则形容词}

古诺尔斯语有三类不规则形容词。第一类形容词形态上比较异常,这类形容词有且仅有一个:annarr; 第二类形容词的原级、比较级和最高级采用了不同的词干,类似于英语中good---better---best;第三类形容词缺少原级,只有比较级和最高级形式,它们是一些副词的派生词。


\subsubsection{不规则形容词annarr}

annarr `other, another; second, next‌'是古诺尔斯语中最不规则的形容词。它不仅在形态上十分特殊,而且缺少比较级和最高级。annarr只按原级的强变格变形,没有弱变格形式,其强变形同时承担了修饰特指和非特指名词的功能。它的变格中有词干ann-和aðr-的交替,这是因为在古诺尔斯语中,-nn-有时在-r前变为-ð-,另见maðr的变格(参见\ref{辅音词干})。annarr完整的变格形式如下:% to do: better ref

\begin{longtable}{llll}
    \toprule
    性   & 阳性    & 阴性     & 中性    \\
    \midrule
    \endhead
    \bottomrule
    \endfoot
    单数 &         &          &         \\
    N    & annarr  & ǫnnur    & annat   \\
    A    & annan   & aðra     & annat   \\
    G    & annars  & annarrar & annars  \\
    D    & ǫðrum   & annarri  & ǫðrum   \\
    复数 &         &          &         \\
    N    & aðrir   & aðrar    & ǫnnur   \\
    A    & aðra    & aðrar    & ǫnnur   \\
    G    & annarra & annarra  & annarra \\
    D    & ǫðrum   & ǫðrum    & ǫðrum   \\
\end{longtable}


\subsubsection{异干互补形容词}

这类形容词的原级、比较级和最高级采用了不同的词干,如下所示:

\begin{longtable}{lll}
    \toprule
    原级词干          & 比较级词干   & 最高级词干 \\
    \midrule
    \endhead
    \bottomrule
    \endfoot
    góð- `good‌'       & betr-        & bezt-      \\
    ill-, vánd- `bad‌' & verr-        & verst-     \\
    mikil- `great‌'    & meir-        & mest-      \\
    lítil- `little‌'   & minn-        & minnst-    \\
    marg- `many‌'      & fleir-       & flest-     \\
    gamal - `old‌'     & eldr-, ellr- & elzt-      \\
\end{longtable}

这些形容词基本都有英语的对应,它们在英语中也是异干互补的。异干互补形容词的比较级和最高级是相对规则的,它们的标志一般是更古老的-r-/-st-词尾,-st-有时和齿音合写为-zt-. 另外,比较语言学的研究表明这些形容词的比较级和最高级反而比较容易在其他印欧语中找到同源词。因此,与其认为这些形容词的比较级和最高级是不规则的创新,更有可能的情况是,这些形容词过去可能是基于比较级/最高级词干的规则形容词,但是新产生的形容词取代了它们的原级。

在词形变化上,有一些值得注意的问题:

\begin{enumerate}
    \item
          góðr的单数中性主格和宾格有同化现象,另外还有元音缩短:góð- + -t
          > gótt > gott,其他形式是规则的。
    \item
          ill-不能再和-r同化,因此阳性单数主格就是illr,illr的变形完全规则。
    \item
          mikill的变格有一些不规则之处。其单数阳性宾格是mikinn而非†mikilan,单数中性主格和宾格是mikit而非†mikilt.
          另外,作为双音节形容词,mikill也规则地发生弱读元音省略,得到如miklum,
          miklir这样的形式。
    \item
          lítill的变格类似于mikill,也有单数阳性宾格lítinn,单数中性主格和宾格lítit.
          同样地,作为一个双音节形容词,它也符合元音省略的条件,但是在这些发生省略的形式中,长元音í缩短为i,因此有litlum,
          litlir这样的形式。
    \item
          margr的单数中性主格和宾格是mart而非†margt.
    \item
          gamall是双音节形容词,也发生元音缩略。
\end{enumerate}

\subsubsection{不完全变化形容词}

一些形容词缺少原级,只有比较级和最高级的形式,这类形容词被称为“不完全变化的”(Defective)。它们大多从一些副词派生而来,其语义就是对应副词含义的比较级和最高级。读者可以理解为这些副词``承担''了这类形容词的原级:

\begin{longtable}{lll}
    \toprule
    副词                & 比较级词干      & 最高级词干       \\
    \midrule
    \endhead
    \bottomrule
    \endfoot
    aptr `back'         & aptar-, eptr-   & aptast-, epzt-   \\
    austr `east‌'        & eystr-          & austast-         \\
    fyrr `before‌'       & fyrr-           & fyrst-           \\
    *hinder `behind'    & hindr-          & hinzt-           \\
    inn `in, into‌'      & innr-           & innst-           \\
    niðr `down'         & neðr-           & nezt-            \\
    norðr `north‌'       & norðar-, nyrðr- & norðast-, nyrztr \\
    suðr, sunnr `south' & syðr-           & synst-           \\
    vestr `west'        & vestr-          & vestast-         \\
    út `out'            & ýtr-            & ýzt-             \\
    nær `near'          & nær-            & næst-            \\
\end{longtable}

说明:

\begin{enumerate}
    \item
          hindri,
          hinztr来源于原始日耳曼语副词*hinder,相当于德语hinter,英语hind(参考behind),但是这个词在古诺尔斯语中没有保留为副词,但有意义相关的名词hindr
          `hindrance'.
    \item
          norðr的两种变形有可能从syðri, synstr类比得来。
    \item
          副词suðr和sunnr的两写造成了比较级和最高级词干的不同,其中sunnr是更早的形式。
\end{enumerate}

\section{副词}\label{副词}

副词是一种与形容词密切相关的词类,它大多数是由形容词派生的,但缺少格、性、数的屈折(副词也有级)。本书先介绍副词的构词法,然后介绍一些含义特殊的副词,最后介绍副词的比较级和最高级。

\subsection{副词的构成}\label{副词的构成}

古诺尔斯语只有一小部分词天然属于副词,如mjǫk `very', svá `thus', `so', þá `then', vel `well',其他绝大多数副词都由形容词派生而来。古诺尔斯语有以下几种常见的派生方法:

\begin{enumerate}
    \item
          使用强变格单数中性宾格作副词,这种构词法常用于含义最基本的形容词,例如:

          \begin{quote}
              mikill > mikit `much, greatly‌'

              lágr > lágt `low, softly‌'

              allr > allt `all the way‌'

              hár > hátt `highly‌'
          \end{quote}

    \item
          添加后缀-a. 这种构词法也是最基本,最普遍的做法,例如:

          \begin{quote}
              illr > illa `badly‌'

              gjarn\footnote{gjarn的强变格中,以r开头的词尾与n同化,接着发生辅音简化脱落。因此有*gjarnn
                  > gjarn.} > gjarna `eagerly‌'

              许多形容词由-ligr词尾派生,这时用-liga构成副词。-liga有时还加到本身不包含-ligr的形容词上,例如:

              harðligr > harðliga `fiercely‌'

              varligar > varliga `scarcely‌'

              glǫggr > glǫggliga `clearly'

              这些副词有时也脱去-ig-,缩短为harðla, varla等。
          \end{quote}

    \item
          其他格,用形容词或名词的其它格作副词的情况比较少见,有时副词的含义也发生改变。这些副词都是固定的用法:

          \begin{quote}
              宾格: megin `side(s)‌' < vegr `way‌'\footnote{megin的m并不是vegr的一部分,而是前一个词的与格词尾被重解的结果,例如þeim megin `on that side' < *þeim veginn}

              属格: alls `of all, at all‌'; stundar `very, quite‌' < stund `time. hour'

              与格: miklu `much, by far‌'; stórum `hugely‌'; næstum `the last time‌' < næstr `nearst'; stundum `sometimes‌'

          \end{quote}
\end{enumerate}

\subsection{肯定副词与否定副词}\label{肯定副词与否定副词}

相当于现代英语yes和no的副词是já和nei. 其他表否定的副词常用后缀-gi构成,如eigi `not‌'; engi `no, not any‌'; hvergi `nowhere, not at all‌'; aldri `never‌' < aldregi. 否定动词的副词可以用eigi, ekki或né. 早期诗歌还常用-a, -at作为动词的否定后缀,例如:
\begin{quote}
    vara `was not' < var + -a

    kannat `knows not' < kann + -at

    vaska `I was not‌' < vas ek + -a
\end{quote}

né有时还可以和这些否定后缀一起使用: sofa né má-k-at `I cannot sleep‌'.

\subsection{方位性副词}\label{方位性副词}

方位性副词通过下列一些后缀构成:

\begin{enumerate}
    \item
          后缀-i. 指示静态的位置:inn `into‌' vs. inni `inside, within‌'
    \item
          后缀-an. 指示从某个位置来:innan `from within‌'
    \item
          后缀-gat/-nig. 指示到某个位置去:hingat/hinnig `to here‌', þangat `to
          there‌'
\end{enumerate}


特别值得注意的是-an型副词,它有两个衍生的用法:
\begin{enumerate}
    \item 和介词fyrir `before, in front‌'连用作为一个介词性的词组,接续宾格表示“在...的...方位”:fyrir vestan
          valhǫll `in the west of Valhalla'

    \item 接续一个属格名词,表示“比...更偏向...”:austan lands `east of coast‌'
\end{enumerate}

% to do 方向性副词除了单独使用外,还经常和介词短语连用,详见(交叉引用)。

\subsection{副词的比较级和最高级}\label{副词的比较级和最高级}

副词的比较级和最高级采用类似于形容词的规则变化,将后缀-ar-/-ast-或者-r-/-st-添加在副词的原型后面构成比较级或最高级。用-r-/-st-构成比较级和最高级时,词根元音也常发生i-变异(但也有不发生i-变异的形式)。有些时候-ar-也可以变成-arr-, -ast-则变为-arst-, 例如:

\begin{quote}
    lengi `for a long time' > lengr, lengst

    opt `often' > optar(r), opta(r)st

    framt `forward' > fremr, fremst或framr(r), frama(r)st
\end{quote}

对于衍生于形容词的中性主格(宾格)的副词,它们的比较级和最高级沿用形容词的比较级和最高级形式,例如:
\begin{quote}
    skjótr > skjótt `swiftly' > skjótara, skjótast `more swift(ly), most swift(ly)'
\end{quote}

少数副词的比较级/最高级也会采用异干互补的构词方式,列举如下:

\begin{longtable}{lll}
    \toprule
    原型               & 比较级      & 最高级 \\
    \midrule
    \endhead
    \bottomrule
    \endfoot
    lítt `little‌'      & minnr, miðr & minst  \\
    mjǫk `much‌'        & meir(r)     & mest   \\
    vel `well‌'         & betr        & bezt   \\
    illa `badly‌'       & verr        & verst  \\
    gjarna `willingly‌' & heldr       & helzt  \\
\end{longtable}

总体来说,古诺尔斯语的副词没有像形容词那样明确的规则,一个形容词有时可以派生出多种形式甚至意义不同的副词,副词的比较级和最高级也可能有不同的形式。不过副词的不规则性一般不造成判读的障碍。

\section{分词}\label{分词}

动词的现在分词和过去分词都按形容词变化。分词可以像形容词一样修饰名词,这种用法和英语非常类似,例如logandi brandr `burning brand‌'.
但是总的来说,古诺尔斯语更倾向于用从句来改写这种使用分词的情况。

\subsection{现在分词}\label{现在分词}

无论是强动词还是弱动词,现在分词都通过在动词不定式的基础上加-nd-构成。它添加形容词的比较级词尾,即只有弱变化形式。

例如,sofa `sleep‌'的现在分词变形如下所示:

\begin{longtable}{llll}
    \toprule
    性   & 阳性     & 阴性     & 中性     \\
    \midrule
    \endhead
    \bottomrule
    \endfoot
    单数 &          &          &          \\
    N    & sofandi  & sofandi  & sofanda  \\
    A    & sofanda  & sofandi  & sofanda  \\
    G    & sofanda  & sofandi  & sofanda  \\
    D    & sofanda  & sofandi  & sofanda  \\
    复数 &          &          &          \\
    N    & sofandi  & sofandi  & sofandi  \\
    A    & sofandi  & sofandi  & sofandi  \\
    G    & sofandi  & sofandi  & sofandi  \\
    D    & sofǫndum & sofǫndum & sofǫndum \\
\end{longtable}

% 分词有时也进一步添加-sk后缀,但这本身是非常少见的用法。

\subsection{过去分词}\label{过去分词}

过去分词相比现在分词来说常见很多,这主要是因为过去分词可以和助动词hafa连用表示完成态,类似于英语的have
done结构。强弱动词的过去分词构成方法并不一致,但它们都可以添加形容词的强变格和弱变格词尾。

强动词的过去分词构成方式是在词干上添加-in,然后添加形容词的词尾。过去分词可以按强变格变化,也可以按弱变格变化,但绝大多数情况下,我们只见到过去分词的强变化形式。以koma `come‌'的过去分词kominn为例,其强变格为:

\begin{longtable}{llll}
    \toprule
    性   & 阳性    & 阴性     & 中性    \\
    \midrule
    \endhead
    \bottomrule
    \endfoot
    单数 &         &          &         \\
    N    & kominn  & komin    & komit   \\
    A    & kominn  & komna    & komit   \\
    G    & komins  & kominnar & komins  \\
    D    & komnum  & kominni  & komnu   \\
    复数 &         &          &         \\
    N    & komnir  & komnar   & komin   \\
    A    & komna   & komnar   & komin   \\
    G    & kominna & kominna  & kominna \\
    D    & komnum  & komnum   & komnum  \\
\end{longtable}

说明:

\begin{enumerate}
    \item
          阳性单数宾格词尾为-n而非-an,得到kominn.
    \item
          中性单数主格和宾格为komit.
    \item
          弱读元音i在元音开头的词尾前省略。
\end{enumerate}

弱变格为:

\begin{longtable}{llll}
    \toprule
    性   & 阳性   & 阴性   & 中性   \\
    \midrule
    \endhead
    \bottomrule
    \endfoot
    单数 &        &        &        \\
    N    & komni  & komna  & komna  \\
    A    & komna  & komnu  & komna  \\
    G    & komna  & komnu  & komna  \\
    D    & komna  & komnu  & komna  \\
    复数 &        &        &        \\
    N    & komnu  & komnu  & komnu  \\
    A    & komnu  & komnu  & komnu  \\
    G    & komnu  & komnu  & komnu  \\
    D    & komnum & komnum & komnum \\
\end{longtable}

弱动词的过去分词主要由第三基本元+ -ð-构成。对于弱动词而言,其第三基本元很大程度上与过去时词干类似。过去分词同样有强弱变形的区分,以elska `love‌'的过去分词elskaðr为例,其强变格形式如下:

\begin{longtable}{llll}
    \toprule
    性   & 阳性     & 阴性      & 中性     \\
    \midrule
    \endhead
    \bottomrule
    \endfoot
    单数 &          &           &          \\
    N    & elskaðr  & elskuð    & elskat   \\
    A    & elskaðan & elskaða   & elskat   \\
    G    & elskaðs  & elskaðrar & elskaðs  \\
    D    & elskuðum & elskaðri  & elskuðu  \\
    复数 &          &           &          \\
    N    & elskaðir & elskaðar  & elskuð   \\
    A    & elskaða  & elskaðar  & elskuð   \\
    G    & elskaðra & elskaðra  & elskaðra \\
    D    & elskuðum & elskuðum  & elskuðum \\
\end{longtable}

其对应的弱变格如下:

\begin{longtable}{llll}
    \toprule
    性   & 阳性     & 阴性     & 中性     \\
    \midrule
    \endhead
    \bottomrule
    \endfoot
    单数 &          &          &          \\
    N    & elskaði  & elskaða  & elskaða  \\
    A    & elskaða  & elskuðu  & elskaða  \\
    G    & elskaða  & elskuðu  & elskaða  \\
    D    & elskaða  & elskuðu  & elskaða  \\
    复数 &          &          &          \\
    N    & elskuðu  & elskuðu  & elskuðu  \\
    A    & elskuðu  & elskuðu  & elskuðu  \\
    G    & elskuðu  & elskuðu  & elskuðu  \\
    D    & elskuðum & elskuðum & elskuðum \\
\end{longtable}
