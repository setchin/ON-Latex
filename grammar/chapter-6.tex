\chapter{数词}

\begin{introduction}[章节要点]
    \item 20以内的基数词与序数词
    \item 大数词的构成
    \item 数词的变格
\end{introduction}

\section{基数词与序数词}
下面是古诺尔斯语中基本的数词:

\begin{longtable}{lll}
    \toprule
    数字   & 基数词                   & 序数词                                \\
    \midrule
    \endhead
    \bottomrule
    \endfoot
    1    & einn                  & fyrstr                             \\
    2    & tveir                 & annarr                             \\
    3    & þrír                  & þriði                              \\
    4    & fjórir                & fjórði                             \\
    5    & fimm                  & fimmti                             \\
    6    & sex                   & sétti                              \\
    7    & sjau                  & sjaundi                            \\
    8    & átta                  & átti, áttandi                      \\
    9    & níu                   & níundi                             \\
    10   & tíu                   & tíundi                             \\
    11   & ellifu                & ellifti                            \\
    12   & tólf                  & tólfti                             \\
    13   & þrettán               & þrettándi                          \\
    14   & fjórtán               & fjórtándi                          \\
    15   & fimmtán               & fimmtándi                          \\
    16   & sextán                & sextándi                           \\
    17   & sjaután               & sjautándi                          \\
    18   & átján                 & átjándi                            \\
    19   & nítján                & nítjándi                           \\
    20   & tuttugu, tvítján      & tuttugandi, tvítjándi, tuttugasti, \\
    21   & tuttugu ok einn       & tuttugandi ok fyrstr               \\
         & einn ok tuttugu       & fyrstr ok tuttugandi               \\
    22   & tuttugu ok tveir      & tuttugandi ok annarr               \\
    30   & þrír tigir            & þrítugandi                         \\
    31   & þrír tigir ok einn    & þrítugandi ok fyrstr               \\
         & einn ok þrír tigir    & fyrstr ok þrítugandi               \\
    40   & fjórir tigir          & fertugandi                         \\
    50   & fimm tigir            & fimmtugandi                        \\
    60   & sex tigir             & sextugandi                         \\
    70   & sjau tigir            & sjautugandi                        \\
    80   & átta tigir            & áttatugandi                        \\
    90   & níu tigir             & nítugandi                          \\
    100  & tíu tigir             & (títugandi)                        \\
    110  & ellifu tigir          & (ellifutugandi)                    \\
    120  & hundrað               & (hundraðasti)                      \\
    200  & hundrað ok átta tigir & (hundraðasti ok áttatugandi)       \\
    240  & tvau hundrað/hundruð  &                                    \\
    960  & átta hundrað/hundruð  &                                    \\
    1200 & þúsund                & (þúsandasti)                       \\
\end{longtable}

括号里的序数词形式是从现代冰岛语借过来的。

有一些情况值得注意:
\begin{enumerate}
    \item
          1-12\footnote{读者可能会好奇为什么11和12的形式也不规则。事实上,ellifu来自于*ainalif,tólf来自于*twalif,它们是由数词1,2加上*-lif
              `left'构成,表示比10多1/2.}的形态是数词变化的基础,需要特别注意。而大于12的数词一般有迹可循。
    \item
          3-12的序数词一般是在词尾上添加-ði/-di/-ti,大于12的序数词一般加-andi/-undi.
    \item
          13-19的基数词是由词尾-tán添加在对应的0-10的基数词上得到的。
    \item
          20以上的整十的基数词由0-10的基数词和tigir构成。tigir是tigr的复数,表示``一组十个''的概念。
    \item
          21-29;31-39等由整十倍的数词和0-10的数词合成,这两个数词哪个在前哪个在后并无影响。
    \item
          hundrað和þúsund与现在的hundred和thousand表意不同,在基督教传入之前,这两个数词在日耳曼语中一般表示的是120/1200。因此200是由120(hundrað)+80(átta
          tigir)表示的。
    \item
          120, 1200的倍数由相应的基数词和hundrað/þúsund的复数构成。这种构造类似于整十倍的数词,但是,hundrað和þúsund有时不用复数形式。
    \item
          20,100,120的序数词也可以添加-asti词尾。
\end{enumerate}

\subsection{数词的变形}

大部分基数词是不可变格的。但1-4按照形容词变格,与修饰的名词的格、性、数保持一致。只有ein-有单数和复数,但作复数时表示``单独的''。其它基数词只有复数形式有完整变格,单数不变格。
\begin{longtable}{llll}
    \toprule
    ein- `one‌' & 阳性    & 阴性     & 中性    \\
    \midrule
    \endhead
    \bottomrule
    \endfoot
    单数         &       &        &       \\
    N          & einn  & ein    & eitt  \\
    A          & einn  & eina   & eitt  \\
    G          & eins  & einnar & eins  \\
    D          & einum & einni  & einu  \\
    复数         &       &        &       \\
    N          & einir & einar  & ein   \\
    A          & eina  & einar  & ein   \\
    G          & einna & einna  & einna \\
    D          & einum & einum  & einum \\
\end{longtable}

\begin{longtable}{llll}
    \toprule
    tveir `two‌' & 阳性       & 阴性       & 中性       \\
    \midrule
    \endhead
    \bottomrule
    \endfoot
    复数          &          &          &          \\
    N           & tveir    & tvær     & tvau     \\
    A           & tvá      & tvær     & tvau     \\
    G           & tveggja  & tveggja  & tveggja  \\
    D           & tveim(r) & tveim(r) & tveim(r) \\
\end{longtable}

\begin{longtable}{llll}
    \toprule
    þrí- `three‌' & 阳性      & 阴性      & 中性      \\
    \midrule
    \endhead
    \bottomrule
    \endfoot
    复数           &         &         &         \\
    N            & þrír    & þrjár   & þrjú    \\
    A            & þrjá    & þrjár   & þrjú    \\
    G            & þriggja & þriggja & þriggja \\
    D            & þrim(r) & þrim(r) & þrim(r) \\
\end{longtable}

\begin{longtable}[]{llll}
    \toprule
    fjór- `four‌' & 阳性       & 阴性       & 中性       \\
    \midrule
    \endhead
    \bottomrule
    \endfoot
    复数           &          &          &          \\
    N            & fjórir   & fjórar   & fjǫgur   \\
    A            & fjóra    & fjórar   & fjǫgur   \\
    G            & fjǫgurra & fjǫgurra & fjǫgurra \\
    D            & fjórum   & fjórum   & fjórum   \\
\end{longtable}

除此之外,只有tigr, hundrað和þúsund可以变格,其他基数词都不变格。

tigr按照u-词干阳性强名词变格,hundrað按照a-词干中性强名词变格,þúsund按照i-词干阴性强名词变格:

\begin{longtable}{llll}
    \toprule
    \multicolumn{4}{c}{数词}              \\
    \midrule
    \endhead
    \bottomrule
    \endfoot
    词干 & tig-u- & hundrað-a & þúsund-i- \\
    单数 &        &           &           \\
    N  & tigr   & hundrað   & þúsund    \\
    A  & tig    & hundrað   & þúsund    \\
    G  & tigar  & hundraðs  & þúsundar  \\
    D  & tigi   & hundraði  & þúsund    \\
    复数 &        &           &           \\
    N  & tigir  & hundruð   & þúsundir  \\
    A  & tigu/tigi   & hundruð   & þúsundir  \\
    G  & tiga   & hundraða  & þúsunda   \\
    D  & tigum  & hundruðum & þúsundum  \\
\end{longtable}