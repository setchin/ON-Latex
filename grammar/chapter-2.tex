\chapter{名词与变格法}
\label{名词与变格法}

\begin{introduction}[章节要点]
  \item 古诺尔斯语的名词系统
  \item 名词的分类
  \item 各类名词的变格法
  \item 名词的特指后缀
\end{introduction}

\section{名词的概述}
\label{名词的概述}
\subsection{语法范畴}
古诺尔斯语是一种形态变化十分丰富的语言,语法中的范畴常由词尾进行表示。

对于名词系统而言,古诺尔斯语有以下几个基本范畴:格、性、数。格(Case)反映名词在短语、从句或句子中所起语法功能;性(Gender)反映一个名词的类别;数(Number)则顾名思义,反映名词指代的个体的数量。

在古诺尔斯语中,名词共有:

\begin{enumerate}

  \item
        2个数。分别为单数和复数,这点和英语完全一致,指代单个物体的名词用单数,指代一个以上的物体时则用复数。PIE名词系统中的双数在日耳曼语中完全消失。
  \item
        3个性。分别是阴性、阳性和中性。和大多数区分性的语言一样,性是一个语法概念,和具体物体的属性关系不大(但在描述生物性十分明确的物体时,语法性和生物性是一样的,例如男人不可能是阴性词)。阳性名词未必是阳刚的,阴性名词也未必是阴柔的,不便于区分生物性的名词也未必都是阴性的。性是一种对名词分类的方式,古诺尔斯语中的每个性都对应着一套相应的词尾。
  \item
        4个格(主格、属格、与格、宾格)。为了方便起见,本书中用缩写表示名词格:
\end{enumerate}

\begin{longtable}{lll}
  \toprule
  \textbf{中文} & \textbf{英文全称} & \textbf{英文缩写} \\
  \midrule
  \endhead
  \bottomrule
  \endfoot
  主格          & Nominative    & N             \\
  属格          & Genitive      & G             \\
  与格          & Dative        & D             \\
  宾格          & Accusative    & A             \\
\end{longtable}

主格用于标记一句话的语法主语,或是主语的补足语。

宾格用于标记动词的直接宾语,或是表示一个既定动作的时间/空间性范围,例如英文中for
several miles‌或for several days在古诺尔斯语中可以用宾格标记several
miles或several days,从而不必用介词for来表示范围。

与格用于标记动词的间接宾语,或是一个动作的方式、手段、工具等。PIE中还有一个工具格(Instrumental),工具格合并到了与格中。古诺尔斯语的与格和古希腊语十分相似。

属格用于标记关系,关系既可以是具体明确的,如所有关系:John\textbf{'s}
book,也可以是更宽泛的,例如作定语:a day\textbf{‌'s}
journey;作同位语:the world tree \textbf{of}
Yggdrasil。少数动词也支配属格的双宾语,如biðja(要求),要求的对象用属格,要求的内容用宾格。

除主格外的格统称间接格(Oblique)。

有格标记(尤其是主格以外的格)的名词天生包含有英语中介词短语的意味,因此英语中对应的介词在古诺尔斯语中常常省略,当然也有一些表达必须要添加介词。介词也可以接续任意简介格。

\subsection{强名词和弱名词}
在上文中已经提到,性是一种为名词分类的方式,但除了性之外,古诺尔斯语中还把名词分为强变化名词和弱变化名词,简称强名词和弱名词。强弱在这里也只是一种给名词分类的方法,并没有更深一层的含义。德国语言学家格林(J. Grimm),即格林定律的发现者和《格林童话》的收集者,在描述日耳曼语语法特征时最早引入了强弱的概念,这套术语一直沿用至今。强名词和弱名词这一分类的唯一标准在于词尾的类型,即强名词支配强变化词尾,弱名词支配弱变化词尾。名词的强弱、性都是其固有属性,不会随数和格的改变而改变。

按照这个规则,名词的变格法可以按照性和强弱进行划分,一种有2×3=6类词尾。即强变化的阴阳中性词尾和弱变化的阴阳中性词尾,它们彼此有所区别。

强名词和弱名词都可以进一步按照词干的词干元音(Thematic
vowel)分类,词干元音依附在词干后,它对名词的形态还有更进一步的影响。按词干元音对名词分类继承于原始日耳曼语,和其他古典语言的词类划分有相似之处,例如古诺尔斯语的a-词干名词和古希腊语的-ο变格法同源。这种分类方法有一点不直观的地方,在古诺尔斯语中,词尾的音节发生大量脱落,以至于词干元音经常消失。a-词干名词gramr
`wrath'中没有一个a,但它的古希腊语同源词χρόμ\textbf{ο}ς(khróm\textbf{o}s)中则可以清晰地看到-ο-。但是,词干元音有助于理解词干中变元音的来源,因此本书在介绍名词时,也将词干元音纳入进来。请读者注意,即便不了解名词的词干元音也完全可以学习古诺尔斯的变格法,甚至绝大多数的字典都不会列出名词的词干元音(少数注重历时比较的词源学字典除外),它们大多只会列出名词的性和少数变格的形式,这样已经完全足够确定名词的变形方式了。本书对于词干元音的介绍只是为了方便初学者了解名词的构成和词形变化中一些不规则现象产生的原因,读者尤其应该记住的不是名词的词干元音,而是六类变格法的规则。

\section{强名词的变格法}
\label{强名词的变格法}

\subsection{强名词的词尾}
\label{强名词的词尾}

强名词根据阴、阳、中性添加下述的词尾:

\begin{longtable}{llll}
  \toprule
  \textbf{性} & \textbf{阳性} & \textbf{中性}   & \textbf{阴性}             \\
  \midrule
  \endhead
  \bottomrule
  \endfoot
  单数         &             &               &                         \\
  N          & -r          & -ø            & (词干u-变异) + -ø, -r       \\
  A          & -ø          & -ø            & (词干u-变异) + -ø, {[}-u{]} \\
  G          & -s, -ar     & -s            & -ar                     \\
  D          & -i          & -i            & (词干u-变异) + -ø, {[}-u{]} \\
  复数         &             &               &                         \\
  N          & 词干元音 + -r   & (词干u-变异) + -ø & 词干元音 + -r               \\
  A          & 词干元音 + -ø   & (词干u-变异) + -ø & 词干元音 + -r               \\
  G          & -a          & -a            & -a                      \\
  D          & -um         & -um           & -um                     \\
\end{longtable}

在上述的列表中,-ø表示零词尾。有可能加不同的词尾的,用逗号隔开。逗号前的词尾一般是更常见、更基础的形式,逗号后的词尾则添加在一些相对少见的名词类别上。这些问题留到后续介绍名词的词干元音划分法时说明。

名词中的u-变异现象非常明显,有一些u-变异出现在零词尾前,这是由于先前造成u-变异的词尾已经脱落。阴性名词中这一现象尤其显著,试比较古诺尔斯语g\textbf{jǫ}f
`gift'和古英语ġief\textbf{u},除古诺尔斯语外的日耳曼语大多都保留了词尾的元音。少数阴性名词的单数宾格和属格中还保留了造成u-变异的-u,这些词尾用中括号{[}{]}标出。复数与格的-um词尾总是规则地造成u-音变,如barn
`child'的复数属格是bǫrnum.

与之相比,名词中i-变异的痕迹则少得多,某些名词的词干元音包含i或j,这使得整个名词词干都发生了i-变异,因而在共时系统中表现地非常规则。值得注意的是,单数属格的-i词尾基本不会造成i-变异,其原因是这个词尾过去是*-ē,后来在i-变异停止后被抬升为-i,试比较古诺尔斯语úlf\textbf{i}
`wolf'和古英语wulf\textbf{e}.

一些规律可供参考:

\begin{info}
  \begin{enumerate}[1.]
    \item -r是主格的标志,除中性词外,单数和复数主格一般都有-r.
    \item 复数属格都是-a,复数与格都是-um.
    \item -s或-ar是单数属格的标志。
    \item 单数宾格都是零词尾,阴性名词的-u词尾已经脱落。
    \item 中性名词不论单数复数,主格和宾格总一样,且均不添加词尾。


  \end{enumerate}
\end{info}

\subsection{a/ja/wa-词干}
\label{a/ja/wa-词干}

简单来说,印欧语的常见构词法(无论动词还是名词)是用一个元音把词根和词尾连接到一起,词尾前的部分都称为词干,因而这个元音被称为词干元音,参考下面的例子:

\begin{quote}
  \begin{tabular}{llll}
    PIE *bʰér-e-ti `he bears' & 词根bʰér-  & 词干元音-e- & 人称词尾-ti \\
    PIE *gʷʰér-o-s `warmth'   & 词根gʷʰér- & 词干元音-o- & 格词尾-s
  \end{tabular}
\end{quote}

词干元音是晚期PIE的一种创新,有词干元音的词类(英文称为Thematic)变化一般比较规则。还有一部分更古老的词是无词干元音的(英文称为Athematic),这些词的变形则比较复杂,涉及到词干的元音交替,甚至是重音位置的变化。

在日耳曼语中,以a/ja/wa-为词干元音的名词是各种日耳曼语中最广大的名词类之一,它们基本都是阳性或中性的。我们把它们简称为a/ja/wa-词干,其中,a-词干是最基本的类型,少数词在a-前插入了一个-j-/-w-。半元音的出现使得ja-词干和wa-词干的变形略有一些费解之处。

\subsubsection{a-词干名词}

a-词干名词规则地适用于 \ref{强名词的词尾} 中介绍的词尾,试比较下面的阳性名词sandr
`sand', himinn `heaven'和中性名词barn `child', kné `knee':

\begin{longtable}{lllll}
  \toprule
     & \multicolumn{3}{c}{\textbf{阳性}} & \multicolumn{1}{c}{\textbf{中性}}                          \\
  \midrule
  \endhead
  \bottomrule
  \endfoot
  词干 & sand-a-                         & himin-a-                        & barn-a- & kné-a-       \\
  单数 &                                 &                                 &         &              \\
  N  & sandr                           & himinn                          & barn    & kné          \\
  A  & sand                            & himin                           & barn    & kné          \\
  G  & sands                           & himins                          & barns   & knés         \\
  D  & sandi                           & himni                           & barni   & kné          \\
  复数 &                                 &                                 &         &              \\
  N  & sandar                          & himnar                          & bǫrn    & kné          \\
  A  & sanda                           & himna                           & bǫrn    & kné          \\
  G  & sanda                           & himna                           & barna   & knjá         \\
  D  & sǫndum                          & himnum                          & bǫrnum  & knjám, knjóm \\
\end{longtable}

在上表中的词干一栏中,我们用-a-标出了这些名词词干原有的样子,但是-a-仅在阳性名词的复数主格和宾格中出现。

说明:

\begin{enumerate}
  \item
        阳性a-词干是最基本、构词力最强的词类。它的标志是主格的-r,许多派生词尾-ingr/-ungr(大致表示属于\ldots 的)都按这种方式变格。有时单词末尾的r是词干的一部分,如angr
        `sorrow', akr
        `meadow',这是因为词尾辅音简化( \ref{辅音简化} )导致的。
  \item
        一些中性名词的词干中有长元音,和词尾连接时可能引起 \ref{元音省略和缩合} 中的元音缩合。比较明显地变化是é+-a/-um
        > já/jó. 以é结尾的中性名词还有tré `tree', hlé `shelter',
        klé `stone' vé `house', fé
        `cattle'等,但不是每个词都发生元音缩合。tré,
        klé的变形和kné一致,但hlé, vé,
        fé中都没有元音缩合,即有类似于véum的形式。
  \item
        单数与格的-i一般不造成i-变异,这是因为它来自于原始日耳曼语的-ai. 但常见词dagr `day'却变化为degi.
  \item
        双音节词干的名词的变化基本类似于himinn,根据 \ref{元音省略和缩合} 中元音省略的规则1,him\textbf{i}n-a-中的弱读元音(粗体标出)在以元音起首的词尾前全部脱落,因此单数与格和整个复数中完全没有弱读元音。不过,这一规则亦有少许例外,如常见的人名Gunnar的单数与格Gunnari保留了-a-.
  \item
        部分以-ill词尾派生的名词,如ketill `kettle'和lykill
        `key'(注意词尾-r的同化),
        它们第一个音节的前元音是后一个音节的i导致的,当非重读的i被省略时,有时第一个音节的元音恢复其本来的形式,如上述两个词的单数与格为katli,lukli.
        这个音变无疑是历时规律的影响,因而在共时系统中常常被类比原则所规则化,因此lukli也可以写成lykli,且现代冰岛语只用lykli。其他一些名词如berill
        < *barilaz
        `vessel'则完全不恢复原本的a;ketill是一个例外,无论是古诺尔斯语还是现代冰岛语,都要恢复a.\footnote{-ill词尾有两个作用:1)用作指小词(Diminutive
          nouns)后缀。从名词上派生出比原始词略小、较弱等概念的新词;2)用作代理名词(Agent
          nouns)后缀。从动词上派生出表示发出这个动作的实体。例如berill可以理解为从bera
          `carry'这个动词上派生,表示``可以运载东西的物体'',衍生为``船''。这些名词和动词关系密切,其中的元音也很可能受到动词的影响。}
        没有一个完美的规律可以解释这其中的不规则现象,读者在遇到相应的词时应该借助字典了解其正确的变形。
  \item
        一小部分名词的单数属格不是-s而是-ar,例如hǫfundr
        `chieftain'的单数属格为hǫfundar.
        -ar的来源尚不明确,一些观点认为它是从别的词类中类比得到,另一部分比较新的观点认为这个词尾本和-s同源,但经历了稍有区别的发展过程。
\end{enumerate}

\subsubsection{ja-词干名词}\label{ja-词干名词}

ja-词干名词的词干元音前有一个j,在日耳曼语中,j的性质非常特殊,它根据词干的长短时而变为辅音,时而变为元音。在古诺尔斯语中,词干的长短和词干音节(这里的词干并不包括j\footnote{从理论上来说,j的确是``词干''的一部分。但它是一个增音,于是在计算音节长短时,不把其计入其中。})的划分有密切的关系,它们是这么定义的:

\begin{info}
  短词干:词干音节只有一个单辅音+不超过一个辅音/一个双元音或长元音\\
  长词干:词干音节为单辅音+辅音簇/双元音或长元音+任意数量的辅音
\end{info}

短词干基本上和传统式音节划分法中的短音节和部分第二类长音节一致;其余的音节都被分到长词干。如果使用的是格律法,那么可以认为不超过两个音拍的词干属于短词干。

ja-词干名词的总体规律是:

\begin{info}
  j在长词干后面表现为i,在短词干后面表现为j。
\end{info}

因此,长词干ja-词干名词有时也称为ija-词干或者īa-词干,短词干ja-词干名词就简单地称作ja-词干。无论是哪种情况,i/j的存在都导致了整个词干发生了i-变异,因此所有的ja-词干名词中只有前元音。不过,即便变格的形式中i/j没有出现,这个前元音依旧保留了下来,从共时特征上已经看不出任何i-变异的痕迹。

试比较下面的长短词干名词:阳性名词niðr `kinsman', hirðir `herdsman',
中性名词kyn `kin', ríki
`kingdom',每一对词中,左边的是短词干名词,右边的是长词干名词。

\begin{longtable}{lllll}
  \toprule
     & \multicolumn{3}{c}{\textbf{阳性}} & \multicolumn{1}{c}{\textbf{中性}}                       \\
  \midrule
  \endhead
  \bottomrule
  \endfoot
  词干 & nið-j-a-                        & hirð-j-a-                       & kyn-j-a- & rík-j-a- \\
  单数 &                                 &                                 &          &          \\
  N  & niðr                            & hirðir                          & kyn      & ríki     \\
  A  & nið                             & hirði                           & kyn      & ríki     \\
  G  & niðs                            & hirðis                          & kyns     & ríkis    \\
  D  & nið                             & hirði                           & kyni     & ríki     \\
  复数 &                                 &                                 &          &          \\
  N  & niðjar                          & hirðar                          & kyn      & ríki     \\
  A  & niðja                           & hirða                           & kyn      & ríki     \\
  G  & niðja                           & hirða                           & kynja    & ríkja    \\
  D  & niðjum                          & hirðum                          & kynjum   & ríkjum   \\
\end{longtable}

说明:

\begin{enumerate}

  \item
        根据上述规律,短词干的ja-词干名词可以被理解为以-j-结尾的a-词干名词,例如niðr的词干部分可被看作是niðj-.
        根据半元音的保持性(见 \ref{半元音的保持性} ),词干末尾的j只能出现在以后元音起首的词尾前。因此整个单数部分中都没有-j-.但在-ar, -a, -um前,-j-保留下来,这个现象发生在阳性名词的复数以及中性名词的复数属格和与格中。
  \item
        类似地,长词干的ja-名词可以被理解为以-i-结尾的a-词干名词,例如hirðir的词干部分可被看作是hirði-.
        根据 \ref{元音省略和缩合} 中元音省略的规则2,-i-在任何以元音起首的词尾前脱落。因此整个复数部分中-i-都没有表现出来。
  \item
        短词干的阳性ja-词干名词极其罕见,除了niðr外,常见的词只有herr
        `army',其他本属于这一类的名词在古诺尔斯语中被类比到了其他种类里。这类词的单数与格没有-i,形态和宾格一致,其原因可能是受到了长词干名词的影响,因为后者的单数属格和宾格总是一致的。
  \item
        ríki一词的复数属格和与格中有-j-而不是-i-,这是古诺尔斯语中的一个例外:
\end{enumerate}
%enumerate环境与info环境交互问题
\begin{info}
  如果长词干以硬颚音-k/-g结尾,则-i-在元音开头的词尾前变回-j-。
\end{info}

\begin{info}
  $\bullet$ \textbf{西弗斯定律(Sievers's Law)}

  \indent
  德国语言学家爱德华·西弗斯(EduardSievers)发现包含辅音簇中的半元音在元音前的发音会根据词干音节的轻重而改变。在原始日耳曼语中,PIE的*y在\textbf{轻词干}后变成*j,\textbf{重词干}后变成*ij(按照格律划分法):\\
  \indent
  轻词干:PIE *kor-yo-s > PGmc. *harjaz > Go.harjis `army'\\
  \indent
  重词干:PIE *ḱerd\textsuperscript{h}-yo-s > PGmc. *hirdijaz > Go. hairdeis `shepherd' (ei读作长音i)\\
  \indent
  西弗斯定律在哥特语中最为明显,在古诺尔斯语中也有保留,上述的ja-词干的情况就是最直接的例子。只不过,在古诺尔斯语中西弗斯定律的演变结果可能在进一步的音变中消失了。\\
  \indent
  古诺尔斯语中长短词干的划分和上述的轻重词干不完全一致,主要是一些长元音词干(按照格律法应该认为是重词干)的名词变形和轻词干一致。其主要原因是古诺尔斯语中的这些长元音大多在原始日耳曼语中是元音+半元音的形式,它们相当于轻词干名词,从而符合西弗斯定律的条件。在原始日耳曼语发展的过程中,半元音和元音结合为了古诺尔斯语的长元音。\\
  \indent
  西弗斯定律不仅在名词中有体现,在一类弱动词中也非常显著(见 \ref{第一弱变位法} )。
\end{info}

\subsubsection{wa-词干名词}

wa-词干名词数量不多,它们相比ja-词干来说规则得多,这是因为西弗斯定律在日耳曼语中仅仅对y有效(虽然在其他一些古典语言,如梵语中,它也影响w)。wa-在古诺尔斯语中写作va-,v一开始仅在同时满足下述两个条件时保留:

\begin{enumerate}
  \item
        紧跟在一个短音节或辅音k/g之后(辅音簇亦可,只需要其最后一个辅音是k/g)。
  \item
        在a或i之前。
\end{enumerate}

但在很多情况下,v又重新按类比原则添加了回去,有时甚至可以出现在u之前。另外,由于绝大部分wa-词干名词都满足第一个条件,而第二个条件和 \ref{半元音的保持性} 中介绍的半元音保持性一致,读者亦可以认为wa-词干名词中v的显现完全是规则的。

类似于ja-词干名词,wa-词干名词也系统地发生了u-变异,因此许多词干中的元音在共时条件下可以认为是ǫ.

阳性名词sǫngr `song', sær/sjór `sea',中性名词hǫgg
`strike'都按照wa-词干变格:

\begin{longtable}{lllll}
  \toprule
     & \multicolumn{3}{c}{\textbf{阳性}} & \textbf{中性}                       \\
  \midrule
  \endhead
  \bottomrule
  \endfoot
  词干 & sang-v-a-                       & sæ-v-a-     &         & hagg-v-a- \\
  单数 &                                 &             &         &           \\
  N  & sǫngr                           & sær         & sjór    & hǫgg      \\
  A  & sǫng                            & sæ          & sjó     & hǫgg      \\
  G  & sǫngs                           & sævar       & sjóvar  & hǫggs     \\
  D  & sǫngvi                          & sæ(vi)      & sjó(vi) & hǫggvi    \\
  复数 &                                 &             &         &           \\
  N  & sǫngvar                         & sævar       & sjóvar  & hǫgg      \\
  A  & sǫngva                          & sæva        & sjóva   & hǫgg      \\
  G  & sǫngva                          & sæva        & sjóva   & hǫggva    \\
  D  & sǫngum                          & sæ(v)um     & sjóvum  & hǫggum    \\
\end{longtable}

sær有两个异体形式sjór,sjár,在早期的文本中,sær较为常见,sjór是后期出现的产物,sjár则较为罕见。这组词也可以按照i-词干变格(见 \ref{阳性i-词干} )。

如果v紧随元音之后,这种词的单数属格一般是-ar,如sævar和sjóvar,但是规则地添加-s也是可行的,类比作用在其中发挥了很大作用。

\subsection{ō/jō/wō-词干}
\label{ō/jō/wō-词干}

ō / jō /
wō-词干名词都是阴性的。由于词干根据原始语分类,在古诺尔斯语出现的时期词干元音已经不以ō形式出现,反而合并为a。ō/jō/wō-词干是a/ja/wa-词干的阴性对应,也是最广泛的阴性名词。

\subsubsection{ō-词干}

阴性名词grǫf `hole', laug `bath', á `river‌', 以及Ingibjǫrg `Ingeborg‌'
是典型的ō-词干名词:

\begin{longtable}{lllll}
  \toprule
     & \multicolumn{4}{c}{\textbf{阴性}}                                 \\
  \midrule
  \endhead
  \bottomrule
  \endfoot
  词干 & graf-a-                         & laug-a- & á-a- & Ingibjarg-a- \\
  单数 &                                 &         &      &              \\
  N  & grǫf                            & laug    & á    & Ingibjǫrg    \\
  A  & grǫf                            & laug    & á    & Ingibjǫrgu   \\
  G  & grafar                          & laugar  & ár   & Ingibjargar  \\
  D  & grǫf                            & laug(u) & á    & Ingibjǫrgu   \\
  复数 &                                 &         &      &              \\
  N  & grafar                          & laugar  & ár   & -            \\
  A  & grafar                          & laugar  & ár   & -            \\
  G  & grafa                           & lauga   & á    & -            \\
  D  & grǫfum                          & laugum  & ám   & -            \\
\end{longtable}

说明:

\begin{enumerate}

  \item
        ō-词干名词的阳性主格总没有-r,这是阴性ō-词干名词的基本特征。-r出现在下面的长词干jō-词干名词中。
  \item
        单数主格、宾格和与格中有u-变异。单数主格u-变异由词干元音造成,其经历了*-ō- > *-u- > -a-的历史音变。在u-变异仍然活跃的时期,词干元音仍是-u-. 这在一些西日耳曼语中有保留,例如古英语中grǫf的同源词为ġiefu(单数主格)。单数与格的u-变异则来自于格词尾-u,经过弱读元音的省略后,现在这个词尾可加可不加。
        单数宾格的u-变异从主格类比得来。一般来说,单音节词干的单数宾格和与格都不加-u词尾。多音节词和一些专有名词中则有时保留-u,如Ingibjǫrgu的宾格和与格都有-u,而一个常见的阴性派生词尾-ing(从动词派生名词,区别阳性的-ingr)的单数变格为-ing,
        -ing, -ingar, -ingu,其与格加-u而宾格不加-u.\footnote{一般认为,ō-词干的单数宾格的变化类比了单数主格,所以大部分名词的宾格实则不添加-u。相比之下,与格加-u的情况更多些。}
  \item
        复数式的构成和阳性非常相似,但注意阴性名词的复数宾格也有-r.
  \item
        á是一个常见的阴性名词,但其词干只有一个长元音,与词尾连接时经常产生元音缩合。
\end{enumerate}

\subsubsection{jō-词干}

jō-词干中也有长短词干的区别,对于这两类词干的处理与ja-词干完全一致,因为两者都是西弗斯定律的结果。

短词干名词 ben `wound',长词干名词heiðr `heath'代表了两类jō-词干名词:

\begin{longtable}{lll}
  \toprule
     & \multicolumn{2}{c}{\textbf{阴性}}             \\
  \midrule
  \endhead
  \bottomrule
  \endfoot
  词干 & ben-j-a-                        & heið-j-a- \\
  单数 &                                 &           \\
  N  & ben                             & heiðr     \\
  A  & ben                             & heiði     \\
  G  & benjar                          & heiðar    \\
  D  & ben                             & heiði     \\
  复数 &                                 &           \\
  N  & benjar                          & heiðar    \\
  A  & benjar                          & heiðar    \\
  G  & benja                           & heiða     \\
  D  & benjum                          & heiðum    \\
\end{longtable}

说明:

\begin{enumerate}

  \item
        只有长词干jō-词干的主格有标记-r,这是从i-词干名词(见 \ref{阴性i-词干} )类比得到的。
  \item
        -j-的保留完全符合 \ref{半元音的保持性} 中的规律。
  \item
        一些短词干名词的单数与格中还保留了-u,构成了-ju的形式。常见的词包括ey
        `island', egg `egg', hel `Hel, death'.
\end{enumerate}

\subsubsection{wō-词干}

另有一小类 wō-词干名词,其变化类似于wa-词干。参考dǫgg `dew', ǫr `arrow'的变格:

\begin{longtable}{lll}
  \toprule
     & \multicolumn{2}{c}{\textbf{阴性}}           \\
  \midrule
  \endhead
  \bottomrule
  \endfoot
  词干 & dǫgg-v-a                        & ǫr-v-a- \\
  单数 &                                 &         \\
  N  & dǫgg                            & ǫr      \\
  A  & dǫgg                            & ǫr      \\
  G  & dǫggvar                         & ǫrvar   \\
  D  & dǫgg(u)                         & ǫr(u)   \\
  复数 &                                 &         \\
  N  & dǫggvar                         & ǫrvar   \\
  A  & dǫggvar                         & ǫrvar   \\
  G  & dǫggva                          & ǫrva    \\
  D  & dǫggum                          & ǫrum    \\
\end{longtable}

说明:

\begin{enumerate}

  \item
        wō-词干名词中-v-的出现完全符合半元音的保持性规则。
  \item
        单数与格的-u是可选的,但不加的情况为多。
\end{enumerate}

\subsection{i-词干}
\label{i-词干}

一部分日耳曼语名词的词干元音是-i-,称之为i-词干名词。由于-i-的存在,部分名词的整个变位中都发生了i-变异,例如gestr
< PGmc. *gastiz (比较哥特语gasts),但在有些名词中,例如staðr,
这一音变没有沿袭下来。\footnote{i-词干名词中不发生i-变异的一般是短词干名词,类似的例子还有dalr
  `dale', salr `room', matr `food', þulr
  `poet'. 长词干名词一般都发生i-变异,少数的例外包括stuldr `theft', sultr
  `hunger', burðr `birth', sauðr `sheep',它们的词干中一般有u.
  i-变异无论是否失效,并不影响读者对于变格的判断,因为这个变格表中要么都发生i-变异,要么都不发生i-变异。}i-词干名词很大程度上是日耳曼语的创新,它将许多PIE中变形规则复杂的无词干元音名词规则化了。i-词干名词可以对应PIE中的各种性的名词,但中性词比较少。到了古诺尔斯语时期,i-词干名词只剩下阳性和阴性两种,原先的中性词被合并到a-词干的阳性词当中。i-词干与a-词干和ō-词干发生了很大程度的交互和类比,一些词尾因此相互借鉴和渗透了。

\subsubsection{阳性i-词干}\label{阳性i-词干}

阳性i-词干名词添加阳性词尾。试比较三个常见名词staðr `place', gestr
`guest', bekkr `bench':

\begin{longtable}{llll}
  \toprule
     & \multicolumn{3}{c}{\textbf{阳性}}                            \\
  \midrule
  \endhead
  \bottomrule
  \endfoot
  词干 & stað-i-                         & gest-i- & bekk-i-        \\
  单数 &                                 &         &                \\
  N  & staðr                           & gestr   & bekkr          \\
  A  & stað                            & gest    & bekk           \\
  G  & staðar                          & gests   & bekks, bekkjar \\
  D  & stað                            & gest(i) & bekk           \\
  复数 &                                 &         &                \\
  N. & staðir                          & gestir  & bekkir         \\
  A  & staði                           & gesti   & bekki          \\
  G  & staða                           & gesta   & bekkja         \\
  D  & stǫðum                          & gestum  & bekkjum        \\
\end{longtable}

说明:

\begin{enumerate}
  \item
        复数主格和宾格中有词干元音-i,因此,i-词干名词的变格和a-词干的区别仅在于词干元音不同。
  \item
        单数属格可能是-s,也可能是-ar,还可能二者皆可。i-词干中这两种词尾出现的频率几乎是相等的,请读者查阅字典解决。\footnote{从古老的卢恩铭文来看,-ar很可能是更古老的词尾。-s词尾由a-词干名词类比得来。}
  \item
        区别于a-词干名词,单数与格一般不加词尾-i,但gesti这种形式无疑受到了a-词干的影响。
  \item
        在-k/-g后,词干元音-i-以-j-的形式出现在后元音前(-ar, -a,
        -um)。这条规律在 \ref{ja-词干名词} 中已经提到过。
\end{enumerate}

\subsubsection{阴性i-词干}\label{阴性i-词干}

大量的阴性i-词干名词与ō-词干发生混杂,以至于许多i-词干名词和ō-词干的单数形式完全相同,这些词的标志是复数主格和宾格的词尾-ir。参见下面几个常见的阴性名词nauð(r)
`distress', ást `love', hǫll `hall':

\begin{longtable}{llll}
  \toprule
     & \multicolumn{3}{c}{\textbf{阴性}}                     \\
  \midrule
  \endhead
  \bottomrule
  \endfoot
  词干 & nauð-i-                         & ást -i- & hall-i- \\
  单数 &                                 &         &         \\
  N  & nauð(r)                         & ást     & hǫll    \\
  A  & nauð                            & ást     & hǫll    \\
  G  & nauðar                          & ástar   & hallar  \\
  D  & nauð                            & ást     & hǫll(u) \\
  复数 &                                 &         &         \\
  N  & nauðir                          & ástir   & hallir  \\
  A  & nauðir                          & ástir   & hallir  \\
  G  & nauða                           & ásta    & halla   \\
  D  & nauðum                          & ástum   & hǫllum  \\
\end{longtable}

说明:

\begin{enumerate}
  \item
        原先的i-词干无论是阴性还是阳性,词尾总是一样的。之后受到a-/ō-词干的影响才使其产生了分歧。nauðr的词尾-r实则是更古老的主格词尾的反映:*-i-z > *-i-r. 在阳性名词中,词尾-r保留了下来(因为a-词干名词也有-r),但在阴性名词中,受到ō-词干的影响,主格词尾大多被省略了。
  \item
        ást反映了最典型的i-词干名词,它和ō-词干阴性名词的区别仅仅在于复数主格和宾格的词尾是-ir,这是词干元音有区别导致的。
  \item
        hǫll一词也反映了i-词干和ō-词干的混淆,它本身是ō-词干的,因此有时在单数与格中出现可选的-u,但在复数式中,又以-ir结尾。
\end{enumerate}

\subsection{u-词干}
\label{u-词干}

u-词干名词的来源和i-词干十分相似。在古诺尔斯语中,u-词干名词都是阳性的,其他性的名词被合并到了其他类别中。u-词干名词的词尾和阳性名词基本一致,但是复数主格是-ir而非*-ur.

u-词干名词有两个明显的特征:

\begin{enumerate}
  \item
        元音分割。根据 \ref{元音分割} 的规则,*e在a前变成ja,u前变成jǫ.
        由于u-词干名词的词尾中有a和u的交替,故两类元音分割都有发生。
  \item
        包含-i-/-u-的词尾发生i-/u-变异。
\end{enumerate}

我们以三个名词skjǫldr `shield', vǫllr `ground', fǫgnuðr `joy'为例:

\begin{longtable}{llll}
  \toprule
     & \multicolumn{3}{c}{\textbf{阳性}}                       \\
  \midrule
  \endhead
  \bottomrule
  \endfoot
  词干 & skeld-u-                        & vall-u- & fagnað-u- \\
  单数 &                                 &         &           \\
  N  & skjǫldr                         & vǫllr   & fǫgnuðr   \\
  A  & skjǫld                          & vǫll    & fǫgnuð    \\
  G  & skjaldar                        & vallar  & fagnaðar  \\
  D  & skildi                          & velli   & fagnaði   \\
  复数 &                                 &         &           \\
  N  & skildir                         & vellir  & fagnaðir  \\
  A  & skjǫldu                         & vǫllu   & fǫgnuðu   \\
  G  & skjalda                         & valla   & fagnaða   \\
  D  & skjǫldum                        & vǫllum  & fǫgnuðum  \\
\end{longtable}

说明:

\begin{enumerate}
  \item
        单数与格、复数主格的词尾包含-i-(原始形式分别为*-iwi, *-iwiz),这导致了词干中的i-变异,与a-词干的与格词尾不同。特别地,词干中的元音*e在这种情况下抬升成了i,其他元音都按 \ref{变元音} 发生变化。
  \item
        在除了单数属格、复数主格的情况下,词干元音是-u或-a,因此这些形式中都可能有元音分割。如果不满足元音分割的条件(词干中的元音不是*e),则在包含u-的词尾前发生u-变异。
  \item
        fǫgnuðr展示了以-uðr词尾派生出的名词,这个词尾本来是-aðr\footnote{-aðr本是形容词词尾,从名词中派生出表示``拥有这个名词的''形容词。如hjalmr
          `helmet' + -aðr > hjalmaðr `having a helmet, helmed'.
          作名词的派生词尾时,构成动词或形容词的同义名词。例如fǫgnuðr由动词fagna
          `rejoice'派生得来。},但词干元音的-u导致了非重读的a
        >
        u,进而引起了词根中重读元音的进一步u-变异。在单数与格、复数主格中,-i对非重读的a无效,因此没有造成i-变异。古诺尔斯语中-uðr
        词尾也可以就写作-aðr而不引起任何u-变异,此时的变形基本是完全规则的(复数宾格词尾按正常的阳性词尾类比为了-i)。故fǫgnuðr也可写作fagnaðr,其变格如下所示:
\end{enumerate}

\begin{longtable}{ll}
  \toprule
  \multicolumn{2}{c}{\textbf{fagnaðr}} \\
  \midrule
  \endhead
  \bottomrule
  \endfoot
  单数 &                                 \\
  N  & fagnaðr                         \\
  A  & fagnað                          \\
  G  & fagnaðar                        \\
  D  & fagnaði                         \\
  复数 &                                 \\
  N  & fagnaðir                        \\
  A  & fagnaði                         \\
  G  & fagnaða                         \\
  D  & fǫgnuðum                        \\
\end{longtable}

\subsection{辅音词干}
\label{辅音词干}

目前所介绍的名词都有一个词干元音,还有一小类词的词干以辅音作结。可以进一步分词几个小类:r-词干;nd-词干;其它辅音词干。这些词几乎都是更古老的语言中保留下来的产物,往往反映出一些非常基本的概念(从r-词干中可见一斑)。这些词在PIE中的变形一般比较复杂,在日耳曼语中这些词受到其他词类的影响逐渐规则化,但它们的形式仍不太符合上述的元音词干名词。

\subsubsection{r-词干}

表示亲属关系的词都属于r-词干,它们的语法性和自然性是一样的,也就是说``母亲''、``女儿''这样的词是阴性的,``父亲''、``兄弟''这类词是阳性的。这类词仅有五个,都表示最亲密的血缘关系。

\begin{longtable}{llllll}
  \toprule
     & \textbf{faðir} & \textbf{móðir} & \textbf{bróðir} & \textbf{dóttir} & \textbf{systir} \\
  \midrule
  \endhead
  \bottomrule
  \endfoot
  单数 &                &                &                 &                 &                 \\
  N  & faðir          & móðir          & bróðir          & dóttir          & systir          \\
  A  & fǫður, feðr    & móður          & bróður          & dóttur          & systur          \\
  G  & fǫður, feðr    & móður          & bróður          & dóttur          & systur          \\
  D  & fǫður, feðr    & móður          & bróður          & dóttur, dœtr    & systur          \\
  复数 &                &                &                 &                 &                 \\
  N  & feðr           & mœðr           & brœðr           & dœtr            & systr           \\
  A  & feðr           & mœðr           & brœðr           & dœtr            & systr           \\
  G  & feðra          & mœðra          & brœðra          & dœtra           & systra          \\
  D  & feðrum         & mœðrum         & brœðrum         & dœtrum          & systrum         \\
\end{longtable}

说明:

\begin{enumerate}
  \item
        这些词的单数主格都是-ir,其余单数间接格-ur并造成u-变异。faðir的间接格还出现了带有i-变异的-r词尾,这可能是更古老的形式。
  \item
        复数中都出现i-变异。主格和宾格的词尾为-r,属格和与格的词尾与其他所有名词一致。
  \item
        请注意:表示儿子的词sonr或sunr按u-词干变形。其中词干中o/u的交替来自于一种古老的a-变异(PIE的*o在PGmc.中与*a合流,使得PGmc.缺少*o音。由于音系的不平衡性,u在a前下降为o,这一过程称为a-变异)。
\end{enumerate}

\subsubsection{nd-词干}

这些名词由动词的现在分词(见 \ref{现在分词} )衍生,最后成为某种固定的表达。由于现在分词的基本词尾是-andi,故得名。

我们以两个名词bóndi `farmer'以及gefandi `giver' 为例说明其词尾:

\begin{longtable}{lll}
  \toprule
     & \textbf{bóndi} & \textbf{gefandi} \\
  \midrule
  \endhead
  \bottomrule
  \endfoot
  单数 &                &                  \\
  N  & bóndi          & gefandi          \\
  A  & bónda          & gefanda          \\
  G  & bónda          & gefanda          \\
  D  & bónda          & gefanda          \\
  复数 &                &                  \\
  N  & bœndr          & gefendr          \\
  A  & bœndr          & gefendr          \\
  G  & bónda          & gefanda          \\
  D  & bóndum         & gefǫndum         \\
\end{longtable}

说明:

\begin{enumerate}
  \item
        单数主格词尾为-i,间接格为-a,这正好和形容词的比较级变格一致。(参见 \ref{形容词的比较级和最高级} )
  \item
        复数主格和宾格的变形和r-词干一致,都发生了i-变异。
\end{enumerate}

\subsubsection{其他辅音词干}

还有一些词不太符合上述的词干划分。这些名词的特征是复数主格和宾格的-r词尾,-r有可能被前面的辅音同化。和上述的两类辅音词干一样,复数主格和宾格中同样出现了i-变异。这些词数量不多,但出现频次未必很低,读者可把它们当作不规则的名词记忆。

这类词包括一些常见的阳性名词:maðr `man', nagl `nail', mónuðr `month',
vetr `winter', fótr `foot':

\begin{longtable}{llllll}
  \toprule
     & \textbf{maðr} & \textbf{nagl} & \textbf{mónuðr} & \textbf{vetr} & \textbf{fótr} \\
  \midrule
  \endhead
  \bottomrule
  \endfoot
  单数 &               &               &                 &               &               \\
  N  & maðr          & nagl          & mónuðr          & vetr          & fótr          \\
  A  & mann          & nagl          & mónuð           & vetr          & fót           \\
  G  & manns         & nagls         & mánaðar         & vetrar        & fótar         \\
  D  & manni         & nagli         & mónuð           & vetr          & fœti          \\
  复数 &               &               &                 &               &               \\
  N  & menn          & negl          & mónuðr          & vetr          & fœtr          \\
  A  & menn          & negl          & mónuðr          & vetr          & fœtr          \\
  G  & manna         & nagla         & mánaða          & vetra         & fóta          \\
  D  & mǫnnum        & nǫglum        & mónuðum         & vetrum        & fótum         \\
\end{longtable}

注意mónuðr有另一拼写mánaðr, fingr `finger‌'的变格与vetr一致。

类似地还有一些阴性名词: bók `book', tǫnn `tooth', nátt `night', kýr
`cow':

\begin{longtable}{lllll}
  \toprule
     & \textbf{bók} & \textbf{tǫnn} & \textbf{nátt, nótt} & \textbf{kýr} \\
  \midrule
  \endhead
  \bottomrule
  \endfoot
  单数 &              &               &                     &              \\
  N  & bók          & tǫnn          & nátt, nótt          & kýr          \\
  A  & bók          & tǫnn          & nátt, nótt          & kú           \\
  G  & bókar, bœkr  & tannar        & náttar, nætr        & kýr          \\
  D  & bók          & tǫnn          & nátt, nótt          & kú           \\
  复数 &              &               &                     &              \\
  N  & bœkr         & tennr, teðr   & nætr                & kýr          \\
  A  & bœkr         & tennr, teðr   & nætr                & kýr          \\
  G  & bóka         & tanna         & nátta               & kúa          \\
  D  & bókum        & tǫnnum        & náttum, nóttum      & kúm          \\
\end{longtable}

\section{弱名词的变格法}
\label{弱名词的变格法}

\subsection{弱名词的词尾}
\label{弱名词的词尾}

弱名词的词尾相比强名词来说简单得多:

\begin{longtable}{llll}
  \toprule
  \textbf{性} & \textbf{阳性} & \textbf{中性} & \textbf{阴性} \\
  \midrule
  \endhead
  \bottomrule
  \endfoot
  单数         &             &             &             \\
  N          & -i          & -a          & -a          \\
  A          & -a          & -a          & -u          \\
  G          & -a          & -a          & -u          \\
  D          & -a          & -a          & -u          \\
  复数         &             &             &             \\
  N          & -ar         & -u          & -ur         \\
  A          & -a          & -u          & -u          \\
  G          & -a          & -na         & -na         \\
  D          & -um         & -um         & -um         \\
\end{longtable}

上述的词尾中包含u的会规则地导致u-变异。但-i并不造成i-变异。

弱名词很大程度上也是日耳曼语的创新,它将另一部分无词干元音名词规则化了。弱名词的词干尾有一个辅音-n,在日耳曼语中又可进一步分为an-词干、ōn-词干和in-词干,类比三种性的强名词词干。

在历史演化过程中,词干尾的-n发生大规模的脱落,在古诺尔斯语中仅保留在部分词的复数属格中。因此,推断弱名词原始的形式是比较困难的。不过,由于弱名词变化简单,只需要知道其性就可以推断出完整的变化,应该不会造成学习上的困难。另外,虽然本书介绍弱名词时仍旧给出了其词干类型,但这只是一种分类的方法,读者无需记忆弱名词的类别。

\subsection{an/jan-词干}
\label{an/jan-词干}

这类词类比a-词干名词,同样都是阳性和中性的,参考下面的例词,阳性词bogi
`bow', bryti `steward', gumi `man', 中性词 hjarta `heart':

\begin{longtable}{lllll}
  \toprule
     & \multicolumn{3}{c}{\textbf{阳性}} & \textbf{中性}                      \\
  \midrule
  \endhead
  \bottomrule
  \endfoot
  词干 & bog-                            & bryt-j-     & gum-     & hjart-  \\
  单数 &                                 &             &          &         \\
  N  & bogi                            & bryti       & gumi     & hjarta  \\
  A  & boga                            & brytja      & guma     & hjarta  \\
  G  & boga                            & brytja      & guma     & hjarta  \\
  D  & boga                            & brytja      & guma     & hjarta  \\
  复数 &                                 &             &          &         \\
  N  & bogar                           & brytjar     & gum(n)ar & hjǫrtu  \\
  A  & boga                            & brytja      & gum(n)a  & hjǫrtu  \\
  G  & boga                            & brytja      & gumna    & hjartna \\
  D  & bogum                           & brytjum     & gum(n)um & hjǫrtum \\
\end{longtable}

说明:

\begin{enumerate}
  \item
        gumi是一个比较古老的词,它的复数属格还是-na词尾,这反映了比较古老的an-词干的形式。以这个-n-为基础,复数的其他形式发生了类推,因此也有包含-n-的词形。其他词形都是规则的。
  \item
        bryti是所谓的jan-词干名词,除单数主格外其他形式中还规则地插入了-j-,但后来这些半元音也都消失了,使得它和其他an-词干名词没有任何区别。这类词数量很少,一些表示人的后缀-ingi,-virki按这种方式变格。
\end{enumerate}

\subsection{ōn/jōn -词干}
\label{ōn/jōn -词干}

这类词类比ō-词干名词,它们都是阴性的。和上面的an/jan-词干一样,这类词中也有含有-j-的子类,参考下面的例词:saga
`tale', stjarna `star', ásjá `help', smiðja `smithy':

\begin{longtable}{lllll}
  \toprule
     & \multicolumn{4}{c}{\textbf{阴性}}                              \\
  \midrule
  \endhead
  \bottomrule
  \endfoot
  词干 & sag-                            & stjarn-  & ásjá- & smið-j- \\
  单数 &                                 &          &       &         \\
  N  & saga                            & stjarna  & ásjá  & smiðja  \\
  A  & sǫgu                            & stjǫrnu  & ásjá  & smiðju  \\
  G  & sǫgu                            & stjǫrnu  & ásjá  & smiðju  \\
  D  & sǫgu                            & stjǫrnu  & ásjá  & smiðju  \\
  复数 &                                 &          &       &         \\
  N  & sǫgur                           & stjǫrnur & ásjár & smiðjur \\
  A  & sǫgur                           & stjǫrnur & ásjár & smiðjur \\
  G  & sagna                           & stjarna  & ásjá  & smiðja  \\
  D  & sǫgum                           & stjǫrnum & ásjám & smiðjum \\
\end{longtable}

说明:

\begin{enumerate}
  \item
        ásjá来自于á `on' + sjá `watch',有一些以介词 +
        sjá的名词都按这种方式变格,它们表达``以某种特定方式看''的含义。除了ásjá外还有:
        \begin{quote}
          eptir-sjá \emph{stare-watch} `the looking with desire after a lost
          thing, hence loss, grief'

          aptr-sjá \emph{after-watch} `regret'

          for-sjá \emph{before-watch} `foresight'

        \end{quote}
        ásjá 也可像á一样变格,这样单数属格就是ásjár, 其余形式没有区别。
  \item
        一个常见的ōn-词干名词kona `woman'的复数属格是kvenna.
        后者实际上是这个词更古老的形式的保留(PGmc.
        *kwenǭ)。造成这个现象的原因是辅音后的w脱落常常导致e/a >
        o的音变。类似的例子在tolf `twelve' < *twalif中也有体现。
\end{enumerate}

\subsection{in-词干}
\label{in-词干}

in-词干均为阴性,大都是概念性的抽象名词。它的词尾和正常阴性词尾略有不同,一切单数形式都以-i结尾,这就相当于单数不变格。抽象名词少有复数形式,如果有,就采取ō-词干的词尾,常见的in-词干名词有elli
`old age' 以及lygi `life':

\begin{longtable}{lll}
  \toprule
     & \multicolumn{2}{c}{\textbf{阴性}}         \\
  \midrule
  \endhead
  \bottomrule
  \endfoot
  词干 & elli-                           & lygi- \\
  单数 &                                 &       \\
  N  & elli                            & lygi  \\
  A  & elli                            & lygi  \\
  G  & elli                            & lygi  \\
  D  & elli                            & lygi  \\
  复数 &                                 &       \\
  N  & -                               & lygar \\
  A  & -                               & lygar \\
  G  & -                               & lyga  \\
  D  & -                               & lygum \\
\end{longtable}

\section{定冠词和特指后缀}
\label{定冠词和特指后缀}

类似于英语的定冠词the,古诺尔斯语也有一个定冠词inn。inn既可以像the一样添加在被修饰的名词前面,但它也可以后缀添加在名词后面。这种-inn后缀就称为名词的特指后缀。

类似于希腊语,古诺尔斯语的定冠词inn也根据性、数、格发生变化,在单独使用时,inn的变格如下所示:

\begin{longtable}{llll}
  \toprule
     & \textbf{阳性} & \textbf{阴性} & \textbf{中性} \\
  \midrule
  \endhead
  \bottomrule
  \endfoot
  单数 &             &             &             \\
  N  & inn         & in          & it          \\
  A  & inn         & ina         & it          \\
  G  & ins         & innar       & ins         \\
  D  & inum        & inni        & inu         \\
  复数 &             &             &             \\
  N  & inir        & inar        & in          \\
  A  & ina         & inar        & in          \\
  G  & inna        & inna        & inna        \\
  D  & inum        & inum        & inum        \\
\end{longtable}

当定冠词作为特指后缀出现时,-inn的形式会根据名词的词尾发生些许变化:

\begin{info}
  \begin{enumerate}
    \item  如果-inn前是一个非重读短元音,-i-一定会脱落(见 \ref{元音的音变} 规则2)。
    \item  对于双音节的-inn的变格(如inna,inum),-i-还在下列两种情况下脱落:
          \begin{enumerate}[i.]
            \item  i紧跟着一个长元音,如á-nni.
            \item  只有一个n的双音节形式(如inum,inar)在辅音后失去i,如skildir-nir,    但阴性单数宾格的-ina除外(见 \ref{元音省略和缩合} 规则1)。
          \end{enumerate}
    \item  对于单音节的形式,-i-在长元音后保留,如kné-in。
  \end{enumerate}
\end{info}

-inn还对格词尾有一定影响。当加后缀-inum时,复数与格的尾音-m脱落。因此有这样的音变:-um
+ -inum > -u + -inum > -u -num。

我们用几个强名词示例:

\begin{longtable}{llll}
  \toprule
     & \textbf{阳性} & \textbf{阴性}  & \textbf{中性} \\
  \midrule
  \endhead
  \bottomrule
  \endfoot
  单数 &             &              &             \\
  N  & úlfr-inn    & gjǫf-in      & tré-it      \\
  A  & úlf-inn     & gjǫf-ina     & tré-it      \\
  G  & úlfs-ins    & gjafar-innar & trés-ins    \\
  D  & úlfi-num    & gjǫf-inni    & tré-nu      \\
  复数 &             &              &             \\
  N  & úlfar-nir   & gjafar-nar   & tré-in      \\
  A  & úlfa-na     & gjafar-nar   & tré-in      \\
  G  & úlfa-nna    & gjafa-nna    & trjá-nna    \\
  D  & úlfu-num    & gjǫfu-num    & trjá-num    \\
\end{longtable}

一些弱名词示例如下,弱名词词尾的元音造成了大量的缩合:

\begin{longtable}{llll}
  \toprule
     & \textbf{阳性} & \textbf{阴性} & \textbf{中性} \\
  \midrule
  \endhead
  \bottomrule
  \endfoot
  单数 &             &             &             \\
  N  & bogi-nn     & kona-n      & auga-t      \\
  A  & boga-nn     & konu-na     & auga-t      \\
  G  & boga-ns     & konu-nnar   & auga-ns     \\
  D  & boga-num    & konu-nni    & augu-nu     \\
  复数 &             &             &             \\
  N  & bogar-nir   & konur-nar   & augu-n      \\
  A  & boga-na     & konur-nar   & augu-n      \\
  G  & boga-nna    & kvenna-nna  & augna-nna   \\
  D  & bogu-num    & konu-num    & augu-num    \\
\end{longtable}

古诺尔斯语和英语对于定冠词(或特指后缀)的用法有很大共性但不完全一致。总的来说古诺尔斯语使用特指的情况比英语要少一些,最显著的一点是在描述读者或作者熟悉的事物时,不需要使用特指标记,特别是例如konungr
`king', dróttning
`queen'这样的名词。另外,属格结构一般也不用加特指后缀,但它表示的也是特指含义。

定冠词或特指后缀有如下的用法:

\begin{enumerate}
  \item 定冠词单独使用时:
        \begin{enumerate}
          \item 置于名词或形容词前,类似于英语:
                \begin{quote}
                  inn maðr `the man'

                  inn blindi maðr `the blind man'
                \end{quote}

          \item 有时也置于副词、形容词比较级之前。在比较级时只用in:
                \begin{quote}
                  it sama `the same, likewise'

                  in heldr `the more'
                \end{quote}

          \item 置于名词/代词和形容词之间:
                \begin{quote}
                  专有名词+称谓:Óláfr inn helgi `Olaf the saint'

                  名词+限定性形容词:hendi inn hœgri `hand the right > the
                  right hand'

                  人称代词+形容词:þú hinn armi `you the wicked > thou
                  wretch!'
                \end{quote}

          \item 置于指示代词和名词短语之间:
                \begin{quote}
                  sá inn blindi maðr `that the blind man > the/this/that
                  blind man'

                  这种说法在英语中不可行。另外,还可以改成类似于用法3的说法:

                  maðr sá inn blindi / sá maðr inn blindi
                \end{quote}

        \end{enumerate}

  \item 作为特指后缀添加在名词后:
        \begin{enumerate}
          \item 缺少形容词时,通常加特指后缀:
                \begin{quote}
                  maðrinn `the man'
                \end{quote}

          \item 后接物主代词时,前面的名词常用特指后缀表示强烈语气;有时不接物主代词单独使用特指后缀也有这样的作用:
                \begin{quote}
                  ástin / ástin mín `O my love!'
                \end{quote}

          \item 特指后缀后,依然可以用其他指示词:
                \begin{quote}
                  hǫndin sú hœgri `hand-the that right > the right hand'
                \end{quote}

        \end{enumerate}

\end{enumerate}
