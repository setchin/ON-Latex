\chapter{介词}
\label{chp:preposition}
\begin{introduction}[章节要点]
    \item 介词的种类及结构
    \item 介词的基本用法
    \item 常见的介词
\end{introduction}

\section{介词的概述}
介词是一类不可变化的词类,它们的词形在各种位置上都保持一致。绝大多数情况下,介词都和名词短语一起构成介词短语,表示各类与时间、地点、方式等形状有关的概念。古诺尔斯语的介词短语结构和英语非常相似。在介词短语中,介词几乎总是出现在名词之前的,这也就是英语\textbf{Pre}position的来源。但偶尔介词也可以出现在名词的后面,这时最好用更准确的说法\textbf{Post}position来称呼这类介词。

介词虽然不能发生变格,但它对接续的名词的格有支配性作用。在英语中,这个现象被称为“介宾结构”,即介词后一律接宾格。在古诺尔斯语中,介词后的名词可以是间接格中的任意一个。同一个介词也可能通过接续不同的格来表达不同的含义,熟悉德语的读者立刻会想到“静三动四”的规则,即表示方位性的介词可以接第三格(即与格)名词表示静态的含义,接第四格(即宾格)名词表示动态的含义,如in+第三/第四格可表示英语中的in/into。在古诺尔斯语中,也有类似的情况。

一些介词也可以单独作副词使用。从严格意义上来说,此时介词的词性已经发生了改变,不过这在语义上并不构成区别。参考下面的句子:
\begin{enumerate}[itemindent=1em]
    \item He walked into the house
    \item He walked in without saying `hello'
\end{enumerate}
古诺尔斯语中最基本的介词包括á `on', af `off', at `at, to', frá `from', í `in', með `with',  um `about, in', við `with',这几个介词无一例外都由原始日耳曼语继承而来。读者也可以立刻发现,它们的英语翻译几乎就是它们的同源词,形态非常类似。还有一类介词由其他实义名词演变而来,如til `till, to'来源于PGmc. *tila `goal', meðal `between'来源于miðr `middle'等等。一般来说,第二类介词(即由名词派生出来的介词)往往和第一类介词(即固有的介词,且主要是á, í, um这三个)连用,如í meðal,这个词组整体起到介词的作用,读者也可以把这个词组理解为 `in the middle of'(=between)。不过,这种词组里的第一类介词并不是必要的,也可以省略。有些第二类介词也总是单独使用,如til.
\section{介词的补语}
名词短语是介词最常见的补语,但不同的介词支配不同的格,因而补语需要根据介词的类型变格。此外,介词还可支配从句补语。
\subsection{支配宾格的介词}
\begin{enumerate}[itemindent=1em, label=\textbf{\arabic*}.]
    \item \textbf{of}

          of是一个古老的介词,几乎只出现在诗歌中。它一般支配宾格,但是也可以接与格,格对语义基本没有影响。

          of的主要意思是“在...之上”,相当于英文`over'.

    \item \textbf{um}

          um和of有密切的联系。这个词最早的形式是umb,在诗歌中umb有时变成了of,但最后um取代了of和umb成为了最常见的形式。um有以下几个意思:
          \begin{enumerate}
              \item 表示环绕

                    相当于英语的`around',这是um最基本的含义:
                    \begin{quote}
                        fara í hring um skipit `go in a ring around the ship = circle around the ship'
                    \end{quote}

              \item 表示表面

                    泛指“经过...的表面”,类似于英语的`over, about',但这个介词主要不是表达位置关系的(见á),参考下面的例子:
                    \begin{quote}
                        um allar sveitir `all over the country'

                        kominn um langan veg `come a long way off'

                        herja um Skotland  `harry Scotland'
                    \end{quote}

              \item 表示越过

                    大致表示“越过,从上面经过”等,和of的意思相似,类似于英语`across, over, beyond, past',和表示动作的动词连用:
                    \begin{quote}
                        sigla vestr um Bretland `sail west past Bretland'

                        ríða um tún `pass by a place/house'

                        bera \k{o}l um eld `bear the ale across the fire'
                    \end{quote}
                    另外,表示这个含义时,um有时也可以接与格。
              \item 表示时间

                    um接续表示时间的名词时有两种用法:
                    \begin{enumerate}
                        \item 接续段时间

                              表示`during',尤强调时间的持续性(相比á和í):
                              \begin{quote}
                                  um nótt `throughout the night'

                                  um alla daga `all day long'
                              \end{quote}

                        \item 接续点时间

                              表示`at':
                              \begin{quote}
                                  um dagmál, náttmál\footnotemark `at morning, night'
                              \end{quote}
                              \footnotetext{dagmál, náttmál的字面义是`day-meal, night-meal',它们用来表示时间,分别相当于早上和晚上的八九点}
                    \end{enumerate}

              \item 引申义

                    um的引申义非常多,它最主要的意思由`around'延申出来,大致表示“关于,涉及”,相当于英语`of, about, concerning':
                    \begin{quote}
                        tala, þræta, spyrja um e-t `speak of, quarrel about, ask about something'
                    \end{quote}
          \end{enumerate}
    \item \textbf{(í) gegnum}

          表示“穿过”,相当于英语的`through'。括号中的í表示这个第一类介词是可加可不加的,即gegnum和í gegnum都正确。
          \begin{quote}
              leggja í gegnum skj\k{o}ldinn `thrust through the shield'
          \end{quote}

    \item \textbf{um fram}

          表示“越过,超过;多于”,类似于英文`beyond, over, more than':
          \begin{quote}
              um fram alla menn `more than any man'
          \end{quote}

\end{enumerate}
\subsection{支配与格的介词}
所有表示“从...而来;离开...”等类似含义的介词都支配与格,因为原始印欧语的离格(Ablative)\footnotemark 合并到了日耳曼语的与格当中。当然,还有一部分其他的介词也支配与格。
\footnotetext{或译为夺格,从格。印欧语中表示“从...离开”这一含义的名词格。}
\begin{enumerate}[itemindent=1em, label=\textbf{\arabic*}.]
    \item \textbf{af}

          af的含义非常多,下面列举几个主要的用法:
          \begin{enumerate}
              \item 表示来源

                    表示“从...表面/上面而来”,相当于`off, from',它尤其对应使用á(`on',参见\ref{sec:prep_with_DA}中相应内容)表示的地点。也就是说,如果表示“在某处”用的是á + swh,那么“从该处来”就用af + swh:
                    \begin{quote}
                        draga gullhring af hendi sér `take the golden ring from his hand'

                        \HandRight \quad hringr er \textit{á} hendi sér `the ring is on his hand'

                        hlaupa af hesti sínum `jump off his horse

                        \HandRight \quad ríða \textit{á} hesti sínum `ride on his horse'
                    \end{quote}
                    但是,有些用í `in', um `around', yfir `over'提示的地点也会用到af,这些是习惯性的表达:
                    \begin{quote}
                        (sjómenn) koma af hafi `(of sailors) come back from the ocean'

                        \HandRight \quad \textit{í} hafi `at sea'(参见í的用法\ref{quote_hafi})

                        taka af herklæðum `take off the armour'

                        \HandRight \quad setja herklæði \textit{yfir} honum `put armour over him'
                    \end{quote}
              \item 表示时间

                    表示“过了...”,相当于英文`past, gone from':
                    \begin{quote}
                        af barnsaldri `having past adolescence'

                        mikit var af nótt 字面义 `much was gone from the night' (`= much of the night was past')
                    \end{quote}

              \item 引申义
                    \begin{enumerate}

                        \item 表示部分

                              表示整体中的一部分,相当于英文`of, off':
                              \begin{quote}
                                  hlutr af Skotlandi `a big part of Scotland'

                                  h\k{o}ggva h\k{o}nd, h\k{o}fuð, fót af honum `cut his hand, head, foot off'
                              \end{quote}

                        \item 表示材质

                              和ór同义,但ór用得更多些:
                              \begin{quote}
                                  hringr af gulli `golden ring'
                              \end{quote}

                        \item 表示原因

                              类似于英文`out of, by reason of':
                              \begin{quote}
                                  af frændsemis s\k{o}kum `for kinship's sake

                                  úbygðr af frosti ok kulda `uninhabited because of frost and cold'
                              \end{quote}

                        \item 表示被动句的施事

                              就一般情况而言,主动句中施事是主语,受事是宾语;在被动句中同一动作的受事提升为主语,而施事常常被省略(因为被动句更关心动作本身或动作的受事),如果在被动句中要表达施事,可以用af,它的作用相当于英文的`by':
                              \begin{quote}
                                  meira virðr af m\k{o}nnum `highly esteemed by men'
                              \end{quote}
                    \end{enumerate}
          \end{enumerate}

    \item \textbf{ór}
          \begin{enumerate}
              \item 表示来源

                    和af相反,ór表达的概念是“从...内部而来”,因此它与í(`in',参见\ref{sec:prep_with_DA}中相应内容)相呼应。因此,它相当于英文的`out of':
                    \begin{quote}
                        hlaupa út ór stofunni `run out of the building'
                    \end{quote}

              \item 表示材料

                    这是ór的引申义,和af表材料时用法、意义都相同。
          \end{enumerate}

    \item \textbf{frá}

          偶尔也有ífrá, áfrá的形式,相当于英文的`from'。区别于af和ór, frá并不和某些表示静态位置的介词严格对应,它宽泛地表示“来源;分离;起始”等概念,既可以接续表示地点的名词,也可以接续表示时间的名词:
          \begin{quote}
              skamt frá ánni `not far from the river'

              frá \th essum degi `from this day')
          \end{quote}

    \item \textbf{undan}
          \begin{enumerate}
              \item 表示方向

                    undan的基本意思是“从下面”,由undir `under'这一形容词演变而来,但增加了离格的含义。它相当于英文的`from beneath':
                    \begin{quote}
                        springa upp undan borðinu `jump up from the table'
                    \end{quote}

              \item 表示远离

                    表示“躲避、撤退、逃离”等含义时,可用undan:
                    \begin{quote}
                        renna undan óvinum `run away from enemies'
                    \end{quote}
          \end{enumerate}

    \item \textbf{at}

          相当于英文中的`at'和`to',有多种含义:
          \begin{enumerate}
              \item 表示方向

                    相当于`at, to, towards',常和表示位移的动词连用,既可以和具体的地点名词连用,也可以和一些抽象名词连用表示“投身于某事”等类似的含义:
                    \begin{quote}
                        fara, ganga, koma at ... `travel to, go to, come to ...'

                        ríða at hrossum, sauðum 字面义`go to the horses, sheep',实际表示照看它们,类似于`attend to'
                    \end{quote}
                    at常有“接近”的含义:
                    \begin{quote}
                        ganga allt at honum `go quite up to him'
                    \end{quote}
                    有时,动词+at含有一定的敌意,和英语类似(比较shout to和shout at):
                    \begin{quote}
                        renna at ... `rush at, assault ...'
                    \end{quote}

              \item 表示地点

                    宽泛地表示“在”,并不提示位置关系,相当于英语的`at',常和表示静态的、存在性的动词连用:
                    \begin{quote}
                        sitja, standa, vera at ... `sit, stay, be at ...'
                    \end{quote}

              \item 表示时间

                    一般接续时刻,特别是表示某事发生的那个瞬间时,常用at,和英文中的用法类似。除了接续表示时间的名词外,如果接续表示事件的名词,则表示这个事件发生的时刻:
                    \begin{quote}
                        at sinni `at present'

                        at skilnaði `at their parting = when they parted'
                    \end{quote}

              \item 引申义
                    \begin{enumerate}
                        \item 表示变化

                              变化的方向常用at提示,相当于英文的`to':
                              \begin{quote}
                                  brenna at \k{o}sku `burn to ashes'

                                  verða at e-u `turn into something'
                              \end{quote}

                        \item 表示来源

                              表示从某人获得信息、知识、物品等时可用at,意思等同于frá:
                              \begin{quote}
                                  nema frœði at e-m `learn knowledge from someone'

                                  kaupa land at  e-m `buy the land from someone'
                              \end{quote}

                        \item 表示遵循

                              相当于`according to':
                              \begin{quote}
                                  at ráði allra vitrustu manna `according to the device from all the wisest men'
                              \end{quote}

                        \item 表示方面

                              由表示地点的at引申而来,类似于`be good at'中的at,表示`as to, in respect of, in regards to':
                              \begin{quote}
                                  auðigr at fé `wealthy of goods'

                                  spekingr at viti `wise man in terms of wits = a man of great intellect'


                              \end{quote}

                    \end{enumerate}
                    此外,在诗歌中也出现过at+宾格表示“跟随”的用法,与eptir含义相同。
          \end{enumerate}

    \item  \textbf{(í) gegn; (á/í) mót(i)}

          这两个介词都表示“相反”,相当于`against, opposite to'。另外请注意和(í) gegn和(í) gegnum的区别,后者是接宾格的。有时这两个介词也有“直冲着”的意思,是从“相反”这个含义进一步演化过来的:
          \begin{quote}
              mæla honum í gegn `speak against him'

              sjá í móti sólu `look straight at the sun'
          \end{quote}

    \item  \textbf{hjá}
          \begin{enumerate}
              \item 表示相邻

                    表示“紧靠着;毗邻”,相当于英文中`by, near'。因此,这个词也有“与...一起”的意思:
                    \begin{quote}
                        setjask niðr hjá honum `take a seat by his side'

                        vera hjá e-m `stay with someone'
                    \end{quote}

              \item 表示经过

                    相当于英文中`passing by':
                    \begin{quote}
                        sneiða hjá ... `pass by ...'

                        farask hjá `pass by one another'
                    \end{quote}
          \end{enumerate}

    \item \textbf{nær}

          表示“接近”,相当于英文`near',可接地点和时间名词。这个介词由形容词的原型演变而来,因此有时出现在名词的后面:
          \begin{quote}
              brautu nær `near the road'

              nær aptni `near night'
          \end{quote}
\end{enumerate}
\subsection{支配属格的介词}
\begin{enumerate}[itemindent=1em, label=\textbf{\arabic*}.]
    \item \textbf{til}

          til是最常见的接续属格的介词。
          \begin{enumerate}
              \item 表示方向

                    相当于`to, towards':
                    \begin{quote}
                        ganga til kirkju, boðs ... `go to  church, banquet ...'

                        ganga til svefns `go to sleep'
                    \end{quote}

              \item 表示时间

                    表示时间上的“直到...”,相当于`til, until':
                    \begin{quote}
                        til dauðadags `til the day of death'
                    \end{quote}

              \item 表示目的

                    表示“目的、能力、性质”等,相当于英语中表示目的的`to do'不定时中的`to'或者`for',参见例句:
                    \begin{quote}
                        hross til reiðar `horse for riding'

                        hlaðinn til hafs `ready for use'

                        sverð öruggt til vápns `sword reliable (enough) to be a weapon = a reliable sword'
                    \end{quote}
          \end{enumerate}

    \item (á/í) meðal/milli/millum

          这三个复合介词都表示同一个意思“在...之间”,相当于英文中`between, among, in the middle of',可以接时间概念也可以接地点概念:
          \begin{quote}
              meðal þín ok annarra `between you and another'

              milli jóla ok f\k{o}stu\footnotemark `between Yule and Lent'

              sigla millum landa `sail from one land to another'
          \end{quote}

          \footnotetext{Jól(英文Yule)是古日耳曼地区在冬至时庆祝的节日,后来在基督教传入后演变为圣诞节;\\Fasta(英文称Lent)是大斋节,基督教传入后带来的节日,大约在每年的二月至三月。}

    \item \textbf{innan}

          表示“在...之内”,相当于`within',可接时间和地点名词:
          \begin{quote}
              innan lítils tíma  `within a short time'

              innan borgar `in town'
          \end{quote}
\end{enumerate}
\subsection{支配与格和宾格的介词}
\label{sec:prep_with_DA}

相当一部分的介词既可以支配与格又可以支配宾格,名词使用哪个格主要根据语义来决定。一般来说,介词短语表示静态的、方位性质(Location)的概念时,名词用其与格形式;表示动态的、方向性质(Motion)的概念时,则用宾格。当然,除了静态——动态这一区分外,有些介词在表示时间概念的时候接宾格,表示地点概念的时候接与格。有些介词既能接与格表示时间也能接宾格表示时间,但支配与格时表意更倾向于一次性的、特定的时间范畴,而接宾格时则有表达重复性的时间概念的含义。

以下是一些常见的可支配与格和宾格的介词,本书给出了它们的一般用法,但许多介词都有多种含义,读者需要经常借助字典解决:
\begin{enumerate}[itemindent=1em, label=\textbf{\arabic*}.]
    \item \textbf{á} \label{prep:on}

          á是一个典型的支配与格和宾格的介词。它有表示时间和地点两方面的用法:
          \begin{enumerate}
              \item 接与格
                    \begin{enumerate}
                        \item 表示地点

                              表示“在...之上”,相当于英语的`on', `upon',如:

                              \begin{quote}
                                  á gólfi `on the floor'

                                  á sjá ok á landi `on sea and land'
                              \end{quote}
                              与`land'及其相关派生词连用时,基本上用á,如á Englandi, Írlandi, Skotlandi.

                              当然,不是所有英语用介词`on'的情况在古诺尔斯语中就用á,许多用法仍是习惯性的。特别地,á和\'{i}(另见下)的用法常常有重叠的地方,如á himni ok j\k{o}r\dh u `in heaven and on earth',但有í helviti `in hell',读者从英文翻译中也能发现语义相近的概念接续的介词不一定相同。
                        \item 表示时间

                              表示时间过程,相当于英语的`during',如:
                              \begin{quote}
                                  á því ári, sumri ... `during that year, summer ...'
                              \end{quote}

                        \item 固定用法,表示身体部位

                              表示某人的身体部位时,一般不用所有格(属格),而常用介词短语。如:
                              \begin{quote}
                                  hendr, eyru ... á mér `my hand, ear ... =(hendr mín,正确但不自然)'
                              \end{quote}
                              但表示人体内的部分(hjarta `heart', auga `eye')时,一般用í.\\
                              有些时候,表示类似于人的肢体和人这样不可分离的所有关系时,也用á表达,如dyrr á húsi `door of the house'.
                    \end{enumerate}
              \item 接宾格
                    \begin{enumerate}
                        \item 表示地点

                              表示动态地“来到...之上”,相当于英语`onto',如:
                              \begin{quote}
                                  ganga á land `come ashore'
                              \end{quote}
                        \item 表示时间

                              表示时间时的,介词的“动态性”比较隐秘,如á morgun表示的是即将到来的早晨,即`tomorrow'。\\
                              除此之外,接与格和接宾格的不同点在于接宾格时一般要用名词的特指形式,而接与格时不用;接宾格时,表意更倾向于重复性的时间概念,因此表示节日、一周中的某天时,也用宾格。
                              \begin{quote}
                                  á vetrinn `every winter'

                                  á Jóladaginn `on Yule's day'

                                  á Sunnudag `on Sunday'
                              \end{quote}


                    \end{enumerate}
          \end{enumerate}

    \item \textbf{í} \label{prep:in}

          í是á的反义词,也有表示时间和地点两方面的用法:
          \begin{enumerate}
              \item 接与格
                    \begin{enumerate}
                        \item 表示地点

                              表示“在...内部”,相当于英语的`in',接与格表示静态的方位含义。特别地,它常和表示岛屿、深谷、峡湾、树林、洼地、江河湖海的名词连用:
                              \begin{quote}
                                  í húsi, h\k{o}ll, skála ... `in the house, hall, lodge ...'

                                  í eyju, dal, fir\dh i, skógi ... `in the island, dale, firth, woods ...'

                                  í lægi, á, lœk, hafi ... `in the bay, river, lake, ocean ...'
                                  \label{quote_hafi}
                              \end{quote}
                        \item 表示时间

                              和á含义类似。表示时间时一般用宾格较多。
                    \end{enumerate}
              \item 接宾格
                    \begin{enumerate}
                        \item 表示地点

                              与接与格的含义相对,表示动态的过程,“进入...的内部”,相当于英语的`into':
                              \begin{quote}
                                  sigla í haf `sail into the sea'

                                  verpa sér í s\k{o}ðulinn `throw oneself into the saddle=(mount)'
                              \end{quote}
                        \item 表示时间

                              大致等于英语的`during, at',í接时间概念时,经常和一天中的部分连用:
                              \begin{quote}
                                  í miðjan morgin `in the middle of morning=(at six o'clock)'
                              \end{quote}
                              它也常常表示“距离现在最近的时刻”,相当于英文中的`this+时间':
                              \begin{quote}
                                  í morgin, kveld, nott, vetr ... `this morning, evening, tonight, winter ...'
                              \end{quote}
                    \end{enumerate}
          \end{enumerate}

    \item \textbf{eptir}

          eptir接与格和宾格时表示不同的含义,它相当于英语的`after'.
          \begin{enumerate}
              \item 接与格

                    表示地点含义的“在...之后,跟着...”时用与格:
                    \begin{quote}
                        rí\dh a eptir \th eim `ride after him'
                    \end{quote}

              \item 接宾格

                    表示时间含义的“在...之后”时用宾格:
                    \begin{quote}
                        ár eptir ár `year after year'
                    \end{quote}
                    eptir e-n常表示“在某人死后;继承某人的遗产”之类的含义:
                    \begin{quote}
                        taka arf eptir f\k{o}\dh ur sinn `take inheritance after his father'
                    \end{quote}
          \end{enumerate}

    \item \textbf{fyrir}

          fyrir大致是eptir的反义词,这个词有多种形式。它一般写作fyrir或firir,但有时也缩略为单音节的fyr.
          \begin{enumerate}
              \item 接与格

                    \begin{enumerate}
                        \item 表示地点

                              表示“在...之前”,相当于英语的`before, for, fore-'。在这个基本意义的基础上,“向前”引申出“在...面前,当面”“引导...”等含义。特别地,与表示言说的动词连用时,fyrir表示的就是说话的对象:
                              \begin{quote}
                                  fyrir dyrum `before the doors, at the doors'

                                  fyrir Guði `before God'

                                  mæla fyrir honum `say before him = tell him, say to him'

                                  á varð fyrir þeim `a river was in front of them'

                                  ráða fyrir landi, ríki ... `reign over the land, kingdom ...'
                              \end{quote}

                        \item 表示时间

                              表示“在...之前”,相当于英语的`ago':
                              \begin{quote}
                                  fyrir þrem nóttum `three nights ago'
                                  fyrir stundu `a while ago'
                              \end{quote}

                        \item 引申义

                              fyrir有很多引申义,这往往要根据上下文决定。
                              \begin{enumerate}
                                  \item 表示对于

                                        尤用在不好的事情上,英语中类似于`it is bad for sb. to do something'中的`for sb.'可用fyrir e-m表示:
                                        \begin{quote}
                                            taka fé fyrir \k{o}ðrum `take anothers' money' fyrir \k{o}ðrum表受害的对象

                                            fara ílla fyrir e-m `become bad, turn out to be ill for someone'
                                        \end{quote}

                                  \item 表示原因

                                        和英文中`for'表示原因同理:
                                        \begin{quote}
                                            deyja fyrir harmi `die because of sorrow'
                                        \end{quote}
                                        进一步地,可以表示“受...影响”时也可用fyrir。在表示对抗、争斗时尤其常见:
                                        \begin{quote}
                                            hafa bana fyrir þeim 字面义:`have death before them',实际表示受他们的影响而死(=`be killed by them')

                                            verða halloki fyrir þeim `become overcome before them = be overcome in their fighting'
                                        \end{quote}
                                  \item 其他引申义

                                        fyrir还有许多其他引申义,往往和动词的语义有关,如leiða fyrir skipi `helm the ship',请读者查阅词典解决。

                              \end{enumerate}
                    \end{enumerate}
              \item 接宾格

                    与接与格时的表意类似,但多了一层动态的含义。
                    \begin{enumerate}
                        \item 表示地点

                              \begin{enumerate}
                                  \item 基本含义

                                        动态地“来到...之前”,一般和表示位移的动词连用:
                                        \begin{quote}
                                            ganga fyrir þeim `go before them'
                                        \end{quote}

                                  \item 复合结构

                                        在fyrir + -an型副词 + 宾格名词这一复合结构中,它表示“沿着某个方向”,这个方向是由-an型副词指定的:
                                        \begin{quote}
                                            r\'{i}\dh a fyrir austan dyrr `ride east towards the door'

                                            spjót kom í skj\k{o}ld fyrir ofan mundriðann `the spear hit the shield above the handle'
                                        \end{quote}

                              \end{enumerate}

                        \item 表示时间

                              相当于英文的`before',与接与格时表示的`ago'有语义上的差别。fyrir + D 是从现在时间往回推算与格名词表示的时间长度,而fyrir + A 表示的是宾格名词代表的时间之前的任意一个时刻。这个区别和英语中的`before'和`ago'是一样的。
                              \begin{quote}
                                  fyrir sól `before sunrise'
                              \end{quote}

                        \item 引申义
                              \begin{enumerate}
                                  \item 表示代替

                                        近似于英语中表示“代替;交换”等含义的`for':
                                        \begin{quote}
                                            ganga fyrir þik `go for you, go in your stead'

                                            gjalda \th rjár merk fyrir hana `pay three marks for her'
                                        \end{quote}

                                  \item 其他引申义

                                        和与格的情况一样,接宾格时有些词组有特殊的含义,请查阅字典解决。
                              \end{enumerate}
                    \end{enumerate}

          \end{enumerate}

    \item \textbf{me\dh}
          \begin{enumerate}
              \item 接与格

                    me\dh 的用法基本和英语`with'相同。
                    \begin{enumerate}
                        \item 表示伴随

                              相当于英文的`with, together',如:
                              \begin{quote}
                                  fara með honum `go with him'
                              \end{quote}

                        \item 表示工具

                              相当于英文的`with'或某些与格名词的用法(古诺尔斯语的与格继承了原始印欧语的工具格,参见交叉引用),如:
                              \begin{quote}
                                  verja sik með sverðum, skj\k{o}ldum `defend oneself with swords, shields'
                              \end{quote}

                        \item 表示方式

                              相当于英文`with, by, using',如:
                              \begin{quote}
                                  með sama hætti `in the same manner'

                                  með hlaupi `by running'
                              \end{quote}

                        \item 表示范围之中

                              相当于英文`among',如:
                              \begin{quote}
                                  siðr með kaupm\k{o}nnum `custom among merchants'
                              \end{quote}

                    \end{enumerate}
              \item 接宾格
                    \begin{enumerate}
                        \item 表示伴随

                              和接与格的表伴随含义类似,但是接宾格时有“携带、支配”之义,而非是简单的“与...一起”。因此,宾格名词一般是无生命的:
                              \begin{quote}
                                  fara með vápn `go carrying a weapon'
                              \end{quote}

                        \item 表示材料

                              相当于`made with',如:
                              \begin{quote}
                                  kirkja með stein `stone church'
                              \end{quote}
                    \end{enumerate}
          \end{enumerate}

    \item \textbf{undir}

          undir的含义和英文`under'类似。
          \begin{enumerate}
              \item 接与格

                    表示静态方位“在...之下”,如:
                    \begin{quote}
                        undir hesti hans `under his horse'
                    \end{quote}
              \item 接宾格

                    表示动作“来到...之下”,如:
                    \begin{quote}
                        fara undir skipit `come from beneath the ship'
                    \end{quote}
          \end{enumerate}
          无论是接与格还是宾格,undir和一些动词连用也有引申含义,请读者查阅字典解决。

    \item \textbf{yfir}

          yfir是undir的反义词,与英文`over'类似。
          \begin{enumerate}
              \item 接与格

                    表示静态位置“在...之上”,如:
                    \begin{quote}
                        stenda yfir Niflheimi `stand over Nifheim'
                    \end{quote}
              \item 接宾格

                    表示动作“来到...之上”,如:
                    \begin{quote}
                        hlaupa yfir netit `jump over the net'

                        vexa yfir hann `grow over him = be taller than him'
                    \end{quote}
          \end{enumerate}
          无论是接与格还是宾格,yfir和一些动词连用也有引申含义,请读者查阅字典解决。

    \item \textbf{við}
          \begin{enumerate}
              \item 接与格
                    \begin{enumerate}
                        \item 表示位置

                              表示“靠着”,相当于英文`against',如:
                              \begin{quote}
                                  slá honum niðr við steininum `smash him down to the stone'
                              \end{quote}

                        \item 表示伴随、工具、方法

                              此时和með同义。með和við在历史上发生了语义上的混淆,因此við + D也可以表示大部分með + D的功能。
                    \end{enumerate}

              \item 接宾格
                    \begin{enumerate}
                        \item 表示位置

                              相当于`near, by',如:
                              \begin{quote}
                                  sitja við elda `sit by fire'

                                  skj\k{o}ldr við skj\k{o}ld `shield to shield (in a row)'
                              \end{quote}

                        \item 表示时间

                              表示“接近...时刻”,类似于英文中`at, towards',如:
                              \begin{quote}
                                  við þat sjálft `at that moment'

                                  við aptan `at night'
                              \end{quote}

                        \item 表示方向

                              相当于`towards, to, respecting, regarding',一般接表示人或物的名词,和动词连用提示动作的对象:
                              \begin{quote}
                                  tala, mæla, segja, ræða við e-n `talk to, consult someone'
                              \end{quote}

                        \item 表示伴随

                              和með接宾格时含义类似。

                        \item 引申义
                              \begin{enumerate}
                                  \item 表示原因

                                        宽泛地表示造成某种现象的原因,类似于`by, at, with'等。此时,við可以和各种词类连用:
                                        \begin{quote}
                                            falla við h\k{o}gg `fall by a stroke'

                                            verða glaðr, reiðr við ... `be glad, angry with ...'

                                            sigla við stj\k{o}rnu-ljós ` sail by star-light'
                                        \end{quote}

                                  \item 表示相符

                                        表示“相符,适合”,类似于英语`according to, after':
                                        \begin{quote}
                                            gera klæði við v\k{o}xt hans `make clothes according to his shape'

                                            skapaðr við sik `shaped according to oneself = be well shaped'
                                        \end{quote}
                              \end{enumerate}
                    \end{enumerate}
          \end{enumerate}
\end{enumerate}

\subsection{其他接多个格的介词}
还有两个介词不属于上述的分类,它们分别是útan和án:
\begin{enumerate}[itemindent=1em, label=\textbf{\arabic*}.]
    \item \textbf{útan}

          útan可接属格和宾格,它有两个含义:
          \begin{enumerate}
              \item 表示外部

                    既可以表示“从外部来”,也可以单纯地表示“在外部”,相当于英文`(from) outside',此时用属格:
                    \begin{quote}
                        fara útan hera\dh s `go outside the district'
                    \end{quote}

              \item 表示缺少

                    相当于英文`without',此时接宾格:
                    \begin{quote}
                        \th eir eru útan sætta `they are without a settlement'
                    \end{quote}
          \end{enumerate}

    \item \textbf{án}

          án相当于英语中的`without',它是唯一一个可以接任意一个间接格的介词,格不影响含义。不过后来基本只用属格。
          \begin{quote}
              Kristnin mátti eigi vera lengi án stjórnarmanninn `The Church could not be long without its leader'
          \end{quote}
\end{enumerate}

\subsection{介词的从句补语}
古诺尔斯语中几乎所有的介词都可以接从句补语,这个从句由at引导,它可以是限定的,也可以是非限定的(参见)。

介词引导非限定从句时,类似于英语中介词+doing的用法:
\begin{exe}
    \ex \gll
    strengir váru hafðir til at festa skip\\
    ropes were used to that fast ship\\
    \trans `Ropes were used for making the ship fast'
\end{exe}

介词引导限定性从句的情况比较少,常见的是类似于til at这样表示结果的习惯说法:
\begin{exe}
    \ex \gll
    skal ek nauðga þeim til at þeir segi it sanna\\
    shall I force them to that they tell-{\sub} the truth\\
    \trans `I shall force them to tell the truth'
\end{exe}

\section{省略补语的介词用法}
如果介词的补语可以从上下文判断,那么此时补语可以省略,使得介词好像副词一般使用。例如一段阐述冰岛语字母体系的文献中有下面这句话:
\begin{exe}
    \ex \gll
    \textit{ór} eru teknir samhljóðendr nǫkkurir ór látínustafrófi, en nǫkkurir \textit{í} gǫrvir\\
    out are taken consonants some from latin-alphabet, but some in made\\
    \trans `Some consonants from Latin alphabet are taken out of our alphabet, but some are added'
\end{exe}

在本句中,两个介词ór和í都省略了váru stafrófi `our alphabet'.

有时候这种省略更加隐蔽,因为介词后出现了其他名词短语,它虽然不是介词的补语,但可能是句子的其他成分,例如:
\begin{exe}
    \ex \gll
    hann finnr, at þar var stungit \textit{í} sverði Sigmundar\\
    He finds, that there was thrust in sword-{\dat} Sigmund-{\gen}\\
    \trans `He finds that it was pierced by Sigmund's sword'
\end{exe}

在这句话中,与格名词sverði甚至和介词í是搭配的,但í sverði `in sword'这种说法是不成立的。因此,只能把sverði Sigmundar理解为方式状语。本句实际上是省略了主语it的被动句,其主动句为Sigmundr stingr sverði hans í honum `Sigmund stabs his sword in him'.

有时古诺尔斯语中还有用两个介词短语(同时是表示方位的)来表示复杂的位置关系。这时第一个介词短语的补语省略,出现了两个介词连在一起的情况。有时这种结构在英语种也存在,有时则是不被英语语法允许的。
\begin{quote}
    út í Skotlandi `from the inland of Scotland'

    ofan ór hlíðinni `down from the slope'
\end{quote}

此时第一个介词可以被理解为副词,这个作副词用的介词也可以放到完整的介词短语的后面,如:
\begin{quote}
    á land upp `up onto land'

    í sjá ofan `down into ocean'
\end{quote}



