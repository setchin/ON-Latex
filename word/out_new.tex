绪论

\textbf{一、古诺尔斯语简介及其在语系中的定位}

古诺尔斯语(Old
Norse)是北日耳曼语族的一个分支,发展自8世纪左右的原始诺尔斯语(Proto-Norse),在大约公元14世纪前,古诺尔斯语通行于斯堪地纳维亚居民以及维京人的海外殖民地。在维京人的口中,古诺尔斯语被称作dǫnsk
tunga,意思是丹麦人的语言,这主要是因为当时民族认知的特性所致。由于其地理位置的影响,古诺尔斯语也被称作古冰岛语或古挪威语,甚至更宽泛的说法:古北欧语或者古斯堪地纳维亚语。只要不引起歧义,上述的说法都是可行的。不过,在更严格的术语定义下,古冰岛语、古挪威语都可算作古诺尔斯语的方言,只是二者的区别比较小。因此,古诺尔斯语是学术界更通行的用法。

古诺尔斯语隶属于日耳曼语族的北支,日耳曼语族又可分为三支:

\begin{enumerate}
  \def\labelenumi{\arabic{enumi}.}
  \item
        北日耳曼语支。古代语言的代表即古诺尔斯语,在约公元14-15世纪后逐渐分化为现今的冰岛语、挪威语、瑞典语、丹麦语等。这些语言相似程度很高,除冰岛语尤其从古外,挪威语、瑞典语、丹麦语的书面语形式可以互通。这些现代语言在早期主要是古诺尔斯语的方言形式,关于这些方言的特征,参见下文。
  \item
        西日耳曼语支。古代语言的代表是古英语、古高地德语、古撒克逊语、古法兰克语等。它们演化为今天的英语、德语和荷兰语等语言。西日耳曼语是日耳曼语族中最大的一支,在现存的约70种日耳曼语和方言中,只有6种属于北日耳曼语,其他均属于西日耳曼语。
  \item
        东日耳曼语支。现已完全灭绝。古代语言的代表是哥特语,亦有零散的汪达尔语等的文本。东日耳曼语是日耳曼语族中最古老的一支,其文献记载是所有日耳曼语中最早的。哥特人曾大规模地向南移居过,因此在克里米亚地区保留了一支哥特人的部落,他们操克里米亚哥特语,但18世纪后亦灭绝。
\end{enumerate}

在古代日耳曼语中,比较有代表性的是古英语(Old
English,OE)、古高地德语(Old High
German,OHG),古诺尔斯语和哥特语(Gothic,Go.),这些语言保留了十分充分的文本资料,在必要的情况下,本书将引用它们和古诺尔斯语进行比较。

这几种语言活跃的时间如下表所示:

\begin{longtable}{lll}
  \toprule
  语言       & 谱系       & 时间     \\
  \midrule
  \endhead
  \bottomrule
  \endfoot
  哥特语     & 东日耳曼语 & 3-10世纪 \\
  古英语     & 西日耳曼语 & 5-12世纪 \\
  古高地德语 & 西日耳曼语 & 8-11世纪 \\
  古诺尔斯语 & 北日耳曼语 & 8-15世纪 \\
\end{longtable}

按照历史语言学的基本观点,所有日耳曼语都演变自一个共同的祖先------原始日耳曼语(Proto-Germanic,PGmc.)。原始语是一种拟构出来的,没有文字资料证实的假象语言,它对于揭示语言间的关联有着重要的作用。原始日耳曼语从一种更加古老、也更加重要的原始语中分化出来,后者称为原始印欧语(Proto-Indo-European,PIE)。原始印欧语和原始日耳曼语的规则都比较复杂,本书仅在必要的时候提供部分相关的知识,拟构出来的原始语用*标出,表示其未被证实存在。从原始日耳曼语中,东日耳曼语分化得最早,在音变等特征上相对比较保守;后分化的西日耳曼语和北日耳曼语则有一定的创新,例如规律的变元音(详见\ref{变元音})。西日耳曼语和北日耳曼语有一些共同的特征,因此一些学者认为它们来自一种原始西北日耳曼语,但尚无定论。

就古诺尔斯语本身而言,可以确定的是它从原始诺尔斯语演变而来。原始诺尔斯语大约流行于公元2-8世纪,虽然也称为``原始'',但与之前所述的原始印欧语和原始日耳曼语不同的是,原始诺尔斯语已经有了文字记载,但处于十分原始的阶段。原始诺尔斯语采用一种叫作卢恩文字的书写系统(区别于后期的拉丁字母),古代北欧人常将卢恩文字铭刻在石头、骨器上用以记录一些简单的信息,譬如人名、物品、祭祀时的祷告等。原始诺尔斯语的书写系统和书面记载都不发达,但它保留了许多词汇原本的特征。下图是卢恩字母表和卢恩石:

\begin{longtable}{llllllll}
  \toprule
  \textbf{ᚠ} & \textbf{ᚢ} & \textbf{ᚦ} & \textbf{ᚨ} & \textbf{ᚱ} & \textbf{ᚲ} & \textbf{ᚷ} & \textbf{ᚹ} \\
  \midrule
  \endhead
  \bottomrule
  \endfoot
  f          & u          & þ          & a          & r          & k          & g          & w          \\
  \textbf{ᚺ} & \textbf{ᚾ} & \textbf{ᛁ} & \textbf{ᛃ} & \textbf{ᛈ} &
  \textbf{ᛇ} & \textbf{ᛉ} & \textbf{ᛊ}                                                                  \\
  h          & n          & i          & j          & p          & ï          & z          & s          \\
  \textbf{ᛏ} & \textbf{ᛒ} & \textbf{ᛖ} & \textbf{ᛗ} & \textbf{ᛚ} &
  \textbf{ᛜ} & \textbf{ᛞ} & \textbf{ᛟ}                                                                  \\
  t          & b          & e          & m          & l          & ŋ          & d          & o          \\
\end{longtable}

\includegraphics[width=2.12693in,height=2.99214in]{media/image1.gif}

\textbf{卢恩文字及其转写}

作为腓尼基字母的一种,卢恩文字也受到了希腊字母和意大利字母的影响

与之密切相关的一点是,卢恩字母在日耳曼文化中有着相当重要的地位,北欧神话中有关于卢恩字母神性的记载,因此在原始诺尔斯语消亡后,卢恩字母也伴随着拉丁字母(主要是基督教传入所致)一起使用了一段时间。

古诺尔斯语有三种方言,其中最重要的是其中的两种:古东诺尔斯语(Old East
Norse)和古西诺尔斯语(Old West
Norse)。前者演变出现今的丹麦语和瑞典语,后者演变出今天的挪威语、冰岛语、法罗语(丹麦属法罗群岛的官方语言,和冰岛语类似)。第三种方言称为古哥特兰语(Old
Gutnish),是哥特兰岛(今属瑞典)民的语言。总体而言,古西诺尔斯语更有北日耳曼语的特点,即它相对于原始日耳曼语发生了更多的变化,而东诺尔斯语和哥特兰语则发展得相对保守。特别地,``哥特兰''一词的词根和``哥特''是完全一样的,哥特兰语也明显受到了哥特语的影响。另外,就文献记载而言,古西诺尔斯语的语料最为丰富,因此古西诺尔斯语是古诺尔斯语的主要方言。在不特别提及古诺尔斯语的方言时,古诺尔斯语一词本身主要指涉的也是古西诺尔斯语。在西诺尔斯语进一步演化的过程中,产生了两支主要的现代语言:冰岛语和挪威语。挪威语由于在斯堪的纳维亚半岛大陆上,与其他语言接触甚广,因此其语法与东诺尔斯语进化出的现代语言有类似之处。与之相反的是,冰岛语在相对闭塞的岛国上几乎不受影响地发展了800年,时至今日,现代冰岛语的语法和书面形式与古诺尔斯语仅有少量的区别。即便是没有受过任何语言学训练的人也能很快总结出冰岛语规则的形态学变化规律,例如阳性词尾-r在冰岛语中演化为了稍有区别的-ur,现代冰岛语的主要创新是语音上的。由于冰岛语良好的从古性,有时也用古冰岛语代替古诺尔斯语,这种说法的好处是可以显式地表达出这种语言的地理分布。

\textbf{二、古诺尔斯语语法提要}

古诺尔斯语是一种屈折语,这意味它具有丰富的形态变化。可发生屈折变化的词类有名词、代词、指示词、形容词和动词。相比于几种古典的印欧语,例如拉丁文、希腊文和梵文,古诺尔斯语的屈折变化已经发生了大量简化。这与其说是古诺尔斯语的特点,倒不如说是日耳曼语的共性。日耳曼语在屈折上的最重要的创新是在动词系统上,具体来说,是用规则的构词法取代了原始印欧语相对复杂的变化。在音系上,北日耳曼语也有一些明显的创新,最主要的是变元音的使用。在古诺尔斯语中,变元音是一种比较规则的共时规律。

\begin{enumerate}
  \def\labelenumi{\Alph{enumi}.}
  \item
        \textbf{古诺尔斯语的音系}
\end{enumerate}

本书使用的发音是约12世纪后的比较成熟的古诺尔斯语发音,它有9个单元音,每个都有长短之分。在标准的正字法中,在短音字母上加锐音符``ˊ''表示长元音。

\begin{quote}
  这10个单元音用国际音标表示为:
\end{quote}

\begin{longtable}{lllllllll}
  \toprule
                   & \multicolumn{4}{c}{前元音} & \multicolumn{4}{c@{}}{后元音}                            \\
  \midrule
  \endhead
  \bottomrule
  \endfoot
                   &
  \multicolumn{2}{>{\raggedright\arraybackslash}p{(\columnwidth - 16\tabcolsep) * \real{0.2150} + 2\tabcolsep}}{%
  \textbf{非圆唇}} &
  \multicolumn{2}{>{\raggedright\arraybackslash}p{(\columnwidth - 16\tabcolsep) * \real{0.1775} + 2\tabcolsep}}{%
  \textbf{圆唇}}   &
  \multicolumn{2}{>{\raggedright\arraybackslash}p{(\columnwidth - 16\tabcolsep) * \real{0.2150} + 2\tabcolsep}}{%
  \textbf{非圆唇}} &
  \multicolumn{2}{>{\raggedright\arraybackslash}p{(\columnwidth - 16\tabcolsep) * \real{0.1775} + 2\tabcolsep}@{}}{%
  \textbf{圆唇}}                                                                                           \\
  \textbf{高元音}  & i                          & iː                            & y & yː &   &    & u & uː \\
  \textbf{中元音}  & e                          & eː                            & ø & øː &   &    & o & oː \\
  \textbf{低元音}  & ɛ                          & ɛː                            &   &    & a & aː & ɔ & ɔː \\
\end{longtable}

古诺尔斯语另有三个双元音:/ɛi/, /ɔu/, /øy/

在元音系统中,最重要的一个特征是元音变异(Umlaut),以此法得到的元音称为变元音。元音变异指的是,后一音节的元音会影响前一音节中元音的性质,使前一个元音的性质趋向于后一个元音。元音变异是一种同化(Assimilation)现象,在北日耳曼语和西日耳曼语中比较常见,但完全不出现在东日耳曼语中。因此,这一定是一个相对比较晚的变化。

古诺尔斯语有两种元音变异:

1)i-变异。后一音节中的i/j有把前一音节的元音抬高(Raising)、前置(Fronting)的趋向。例如把a抬高、前置到e,kr\textbf{a}f-
+ -ja \textgreater{} kr\textbf{e}fja; g\textbf{a}st- + -i \textgreater{}
g\textbf{e}sti.

2)u-变异。后一音节中的u/w有把前一音节的元音圆唇化的趋向(Rounding)。例如把a圆唇化为ǫ(读作ɔ),b\textbf{a}rn-
+ -um \textgreater{} b\textbf{ǫ}rnum.

除此之外,古诺尔斯语比较显著的共时规律是:

\begin{enumerate}
  \def\labelenumi{\arabic{enumi}.}
  \item
        词中的半元音j/w根据后续元音的性质经常消失。
  \item
        大量的辅音同化。
  \item
        非重读元音经常省略。
\end{enumerate}

古诺尔斯语比较重要的历时规律是词尾音节的脱落,这时常导致造成变元音的i/u脱落。

\begin{enumerate}
  \def\labelenumi{\Alph{enumi}.}
  \setcounter{enumi}{1}
  \item
        \textbf{动词}
\end{enumerate}

古诺尔斯语的动词根据人称、数、时态、语态和语气发生变化,这种变化称为动词的变位(Conjugation)。古诺尔斯语有三个人称,分别是第一人称、第二人称和第三人称,和英语等一致;两个数,分别是单数和复数。PIE中原本还有双数,但在西北日耳曼语中完全消失了,仅在哥特语中还有保留;两个时态:现在时和过去时。其中PIE的现在时、将来时都合并到了古诺尔斯语的现在时,非现在时都合并到了过去式;两种语态:主动态和中动态;三种语气:直陈、虚拟和祈使。

古诺尔斯语基本的动词变位方法时根据人称、数、时态、语态和语气在动词词干后添加词尾,但一部分动词在过去时的变位中还需要改变词根中的元音。根据动词过去时的构成是否涉及元音的变换,可将动词分为两种:强动词和弱动词。

强动词的词根元音在过去时和过去分词中发生变换,类似于英语中sing-sang-sung这样的不规则动词。这种利用元音变化进行动词屈折的现象叫作元音交替(Ablaut)。元音交替是古老的语法现象,在PIE中非常常见,而且不同的交替模式(例如sing-sang-sung和run-ran-run是两种交替模式)比较复杂,在日耳曼语中,元音交替的模式基本固定为7大类。词根中的元音按这7种模式进行变换,构成了7类不同的变位法。

强动词继承了PIE中的元音交替,因而这些动词一般比较古老。新产生的动词不再适用元音交替这种规则,而是规则地加入一个塞音中缀构成过去时,这种动词称为弱动词。弱动词是日耳曼语的创新,有很强的构词力,因此弱动词是一个开放的类,新造的动词都按弱动词的方式变化;强动词是一个封闭的类,其数量基本已经固定,新动词不会再按照强动词变化,反而有一些强动词按类比的原则变成了弱动词。

古诺尔斯语中的弱动词塞音后缀是-ð-,英语中的对应是-ed(规则动词)。弱动词采用一套不同于强动词的词尾,根据其词尾的不同,还可以将弱动词分成3类,但其中区别相比强动词而言小得多。

古诺尔斯语还有一些不规则动词。一部分称为过去-现在混合动词,它们的现在时采用强动词的过去词尾,过去时按弱动词变位,这是PIE演化为PGmc.时造成的不对称现象。这类动词只有10个;还有一些如系动词这样的高度不规则动词。

\begin{enumerate}
  \def\labelenumi{\Alph{enumi}.}
  \setcounter{enumi}{2}
  \item
        \textbf{名词、形容词和代词}
\end{enumerate}

名词按照数和格发生变化。和动词类似,名词也只有单数和复数,双数已经完全消失;古诺尔斯语中有四个格:主格、属格、与格、宾格。格大略地可认为是名词在句子中充当的成分,相比与PIE的8个格,古诺尔斯语已经大量简化了名词格,因此一些格有比较复杂的用法。

名词还有两个不变的属性:一是名词的性,包括阴、阳、中三性;二是名词的变化方法,后者称为变格法。名词所属哪一种变格法和名词的性并不是一对一的关系,某些变格法涉及多个性的名词,同理,同属于一种性的名词又对应于不同的变格法。

变格法也分为两类,分别是强变格法和弱变格法,这个术语由动词系统类比而来。因此名词可分为强变化和弱变化名词。强变化名词曾经的词尾音节中有一个元音,例如a,o,i,按照元音的不同,又可以对强变化名词进一步分类;而弱变化名词曾经的词尾音节中不仅有一个元音,这个元音还总是接续一个鼻音n,如an,in.
类似地,也可以按元音的类别对弱变化名词进行划分。一共有5类强变化名词和3类弱变化名词。

古诺尔斯语中的名词可以添加一个-inn后缀使之称为特指名词,-inn相当于英文的the,但-inn要根据格、性、数发生变化,和下述的形容词有类似之处。一个名词是否是特指的在古诺尔斯语中有形态学上的作用,这会影响修饰它的形容词。

形容词和名词一样也要按数和格发生变化,但它还可以按性发生变化,即一个形容词可以变成阴、阳、中性。此外,形容词还可以根据它所修饰的名词是否为特指名词发生变化。此时可以把形容词的变格法分为强变格和弱变格两大类。如果修饰的名词是特指名词,要用形容词的弱变格形式,反之要用强变格形式。因此,一个形容词没有固定的属性,既没有固定的格、性、数,也没有固定的变格法。任何形容词都有两套变格法,要采取变格法中合适的形态使之与名词的格、性、数、特指性保持一致。

代词和形容词在语义和句法上的区别比较小,物主代词、指示代词、疑问代词等几乎都可以像形容词一样使用。因而代词与形容词有许多相似之处,很多形容词性的代词都按形容词的强变格法变化。但是,代词不区分特指性,因而没有强弱之分。

\textbf{三、本书的结构}

待完成

语法

\chapter{拼写与语音}\label{拼写与语音}

\begin{quote}
  \textbf{章节要点:}
\end{quote}

\begin{itemize}
  \item
        \begin{quote}
          古诺尔斯语的两种书写系统
        \end{quote}
  \item
        \begin{quote}
          古诺尔斯语字母的音值及单词拼读
        \end{quote}
  \item
        \begin{quote}
          音节划分规则
        \end{quote}
  \item
        \begin{quote}
          古诺尔斯语的音变
        \end{quote}
\end{itemize}

\section{书写系统和读音}\label{书写系统和读音}

古诺尔斯语主要使用两种字母书写。其一是较早期的卢恩字母(Rune),后来则采用拉丁字母。最早发现的卢恩文字可追溯到公元2世纪。此时的古诺尔斯语尚处在非常原始的时期,故称为原始诺尔斯语。卢恩一词在日耳曼语中的意思是``秘密'',据神话记载,奥丁曾将自身作为祭品倒挂在世界之树上,在历经九夜的折磨后终于拾起了卢恩文字。这个神话的象征是奥丁通过苦行获得了智慧和奥义,因而卢恩的含义远不止一种书写系统那么简单。维京人认为卢恩可以用于占卜,到了中世纪晚期,北欧的文化已经受到了严重的基督教影响,其文字大量被拉丁化,卢恩字母丧失了日常沟通的功能,反而更加往神秘学的方向发展。

卢恩文字最初有24个,这套字母表称之为Elder
Futhark,futhark是前六个字母的读音,和alphabet的含义(希腊字母表的前两个字母)类似。后来卢恩字母也发展出了16个字母的版本,称为Younger
Futhark.

Elder Futhark

\begin{longtable}{llllllllllllllllllllllll}
  \toprule
  ᚠ & ᚢ & ᚦ & ᚨ & ᚱ & ᚲ & ᚷ & ᚹ & ᚺ & ᚾ & ᛁ & ᛃ & ᛈ & ᛇ & ᛉ & ᛊ & ᛏ & ᛒ & ᛖ & ᛗ & ᛚ & ᛜ & ᛞ & ᛟ \\
  \midrule
  \endhead
  \bottomrule
  \endfoot
  f & u & þ & a & r & k & g & w & h & n & i & j & p & ï & z & s & t & b &
  e & m & l & ŋ & d & o                                                                         \\
\end{longtable}

Younger Futhark

\begin{longtable}{llllllllllllllll}
  \toprule
  ᚠ       & ᚢ      & ᚦ       & ᚬ & ᚱ       & ᚴ & ᚼ/ᚽ & ᚾ/ᚿ & ᛁ       & ᛅ/ᛆ & ᛦ & ᛋ/ᛌ & ᛏ/ᛐ & ᛒ & ᛘ & ᛚ \\
  \midrule
  \endhead
  \bottomrule
  \endfoot
  f/v     & u/v/w,

  y, o, ø & þ, ð   & ą, o, æ & r & k, g, ŋ & h & n   & e   & a, æ, e & R   & s &
  t, d    & p      & m       & l                                                                       \\
\end{longtable}

卢恩文字或许有非常隐秘的作用,但这不在本书的讨论范围内。对于大部分中世纪的手稿而言,古诺尔斯语已经用上了我们熟悉的拉丁字母。

除了有最常见的26个字母外,古诺尔斯语的字母表中还包括几个特殊的辅音字母、元音字母和长音字母。这里我们只谈标准正字法下的字母,关于原始手稿中更复杂的情况,将在读本中进一步探索。

\begin{longtable}{llll}
  \toprule
  小写字母 & 大写字母 & 发音(国际音标) & 环境                                                     \\
  \midrule
  \endhead
  \bottomrule
  \endfoot
  á        & Á        & ɔː               &                                                          \\
  a        & A        & ɑ                &                                                          \\
  b        & B        & b                &                                                          \\
  c        & C        & k                &                                                          \\
  d        & D        & d                &                                                          \\
  ð        & Ð        & ð                &                                                          \\
  é        & É        & eː               &                                                          \\
  e        & E        & e                &                                                          \\
  f        & F        & (1) f            & 词首                                                     \\
           &          & (2) v            & 除词首外的其他位置                                       \\
  g        & G        & (1) g            & 词首,双写时,或在\textless gn\textgreater 中            \\
           &          & (2) x            & 在\textless gs\textgreater 或\textless gt\textgreater 中 \\
           &          & (3) ɣ            & 在\textless gh\textgreater 中                            \\
  h        & H        & h                &                                                          \\
  í        & Í        & iː               &                                                          \\
  i        & I        & i                &                                                          \\
  j        & J        & j                &                                                          \\
  k        & K        & (1)k             & 除了下面的情况                                           \\
           &          & (2)x             & 在\textless ks\textgreater 或\textless kt\textgreater 中 \\
  l        & L        & l                &                                                          \\
  m        & M        & m                &                                                          \\
  n        & N        & n                &                                                          \\
  ó        & Ó        & oː               &                                                          \\
  o        & O        & o                &                                                          \\
  p        & P        & (1) p            & 除了下面的情况                                           \\
           &          & (2) f            & \textless ps\textgreater 或\textless pt\textgreater 中   \\
  q        & Q        & k                & 总和u一起出现,qu和kv是一样的                            \\
  r        & R        & r                &                                                          \\
  s        & S        & s                &                                                          \\
  t        & T        & t                &                                                          \\
  ú        & Ú        & uː               &                                                          \\
  u        & U        & u                &                                                          \\
  v        & V        & v                &                                                          \\
  w        & W        & w                &                                                          \\
  x        & X        & xs               &                                                          \\
  ý        & Ý        & yː               &                                                          \\
  y        & Y        & y                &                                                          \\
  z        & Z        & ts               & 极少出现,主要是-t/-d/-ð和-s的合写                       \\
  þ        & Þ        & θ                &                                                          \\
  æ        & Æ        & ɛː               &                                                          \\
  ǫ́        & Ǫ́        & ɔː               &                                                          \\
  ǫ        & Ǫ        & ɔ                &                                                          \\
  ø        & Ø        & ø                &                                                          \\
  œ        & Œ        & øː               &                                                          \\
\end{longtable}

总体来说,古诺尔斯语有9个对立的基本元音音素,每个元音都有一个对应地长音。要构成长音,只需要在短音字母上添加锐音符``ˊ''。但有3个例外:

\begin{longtable}{l}
  \toprule
  \begin{enumerate}\def\labelenumi{\arabic{enumi}.}\item  a的长音á,在12世纪的古诺尔斯语已经与ǫ́合流,所以当时的音系中并没有一个长的a  /ɑ:/。\end{enumerate} \\
  \midrule
  \endhead
  \bottomrule
  \endfoot
  \begin{minipage}[t]{\linewidth}\raggedright
    \begin{enumerate}
      \def\labelenumi{\arabic{enumi}.}
      \setcounter{enumi}{1}
      \item
            æ总是长元音,没有短元音与之对应\footnote{更早期的古诺尔斯语实际上有短的/ɛ/,这是a发生i-变异(参见1.3)的结果。}。
    \end{enumerate}
  \end{minipage}                                                                                                                        \\
  \begin{minipage}[t]{\linewidth}\raggedright
    \begin{enumerate}
      \def\labelenumi{\arabic{enumi}.}
      \setcounter{enumi}{2}
      \item
            ø的长元音是œ,一般不写†\includegraphics{media/image2.wmf}这种字母(但手稿中也有记载)。
    \end{enumerate}
  \end{minipage}                                                                                                                        \\
\end{longtable}

这些音素,以及对应的字母在下表中标出,后面用尖括号\textless\textgreater 标出的是这个元音的写法。关于前元音、后元音等术语,不熟悉的读者可以参照\ref{变元音}节中关于元音性质的描述。

\begin{longtable}{lllllllll}
  \toprule
                            & \multicolumn{4}{c}{前元音}   & \multicolumn{4}{c}{后元音}               \\
  \midrule
  \endhead
  \bottomrule
  \endfoot
                            &
  \multicolumn{2}{>{\raggedright\arraybackslash}p{(\columnwidth - 16\tabcolsep) * \real{0.2346} + 2\tabcolsep}}{%
  非圆唇}                   &
  \multicolumn{2}{>{\raggedright\arraybackslash}p{(\columnwidth - 16\tabcolsep) * \real{0.2430} + 2\tabcolsep}}{%
  圆唇}                     &
  \multicolumn{2}{>{\raggedright\arraybackslash}p{(\columnwidth - 16\tabcolsep) * \real{0.1532} + 2\tabcolsep}}{%
  非圆唇}                   &
  \multicolumn{2}{>{\raggedright\arraybackslash}p{(\columnwidth - 16\tabcolsep) * \real{0.2355} + 2\tabcolsep}@{}}{%
  圆唇}                                                                                               \\
  高元音                    & i \textless i\textgreater{}  & iː \textless í\textgreater{} & y
  \textless y\textgreater{} & yː \textless ý\textgreater{} &                              &   & u
  \textless u\textgreater{} & uː \textless ú\textgreater{}                                            \\
  中元音                    & e \textless e\textgreater{}  & eː \textless é\textgreater{} & ø
  \textless ø\textgreater{} & øː \textless œ\textgreater{} &                              &   & o
  \textless o\textgreater{} & oː \textless ó\textgreater{}                                            \\
  低元音                    & ɛ                            & ɛː \textless æ\textgreater{} &   &   & a
  \textless a\textgreater{} & \textbackslash{}             & ɔ
  \textless ǫ\textgreater{} & ɔː \textless ǫ́\textgreater{}                                            \\
\end{longtable}

古诺尔斯语还有3个双元音/ɛi/, /ɔu/, /øy / 拼作 \textless ei\textgreater,
\textless au\textgreater, \textless ey\textgreater.

低元音/ɛ/只出现在上述的双元音中。

古诺尔斯语的辅音比较规则,少数的例外一般表现为:

\begin{longtable}{l}
  \toprule
  s/t之前的塞音会变成对应发音部位的清擦音。\footnote{这个规律是原始语发生的日耳曼语擦音定律(Germanic  spirant  law)的残留。擦音定律和格林定律、维尔纳定律密切相关,涉及较为复杂的历史音变,请有兴趣的读者自查。\textsubscript{­}} \\
  \midrule
  \endhead
  \bottomrule
  \endfoot
\end{longtable}

古诺尔斯语的辅音也成对出现,双辅音与单辅音的区别仅在于前者的音值更长一些。j和v是半元音,它们的性质分别与i和u相似,在古诺尔斯语中经常发生音变(见\ref{半元音的保持性})。

\section{音节和重音}\label{音节和重音}

音节是构成语音序列的单位,也是语音中最自然的语音结构单位。英文中的water就分为wa-和-ter两个音节。以英语为母语的人在拼读这个词的时候可能在t前停顿,但不可能在t后停顿,把water读成wat'er这样的形式。不同语言常有不同的音节划分规则,音节的类型也影响语言的韵律甚至是形态。

包括古诺尔斯语在内的大多数早期日耳曼语有复杂的音节划分模式,目前尚无一个统一的理论能够解释这些语言中的所有现象。语言学家主要从两个方面推测古代语言的音节划分,一是根据诗歌的规则;二是通过观察手稿中单词的记法。特别地,当一个单词出现在一行的末尾而恰好写不下时,这个词如何被拆分能够很好地反映它的音节情况。

就语法而言,古诺尔斯语的音节类型会影响部分动词(见\ref{第一弱变位法}
ja-词干名词)的变形,因此掌握音节的划分十分重要。有两种音节划分的方法可供参考,一种称为传统式,另一种称为格律式。如果读者之前没有接触过音韵学的知识,用传统式的划分已经可以很好地解决古诺尔斯语的问题(但通常这种方式和其他语言的音节划分不一样)。格律式划分和希腊语、拉丁语等划分一致,也更偏向于现代语音学的划分方法,因此介绍起来相对复杂些。

\textbf{传统式}

古诺尔斯语中有许多单音节词,例如á, til, at, rann.
单音节词只有一个元音,可以是长元音也可以是短元音。在多音节词,如果这个词不是合成词,那么根据元音的位置划分音节(即非词首音节总以元音开头),例如far-a,
kall-a, gǫrð-um, gam-all-a,
hundr-að-a。在合成词中,根据组词的语素划分音节,例如vápn-lauss
(\textless{} vápn + lauss, weapon-less), vík-ing-a-hǫfð-ing-i
(\textless{} víkinga + hǫfðingi, Viking's chieftain).
由于元音和辅音有长短之分,音节也被分成以下四类:

\begin{longtable}{llll}
  \toprule
  \multicolumn{2}{c}{种类} & 描述 & 举例                                 \\
  \midrule
  \endhead
  \bottomrule
  \endfoot
  1                        & 短   & 短元音+短辅音        & bað           \\
  2                        & 长   & 短元音+辅音簇        & rann, ǫnd     \\
  3                        & 长   & 长元音+短辅音/零辅音 & hús, fé, gnúa \\
  4                        & 加长 & 长元音+辅音簇        & nótt, blástr  \\
\end{longtable}

辅音簇(Consonant cluster)的意思是多个辅音的集合。

\textbf{格律式}

格律式的划分方法与大多数音系学的理论一致,一个音节一般包括以下3个结构:

\begin{enumerate}
  \def\labelenumi{\arabic{enumi}.}
  \item
        音节首(Onset)
\end{enumerate}

\begin{quote}
  音节首总是由辅音充当。音节首可以是单辅音,也可以是多个辅音(辅音簇)。古诺尔斯语有些单音节词以元音开头,这时没有音节首。
\end{quote}

\begin{enumerate}
  \def\labelenumi{\arabic{enumi}.}
  \setcounter{enumi}{1}
  \item
        音节核(Nucleus)
\end{enumerate}

\begin{quote}
  音节核是一个响音,即可以是元音或者成音节的辅音。这是大多数语言的必有成分。
\end{quote}

\begin{enumerate}
  \def\labelenumi{\arabic{enumi}.}
  \setcounter{enumi}{2}
  \item
        音节尾(Coda)
\end{enumerate}

\begin{quote}
  由辅音充当,没有音节尾的音节是开音节,反之是闭音节。
\end{quote}

在这种划分下,上述的fara, kalla, gǫrðum, gamalla,
hundraða就要重新划分为fa-ra, kal-la, gǫr-ðum, ga-mal-la, hun-dra-ða.
细心的读者可能会想到面对词中的辅音簇时该如何划分?譬如是分成kal-la还是ka-lla?这个问题经常被讨论,但没有十分确切的定论。一般来说,有以下几个规则可供参考:

\begin{longtable}{l}
  \toprule
  \begin{enumerate}\def\labelenumi{\arabic{enumi}.}\item  多音节词除了第一个音节外必有音节首。\end{enumerate} \\
  \midrule
  \endhead
  \bottomrule
  \endfoot
  \begin{minipage}[t]{\linewidth}\raggedright
    \begin{enumerate}
      \def\labelenumi{\arabic{enumi}.}
      \setcounter{enumi}{1}
      \item
            多个辅音一般划分到两个音节。因而kal-la好于ka-lla.
    \end{enumerate}
  \end{minipage}                                                                          \\
  \begin{minipage}[t]{\linewidth}\raggedright
    \begin{enumerate}
      \def\labelenumi{\arabic{enumi}.}
      \setcounter{enumi}{2}
      \item
            规则1的例外是p,t,k,s和j,v,r一般划分到一个音节。因而si-tja好于sit-ja.
    \end{enumerate}
  \end{minipage}                                                                          \\
  \begin{minipage}[t]{\linewidth}\raggedright
    \begin{enumerate}
      \def\labelenumi{\arabic{enumi}.}
      \setcounter{enumi}{3}
      \item
            合成词在词素划分出区分一个音节。
    \end{enumerate}
  \end{minipage}                                                                          \\
\end{longtable}

按此法划分的音节按照音拍(Mora)的数量对其进行分类。音拍大致指的是语音中一个等时的单位。在古诺尔斯语中,音拍从音节核开始计算,音节首从不计入音拍。音节核和音节尾的音拍按下面的方式计算:

\begin{longtable}{l}
  \toprule
  \begin{enumerate}\def\labelenumi{\arabic{enumi}.}\item  音节核中的单元音算一个音拍;双元音以及长元音算两个音拍。\end{enumerate} \\
  \midrule
  \endhead
  \bottomrule
  \endfoot
  \begin{minipage}[t]{\linewidth}\raggedright
    \begin{enumerate}
      \def\labelenumi{\arabic{enumi}.}
      \setcounter{enumi}{1}
      \item
            音节尾如果是单辅音,不算音拍;如果有辅音簇,算一个音拍。
    \end{enumerate}
  \end{minipage}                                                                                               \\
\end{longtable}

在这种划分下,音节可分为轻音节和重音节两种。轻音节只包括一个音拍,重音节则包括多个音拍:

\begin{longtable}{lll}
  \toprule
  种类                        & 音节核和音节尾可能的类型        & 举例 \\
  \midrule
  \endhead
  \bottomrule
  \endfoot
  轻音节                      & 1. 短元音

  2. 短元音+单辅音            & \textbf{fa}-ra, \textbf{si}-tja

  \textbf{kal}-la, \textbf{bað}                                        \\
  重音节                      & 1. 短元音+辅音簇

  2. 长/双元音+任意数量的辅音 & \textbf{rann, ungr}

  \textbf{hús}, \textbf{fé}, \textbf{blástr}                           \\
\end{longtable}

一个例外情况是长元音+单辅音的组合在诗歌中被当作轻音节,例如búa,其中的成因尚不明确。

古诺尔斯语的重音落在第一个音节上。在两个语素的合成词中,一级重音落在第一个词的第一个音节上,第二个重音落在第二个词的第一个音节上(即总是落在词根上)。在三语素的合成词中,一级重音落在第一个词上,二级重音落在最后一个词上,三级重音落在中间词上,每个词的重音都遵循上述的规则。

重读音节中的元音是任意的,但在弱读音节中,只能是a, i,
u这三个。弱读音节经常发生元音的省略,详见\ref{语音规则}。

\section{变元音}\label{变元音}

在古诺尔斯语的形态学中,变元音或者说元音变异(Umlaut)的现象是普遍存在的。变元音指的是两个元音同化的过程。具体来说,一个给定音节中元音的发音会因为说话者对下一个元音的预期而改变。由于这种预期,后面一个元音会影响到前面一个元音的发音,从而在本质上产生兼具两者特征的新元音。在古诺尔斯语中,i和u是最常见地引起这种音变的元音,因此把对应的过程分别称为i-变异和u-变异。

变元音的痕迹遍布整个古诺尔斯语的屈折系统,但变元音现象主要发生在原始诺尔斯语时期。读者必须认识到这个变化在用古诺尔斯语写作的时期之前已经发生完毕并且不再持续下去了。这就是说,并不是每个i或者u都会导致相应的变化,相反,i-变异和u-变异仅仅作为一种固定下来了的形态规则,保留在\textbf{部分的}名词的变格和动词的变位中。一些符合变元音条件的地方可能并没有按预期的那样发生变元音,有些没有变元音发生条件的地方却反而没有变元音了。这两种现象都和古诺尔斯语的历史音变有关,而后者则尤其普遍,这主要是因为古诺尔斯语的词尾音节脱落得十分严重,许多本来能造成元音变异的音节在古诺尔斯语发展的过程中丢失了,只有其造成的变元音还保留下来。

变元音的本质是一种元音的同化现象。在介绍两种音变前,首先需要了解元音的一些基本理论。试着发i\textasciitilde a的音,会发现口腔逐渐打开,舌尖逐渐向口腔后部退缩,舌苔逐渐远离上颚。再试着发a\textasciitilde u的音,会发现口腔逐渐闭合,舌尖前后位置几乎不变,嘴唇逐渐收敛成o型。由此我们可以发现元音的性质至少和以下信息有关:舌尖位置,舌头最高处里上颚的距离,嘴型。根据发音时舌头在口腔中的相对位置,语言学家绘制了元音表:

\includegraphics[width=3.44236in,height=2.93264in]{media/image3.png}

图表的纵轴称为元音高度,反映了舌头和口腔上部或两颚的距离;舌头位置较低的元音被放在元音图底部,而位置较高者则在元音图顶部。例如,{[}a{]}(相当于汉语拼音的``a'')被置于元音图下方,{[}i{]}(相当于汉语拼音的``i'')则被置于元音图上方。类似地,图表中的横轴称为元音舌位,反映了舌头的前后位置;头位置较靠前的元音被放在元音图左侧,而位置较靠后者则在元音图右侧。例如,{[}y{]}(相当于汉语拼音的``yu'')被置于元音图左方,{[}u{]}(相当于汉语拼音的``u'')则被置于元音图右方。当相同高度、舌位的元音成对出现时,右侧的是圆唇元音,左侧的则是非圆唇元音。发圆唇元音时,嘴唇形成一个圆形的开口,使嘴巴内侧的表面露出,如{[}y{]};而不圆唇元音发音时,嘴巴四周向后绷紧,嘴唇亦向后收缩,仅露出嘴唇的外部表面,如{[}i{]}。

一个元音的性质于是可以由三个维度表征:高度、舌位和圆唇度。元音高度区分高元音、低元音(或称闭元音、开元音);元音舌位区分前元音、后元音;圆唇度使得同一发音位置的元音总有圆唇元音和非圆唇元音的对立。

两种元音变异现象分别是:

1)i-变异。又叫前元音变异(Front
mutation),指的是后一个音节中的i或j导致前一个音节中的元音舌位前移,高度抬升,但嘴型(圆唇/非圆唇)不改变的现象。由于i/j对应的是元音表中的前元音、高元音,i-变异就有把前一音节的元音向表中的左上角移动的倾向。

i-变异造成的效果是:

\begin{longtable}{ll}
  \toprule
  变化前 & 变化后 \\
  \midrule
  \endhead
  \bottomrule
  \endfoot
  a      & e      \\
  á      & æ      \\
  o      & ø      \\
  ó      & œ      \\
  u      & y      \\
  ú      & ý      \\
  au     & ey     \\
\end{longtable}

i-变异前后音长并不发生变化,因此变化后的元音也是成对的长短元音,但注意在12世纪的古诺尔斯语中,a和á的发音部位不同,e和æ的发音部位也略有区别。也有一些手稿把i-变异后的a写作ę,以区分其他的e,这个音最早应该读作/ɛ/,它是æ的短音。

以下情况中不发生i-变异:以i作为格标记的阳性中性名词(见\ref{a/ja/wa-词干})。i-变异失效的原因有很多:一种最简单的情况是原本的元音是*e,但在i-变异停止后才抬升为i;还有一种可能是,没有发生i-变异的词形被类推到了整个变形表中;但仍有一些问题无法解决。

2)u-变异。又叫唇化变异(Labial
mutation),指的是后一个音节的u或v导致前一个音节中的元音被圆唇化,同时保持舌位不变。u-变异的效果是:

\begin{longtable}{ll}
  \toprule
  变化前    & 变化后               \\
  \midrule
  \endhead
  \bottomrule
  \endfoot
  a         & ǫ                    \\
  á         & (ǫ́) \textgreater{} á \\
  非重读的a & u                    \\
\end{longtable}

u-变异在古诺尔斯语的形态学中几乎只对a有效。但在其历史演变过程中,也造成了这样的音变:

\begin{longtable}{ll}
  \toprule
  变化前 & 变化后 \\
  \midrule
  \endhead
  \bottomrule
  \endfoot
  e      & ø      \\
  é      & œ      \\
  i      & y      \\
  í      & ý      \\
\end{longtable}

后一张表中的音变已经完全固定,即无论怎么添加含u的词尾,都不能再导致这样的变形了。只有在少量的词汇中可以看到这种这些音变保留的痕迹,例如英文``sing''的对应词†singva变成了古诺尔斯语的s\textbf{y}ng\textbf{v}a。

读者学习u-变异时,只需记住前一张表格中的内容,请注意,a在重读和非重读情况下造成的音变不同。长音的á由于性质特殊(回忆\ref{书写系统和读音}中元音的3个例外),在很早的时期就与ǫ́合流,以至于变元音的结果没有保留下来。

\section{语音规则}\label{语音规则}

古诺尔斯语有一些规则的音变,下面将从元音和辅音两个方面进行说明。其中,辅音的规则音变相对更为普遍。

\subsection{元音的音变}\label{元音的音变}

古诺尔斯语的共时系统中有下述的常见变化:

\begin{enumerate}
  \def\labelenumi{\Alph{enumi}.}
  \item
        \phantomsection\label{_Ref117017033}{}\textbf{元音变异}
\end{enumerate}

元音变异的原因和结果已经在\ref{变元音}节中详尽地阐释了,现在提供一些案例以供参考。

u-变异是一种比i-变异相对更``规则''的变化,许多造成u-音变的音节在古诺尔斯语中还保留了下来,但造成i-音变的音节大量脱落:

\begin{quote}
  arm + um \textgreater{} ormum `arms'

  sag + ur \textgreater{} sogur `stories'

  kall + að + u \textgreater{} kolluðu `called'
\end{quote}

请注意,ka\textsuperscript{1}lla\textsuperscript{2}ðu \textgreater{}
kolluðu是一个连续的音变(在两个a上加了上标加以区别):词尾的-u首先造成非重读的a\textsuperscript{2}变成了u,这个u又为词根中的重读元音a\textsuperscript{1}提供了音变条件,这使得一个词中发生了两个u-音变,且结果不同。

i-变异有时发生的比较隐秘,试比较一个词干加上词尾后的结果,其中由*标出的词尾是古诺尔斯语的原始形式,但后来其中的元音脱落了:

\begin{quote}
  sat + ja \textgreater{} setja `set'

  vall + ir \textgreater{} vellir `fields'

  mús + *ir \textgreater{} mýss `mice'

  lát + *ir \textgreater{} lætr `lets'
\end{quote}

\begin{enumerate}
  \def\labelenumi{\Alph{enumi}.}
  \setcounter{enumi}{1}
  \item
        \phantomsection\label{_Ref115693879}{}\textbf{元音省略和缩合}
\end{enumerate}

元音省略(Syncope)指的是非重读的短元音脱落的情况,最常造成语音省略的情况是:

\begin{longtable}{l}
  \toprule
  \begin{enumerate}\def\labelenumi{\arabic{enumi}.}\item  \phantomsection\label{_Ref115694569}{}弱读元音在辅音+元音前脱落。\end{enumerate} \\
  \midrule
  \endhead
  \bottomrule
  \endfoot
  \begin{minipage}[t]{\linewidth}\raggedright
    \begin{enumerate}
      \def\labelenumi{\arabic{enumi}.}
      \setcounter{enumi}{1}
      \item
            \phantomsection\label{_Ref115709879}{}以-i结尾的词干在与以元音开头的词尾接触时,i脱落。
    \end{enumerate}
  \end{minipage}                                                                                                        \\
\end{longtable}

常见的例子有:

\begin{quote}
  aptan + ar \textgreater{} aptnar `evenings'

  gamal + an \textgreater{} gamlan `old'

  komin + a \textgreater{} komna `come'

  lifi + um \textgreater{} lifum `live'
\end{quote}

元音缩合(Contraction)指的是一个非重读元音紧跟一个重读元音时发生的音变,如果两者都是前元音或者后元音(除了úa,
óa,以及极少数情况下的 úu),,就会合并为单个长音:

\begin{quote}
  tré + i \textgreater{} tré `tree'

  á + ar \textgreater{} ár `river'

  á + um \textgreater{} ám `rivers'

  trú + um \textgreater{} trúm `faithful'
\end{quote}

在前元音之前的后元音不发生音变,例如 búinn.
在后元音之前的前元音经常形成双元音:

\begin{quote}
  fé + ar \textgreater{} fjár `cattle'

  kné + um \textgreater{} knjám/knjóm `knee'
\end{quote}

\begin{enumerate}
  \def\labelenumi{\Alph{enumi}.}
  \setcounter{enumi}{2}
  \item
        \textbf{元音延长或缩短}
\end{enumerate}

词尾的重读元音(开音节中)被延长,例如þú.
当词尾的辅音脱落而导致原先处于中间位置的元音变成词尾元音时,规则同样成立,例如
†vag \textgreater{} †va \textgreater{} vá.

辅音簇前的长元音常常被缩短,这种现象在双辅音前尤为明显。双元音ei缩短的结果是e,试比较下面一些词的原型和变格的形式:

\begin{quote}
  góðr --- gott `god'

  mín --- minn `my'

  heilagr --- helgan `holy'
\end{quote}

\begin{longtable}{l}
  \toprule
  \begin{itemize}\item  \textbf{普罗科什定律(Prokosch's Law)}\end{itemize}为什么古诺尔斯语频繁地发生元音的延长和缩略?美国语言学家普罗科什(EduardProkosch)发现,古日耳曼语的音节有这样的规律(用格律划分法):\textbf{重音节有趋向2音拍的趋势;非重音节有趋向1音拍的趋势。}因此gott中的元音必须是短的,否则†gótt就有三个音拍了\footnote{但是,这个规则也有例外,例如óðr  ---ótt.  类比这个词,góðr早期确实也有gótt的变形。普罗科什定律更类似于一种趋势而非一个必须遵守的规则。}。读者在记忆这一规则时,可以用普罗科什定律加以解释。 \\
  \midrule
  \endhead
  \bottomrule
  \endfoot
\end{longtable}

古诺尔斯语的元音系统还有一些值得关注的历史音变,这些音变基本发生在原始诺尔斯语时期,古诺尔斯语成文时已经停止,但这个音变的影响在许多词类中都有体现。

\begin{enumerate}
  \def\labelenumi{\Alph{enumi}.}
  \setcounter{enumi}{3}
  \item
        元音分割
\end{enumerate}

元音分割(Breaking)指的是特定环境下单元音变为双元音的过程。在古诺尔斯语中,唯一会发生这种音变的是来自原始诺尔斯语的*e. 元音分割的规则是:

\begin{longtable}{l}
  \toprule
  1. 如果下一个音节有a,则*e \textgreater{} ja;若下个音节有u,则*e\textgreater{} jǫ. \\
  \midrule
  \endhead
  \bottomrule
  \endfoot
  2. 当*e紧跟在l, v, r后面时,元音分割不发生。                                        \\
  3. 当jǫ后面有双辅音时,jǫ \textgreater{} jo.                                        \\
\end{longtable}

元音分割的结果是:

\begin{quote}
  *herta \textgreater{} hjarta `heart'

  *skeldu \textgreater{} skjǫldu `shields'

  *brestaną \textgreater{} bresta `burst'

  *þekkuz \textgreater{} þjokkr `thick'
\end{quote}

这些变化在名词和动词的词根中都非常常见,但大多数词中元音分割贯穿了整个变形表,所以从共时层面来看,读者无法发现这一历史音变的痕迹。在有些词类中,特别是u-词干名词(见\ref{u-词干}),它们的词尾对应了不同的元音分割条件,使得某些形式中不发生元音分割,有的分割为ja,有的又分割jǫ.

\begin{enumerate}
  \def\labelenumi{\Alph{enumi}.}
  \setcounter{enumi}{4}
  \item
        抬升
\end{enumerate}

古诺尔斯语中常有e和i的交替,这一现象成因复杂。一方面,e \textgreater{}
i的抬升可能是i-变异的影响;另一方面,在原始日耳曼语中也有抬升的痕迹。抬升的规则非常复杂,本书仅在发生这种现象的时候进行说明。

\subsection{辅音的音变}\label{辅音的音变}

在某些情况下,辅音会发生规则的音变。最显著的几组如下所示,其中辅音同化的情况在古诺尔斯语中最为常见:

\begin{enumerate}
  \def\labelenumi{\Alph{enumi}.}
  \item
        \phantomsection\label{_Ref117517666}{}\textbf{辅音同化}
\end{enumerate}

辅音同化指的是两个不同性质的辅音变成同一性质的音(经常造成双辅音),有五种重要的辅音同化现象:

\begin{enumerate}
  \def\labelenumi{(\alph{enumi})}
  \item
        \phantomsection\label{_Ref117517668}{}-r 与前面的长音节中的 l, s或 n
        同化,由于-r是最常见的词尾之一,无论是动词、形容词还是名词中都能看到这一音变的痕迹:
\end{enumerate}

\begin{quote}
  sæl- + -r \textgreater{} sæll `happy'

  væn- + -r \textgreater{} vænn `hopeful'

  fús- + -r \textgreater{} fuss `willing'
\end{quote}

\begin{enumerate}
  \def\labelenumi{(\alph{enumi})}
  \setcounter{enumi}{1}
  \item
        -ð-与前面的ð-同化为-dd-,与-t-或-s-同化为-tt-,
        -st-,-ð-是动词过去式的标记,这一现象亦非常常见,下面的例子都是动词和它的过去式形式:
\end{enumerate}

\begin{quote}
  eyða \textgreater{} eyddi `wasted'

  setja \textgreater{} setti `seated'

  kyssa \textgreater{} kyssti `kissed'

  kneikja \textgreater{} kneikti `bent';但merkja \textgreater{} merkði
  `marked'不发生同化
\end{quote}

\begin{enumerate}
  \def\labelenumi{(\alph{enumi})}
  \setcounter{enumi}{2}
  \item
        齿音(-ð-/-d-/-n-)与后续的-t同化为-tt,这种情况通常发生简化变成单独的-t:
\end{enumerate}

\begin{quote}
  kald- + -t \textgreater{} kaltt \textgreater{} kalt `cold'

  harð- + -t \textgreater{} hartt \textgreater{} hart `hard'

  hin- + -t \textgreater{} hitt `the'
\end{quote}

\begin{enumerate}
  \def\labelenumi{(\alph{enumi})}
  \setcounter{enumi}{3}
  \item
        -nn-有时在-r前变为-ð,这个变化最常见于两个非常基本的词中:
\end{enumerate}

\begin{quote}
  mann- + -r \textgreater{} maðr `man'

  ann(a)r- + -ar \textgreater{} aðrar `another'
\end{quote}

\begin{enumerate}
  \def\labelenumi{(\alph{enumi})}
  \setcounter{enumi}{4}
  \item
        -k-+
        -ð-有时同化为-tt-,但有时也变成-kt-,还有时不发生改变。这些音变都发生在过去式中,造成这种交替的很可能是类比的结果:
\end{enumerate}

\begin{quote}
  sœkja \textgreater{} sótti `sought'

  þykkja \textgreater{} þótti `thought'

  kneikja \textgreater{} kneikti `bent'

  但merkja \textgreater{} merkði `marked'不发生同化
\end{quote}

\begin{enumerate}
  \def\labelenumi{\Alph{enumi}.}
  \setcounter{enumi}{1}
  \item
        \textbf{词尾清化}
\end{enumerate}

结尾的辅音簇先清化,有时还进一步发生同化,主要出现在部分强动词的过去式中,这些音变包括-nd
\textgreater{} -nt \textgreater{} -tt; -ng \textgreater{} -nk
\textgreater{} -kk; -ld \textgreater{} -lt:

\begin{quote}
  binda \textgreater{} batt `bound'

  stinga \textgreater{} stakk `stung'

  gjalda \textgreater{} galt `paid'

  ganga \textgreater{} gekk `went'

  halda \textgreater{} helt `held'
\end{quote}

\begin{enumerate}
  \def\labelenumi{\Alph{enumi}.}
  \setcounter{enumi}{2}
  \item
        \textbf{辅音脱落}
\end{enumerate}

非重读音节中的-n-或-l-有时在词尾的-t前脱落:

\begin{quote}
  mikil- + -t \textgreater{} mikit `large'

  búin- + -t \textgreater{} búit `lived'

  但gamal- + -t \textgreater{} gamalt `old'不变
\end{quote}

\begin{enumerate}
  \def\labelenumi{\Alph{enumi}.}
  \setcounter{enumi}{3}
  \item
        \phantomsection\label{_Ref116211616}{}\textbf{辅音延长}
\end{enumerate}

词尾的-t或-r加在长元音词干后会拖长一拍,请注意这和前述的普罗科什定律有区别,前者通常适用于CV:C型词干,而这里属于CV:型词干,这种情况下并不造成元音的缩短。这种辅音延长更可能是受到前面的长元音的影响,如果长元音缩短,这种音变的条件就消失了:

\begin{quote}
  ný- + -t \textgreater{} nýtt `new'

  fá- + -ri\textgreater{} fárri `few'
\end{quote}

\begin{enumerate}
  \def\labelenumi{\Alph{enumi}.}
  \setcounter{enumi}{4}
  \item
        \phantomsection\label{_Ref115765758}{}\textbf{辅音简化}
\end{enumerate}

辅音簇后接辅音时,如果形成了双辅音,则双辅音简化为单辅音:

\begin{quote}
  akr- + -r \textgreater{} akrr \textgreater{} akr `meadow'

  fagr- + -rar \textgreater{} fagrrar \textgreater{} fagrar `fair'

  jarl- + -r \textgreater{} jarll \textgreater{} jarl `noble'
\end{quote}

\subsection{半元音的保持性}\label{半元音的保持性}

古诺尔斯语有两个半元音j和v,这两个半元音的由来涉及到比较复杂的历史音变,在古诺尔斯语中,半元音经常发生脱落。本节首先介绍几个共时系统中的规则,然后提供一些简单的历时规律以供读者进一步参考。

在共时系统中,半元音的遵循的总体规则是:

\begin{longtable}{l}
  \toprule
  半元音在音质``相近''的元音前脱落,在音质``相异''的元音前保留。j在出现在后元音(a, o, u, ǫ及其长音)前;v出现在非圆唇元音(i, e, a及其长音、æ)前(但有例外ǫ和ó)。 \\
  \midrule
  \endhead
  \bottomrule
  \endfoot
\end{longtable}

根据这个规律,在任何情况下,类似于†ji或†vu的形式是不允许出现的,动词系统中经常涉及元音的变化,所以半元音常消失:

\begin{quote}
  krjúp + *ir \textgreater{} *krjýpr \textgreater{} krýpr `crawls'

  *vurðu \textgreater{} urðu `became'
\end{quote}

许多动词、名词的词干尾保留了一个半元音,这里的半元音不仅要遵守上述的规则,而且还在辅音词尾或者零词尾(即不加任何词尾)前脱落,即:

\begin{longtable}{l}
  \toprule
  词干尾的半元音只可能保留在以音质``相异''的元音开头的词尾前。 \\
  \midrule
  \endhead
  \bottomrule
  \endfoot
\end{longtable}

\begin{quote}
  deyj- + -r \textgreater{} deyr `dies'

  telj- + -i \textgreater{} teli `would tell'

  verj- + -um \textgreater{} verjum `defended'

  songv + Ø \textgreater{} song `song'

  hoggv + um \textgreater{} hoggum `strike'

  sæv- + -ar \textgreater{} sævar `strike'
\end{quote}

古诺尔斯语的j和v曾经发生了历史性的脱落,这主要表现为:

1.
在词首,j全部脱落;v在l或者r前脱落。古诺尔斯语词首的j全部由元音分割而来,比较几个古诺尔斯语和英语的同源词:

\begin{quote}
  ungr------\textbf{y}oung (j脱落)

  \textbf{ja}fn------\textbf{e}ven (元音分割)

  ríta------\textbf{wr}ite (v脱落)
\end{quote}

2. 在词中,w在ó和ú之后脱落:

\begin{quote}
  glóa------glo\textbf{w}
\end{quote}

\chapter{名词与变格法}\label{名词与变格法}

\begin{quote}
  \textbf{章节要点:}
\end{quote}

\begin{itemize}
  \item
        \begin{quote}
          古诺尔斯语的名词系统
        \end{quote}
  \item
        \begin{quote}
          名词的分类
        \end{quote}
  \item
        \begin{quote}
          各类名词的变格法
        \end{quote}
  \item
        \begin{quote}
          名词的特指后缀
        \end{quote}
\end{itemize}

\section{名词的概述}\label{名词的概述}

古诺尔斯语是一种形态变化十分丰富的语言,语法中的范畴常由词尾进行表示。

对于名词系统而言,古诺尔斯语有以下几个基本范畴:格、性、数。格(Case)反映名词在短语、从句或句子中所起语法功能;性(Gender)反映一个名词的类别;数(Number)则顾名思义,反映名词指代的个体的数量。

在古诺尔斯语中,名词共有:

\begin{enumerate}
  \def\labelenumi{\arabic{enumi})}
  \item
        2个数。分别为单数和复数,这点和英语完全一致,指代单个物体的名词用单数,指代一个以上的物体时则用复数。PIE名词系统中的双数在日耳曼语中完全消失。
  \item
        3个性。分别是阴性、阳性和中性。和大多数区分性的语言一样,性是一个语法概念,和具体物体的属性关系不大(但在描述生物性十分明确的物体时,语法性和生物性是一样的,例如男人不可能是阴性词)。阳性名词未必是阳刚的,阴性名词也未必是阴柔的,不便于区分生物性的名词也未必都是阴性的。性是一种对名词分类的方式,古诺尔斯语中的每个性都对应着一套相应的词尾。
  \item
        4个格(主格、属格、与格、宾格)。为了方便起见,本书中用缩写表示名词格:
\end{enumerate}

\begin{longtable}{lll}
  \toprule
  中文 & 英文全程   & 英文缩写 \\
  \midrule
  \endhead
  \bottomrule
  \endfoot
  主格 & Nominative & N        \\
  属格 & Genitive   & G        \\
  与格 & Dative     & D        \\
  宾格 & Accusative & A        \\
\end{longtable}

主格用于标记一句话的语法主语,或是主语的补足语。

宾格用于标记动词的直接宾语,或是表示一个既定动作的时间/空间性范围,例如英文中for
several miles‌或for several days在古诺尔斯语中可以用宾格标记several
miles或several days,从而不必用介词for来表示范围。

与格用于标记动词的间接宾语,或是一个动作的方式、手段、工具等。PIE中还有一个工具格(Instrumental),工具格合并到了与格中。古诺尔斯语的与格和古希腊语十分相似。

属格用于标记关系,关系既可以是具体明确的,如所有关系:John\textbf{'s}
book,也可以是更宽泛的,例如作定语:a day\textbf{‌'s}
journey;作同位语:the world tree \textbf{of}
Yggdrasil。少数动词也支配属格的双宾语,如biðja(要求),要求的对象用属格,要求的内容用宾格。

除主格外的格统称间接格(Oblique)。

有格标记(尤其是主格以外的格)的名词(N)天生包含有英语中介词短语(Prep.+N)的意味,因此英语中对应的介词在古诺尔斯语中常常省略,当然也有一些表达必须要添加介词。介词也可以接续主格外的任一格。

在上文中已经提到,性是一种为名词分类的方式,但除了性之外,古诺尔斯语中还把名词分为强变化名词和弱变化名词,简称强名词和弱名词。强弱在这里也只是一种给名词分类的方法,并没有更深一层的含义。德国语言学家格林(J.
Grimm),即格林定律的发现者和《格林童话》的收集者,在描述日耳曼语语法特征时最早引入了强弱的概念,这套术语一直沿用至今。强名词和弱名词这一分类的唯一标准在于词尾的类型,即强名词支配强变化词尾,弱名词支配弱变化词尾。名词的强弱、性都是其固有属性,不会随数和格的改变而改变。

按照这个规则,名词的变格法可以按照性和强弱进行划分,一种有2×3=6类词尾。即强变化的阴阳中性词尾和弱变化的阴阳中性词尾,它们彼此有所区别。

强名词和弱名词都可以进一步按照词干的词干元音(Thematic
vowel)分类,词干元音依附在词干后,它对名词的形态还有更进一步的影响。按词干元音对名词分类继承于原始日耳曼语,和其他古典语言的词类划分有相似之处,例如古诺尔斯语的a-词干名词和古希腊语的-ο变格法同源。这种分类方法有一点不直观的地方,在古诺尔斯语中,词尾的音节发生大量脱落,以至于词干元音经常消失。a-词干名词gramr
`wrath'中没有一个a,但它的古希腊语同源词χρόμ\textbf{ο}ς(khróm\textbf{o}s)中则可以清晰地看到-ο-。但是,词干元音有助于理解词干中变元音的来源,因此本书在介绍名词时,也将词干元音纳入进来。请读者注意,即便不了解名词的词干元音也完全可以学习古诺尔斯的变格法,甚至绝大多数的字典都不会列出名词的词干元音(少数注重历时比较的词源学字典除外),它们大多只会列出名词的性和少数变格的形式,这样已经完全足够确定名词的变形方式了。本书对于词干元音的介绍只是为了方便初学者了解名词的构成和词形变化中一些不规则现象产生的原因,读者尤其应该记住的不是名词的词干元音,而是6类变格法的规则。

\section{强名词的变格法}\label{强名词的变格法}

\subsection{强名词的词尾}\label{强名词的词尾}

强名词根据阴、阳、中性添加下述的词尾:

\begin{longtable}{llll}
  \toprule
  性   & 阳性          & 中性              & 阴性                        \\
  \midrule
  \endhead
  \bottomrule
  \endfoot
  单数 &               &                   &                             \\
  N    & -r            & -ø                & (词干u-变异) + -ø, -r       \\
  A    & -ø            & -ø                & (词干u-变异) + -ø, {[}-u{]} \\
  G    & -s, -ar       & -s                & -ar                         \\
  D    & -i            & -i                & (词干u-变异) + -ø, {[}-u{]} \\
  复数 &               &                   &                             \\
  N    & 词干元音 + -r & (词干u-变异) + -ø & 词干元音 + -r               \\
  A    & 词干元音 + -ø & (词干u-变异) + -ø & 词干元音 + -r               \\
  G    & -a            & -a                & -a                          \\
  D    & -um           & -um               & -um                         \\
\end{longtable}

在上述的列表中,-ø表示零词尾。有可能加不同的词尾的,用逗号隔开。逗号前的词尾一般是更常见、更基础的形式,逗号后的词尾则添加在一些相对少见的名词类别上。这些问题留到后续介绍名词的词干元音划分法时说明。

名词中的u-变异现象非常明显,有一些u-变异出现在零词尾前,这是由于先前造成u-变异的词尾已经脱落。阴性名词中这一现象尤其显著,试比较古诺尔斯语g\textbf{jǫ}f
`gift'和古英语ġief\textbf{u},除古诺尔斯语外的日耳曼语大多都保留了词尾的元音。少数阴性名词的单数宾格和属格中还保留了造成u-变异的-u,这些词尾用中括号{[}{]}标出。复数与格的-um词尾总是规则地造成u-音变,如barn
`child'的复数属格是bǫrnum.

与之相比,名词中i-变异的痕迹则少得多,某些名词的词干元音包含i或j,这使得整个名词词干都发生了i-变异,因而在共时系统中表现地非常规则。值得注意的是,单数属格的-i词尾基本不会造成i-变异,其原因是这个词尾过去是*-ē,后来在i-变异停止后被抬升为-i,试比较古诺尔斯语úlf\textbf{i}
`wolf'和古英语wulf\textbf{e}.

一些规律可供参考:

\begin{longtable}{l}
  \toprule
  1. -r是主格的标志,除中性词外,单数和复数主格一般都有-r.    \\
  \midrule
  \endhead
  \bottomrule
  \endfoot
  2. 复数属格都是-a,复数与格都是-um.                         \\
  3. -s或-ar是单数属格的标志。                                \\
  4. 单数宾格都是零词尾,阴性名词的-u词尾已经脱落。           \\
  5. 中性名词不论单数复数,主格和宾格总一样,且均不添加词尾。 \\
\end{longtable}

\subsection{a/ja/wa-词干}\label{a/ja/wa-词干}

简单来说,印欧语的常见构词法(无论动词还是名词)是用一个元音把词根和词尾连接到一起,词尾前的部分都称为词干,因而这个元音被称为词干元音,参考下面的例子:

\begin{quote}
  PIE *bʰér-e-ti `he bears' 词根bʰér- 词干元音-e- 人称词尾-ti

  PIE *gʷʰér-o-s `warmth' 词根gʷʰér- 词干元音-o- 格词尾-s
\end{quote}

词干元音是晚期PIE的一种创新,有词干元音的词类(英文称为Thematic)变化一般比较规则。还有一部分更古老的词是无词干元音的(英文称为Athematic),这些词的变形则比较复杂,涉及到词干的元音交替,甚至是重音位置的变化。

在日耳曼语中,以a/ja/wa-为词干元音的名词是各种日耳曼语中最广大的名词类之一,它们基本都是阳性或中性的。我们把它们简称为a/ja/wa-词干,其中,a-词干是最基本的类型,少数词在a-前插入了一个-j-/-w-。半元音的出现使得ja-词干和wa-词干的变形略有一些费解之处。

\textbf{a-词干名词}

a-词干名词规则地适用于\ref{强名词的词尾}中介绍的词尾,试比较下面的阳性名词sandr
`sand', himinn `heaven'和中性名词barn `child', kné `knee':

\begin{longtable}{lllll}
  \toprule
       & \multicolumn{2}{c}{阳性} & \multicolumn{2}{c}{中性}                        \\
  \midrule
  \endhead
  \bottomrule
  \endfoot
  词干 & sand-a-                  & himin-a-                 & barn-a & kné-a       \\
  单数 &                          &                          &        &             \\
  N    & sandr                    & himinn                   & barn   & kné         \\
  A    & sand                     & himin                    & barn   & kné         \\
  G    & sands                    & himins                   & barns  & knés        \\
  D    & sandi                    & himni                    & barni  & kné         \\
  复数 &                          &                          &        &             \\
  N    & sandar                   & himnar                   & bǫrn   & kné         \\
  A    & sanda                    & himna                    & bǫrn   & kné         \\
  G    & sanda                    & himna                    & barna  & knjá        \\
  D    & sǫndum                   & himnum                   & bǫrnum & knjám,knjóm \\
\end{longtable}

在上表中的词干一栏中,我们有-a-标出了这些名词词干原有的样子,但是-a-仅在阳性名词的复数主格和宾格中出现。

说明:

\begin{enumerate}
  \def\labelenumi{\arabic{enumi})}
  \item
        阳性a-词干是最基本、构词力最强的词类。它的标志是主格的-r,许多派生词尾-ingr/-ungr(大致表示属于\ldots 的)都按这种方式变格。有时单词末尾的r是词干的一部分,如angr
        `sorrow', akr
        `meadow',这是因为词尾辅音简化(\ref{辅音的音变})导致的。
  \item
        一些中性名词的词干中有长元音,和词尾连接时可能引起\ref{元音的音变}中的元音缩合。比较明显地变化是é+-a/-um
        \textgreater{} já/jó. 以é结尾的中性名词还有tré `tree', hlé `shelter',
        klé `stone' vé `house', fé
        `cattle'等,但不是每个词都发生元音缩合。tré,
        klé的变形和kné一致,但hlé, vé,
        fé中都没有元音缩合,即有类似于véum的形式。
  \item
        单数与格的-i一般不造成i-变异,但常见词dagr `day'却变化为degi.
  \item
        双音节词干的名词的变化基本类似于himinn,根据\ref{元音的音变}n-a-中的弱读元音(粗体标出)在以元音起首的词尾前全部脱落,因此单数与格和整个复数中完全没有弱读元音。不过,这一规则亦有少许例外,如常见的人名Gunnar的单数与格Gunnari保留了-a-.
  \item
        部分以-ill词尾派生的名词,如ketill `kettle'和lykill
        `key'(注意词尾-r的同化),
        它们第一个音节的前元音是后一个音节的i导致的,当非重读的i被省略时,有时第一个音节的元音恢复其本来的形式,如上述两个词的单数与格为katli,lukli.
        这个音变无疑是历时规律的影响,因而在共时系统中常常被类比原则所规则化,因此lukli也可以写成lykli,且现代冰岛语只用lykli。其他一些名词如berill
        \textless{} *barilaz
        `vessel'则完全不恢复原本的a;ketill是一个例外,无论是古诺尔斯语还是现代冰岛语,都要恢复a.\footnote{-ill词尾有两个作用:1)用作指小词(Diminutive
          nouns)后缀。从名词上派生出比原始词略小、较弱等概念的新词;2)用作代理名词(Agent
          nouns)后缀。从动词上派生出表示发出这个动作的实体。例如berill可以理解为从bera
          `carry'这个动词上派生,表示``可以运载东西的物体'',衍生为``船''。这些名词和动词关系密切,其中的元音也很可能受到动词的影响。}
        没有一个完美的规律可以解释这其中的不规则现象,读者在遇到相应的词时应该借助字典了解其正确的变形。
  \item
        一小部分名词的单数属格不是-s而是-ar,例如hǫfundr
        `chieftain'的单数属格为hǫfundar.
        -ar的来源尚不明确,一些观点认为它是从别的词类中类比得到,另一部分比较新的观点认为这个词尾本和-s同源,但经历了稍有区别的发展过程。
\end{enumerate}

\textbf{ja-词干名词}

ja-词干名词的词干元音前有一个j,在日耳曼语中,j的性质非常特殊,它根据词干的长短时而变为辅音,时而变为原因。在古诺尔斯语中,词干的长短和词干音节(这里的词干并不包括j\footnote{从理论上来说,j的确是``词干''的一部分,但是计算音节长短时,不把j计入其中。})的划分有密切的关系,它们是这么定义的:

\begin{longtable}{l}
  \toprule
  短词干:词干音节只有一个单辅音+不超过一个辅音/一个双元音或长元音 \\
  \midrule
  \endhead
  \bottomrule
  \endfoot
  长词干:词干音节为单辅音+辅音簇/双元音或长元音+任意数量的辅音    \\
\end{longtable}

短词干基本上和传统式音节划分法中的短音节和部分第二类长音节一致;其余的音节都被分到长词干。如果使用的是格律法,那么可以认为不超过两个音拍的词干属于短词干。

ja-词干名词的总体规律是:

\begin{longtable}{l}
  \toprule
  j在长词干后面表现为i,在短词干后面表现为j。 \\
  \midrule
  \endhead
  \bottomrule
  \endfoot
\end{longtable}

因此,长词干ja-词干名词有时也称为ija-词干或者īa-词干,短词干ja-词干名词就简单地称作ja-词干。无论是哪种情况,i/j的存在都导致了整个词干发生了i-变异,因此所有的ja-词干名词中只有前元音。不过,即便变格的形式中i/j没有出现,这个前元音依旧保留了下来,从共时特征上已经看不出任何i-变异的痕迹。

试比较下面的长短词干名词:阳性名词niðr `kinsman', hirðir `herdsman',
中性名词kyn `kin', ríki
`kingdom',每一对词中,左边的是短词干名词,右边的是长词干名词。

\begin{longtable}{lllll}
  \toprule
       & \multicolumn{2}{c}{阳性} & \multicolumn{2}{c}{中性}                     \\
  \midrule
  \endhead
  \bottomrule
  \endfoot
  词干 & nið-j-a-                 & hirð-j-a-                & kyn-j-a & rík-j-a \\
  单数 &                          &                          &         &         \\
  N    & niðr                     & hirðir                   & kyn     & ríki    \\
  A    & nið                      & hirði                    & kyn     & ríki    \\
  G    & niðs                     & hirðis                   & kyns    & ríkis   \\
  D    & nið                      & hirði                    & kyni    & ríki    \\
  复数 &                          &                          &         &         \\
  N    & niðjar                   & hirðar                   & kyn     & ríki    \\
  A    & niðja                    & hirða                    & kyn     & ríki    \\
  G    & niðja                    & hirða                    & kynja   & ríkja   \\
  D    & niðjum                   & hirðum                   & kynjum  & ríkjum  \\
\end{longtable}

说明:

\begin{enumerate}
  \def\labelenumi{\arabic{enumi})}
  \item
        根据上述规律,短词干的ja-词干名词可以被理解为以-j结尾的a-词干名词,例如niðr的词干部分可被看作是niðj.
        根据半元音的保持性(见\ref{半元音的保持性}),词干末尾的j只能出现在以后元音起首的词尾前。因此整个单数部分中都没有j,但在-ar,
        -a, -um前,j保留下来,参见阳性名词的复数和中性名词的复数属格和与格。
  \item
        类似地,长词干的ja-名词可以被理解为以-i结尾的a-词干名词,例如hirðir的词干部分可被看作是hirði.
        根据\ref{元音的音变},i在任何以元音起首的词尾前脱落。因此整个复数部分中-i都没有表现出来。
  \item
        短词干的阳性ja-词干名词极其罕见,除了niðr外,常见的词只有herr
        `army',其他本属于这一类的名词在古诺尔斯语中被类比到了其他种类里。这类词的单数与格没有-i,形态和宾格一致,其原因可能是受到了长词干名词的影响,因为后者的单数属格和宾格总是一致的。
  \item
        ríki一词的复数属格和与格中有-j-而不是-i-,这是古诺尔斯语中的一个例外:
\end{enumerate}

\begin{longtable}{l}
  \toprule
  如果长词干以硬颚音k/g结尾,则i在元音开头的词尾前变回j。 \\
  \midrule
  \endhead
  \bottomrule
  \endfoot
\end{longtable}

\begin{longtable}{l}
  \toprule
  \begin{itemize}\item  西弗斯定律(Sievers\textbf{'}s Law)\end{itemize}德国语言学家爱德华·西弗斯(EduardSievers)发现包含辅音簇中的半元音在元音前的发音会根据词干音节的轻重而改变。在原始日耳曼语中,PIE的*y在\textbf{轻词干}后变成*j,\textbf{重词干}后变成*ij(按照格律划分法):轻词干:PIE *kor-yo-s \textgreater{} PGmc. *harjaz \textgreater{} Go.harjis `army'重词干:PIE *ḱerd\textsuperscript{h}-yo-s \textgreater{} PGmc. *hirdijaz\textgreater{} Go. hairdeis `shepherd' (ei读作长音i) \\
  \midrule
  \endhead
  \bottomrule
  \endfoot
  西弗斯定律在哥特语中最为明显,在古诺尔斯语中也有保留,上述的ja-词干的情况就是最直接的例子。只不过,在古诺尔斯语中西弗斯定律的演变结果可能在进一步的音变中消失了。

  古诺尔斯语中长短词干的划分和上述的轻重词干不完全一致,主要是一些长元音词干(按照格律法应该认为是重词干)的名词变形和轻词干一致。其主要原因是古诺尔斯语中的这些长元音大多在原始日耳曼语中是元音+半元音的形式,它们相当于轻词干名词,从而符合西弗斯定律的条件。在原始日耳曼语发展的过程中,半元音和元音结合为了古诺尔斯语的长元音。

  西弗斯定律不仅在名词中有体现,在一类弱动词中也非常显著(见\ref{第一弱变位法})。                                                                                                                                                                                                                                                                                                                                                                                                                    \\
\end{longtable}

\textbf{wa-词干名词}

wa-词干名词数量不多,它们相比ja-词干来说规则得多,这是因为西弗斯定律在日耳曼语中仅仅对y有效(在梵语中,它也影响w)。wa-在古诺尔斯语中写作va-,v一开始仅在同时满足下述两个条件时保留:

\begin{enumerate}
  \def\labelenumi{\arabic{enumi})}
  \item
        紧跟在一个短音节或辅音k/g之后(辅音簇亦可,只需要其最后一个辅音是k/g)。
  \item
        在a或i之前。
\end{enumerate}

但在很多情况下,v又重新按类比原则添加了回去,有时甚至可以出现在u之前。另外,由于绝大部分wa-词干名词都满足第一个条件,而第二个条件和\ref{半元音的保持性}中介绍的一致,读者亦可以认为wa-词干名词v的显现完全是规则的。

类似于ja-词干名词,wa-词干名词也系统地发生了u-变异,因此许多词干中都有ǫ.

阳性名词sǫngr `song', sær/sjór `sea',中性名词hǫgg
`strike'都按照wa-词干变格:

\begin{longtable}{lllll}
  \toprule
       & \multicolumn{3}{c}{阳性} & 中性                         \\
  \midrule
  \endhead
  \bottomrule
  \endfoot
  词干 & song-v-a-                & sæ-v-a- &         & hǫgg-v-a \\
  单数 &                          &         &         &          \\
  N    & sǫngr                    & sær     & sjór    & hǫgg     \\
  A    & sǫng                     & sæ      & sjó     & hǫgg     \\
  G    & sǫngs                    & sævar   & sjóvar  & hǫggs    \\
  D    & sǫngvi                   & sæ(vi)  & sjó(vi) & hǫggvi   \\
  复数 &                          &         &         &          \\
  N    & sǫngvar                  & sævar   & sjóvar  & hǫgg     \\
  A    & sǫngva                   & sæva    & sjóva   & hǫgg     \\
  G    & sǫngva                   & sæva    & sjóva   & hǫggva   \\
  D    & sǫngum                   & sæ(v)um & sjóvum  & hǫggum   \\
\end{longtable}

sær有两个异体形式sjór,sjár,在早期的文本中,sær较为常见,sjór是后期出现的产物,sjár则较为罕见。这组词也可以按照i-词干变格(见\hyperref[_Ref115770706]{2.2.4}中阳性部分)。

如果v紧随元音之后,这种词的单数属格一般是-ar,如sævar和sjóvar,但是规则地添加-s也是可行的,类比作用在其中发挥了很大作用。

\subsection{ō/jō/wō-词干}\label{ō/jō/wō-词干}

ō / jō /
wō-词干名词都是阴性的。由于词干根据原始语分类,在古诺尔斯语出现的时期词干元音已经不以ō形式出现,反而合并为a。ō/jō/wō-词干是a/ja/wa-词干的阴性对应,也是最广泛的阴性名词。

\textbf{ō-词干}

阴性名词grǫf `hole', laug `bath', á `river‌', 以及Ingibjǫrg `Ingeborg‌'
是典型的ō-词干名词:

\begin{longtable}{lllll}
  \toprule
       & \multicolumn{4}{c}{词干}                                 \\
  \midrule
  \endhead
  \bottomrule
  \endfoot
  词干 & graf-a-                  & laug-a- & á-a- & Ingibjarg-a- \\
  单数 &                          &         &      &              \\
  N    & grǫf                     & laug    & á    & Ingibjǫrg    \\
  A    & grǫf                     & laug    & á    & Ingibjǫrgu   \\
  G    & grafar                   & laugar  & ár   & Ingibjargar  \\
  D    & grǫf                     & laug    & á    & Ingibjǫrgu   \\
  复数 &                          &         &      &              \\
  N    & grafar                   & laugar  & ár   &              \\
  A    & grafar                   & laugar  & ár   &              \\
  G    & grafa                    & lauga   & á    &              \\
  D    & grǫfum                   & laugum  & ám   &              \\
\end{longtable}

说明:

\begin{enumerate}
  \def\labelenumi{\arabic{enumi})}
  \item
        ō-词干名词的阳性主格总没有-r,这是阴性ō-词干名词的基本特征。-r出现在下面的长词干jō-词干名词中。
  \item
        单数主格、属格和与格中有u-变异。单数主格和与格的u-变异由词干元音造成,虽然ō在古诺尔斯语中表现为a,但在原始诺尔斯语中仍是-u,这在一些西日耳曼语中有保留,例如古英语中grǫf的同源词ġiefu.
        单数宾格的u-变异从主格类比得来。一般来说,单音节词干的单数宾格和与格都不加-u。多音节词和一些专有名词中这个-u被保留了下来,如Ingibjǫrgu.\footnote{更准确地来说,宾格的-u是被加回去的,用以解释词干中的u-变异。因此,属格中有-u的情况多于宾格中有-u的情况,例如}常见的阴性派生词尾-ing(从动词派生名词,区别阳性的-ingr)的单数变格为-ing,
        ing, -ingar, ingu,宾格不加-u.
  \item
        复数式的构成和阳性非常相似,但注意宾格有-r.
  \item
        á是一个常见的阴性名词,与词尾连接时经常产生元音缩合。
\end{enumerate}

\textbf{jō-词干}

jō-词干中也有长短词干的区别,对于这两类词干的处理与ja-词干完全一致,因为两者都是西弗斯定律的结果。

短词干名词 ben `wound',长词干名词heiðr `heath'代表了两类jō-词干名词:

\begin{longtable}{lll}
  \toprule
       & \multicolumn{2}{c}{阴性}             \\
  \midrule
  \endhead
  \bottomrule
  \endfoot
  词干 & ben-j-a-                 & heið-j-a- \\
  单数 &                          &           \\
  N    & ben                      & heiðr     \\
  A    & ben                      & heiði     \\
  G    & benjar                   & heiðar    \\
  D    & ben                      & heiði     \\
  复数 &                          &           \\
  N    & benjar                   & heiðar    \\
  A    & benjar                   & heiðar    \\
  G    & benja                    & heiða     \\
  D    & benjum                   & heiðum    \\
\end{longtable}

说明:

\begin{enumerate}
  \def\labelenumi{\arabic{enumi})}
  \item
        只有长词干jō-词干的主格有标记-r,这是从i-词干名词(见\hyperref[_Ref115770706]{2.2.4}中阴性部分)类比得到的。
  \item
        -j-的保留完全符合\ref{半元音的保持性}中的规律。
  \item
        一些短词干名词的单数与格中还保留了-u,构成了-ju的形式。常见的词包括ey
        `island', egg `egg', hel `Hel, death'.
\end{enumerate}

\textbf{wō-词干}

另有一小类 wō-词干名词,如dǫgg `dew', ǫr `arrow'

\begin{longtable}{lll}
  \toprule
       & \multicolumn{2}{c}{词干}           \\
  \midrule
  \endhead
  \bottomrule
  \endfoot
  词干 & dǫgg-v-a                 & ǫr-v-a- \\
  单数 &                          &         \\
  N    & dǫgg                     & ǫr      \\
  A    & dǫgg                     & ǫr      \\
  G    & dǫggvar                  & ǫrvar   \\
  D    & dǫgg(u)                  & ǫr(u)   \\
  复数 &                          &         \\
  N    & dǫggvar                  & ǫrvar   \\
  A    & dǫggvar                  & ǫrvar   \\
  G    & dǫggva                   & ǫrva    \\
  D    & dǫggum                   & ǫrum    \\
\end{longtable}

说明:

\begin{enumerate}
  \def\labelenumi{\arabic{enumi})}
  \item
        wō-词干名词中-v-的出现和wa-词干名词完全一致。
  \item
        单数属格的-u是可选的,但不加的情况为多。
\end{enumerate}

\subsection{i-词干}\label{i-词干}

一部分日耳曼语名词的词干元音是-i,称之为i-词干名词。由于i的存在,部分名词的整个变位中都发生了i-变异,例如gestr
\textless{} PGmc. *gastiz (比较哥特语gasts),但在有些名词中,例如staðr,
这一音变没有沿袭下来。\footnote{i-词干名词中不发生i-变异的一般是短词干名词,类似的例子还有dalr
  `dale', salr `room', matr `food', þulr
  `poet',长词干名词一般都发生i-变异,一些例外包括stuldr `theft', sultr
  `hunger', burðr `birth', sauðr `sheep',它们的词干中一般有u.
  i-变异无论是否失效,并不影响读者对于变格的判断,因为这个变格表中要么都发生i-变异,要么都不发生i-变异。}i-词干名词很大程度上是日耳曼语的创新,它将许多PIE中变形规则复杂的无词干元音名词规则化了。i-词干名词对应了PIE中的各种性的名词,但中性词比较少。到了古诺尔斯语时期,i-词干名词只有阳性和阴性两种了,中性词基本被合并到阳性词。i-词干与a-词干和ō-词干发生了很大程度的交互,一些词尾相互借鉴和渗透了。

\textbf{阳性i-词干}

阳性i-词干名词添加阳性词尾。试比较三个常见名词staðr `place', gestr
`guest', bekkr `bench':

\begin{longtable}{llll}
  \toprule
       & \multicolumn{3}{c}{阳性}                            \\
  \midrule
  \endhead
  \bottomrule
  \endfoot
  词干 & stað-i-                  & gest-i- & bekk-i-        \\
  单数 &                          &         &                \\
  N    & staðr                    & gestr   & bekkr          \\
  A    & stað                     & gest    & bekk           \\
  G    & staðar                   & gests   & bekks, bekkjar \\
  D    & stað                     & gest(i) & bekk           \\
  复数 &                          &         &                \\
  N.   & staðir                   & gestir  & bekkir         \\
  A    & staði                    & gesti   & bekki          \\
  G    & staða                    & gesta   & bekkja         \\
  D    & stǫðum                   & gestum  & bekkjum        \\
\end{longtable}

说明:

\begin{enumerate}
  \def\labelenumi{\arabic{enumi})}
  \item
        复数主格和宾格中有词干元音-i,因此,i-词干名词的变格和a-词干的区别仅在于词干元音不同。
  \item
        单数属格可能是-s,也可能是-ar,还可能二者皆可。i-词干中这两种词尾出现的频率几乎是相等的,请读者查阅字典解决。\footnote{从古老的卢恩铭文来看,-ar很可能是更古老的词尾。-s词尾由a-词干名词类比得来。}
  \item
        区别于a-词干名词,单数与格一般不加词尾-i,但gesti这种形式无疑受到了a-词干的影响。
  \item
        在k/g后,词干元音-i-以-j-的形式出现在后元音前(-ar, -a,
        -um)。这条规律在\ref{a/ja/wa-词干}节中已经提到过。
\end{enumerate}

\textbf{阴性i-词干}

大量的阴性i-词干名词与ō-词干发生混杂,以至于许多i-词干名词和ō-词干的单数形式完全相同,这些词的标志是复数主格和宾格的词尾-ir。参见下面几个常见的阴性名词nauð(r)
`distress', ást `love', hǫll `hall':

\begin{longtable}{llll}
  \toprule
       & \multicolumn{3}{c}{阴性}                     \\
  \midrule
  \endhead
  \bottomrule
  \endfoot
  词干 & nauð-i-                  & ást -i- & hall-i- \\
  单数 &                          &         &         \\
  N    & nauð(r)                  & ást     & hǫll    \\
  A    & nauð                     & ást     & hǫll    \\
  G    & nauðar                   & ástar   & hallar  \\
  D    & nauð                     & ást     & hǫll(u) \\
  复数 &                          &         &         \\
  N    & nauðir                   & ástir   & hallir  \\
  A    & nauðir                   & ástir   & hallir  \\
  G    & nauða                    & ásta    & halla   \\
  D    & nauðum                   & ástum   & hǫllum  \\
\end{longtable}

说明:

\begin{enumerate}
  \def\labelenumi{\arabic{enumi})}
  \item
        PIE中这类词无论阴性还是阳性,词尾总是一样的。nauðr的词尾-r实则是更古老的形式,但由于和ō-词干阴性名词的类比,-r后来逐渐丢失了。阳性名词的词尾正好都有-r,所以i-词干的-r完好地保留了下来。
  \item
        ást反映了最典型的i-词干名词,它和ō-词干阴性名词的区别仅仅在于复数主格和宾格的词尾是-ir,这是词干元音有区别导致的。
  \item
        hǫll一词也反映了i-词干和ō-词干的混淆,它本身是ō-词干的,因此有时在单数与格中出现可选的-u,但在复数式中,又以-ir结尾。
\end{enumerate}

\subsection{u-词干}\label{u-词干}

u-词干名词的来源和i-词干十分相似。但在古诺尔斯语中,剩下的u-词干名词都是阳性的,其他性的名词被合并到了其他类别中。u-词干名词的词尾和阳性名词基本一致,但是复数主格是-ir而非†-ur.

u-词干名词有三个明显的特征:

\begin{enumerate}
  \def\labelenumi{\arabic{enumi})}
  \item
        元音分割。根据\ref{元音的音变}的规则,*e在a前变成ja,u前变成jǫ.
        由于u-词干名词的词尾中有a和u的交替,故两类元音分割都有发生。在其他此类中,例如a-词干名词,元音分割由词干
  \item
        包含-i-/-u-的词尾发生i-/u-变异。
\end{enumerate}

我们以三个名词skjǫldr `shield', vǫllr `ground', fǫgnuðr `joy'为例:

\begin{longtable}{llll}
  \toprule
       & \multicolumn{3}{c}{阳性}                       \\
  \midrule
  \endhead
  \bottomrule
  \endfoot
  词干 & skeld-u-                 & vall-u- & fagnað-u- \\
  单数 &                          &         &           \\
  N    & skjǫldr                  & vǫllr   & fǫgnuðr   \\
  A    & skjǫld                   & vǫll    & fǫgnuð    \\
  G    & skjaldar                 & vallar  & fagnaðar  \\
  D    & skildi                   & velli   & fagnaði   \\
  复数 &                          &         &           \\
  N    & skildir                  & vellir  & fagnaðir  \\
  A    & skjǫldu                  & vǫllu   & fǫgnuðu   \\
  G    & skjalda                  & valla   & fagnaða   \\
  D    & skjǫldum                 & vǫllum  & fǫgnuðum  \\
\end{longtable}

说明:

\begin{enumerate}
  \def\labelenumi{\arabic{enumi})}
  \item
        单数属格、复数主格的词干元音是-i,这导致了词干中的i-变异。特别地,词干中的元音*e在这种情况下抬升成了i,其他元音都按\ref{变元音}发生变化。
  \item
        在除了单数属格、复数主格的情况下,词干元音是-u或-a,因此这些形式中都可能有元音分割。如果不满足元音分割的条件(词干中的元音不是*e),则在包含u-的词尾前发生u-变异。
  \item
        fǫgnuðr展示了以-uðr词尾派生出的名词,这个词尾本来是-aðr\footnote{-aðr本是形容词词尾,从名词中派生出表示``拥有这个名词的''形容词。如hjalmr
          `helmet' + -aðr → hjalmaðr `having a helmet, helmed'.
          作名词的派生词尾时,构成动词或形容词的同义名词。例如fǫgnuðr由动词fagna
          `rejoice'派生得来。这个词尾被归结为u-词干,是因为其原始诺尔斯语的形式*-oþuz中有u.},但词干元音的-u导致了非重读的a
        \textgreater{}
        u,进而引起了词根中重读元音的进一步u-变异。在单数属格、复数主格中,-i对非重读的a无效,因此没有造成i-变异。古诺尔斯语中-uðr
        词尾也可以就写作-aðr而不引起任何u-变异,此时的变形基本是完全规则的(复数宾格词尾按正常的阳性词尾类比为了-i)。故fǫgnuðr也可写作fagnaðr,其变格如下所示:
\end{enumerate}

\begin{longtable}{ll}
  \toprule
  \multicolumn{2}{c}{fagnaðr} \\
  \midrule
  \endhead
  \bottomrule
  \endfoot
  单数 &                      \\
  N    & fagnaðr              \\
  A    & fagnað               \\
  G    & fagnaðar             \\
  D    & fagnaði              \\
  复数 &                      \\
  N    & fagnaðir             \\
  A    & fagnaði              \\
  G    & fagnaða              \\
  D    & fǫgnuðum             \\
\end{longtable}

\subsection{辅音词干}\label{辅音词干}

目前所介绍的名词都有一个词干元音,还有一小类词的词干以辅音作结。可以进一步分词几个小类:r-词干;nd-词干;其它辅音词干。这些词几乎都是更古老的语言中保留下来的产物,往往反映出一些非常基本的概念(从r-词干中可见一斑)。这些词在PIE中的变形一般比较复杂,在日耳曼语中这些词受到其他词类的影响逐渐规则化,但它们的形式仍不太符合上述的元音词干名词。

\textbf{r-词干}

表示亲属关系的词都属于r-词干,它们的语法性和自然性是一样的,也就是说``母亲''、``女儿''这样的词是阴性的,``父亲''、``兄弟''这类词是阳性的。这类词仅有五个,都表示最亲密的血缘关系。

\begin{longtable}{llllll}
  \toprule
                    & faðir                        & móðir                      & bróðir    & dóttir       & systir  \\
  \midrule
  \endhead
  \bottomrule
  \endfoot
                    & `father' \emph{\textbf{m}}   & `mother' \emph{\textbf{f}} & `brother'
  \emph{\textbf{m}} & `daughter' \emph{\textbf{f}} & `sister'
  \emph{\textbf{f}}                                                                                                  \\
  单数              &                              &                            &           &              &         \\
  N                 & faðir                        & móðir                      & bróðir    & dóttir       & systir  \\
  A                 & fǫður, feðr                  & móður                      & bróður    & dóttur       & systur  \\
  G                 & fǫður, feðr                  & móður                      & bróður    & dóttur       & systur  \\
  D                 & fǫður, feðr                  & móður                      & bróður    & dóttur, dœtr & systur  \\
  复数              &                              &                            &           &              &         \\
  N                 & feðr                         & mœðr                       & brœðr     & dœtr         & systr   \\
  A                 & feðr                         & mœðr                       & brœðr     & dœtr         & systr   \\
  G                 & feðra                        & mœðra                      & brœðra    & dœtra        & systra  \\
  D                 & feðrum                       & mœðrum                     & brœðrum   & dœtrum       & systrum \\
\end{longtable}

说明:

\begin{enumerate}
  \def\labelenumi{\arabic{enumi})}
  \item
        这些词的单数主格都是-ir,其余单数间接格-ur并造成u-变异。faðir的间接格还出现了带有i-变异的-r词尾,这可能是更古老的形式。
  \item
        复数中都出现i-变异。主格和宾格的词尾为-r,属格和与格的词尾与其他所有名词一致。
  \item
        请注意:表示儿子的词sonr或sunr按u-词干变形。其中词干中o/u的交替来自于一种古老的a-变异(PIE的*o在PGmc.中与*a合流,使得PGmc.缺少*o音,导致u在a前下降为o)。
\end{enumerate}

\textbf{nd-词干}

形式上由动词的现在分词衍生,成为某种固定下来的表达的名词属于-nd词干。

我们以两个名词bóndi `farmer'以及gefandi `giver' 为例说明其词尾:

\begin{longtable}{lll}
  \toprule
       & bóndi  & gefandi  \\
  \midrule
  \endhead
  \bottomrule
  \endfoot
  单数 &        &          \\
  N    & bóndi  & gefandi  \\
  A    & bónda  & gefanda  \\
  G    & bónda  & gefanda  \\
  D    & bónda  & gefanda  \\
  复数 &        &          \\
  N    & bœndr  & gefendr  \\
  A    & bœndr  & gefendr  \\
  G    & bónda  & gefanda  \\
  D    & bóndum & gefǫndum \\
\end{longtable}

说明:

\begin{enumerate}
  \def\labelenumi{\arabic{enumi})}
  \item
        单数主格词尾为-i,间接格为-a,这正好和形容词的比较级变格一致。(交叉)
  \item
        复数主格和宾格的变形和r-词干一致,都发生了i-变异。
\end{enumerate}

\textbf{其他辅音词干}

还有一些词不太符合上述的词干划分。这些名词的特征是复数主格和宾格的-r词尾,-r有可能被前面的辅音同化。和上述的两类辅音词干一样,复数主格和宾格中同样出现了i-变异。这些词数量不多,但出现频次未必很低,读者可把它们当作不规则的名词记忆。

这类词包括一些常见的阳性名词:maðr `man', nagl `nail', mónuðr `month',
vetr `winter', fótr `foot':

\begin{longtable}{llllll}
  \toprule
       & maðr   & nagl   & mónuðr  & vetr   & fótr  \\
  \midrule
  \endhead
  \bottomrule
  \endfoot
  单数 &        &        &         &        &       \\
  N    & maðr   & nagl   & mónuðr  & vetr   & fótr  \\
  A    & mann   & nagl   & mónuð   & vetr   & fót   \\
  G    & manns  & nagls  & mánaðar & vetrar & fótar \\
  D    & manni  & nagli  & mónuð   & vetr   & fœti  \\
  复数 &        &        &         &        &       \\
  N    & menn   & negl   & mónuðr  & vetr   & fœtr  \\
  A    & menn   & negl   & mónuðr  & vetr   & fœtr  \\
  G    & manna  & nagla  & mánaða  & vetra  & fóta  \\
  D    & mǫnnum & nǫglum & mónuðum & vetrum & fótum \\
\end{longtable}

注意mónuðr有另一拼写mánaðr, fingr `finger‌'的变格与vetr一致。

类似地还有一些阴性名词: bók `book', tǫnn `tooth', nátt `night', kýr
`cow':

\begin{longtable}{lllll}
  \toprule
       & bók         & tǫnn        & nátt, nótt     & kýr \\
  \midrule
  \endhead
  \bottomrule
  \endfoot
  单数 &             &             &                &     \\
  N    & bók         & tǫnn        & nátt, nótt     & kýr \\
  A    & bók         & tǫnn        & nátt, nótt     & kú  \\
  G    & bókar, bœkr & tannar      & náttar, nætr   & kýr \\
  D    & bók         & tǫnn        & nátt, nótt     & kú  \\
  复数 &             &             &                &     \\
  N    & bœkr        & tennr, teðr & nætr           & kýr \\
  A    & bœkr        & tennr, teðr & nætr           & kýr \\
  G    & bóka        & tanna       & nátta          & kúa \\
  D    & bókum       & tǫnnum      & náttum, nóttum & kúm \\
\end{longtable}

\section{弱名词的变格法}\label{弱名词的变格法}

\subsection{弱名词的词尾}\label{弱名词的词尾}

弱名词的词尾相比强名词来说简单得多:

\begin{longtable}{llll}
  \toprule
  性   & 阳性 & 中性 & 阴性 \\
  \midrule
  \endhead
  \bottomrule
  \endfoot
  单数 &      &      &      \\
  N    & -i   & -a   & -a   \\
  A    & -a   & -a   & -u   \\
  G    & -a   & -a   & -u   \\
  D    & -a   & -a   & -u   \\
  复数 &      &      &      \\
  N    & -ar  & -u   & -ur  \\
  A    & -a   & -u   & -u   \\
  G    & -a   & -na  & -na  \\
  D    & -um  & -um  & -um  \\
\end{longtable}

上述的词尾中包含u的会规则地导致u-变异。但-i并不造成i-变异。

弱名词很大程度上也是日耳曼语的创新,它将另一部分无词干元音名词规则化了。弱名词的词干尾有一个辅音-n,在日耳曼语中又可进一步分为an-词干、ōn-词干和in-词干,类比阴阳中性的基本词干。

日耳曼语中的-n经常被重新理解为词尾的一部分,因此基本从词干上消失了,在古诺尔斯语中仅保留在部分词的复数属格中。弱名词的词尾有时也受到强名词的影响,因此推断弱名词原始的变化形式是比较困难的。不过,在古诺尔斯语中弱名词变化简单,应该不造成学习上的困难。另外,虽然本书介绍时仍旧给出了词干类型,但这只是一种分类的方法,事实上读者通过名词的性就基本已经可以推断其变格。

\subsection{an/jan-词干}\label{an/jan-词干}

这类词类比a-词干名词,同样都是阳性和中性的,参考下面的例词,阳性词bogi
`bow', bryti `steward', gumi `man', 中性词 hjarta `heart':

\begin{longtable}{lllll}
  \toprule
       & \multicolumn{3}{c}{阳性} & 中性                         \\
  \midrule
  \endhead
  \bottomrule
  \endfoot
  词干 & bog-                     & bryt-j- & gum-     & hjart-  \\
  单数 &                          &         &          &         \\
  N    & bogi                     & bryti   & gumi     & hjarta  \\
  A    & boga                     & brytja  & guma     & hjarta  \\
  G    & boga                     & brytja  & guma     & hjarta  \\
  D    & boga                     & brytja  & guma     & hjarta  \\
  复数 &                          &         &          &         \\
  N    & bogar                    & brytjar & gum(n)ar & hjǫrtu  \\
  A    & boga                     & brytja  & gum(n)a  & hjǫrtu  \\
  G    & boga                     & brytja  & gumna    & hjartna \\
  D    & bogum                    & brytjum & gum(n)um & hjǫrtum \\
\end{longtable}

说明:

\begin{enumerate}
  \def\labelenumi{\arabic{enumi})}
  \item
        gumi是一个比较古老的词,它的复数属格还是-na词尾,这反映了比较古老的an-词干的形式。以这个-n-为基础,复数的其他形式发生了类推,因此也有包含-n-的词形。其他词形都是规则的。
  \item
        bryti是所谓的jan-词干名词,除单数主格外其他形式中还规则地插入了-j-,但后来这些半元音也都消失了,使得它和其他an-词干名词没有任何区别。这类词数量很少,一些表示人的后缀-ingi,-virki按这种方式变格。
\end{enumerate}

\subsection{ōn/jōn -词干}\label{ōn/jōn -词干}

这类词类比ō-词干名词,它们都是阴性的。和上面的an/jan-词干一样,这类词中也有含有-j-的子类,参考下面的例词:saga
`tale', stjarna `star', ásjá `help', smiðja `smithy':

\begin{longtable}{lllll}
  \toprule
       & \multicolumn{4}{c}{阴性}                              \\
  \midrule
  \endhead
  \bottomrule
  \endfoot
  词干 & sag-                     & stjarn-  & ásjá- & smið-j- \\
  单数 &                          &          &       &         \\
  N    & saga                     & stjarna  & ásjá  & smiðja  \\
  A    & sǫgu                     & stjǫrnu  & ásjá  & smiðju  \\
  G    & sǫgu                     & stjǫrnu  & ásjá  & smiðju  \\
  D    & sǫgu                     & stjǫrnu  & ásjá  & smiðju  \\
  复数 &                          &          &       &         \\
  N    & sǫgur                    & stjǫrnur & ásjár & smiðjur \\
  A    & sǫgur                    & stjǫrnur & ásjár & smiðjur \\
  G    & sagna                    & stjarna  & ásjá  & smiðja  \\
  D    & sǫgum                    & stjǫrnum & ásjám & smiðjum \\
\end{longtable}

说明:

\begin{enumerate}
  \def\labelenumi{\arabic{enumi})}
  \item
        ásjá来自于á `on' + sjá `watch',有一些以介词 +
        sjá的名词都按这种方式变格,它们表达``以某种特定方式看''的含义。除了ásjá外还有:
\end{enumerate}

\begin{quote}
  eptir-sjá \emph{stare-watch} `the looking with desire after a lost
  thing, hence loss, grief'

  aptr-sjá \emph{after-watch} `regret'

  for-sjá \emph{before-watch} `foresight'

  ásjá 也可像á一样变格,这样单数属格就是ásjár,其余形式没有区别。
\end{quote}

\begin{enumerate}
  \def\labelenumi{\arabic{enumi})}
  \setcounter{enumi}{1}
  \item
        一个常见的ōn-词干名词kona `woman'的复数属格是kvenna.
        后者实际上是这个词更古老的形式的保留(PGmc.
        *kwenǭ)。造成这个现象的原因是辅音后的w脱落常常导致e/a \textgreater{}
        o的音变。类似的例子在tolf `twelve' \textless{} *twalif中也有体现。
\end{enumerate}

\subsection{in-词干}\label{in-词干}

in-词干均为阴性,大都是概念性的抽象名词。它的词尾和正常阴性词尾略有不同,一切单数形式都以-i结尾,这就相当于单数不变格。抽象名词少有复数形式,如果有,就采取ō-词干的词尾,常见的in-词干名词有elli
`old age' 以及lygi `life':

\begin{longtable}{lll}
  \toprule
       & \multicolumn{2}{c}{阴性}         \\
  \midrule
  \endhead
  \bottomrule
  \endfoot
  词干 & elli-                    & lygi- \\
  单数 &                          &       \\
  N    & elli                     & lygi  \\
  A    & elli                     & lygi  \\
  G    & elli                     & lygi  \\
  D    & elli                     & lygi  \\
  复数 &                          &       \\
  N    & -                        & lygar \\
  A    & -                        & lygar \\
  G    & -                        & lyga  \\
  D    & -                        & lygum \\
\end{longtable}

\section{定冠词和特指后缀}\label{定冠词和特指后缀}

类似于英语的定冠词the,古诺尔斯语也有一个定冠词inn。inn既可以像the一样添加在被修饰的名词前面,但它也可以后缀添加在名词后面。这种-inn后缀就称为名词的特指后缀。

类似于希腊语,古诺尔斯语的定冠词inn也根据性、数、格发生变化,在单独使用时,inn的变格如下所示:

\begin{longtable}{llll}
  \toprule
       & 阳性 & 阴性  & 中性 \\
  \midrule
  \endhead
  \bottomrule
  \endfoot
  单数 &      &       &      \\
  N    & inn  & in    & it   \\
  A    & inn  & ina   & it   \\
  G    & ins  & innar & ins  \\
  D    & inum & inni  & inu  \\
  复数 &      &       &      \\
  N    & inir & inar  & in   \\
  A    & ina  & inar  & in   \\
  G    & inna & inna  & inna \\
  D    & inum & inum  & inum \\
\end{longtable}

当定冠词作为特指后缀出现时,-inn的形式会根据名词的词尾发生些许变化:

\begin{longtable}{l}
  \toprule
  \begin{enumerate}\def\labelenumi{\arabic{enumi})}\item  如果-inn前是一个非重读短元音,-i-一定会脱落(见\ref{元音的音变}规则2)。\item  对于双音节的-inn的变格(如inna,inum),-i-还在下列两种情况下脱落:  \begin{enumerate}  \def\labelenumii{\roman{enumii}.}  \item    i紧跟着一个长元音,如á-nni.  \item    只有一个n的双音节形式(如inum,inar)在辅音后失去i,如skildir-nir,    但阴性单数宾格的-ina除外(见\ref{元音的音变}规则1)。  \end{enumerate}\item  对于单音节的形式,-i-在长元音后保留,如kné-in。\end{enumerate} \\
  \midrule
  \endhead
  \bottomrule
  \endfoot
\end{longtable}

-inn还对格词尾有一定影响。当加后缀-inum时,复数与格的尾音-m脱落。因此有这样的音变:-um
+ -inum \textgreater{} -u + -inum \textgreater{} -u -num。

我们用几个强名词示例:

\begin{longtable}{llll}
  \toprule
       & 阳性      & 阴性         & 中性     \\
  \midrule
  \endhead
  \bottomrule
  \endfoot
  单数 &           &              &          \\
  N    & úlfr-inn  & gjǫf-in      & tré-it   \\
  A    & úlf-inn   & gjǫf-ina     & tré-it   \\
  G    & úlfs-ins  & gjafar-innar & trés-ins \\
  D    & úlfi-num  & gjǫf-inni    & tré-nu   \\
  复数 &           &              &          \\
  N    & úlfar-nir & gjafar-nar   & tré-in   \\
  A    & úlfa-na   & gjafar-nar   & tré-in   \\
  G    & úlfa-nna  & gjafa-nna    & trjá-nna \\
  D    & úlfu-num  & gjǫfu-num    & trjá-num \\
\end{longtable}

一些弱名词示例如下,弱名词词尾的元音造成了大量的缩合:

\begin{longtable}{llll}
  \toprule
       & 阳性      & 阴性       & 中性      \\
  \midrule
  \endhead
  \bottomrule
  \endfoot
  单数 &           &            &           \\
  N    & bogi-nn   & kona-n     & auga-t    \\
  A    & boga-nn   & konu-na    & auga-t    \\
  G    & boga-ns   & konu-nnar  & auga-ns   \\
  D    & boga-num  & konu-nni   & augu-nu   \\
  复数 &           &            &           \\
  N    & bogar-nir & konur-nar  & augu-n    \\
  A    & boga-na   & konur-nar  & augu-n    \\
  G    & boga-nna  & kvenna-nna & augna-nna \\
  D    & bogu-num  & konu-num   & augu-num  \\
\end{longtable}

古诺尔斯语和英语对于定冠词(或特指后缀)的用法有很大共性但不完全一致。总的来说古诺尔斯语使用特指的情况比英语要少一些,最显著的一点是在描述读者或作者熟悉的事物时,不需要使用特指标记,特别是例如konungr
`king', dróttning
`queen'这样的名词。另外,属格结构一般也不用加特指后缀,但它表示的也是特指含义。

定冠词或特指后缀有如下的用法:

\begin{enumerate}
  \def\labelenumi{\Alph{enumi}.}
  \item
        定冠词单独使用时:
\end{enumerate}

\begin{enumerate}
  \def\labelenumi{\arabic{enumi})}
  \item
        置于名词或形容词前,类似于英语:
\end{enumerate}

\begin{quote}
  inn maðr `the man'

  inn blindi maðr `the blind man'
\end{quote}

\begin{enumerate}
  \def\labelenumi{\arabic{enumi})}
  \setcounter{enumi}{1}
  \item
        有时也置于副词、形容词比较级之前。在比较级时只用in:
\end{enumerate}

\begin{quote}
  it sama `the same, likewise'

  in heldr `the more'
\end{quote}

\begin{enumerate}
  \def\labelenumi{\arabic{enumi})}
  \setcounter{enumi}{2}
  \item
        置于名词/代词和形容词之间:
\end{enumerate}

\begin{quote}
  专有名词+称谓:Óláfr inn helgi `Olaf the saint'

  名词+限定性形容词:hendi inn hœgri `hand the right \textgreater{} the
  right hand'

  人称代词+形容词:þú hinn armi `you the wicked \textgreater{} thou
  wretch!'
\end{quote}

\begin{enumerate}
  \def\labelenumi{\arabic{enumi})}
  \setcounter{enumi}{3}
  \item
        置于指示代词和名词短语之间:
\end{enumerate}

\begin{quote}
  sá inn blindi maðr `that the blind man \textgreater{} the/this/that
  blind man'

  这种说法在英语中不可行。另外,还可以改成类似于用法3的说法:

  maðr sá inn blindi / sá maðr inn blindi
\end{quote}

\begin{enumerate}
  \def\labelenumi{\Alph{enumi}.}
  \setcounter{enumi}{1}
  \item
        作为特指后缀添加在名词后:
\end{enumerate}

\begin{enumerate}
  \def\labelenumi{\arabic{enumi})}
  \item
        缺少形容词时,通常加特指后缀:
\end{enumerate}

\begin{quote}
  maðrinn `the man'
\end{quote}

\begin{enumerate}
  \def\labelenumi{\arabic{enumi})}
  \setcounter{enumi}{1}
  \item
        后接物主代词时,前面的名词常用特指后缀表示强烈语气;有时不接物主代词单独使用特指后缀也有这样的作用:
\end{enumerate}

\begin{quote}
  ástin / ástin mín `O my love!'
\end{quote}

\begin{enumerate}
  \def\labelenumi{\arabic{enumi})}
  \setcounter{enumi}{2}
  \item
        特指后缀后,依然可以用其他指示词:
\end{enumerate}

\begin{quote}
  hǫndin sú hœgri `hand-the that right \textgreater{} the right hand'
\end{quote}

\chapter{动词与变位法}\label{动词与变位法}

\begin{quote}
  \textbf{章节要点:}
\end{quote}

\begin{itemize}
  \item
        \begin{quote}
          古诺尔斯语的动词系统
        \end{quote}
  \item
        \begin{quote}
          动词的时态、体态
        \end{quote}
  \item
        \begin{quote}
          动词的分类
        \end{quote}
  \item
        \begin{quote}
          各类动词的变位法
        \end{quote}
  \item
        \begin{quote}
          不规则动词
        \end{quote}
\end{itemize}

\section{动词的概述}\label{动词的概述}

和我们已经介绍过的名词一样,动词有强和弱两种类型。不过需要注意,动词的强弱和名词的强弱指的不是同一个东西。简单来说,强动词可以理解为以元音交替(Ablaut)为特征的``不规则动词''(不规则动词是借用英语的概念来说的,通过下面的内容可以知道,这些变化实际上是规则的);对应地,弱动词可以类比为英语的规则动词。和名词类似,动词的强弱也只有类型学上的意义,它们完全可以用``I型动词''和``Ⅱ型动词''这样的术语来代替。

我们在这里简单讲解一下元音交替的来历。元音交替是原始印欧语的重要特征,它可以用于同根词的进一步变形。元音交替指的是词干中的元音(可以理解为词根元音)在不同词形下发生变换的现象,请注意这指的不是词干元音(Thematic
vowel)的变化。在名词中,元音交替可以区分变格;在动词中它则可以区分变位。参考古希腊文中πατήρ
`father'的元音交替:

\begin{longtable}{llll}
  \toprule
  古希腊语          &                   & 格       & 元音   \\
  \midrule
  \endhead
  \bottomrule
  \endfoot
  πα-τ\textbf{έ}ρ-α & pa-t\textbf{é}r-a & 单数宾格 & 短e    \\
  πα-τ\textbf{ή}ρ   & pa-t\textbf{ḗ}r   & 单数主格 & 长e    \\
  πα-τρ-ός          & pa-tr-ós          & 单数属格 & 无元音 \\
\end{longtable}

在动词中,元音交替的现象完好地保留在梵语等更早的语言(梵语成文时期比古诺尔斯语早十余个世纪)中。在英语中,能体现元音交替是一些所谓的不规则动词:

\begin{quote}
  r\textbf{i}de r\textbf{o}de r\textbf{i}dden

  s\textbf{i}ng s\textbf{a}ng s\textbf{u}ng

  fl\textbf{y} fl\textbf{ew} fl\textbf{o}wn
\end{quote}

但元音交替作为一种非常古老的现象,在原始日耳曼语从原始印欧语分离出来的时候,这种构词方法已经逐渐被舍弃了,因此原始日耳曼语采用了新的方法来衍生动词,这就是``弱变化''规则(英文对应-ed式规则动词)。继承于原始印欧语的更古老的动词仍然保留了强变化规则,不过随着时间的推移,强变化愈发被人遗忘,许多历史上的强动词也归入弱动词之中。

古诺尔斯语有两个时态(Tense):现在(Present)和过去(Preterite)。读者最好把二者理解为过去和非过去,因为简单来说,现在时既支配现在的动作也支配未来的动作。相同地,古诺尔斯语的过去式同样可以和英语的若干与过去有关的时态有关。有英语基础的读者会经常联想到英语中复杂的时态系统,但事实上,这些表达严格来说叫作``体''或``体貌''(Aspect),例如英语中的完成时实际上是一种体,而非时态。时态用来区分动词在时间尺度上的位置;体貌则描述关于该动作的开始、持续、完成或重复等方面的情况,但不涉及该动作发生的时间。当然,许多动词的表达既涉及时态又涉及体貌,例如英语中的现在完成时描述了完成体,但隐含的意思是一个现在的状态。因此在许多印欧语中,这个时态(暂且仍粗略地使用这个不太准确的术语)采取动词现在时的词干并加上一些派生后缀。在古诺尔斯语中,表达体态的方法是添加助动词,而非词形屈折。

古诺尔斯语有两个语态(Voice):主动(Active)和中动(Middle/Mediopassive)。中动态在古诺尔斯语中一方面有一定被动的含义,另一方面还表达一些反身的动作,详见\ref{强动词的中动词尾}。

古诺尔斯语有三种语气(Mood):直陈(Indicative),虚拟(Subjunctive),祈使(Imperative)。直陈语气表示一般地陈述;虚拟语气主要表示可能发生但尚未发生的动作或愿望;祈使语气表示命令。

动词除了根据时态、语态、语气发生变形外,也根据人称变化。古诺尔斯有三个人称,但祈使语气只有第一人称和第二人称的形式。

\section{强动词的变位法}\label{强动词的变位法}

强动词的特征是元音交替。如\ref{动词的概述}所述,元音的交替仍然保留在英语中,例如sing-sang-sung-song.
词干部分s-ng 加上i得现在时,加a得过去式,加u得过去分词,
加o得衍生的名词。但不是每个动词的元音交替模式都是一致的,比如hang-hung-hung,它不仅采取不同的元音添加,也没有对应的衍生名词。

要了解词形的变化,必须首先搞清楚古诺尔斯语动词的最基本形式:不定式。不定式在语法上属于非限定动词的一种,即这种动词还没有人称、时态等的限定,但我们在这一章谈到的不定式只是动词词形的一种形式,它最简单,未经过变形,根据动词不定式能推导出动词的其他形式。有时情况下,也把这种形式称为词典形(Dictionary
form),即词典上提供的基本形态。古诺尔斯语的不定式可以和以下几种语言的``不定式''\footnote{有许多语言的词典形就是属于非限定动词的不定式,但还有一些语言的词典形是限定动词,例如古希腊语常用第一人称单数式作为词典形。}类比:英语eat,拉丁语portare,德语liben,日语考える。动词变化丰富的语言基本都有不定式的标记,用蓝色字体标出,蓝色之外的部分可以认为是词干。就古诺尔斯语而言,不定式的标记是-a,也有时候-a前紧跟着-j-或者-v-,称之为ja/va-不定式\footnote{j/v事实上都是词干的一部分,即不定式标记总是-a,-a之前的部分为词干。}。和ja-/jō-词干一样,j的出现有时会引起进一步的音变(西弗斯定律的作用,见\ref{a/ja/wa-词干}),但va-不定式一般比较规则。进一步来讲,古诺尔斯语的不定式也和英语一样有两种,第一种是单独的以-a结尾的词典形,第二种是以at+词典形引导的at不定式。这两种不同的不定式都属于非限定动词,它们的语法功能和英文中带不带to的不定式类似,都可以作某些动词的补足语等。

根据强动词元音交替的模式(例如上面的i-a-u和a-u-u),可以将强动词分为七类。前六类动词比较规则,第七类动词则是一些历史上不太规则的动词的残留。为了明确元音交替的模式,需要选择动词的一些形式作为基本元。请先参考下面六类动词的一些变形,bíta
`bite', skjóta `shoot', bresta `burst', bera `bear, carry', reka
`drive‌', fara `go/fare‌':

\begin{longtable}{lllllll}
  \toprule
  类  & 不定式 & 三单现在时 & 三单过去时 & 三复过去时 & 三单过去虚拟式 & 过去分词 \\
  \midrule
  \endhead
  \bottomrule
  \endfoot
  I   & bíta   & bítr       & beit       & bitu       & biti           & bitinn   \\
  II  & skjóta & skýtr      & skaut      & skutu      & skyti          & skotinn  \\
  III & bresta & brestr     & brast      & brustu     & brysti         & brostinn \\
  IV  & bera   & berr       & bar        & báru       & bæri           & borinn   \\
  V   & reka   & rekr       & rak        & ráku       & ræki           & rekinn   \\
  VI  & fara   & ferr       & fór        & fóru       & fœri           & farinn   \\
\end{longtable}

通过上表可以发现:

\begin{longtable}{l}
  \toprule
  \begin{enumerate}\def\labelenumi{\arabic{enumi})}\item  单数现在时的元音要么和不定式一致,要么由它发生i-变异得到。\item  单数过去式的元音是独立的。\item  复数过去式的元音是独立的。\item  过去虚拟式的元音由直陈复数过去式的元音i-变异得到。\item  过去分词的元音是独立的。\end{enumerate} \\
  \midrule
  \endhead
  \bottomrule
  \endfoot
\end{longtable}

因此,古诺尔斯语的强动词系统中有四种元音的交替,那么,最少用四个动词形式即可确定整个变位表的形式,这四个形式称为四个基本元。这四个基本元在词典上一般选用:

\begin{quote}
  第一基本元:不定式,或称词典形,词典的索引;

  第二基本元:第三人称单数过去式;

  第三基本元:第三人称复数过去式;

  第四基本元:过去分词。
\end{quote}

有时词典中也标记第三人称单数现在时。例如bera在词典上就记为:

\begin{quote}
  Cleasby \& Vigfússon\footnote{Richard Cleasby and Guðbrandur Vigfússon
    (1874), An Icelandic-English Dictionary}:\emph{BERA, bar, báru,
    borit, pres. berr}
\end{quote}

\subsection{强动词的主动词尾}\label{强动词的主动词尾}

古诺尔斯语的动词分为主动词尾和中动词尾,主动态动词添加主动词尾,其含义和英文的主动态没有区别。强动词的主动词尾如下所示:

\begin{longtable}{llll}
  \toprule
  强动词     & 直陈 & 虚拟 & 祈使 \\
  \midrule
  \endhead
  \bottomrule
  \endfoot
  单数现在时 &      &      &      \\
  1          & -ø   & -a   &      \\
  2          & -r   & -ir  & -ø   \\
  3          & -r   & -i   &      \\
  复数现在时 &      &      &      \\
  1          & -um  & -im  & -um  \\
  2          & -ið  & -ið  & -ið  \\
  3          & -a   & -i   &      \\
  单数过去时 &      &      &      \\
  1          & -ø   & -a   &      \\
  2          & -t   & -ir  &      \\
  3          & -ø   & -i   &      \\
  复数过去时 &      &      &      \\
  1          & -um  & -im  &      \\
  2          & -uð  & -ið  &      \\
  3          & -u   & -i   &      \\
\end{longtable}

-ø表示无须添加词尾,表格中的空白指这种形式不存在。这完全体现在动词的祈使式上,只有单复数的现在时第二人称、复数现在时第一人称有祈使式。即祈使式可以用在下面两种情况下:

\begin{enumerate}
  \def\labelenumi{\arabic{enumi})}
  \item
        要求你、你们做,参考英语 `do it!';
  \item
        提议我们做,参考英语 `let's do it!'.
\end{enumerate}

在不同时态、数的词尾前,要选取不同的基本元,其变化方式如下:

\begin{longtable}{l}
  \toprule
  单数现在直陈式: 取不定式词干,如果词干有后元音,则施加i-变异,加词尾; \\
  \midrule
  \endhead
  \bottomrule
  \endfoot
  复数现在直陈式/一切现在虚拟式: 取不定式词干,加词尾;                  \\
  单数过去直陈式: 取第二基本元词干,加词尾;                             \\
  复数过去直陈式: 取第三基本元词干,加词尾;                             \\
  一切过去虚拟式: 取第三基本元词干,如果词干有后元音,则施加i-变异。
  加词尾;                                                               \\
  一切祈使式: 取不定式词干,加词尾;                                     \\
  现在分词: 取不定式词干,加词尾;                                       \\
  过去分词: 取第四基本元词干,加词尾。                                   \\
\end{longtable}

注意,某些情况下,非圆唇元音受-um的影响有可能会发生u-变异,这是规则音变的结果。用
(1), (2), (3), (4)分别标记四个基本元,变形方式如下:

\begin{longtable}{llll}
  \toprule
  强动词     & 直陈              & 虚拟               & 祈使              \\
  \midrule
  \endhead
  \bottomrule
  \endfoot
  单数现在时 &                   &                    &                   \\
  1          & (1) + (i-变异) -ø & (1) +-a            &                   \\
  2          & (1) + (i-变异) -r & (1) +-ir           & (1) +-ø           \\
  3          & (1) + (i-变异) -r & (1) +-i            &                   \\
  复数现在时 &                   &                    &                   \\
  1          & (1) +(u-变异) -um & (1) +-im           & (1) +(u-变异) -um \\
  2          & (1) +-ið          & (1) +-ið           & (1) +-ið          \\
  3          & (1) +-a           & (1) +-i            &                   \\
  单数过去时 &                   &                    &                   \\
  1          & (2) +-ø           & (3) + (i-变异) -a  &                   \\
  2          & (2) +-t           & (3) + (i-变异) -ir &                   \\
  3          & (2) +-ø           & (3) + (i-变异) -i  &                   \\
  复数过去时 &                   &                    &                   \\
  1          & (3) +-um          & (3) + (i-变异) -im &                   \\
  2          & (3) +-uð          & (3) + (i-变异) -ið &                   \\
  3          & (3) +-u           & (3) + (i-变异) -i  &                   \\
  不定式     & (1) + -a          &                    &                   \\
  现在分词   & (1) + -andi       &                    &                   \\
  过去分词   & (4) + -inn        &                    &                   \\
\end{longtable}

说明:

\begin{enumerate}
  \def\labelenumi{\arabic{enumi})}
  \item
        现在时的单数直陈式中普遍地出现了i-变异,但没有任何i的痕迹。在其他西日耳曼语中,只有第二人称和第三人称出现了i-变异,例如古英语的单数现在时bēode-bīetst-
        bīetst,这表明古诺尔斯语中第一人称的i-变异是类推的影响。从中动态(见\ref{强动词的中动词尾})的词尾来看,第一人称的词尾本是*u.
        第二人称的-r由-ir演变而来,第三人称的-r也由第二人称类推得到,在卢恩铭文中记载到了早期的-iþ形式。
  \item
        绝大多数情况下(除了过去虚拟式)词尾-ið不造成i-变异。在原始日耳曼语中,这个词尾已经是*-id了,其不能造成i-变异应当是受同时态的其他词形的影响。
  \item
        \phantomsection\label{_Ref116919964}{}虚拟式的词尾都是一样的,但现在虚拟式和过去虚拟式由词干的i-变异所区分。在原始日耳曼语中,现在时的词尾包含一个双元音*ai(相当于比过去式的词尾多了*-a-),双元音发生缩略变为*e,后来抬升为i.
        这样,例如第二人称现在时词尾*-aiz \textgreater{} *-ez \textgreater{}
        -ir就没有i-变异条件。在过去式中则不存在这样的双元音,i-变异正常发生。无疑,类推作用使得过去式中某些本不能造成i-变异的词尾也引起了元音变异,这样才能与现在时区分开来。
  \item
        -um词尾比较规则地引起u-变异,这和名词中的情况类似。复数过去直陈式的词尾都含有-u,理论上都能造成u-变异,但是古诺尔斯语中保留下来的复数过去式词干没有可以发生u-变异的元音。
\end{enumerate}

\subsection{强动词的中动词尾}\label{强动词的中动词尾}

动词的语态的概念涉及给定陈述中施事和受事发挥作用的方式。施事是一个动作的逻辑执行者;受事是动作的逻辑接受者,或其对象。施事可能是但也可能不是其分句的语法主语,同样地,受事也可能是也可能不是其分句的直接宾语。当述语处于主动态时,施事也是语法主语;在这种情况下,如果存在受事,它就是直接宾语。例如``狗咬人''一句中,狗既是行动的逻辑执行者,也即施事,同时也是动词``咬''的语法主语。因此,这个表述是主动的;受事``人'',是直接宾语。当述语处于被动态时,受事成为语法主语。同样的句子可以改写为``人被狗咬''。动作的逻辑执行者``狗''仍然是施事。但它不再是语法主语了;而受事``人'',是动词``被咬''的语法主语,使之成为被动陈述。

中动态,正如其名字暗示的那样,处于主动和被动之间,是一种很难准确定义的语态。中动态表达的动作是对施事产生某种影响的,比如简单的反身动作(I
washed myself‌),影响个人利益的动作(I had a sacrifice
performed),或是内在变化(I called to mind what he
said)还有一些其它差异不大的表达。在古诺尔斯语中,中动态也有被动态的含义。

中动态的表达最初是由主动态增加人称代词。第一人称单数添加mik `me',
其他人称都添加反身代词sik,参见\ref{人称代词}。后来这个表达固定成了词尾,其形式一般就是在主动态后加-sk,第一人称单数则是加-mk,同时进行一些语音变化。比较明显的有:

\begin{enumerate}
  \def\labelenumi{\arabic{enumi})}
  \item
        -r + -sk \textgreater{} *-ssk \textgreater-sk
  \item
        -t + -sk \textgreater{} -zk
  \item
        -ð + -sk \textgreater{} -zk
\end{enumerate}

\begin{longtable}{llll}
  \toprule
  强动词     & 直陈  & 虚拟  & 祈使  \\
  \midrule
  \endhead
  \bottomrule
  \endfoot
  单数现在时 &       &       &       \\
  1          & -umk  & -umk  &       \\
  2          & -sk   & -isk  & -sk   \\
  3          & -sk   & -isk  &       \\
  复数现在时 &       &       &       \\
  1          & -umsk & -imsk & -umsk \\
  2          & -izk  & -izk  & -izk  \\
  3          & -ask  & -isk  &       \\
  单数过去时 &       &       &       \\
  1          & -umk  & -umk  &       \\
  2          & -zk   & -isk  &       \\
  3          & -sk   & -isk  &       \\
  复数过去时 &       &       &       \\
  1          & -umsk & -imsk &       \\
  2          & -uzk  & -izk  &       \\
  3          & -usk  & -isk  &       \\
\end{longtable}

说明:

\begin{enumerate}
  \def\labelenumi{\arabic{enumi})}
  \item
        单数第一人称的词尾-umk反应了原始语直陈式的元音词尾*-u,而在主动态中,-u已经脱落。虚拟式的-umk可能是从直陈式借来的。
  \item
        *-rsk \textgreater{}
        -sk的变化实则有一点可疑,因为s在r之后本不会发生同化音变,例如sumar
        `summer'的属格sumars中r和s不同化。
\end{enumerate}

-umk可能会导致前面的前元音发生u-变异,加上词干后,其变化表如下:

\begin{longtable}{llll}
  \toprule
  强动词     & 直陈                  & 虚拟                 & 祈使                  \\
  \midrule
  \endhead
  \bottomrule
  \endfoot
  单数现在时 &                       &                      &                       \\
  1          & (1) + (u-变异) -umk   & (1) + (u-变异) -umk  &                       \\
  2          & (1) + (i-变异) -sk    & (1) +-isk            & (1) +-sk              \\
  3          & (1) + (i-变异) -sk    & (1) +-isk            &                       \\
  复数现在时 &                       &                      &                       \\
  1          & (1) +(u-变异) + -umsk & (1) +-imsk           & (1) +(u-变异) + -umsk \\
  2          & (1) +-izk             & (1) +-izk            & (1) +-izk             \\
  3          & (1) +-ask             & (1) +-isk            &                       \\
  单数过去时 &                       &                      &                       \\
  1          & (3) + (u-变异) -umk   & (3) + (i-变异) -umk  &                       \\
  2          & (2) +-zk              & (3) + (i-变异) -isk  &                       \\
  3          & (2) +-sk              & (3) + (i-变异) -isk  &                       \\
  复数过去时 &                       &                      &                       \\
  1          & (3) + (u-变异) + -um  & (3) + (i-变异) -imsk &                       \\
  2          & (3) +-uzk             & (3) + (i-变异) -izk  &                       \\
  3          & (3) +-usk             & (3) + (i-变异) -isk  &                       \\
  不定式     & (1) + -ask            &                      &                       \\
  现在分词   & (1) + -andisk         &                      &                       \\
  过去分词   & (4) + -izk            &                      &                       \\
\end{longtable}

说明:

\begin{enumerate}
  \def\labelenumi{\arabic{enumi})}
  \item
        \phantomsection\label{_Ref117719619}{}第一人称单数直陈过去式用了复数词干。造成这个现象的原因是对词尾成分的误读和重解。由于第一人称单数在主动直陈式中没有词尾,在主动虚拟式中只有词尾-a,中动态词尾-umk可能被重新理解为-um
        +
        -k,而-um是复数式第一人称的词尾(无论是现在时还是过去时),因此所有的单数第一人称的词干全部采用了复数形式。在现在时中,单复数都采用了同一种词干,因此无法看出任何异常,而在过去时中,单数词干和复数词干不一致,就出现了单数中两种词干的混用。
  \item
        第一人称单数虚拟过去式的-umk没有造成u-变异,反而造成了i-变异。读者应该不难想到,这个i-变异是从其他过去时类推的结果,用以和现在时进行区分。否则,第一人称的现在时和过去时将无法区分。
\end{enumerate}

\subsection{第一强变位法}\label{第一强变位法}

古诺尔斯语的每一种强变位法都有自己的特征元音交替模式,第一强变位法的动词最基本的元音交替特征如下:

\textbf{í -\/- ei -- i -\/- i}

少数不定式词干以-g结尾的一类强动词hníga `fall' , míga `piss', síga
`sink', stíga
`step'的第一或第三人称单数过去式还有另一种形式。这时,元音í变为长元音é而不是ei,同时词干结尾的g脱去\footnote{造成词尾-g脱落的原因是浊塞音g在原始诺尔斯语的词尾变成其音位变体h,但后来又系统地发生了词尾-h的脱落,如古诺尔斯语þó(比较Go.
  þau\textbf{h}; OE þēa\textbf{h};
  现代英语thou\textbf{gh}),因此例如sté这样的形式是更早期的词形的反映。ei
  \textgreater{}
  é的音变是一种元音缩合,但主要只出现在这类动词变形中,原始诺尔斯语还有大量的ai
  \textgreater{} á的缩合(ON. sár vs Go. sair
  `wound'),因此这个音变并不奇怪。}.:

\begin{quote}
  stíga \textgreater{} steig, steigt, steig或sté, stétt, sté
\end{quote}

注意这里变化得到的长元音加词尾-t时会触发\ref{辅音的音变}中的辅音延长音变,因此第二人称单数直陈过去式为stétt而不是†stét.

我们以一类强动词bíta
`bite'为例,展示其完整的变位。词典中会给出其四个基本元bíta -\/- beit
-\/- bitu -- bitinn.

主动态:

\begin{longtable}{llll}
  \toprule
  一类强动词 & 直陈    & 虚拟  & 祈使  \\
  \midrule
  \endhead
  \bottomrule
  \endfoot
  单数现在时 &         &       &       \\
  1          & bít     & bíta  &       \\
  2          & bítr    & bítir & bít   \\
  3          & bítr    & bíti  &       \\
  复数现在时 &         &       &       \\
  1          & bítum   & bítim & bítum \\
  2          & bítið   & bítið & bítið \\
  3          & bíta    & bíti  &       \\
  单数过去时 &         &       &       \\
  1          & beit    & bita  &       \\
  2          & beizt   & bitir &       \\
  3          & beit    & biti  &       \\
  复数过去时 &         &       &       \\
  1          & bitum   & bitim &       \\
  2          & bituð   & bitið &       \\
  3          & bitu    & biti  &       \\
  不定式     & bíta    &       &       \\
  现在分词   & bítandi &       &       \\
  过去分词   & bitinn  &       &       \\
\end{longtable}

中动态:

\begin{longtable}{llll}
  \toprule
  一类强动词 & 直陈      & 虚拟    & 祈使    \\
  \midrule
  \endhead
  \bottomrule
  \endfoot
  单数现在时 &           &         &         \\
  1          & bítumk    & bítumk  &         \\
  2          & bízk      & bítisk  & bízk    \\
  3          & bízk      & bítisk  &         \\
  复数现在时 &           &         &         \\
  1          & bítumsk   & bítimsk & bítumsk \\
  2          & bítizk    & bítizk  & bítizk  \\
  3          & bítask    & bítisk  &         \\
  单数过去时 &           &         &         \\
  1          & bitumk    & bitumk  &         \\
  2          & beizk     & bitisk  &         \\
  3          & beizk     & bitisk  &         \\
  复数过去时 &           &         &         \\
  1          & bitumsk   & bitimsk &         \\
  2          & bituzk    & bitizk  &         \\
  3          & bitusk    & bitisk  &         \\
  不定式     & bítask    &         &         \\
  现在分词   & bítandisk &         &         \\
  过去分词   & bitizk    &         &         \\
\end{longtable}

说明:

\begin{enumerate}
  \def\labelenumi{\arabic{enumi})}
  \item
        以-t为词干末尾的动词接续-t词尾时,发生*-tt \textgreater{} *-tst
        \textgreater{}
        -zt,这是古老的日耳曼语擦音定律的残留。中动态-zk发生了类似的音变。这种现象十分常见,在后续出现时,不做过多说明。
\end{enumerate}

\subsection{第二强变位法}\label{第二强变位法}

第二变位法的不定式词干的特征元音是双元音jú,其元音交替模式如下:

\textbf{jú, jó(ý) -\/- au -\/- u(y) -\/- o}

其中,用圆括号标记的是i-变异下的词根元音音变后的形式。

当跟随词根元音的辅音是齿音(ð, d, s,
t)时,jú分割成jó,例如下列动词brjóta `break‌', ljósta `smite‌', skjóta
`shoot‌', bjóða `offer‌', þrjóta `come to an end‌', kjósa
`choose‌',由于双元音jú,
jó都是后元音,再加某些词尾前需要将其i-变异。jú首先变成*jý,进一步变成ý,这是由于j在y前规则地脱去,我们已经在\ref{半元音的保持性}提到了。

在一类强动词中提到的词尾-g脱落现象在二类强动词中仍然存在,这种情况下,对应的元音变成ó。例如动词fljúga
`fly‌'的词干fljúg-在单数过去式中有fló-和flaug-两种形式,类似地,smjúg-
`creep‌'
有smó-和smaug-两种形式,如\ref{第一强变位法}所讲的一样,这里的长元音也会触发词尾的辅音延长音变。

二类动词中有以下一些不太规则的动词:

\begin{enumerate}
  \def\labelenumi{\arabic{enumi})}
  \item
        以咝音s结尾的动词frjósa `freeze‌'和kjósa
        `choose‌'的过去式可以按照弱变位法(见\ref{第二弱变位法})变位,词干分别是frør-和kór-,出现在东部的冰岛方言中。例如单数过去式frøra,
        frørir, frøri,这里没有塞音的痕迹。
  \item
        三个动词lúka `finish‌', súpa `sip‌', lúta `bow‌'词干中没有j.\footnote{这些动词来源古老,仅在印度语支和日耳曼语中有所保留。简单来说,它们的现在时词干和正常的东西有所区别(因此在古诺尔斯语中反映为元音的不同),以表达某些一次性事件。介绍这些动词的元音交替需要大量PIE知识,读者可自查所谓的tudáti-type(命名来自梵语).}
\end{enumerate}

因此,我们选用规则的动词skjóta来展示二类动词的强变位法。

主动态:

\begin{longtable}{llll}
  \toprule
  二类强动词 & 直陈      & 虚拟    & 祈使    \\
  \midrule
  \endhead
  \bottomrule
  \endfoot
  单数现在时 &           &         &         \\
  1          & skýt      & skjóta  &         \\
  2          & skýtr     & skjótir & skjót   \\
  3          & skýtr     & skjóti  &         \\
  复数现在时 &           &         &         \\
  1          & skjótum   & skjótim & skjótum \\
  2          & skjótið   & skjótið & skjótið \\
  3          & skjóta    & skjóti  &         \\
  单数过去时 &           &         &         \\
  1          & skaut     & skyta   &         \\
  2          & skauzt    & skytir  &         \\
  3          & skaut     & skyti   &         \\
  复数过去时 &           &         &         \\
  1          & skutum    & skytim  &         \\
  2          & skutuð    & skytið  &         \\
  3          & skutu     & skyti   &         \\
  不定式     & skjóta    &         &         \\
  现在分词   & skjótandi &         &         \\
  过去分词   & skotinn   &         &         \\
\end{longtable}

中动态:

\begin{longtable}{llll}
  \toprule
  二类强动词 & 直陈        & 虚拟      & 祈使   \\
  \midrule
  \endhead
  \bottomrule
  \endfoot
  单数现在时 &             &           &        \\
  1          & skýtumk     & skjótumk  &        \\
  2          & skýzk       & skjótisk  & skjózk \\
  3          & skýzk       & skjótisk  &        \\
  复数现在时 &             &           &        \\
  1          & skjótumsk   & skjótimsk & skjót  \\
  2          & skjótizk    & skjótizk  & skjót  \\
  3          & skjótask    & skjótisk  &        \\
  单数过去时 &             &           &        \\
  1          & skutumk     & skytumk   &        \\
  2          & skauzk      & skytisk   &        \\
  3          & skauzk      & skytisk   &        \\
  复数过去时 &             &           &        \\
  1          & skutumsk    & skytim    &        \\
  2          & skutuzk     & skytizk   &        \\
  3          & skutusk     & skytisk   &        \\
  不定式     & skjótask    &           &        \\
  现在分词   & skjótandisk &           &        \\
  过去分词   & skotizk     &           &        \\
\end{longtable}

\subsection{第三强变位法}\label{第三强变位法}

第三强变位法动词不定式的词干总是以一个双辅音结尾,其词根的元音是e或i,其元音交替模式如下:

\textbf{e, i -\/- a -\/- u(y) -\/- u, o}

三类强动词的基础元音是e,但如果元音后面紧随着n,这个元音抬升为i.
同样重要的元音音变发生在过去分词中,一般来说过去分词词根的元音都是u,但当它后面紧随着l或r时,u变为o.
第三强变位法涉及一些重要的语音规则,最显著的几个列举如下:

\begin{enumerate}
  \def\labelenumi{\arabic{enumi})}
  \item
        单数过去式中词干结尾的-nd \textgreater- tt;-ng \textgreater- kk:
\end{enumerate}

\begin{quote}
  binda `bind‌' \textgreater{} batt, batzt, batt

  springa `spring, jump‌' \textgreater{} sprakk, sprakkt, sprakk

  注意词尾和词干相接触时 *-tt \textgreater{} -zt的变化
\end{quote}

\begin{enumerate}
  \def\labelenumi{\arabic{enumi})}
  \setcounter{enumi}{1}
  \item
        单数过去式中词干结尾的-ld \textgreater- lt:
\end{enumerate}

\begin{quote}
  gjalda `pay‌' \textgreater{} galt, galzt, galt

  复数形式guldum, gulduð, guldu不变
\end{quote}

\begin{enumerate}
  \def\labelenumi{\arabic{enumi})}
  \setcounter{enumi}{2}
  \item
        在即以l或r开头的辅音簇前,e分割为ja.
        这条规则对单数直陈现在时以外的词形都有效,因此可以理解为这类动词的单数直陈现在时是特殊的:
\end{enumerate}

\begin{quote}
  不定式gjalda \textgreater{} 单数现在时geld, geldr, geldr

  复数现在时gjǫldum, gjaldið, gjalda ǫ是u-变异后的结果

  不定式bjarga `rescue‌' \textgreater{} bergr vs. bjargið

  一个特殊的动词是skjálfa `shiver‌'
  这里e分割为长元音já,例如第一人称复数skjálfum, 而其他形式如skelf, skalf,
  skulfum, skolfinn等没有长元音。
\end{quote}

\begin{enumerate}
  \def\labelenumi{\arabic{enumi})}
  \setcounter{enumi}{3}
  \item
        v在o或u前脱去。这条规则在古诺尔斯语中是通用的,但在第三强变格法中值得特别提及。例如verða
        `become‌' 的四个基本元的词干分别是verð-, varð-, urð-, orð-.
  \item
        一部分动词的不定式标记-a前有插入音-v-,称这类动词为-va不定式动词。这些动词中保留了一些古老的u-变异:a
        \textgreater{} ǫ, i \textgreater{} y, e \textgreater{} ø。因此søkkva
        `sink‌'的基本元是sekk-v-,sǫkk-v-,sukk-;sangv- \textgreater sǫng.
        (这里的-ng不会变成-kk,因为词干是包括-v的)
\end{enumerate}

上述规则也有一些例外:

\begin{enumerate}
  \def\labelenumi{\arabic{enumi})}
  \item
        brenna `burn‌' 和renna `run‌'的现在时词干中的e并没有变成i.
  \item
        \phantomsection\label{_Ref116921872}{}常见动词finna
        `find‌'的词干本是*finþ-, 但原始诺尔斯语晚期发生了*nþ \textgreater{}
        nn的音变。在第三第四基本元中,按照维尔纳定律发生了*nþ \textgreater{}
        nd的浊化,因此在古诺尔斯语中有不规则的词干finn-, fann-,
        \textbf{fund-}, \textbf{fund-}.
  \item
        bregða
        `hasten‌'的第二基本元可以像第二变位法一样先脱去词尾辅音再变成长元音,得到不规则的brá-,且有brá,
        brátt, brá的不规则变形。
\end{enumerate}

以规则动词springa为例,展示第三强变位法。

主动态如下:

\begin{longtable}{llll}
  \toprule
  三类强动词 & 直陈       & 虚拟     & 祈使     \\
  \midrule
  \endhead
  \bottomrule
  \endfoot
  单数现在时 &            &          &          \\
  1          & spring     & springa  &          \\
  2          & springr    & springir & spring   \\
  3          & springr    & springi  &          \\
  复数现在时 &            &          &          \\
  1          & springum   & springim & springum \\
  2          & springið   & springið & springið \\
  3          & springa    & springi  &          \\
  单数过去时 &            &          &          \\
  1          & sprakk     & sprynga  &          \\
  2          & sprakkt    & spryngir &          \\
  3          & sprakk     & spryngi  &          \\
  复数过去时 &            &          &          \\
  1          & sprungum   & spryngim &          \\
  2          & sprunguð   & spryngið &          \\
  3          & sprungu    & spryngi  &          \\
  不定式     & springa    &          &          \\
  现在分词   & springandi &          &          \\
  过去分词   & sprunginn  &          &          \\
\end{longtable}

中动态如下:

\begin{longtable}{llll}
  \toprule
  三类强动词 & 直陈         & 虚拟       & 祈使       \\
  \midrule
  \endhead
  \bottomrule
  \endfoot
  单数现在时 &              &            &            \\
  1          & springumk    & springumk  &            \\
  2          & springsk     & springisk  & springsk   \\
  3          & springsk     & springisk  &            \\
  复数现在时 &              &            &            \\
  1          & springumsk   & springimsk & springumsk \\
  2          & springizk    & springizk  & springizk  \\
  3          & springask    & springisk  &            \\
  单数过去时 &              &            &            \\
  1          & sprungumk    & spryngumk  &            \\
  2          & sprakkzk     & spryngisk  &            \\
  3          & sprakksk     & spryngisk  &            \\
  复数过去时 &              &            &            \\
  1          & sprungumsk   & spryngimsk &            \\
  2          & sprunguzk    & spryngizk  &            \\
  3          & sprungusk    & spryngisk  &            \\
  不定式     & springask    &            &            \\
  现在分词   & springandisk &            &            \\
  过去分词   & sprungizk    &            &            \\
\end{longtable}

\subsection{第四强变位法}\label{第四强变位法}

第四类强动词数量不多,它们的不定式词干以响音l,m,r,n结尾,词根的元音是e,其元音交替模式是:

\textbf{e -\/- a -\/- á(æ) -\/- o}

一般来说,过去分词词根的元音是o,极少数情况下是u\footnote{PGmc.过去分词的元音曾经是*u,但按照音变规律*u受到后面词尾中*a的作用应该规则地下降为o(这个音变叫作a-Umlaut).
  nema在一些古挪威语的文献中也有过去分词nominn的形式,但在冰岛语中似乎恢复了古老的*u.
  除此之外唯一常见的过去分词元音是u的四类强动词是svima
  `swim',其中词干的i是有e下降所得。}。比较下列的两个动词:

\begin{quote}
  bera `carry‌',bera -\/- bar -\/- báru -\/- borinn------分词元音是o
\end{quote}

nema `take‌',nema -\/- nam -\/- námu -\/- numinn------分词元音是u

四类强动词中有一个常见动词略不规则:koma `come' \textless{} PGmc.
*kwemaną, 它的直陈式变位如下:

\begin{longtable}{lll}
  \toprule
  直陈式 & 现在       & 过去          \\
  \midrule
  \endhead
  \bottomrule
  \endfoot
  1单    & køm, kem   & kom, kvam     \\
  2单    & kømr, kemr & komt, kvamt   \\
  3单    & kømr, kemr & kom, kvam     \\
  1复    & komum      & kómum, kvámum \\
  2复    & komið      & kómuð, kvámuð \\
  3复    & koma       & kómu, kvámu   \\
\end{longtable}

也就是说,koma的四个基本元可以是:(1)kom- (2)kom-/kvam- (3) kóm-/kvám
(4)kom-.现在时中ø和e的交替是古诺尔斯语,特别是古冰岛语的常见现象,没有完全的规则能解释其发生的条件。

以规则动词bera为例,变位如下:

主动态:

\begin{longtable}{llll}
  \toprule
  四类强动词 & 直陈    & 虚拟  & 祈使  \\
  \midrule
  \endhead
  \bottomrule
  \endfoot
  单数现在时 &         &       &       \\
  1          & ber     & bera  &       \\
  2          & berr    & berir & ber   \\
  3          & berr    & beri  &       \\
  复数现在时 &         &       &       \\
  1          & berum   & berim & berum \\
  2          & berið   & berið & berið \\
  3          & bera    & beri  &       \\
  单数过去时 &         &       &       \\
  1          & bar     & bæra  &       \\
  2          & bart    & bærir &       \\
  3          & bar     & bæri  &       \\
  复数过去时 &         &       &       \\
  1          & bárum   & bærim &       \\
  2          & báruð   & bærið &       \\
  3          & báru    & bæri  &       \\
  不定式     & bera    &       &       \\
  现在分词   & berandi &       &       \\
  过去分词   & borinn  &       &       \\
\end{longtable}

中动态:

\begin{longtable}{llll}
  \toprule
  四类强动词 & 直陈      & 虚拟    & 祈使    \\
  \midrule
  \endhead
  \bottomrule
  \endfoot
  单数现在时 &           &         &         \\
  1          & berumk    & berumk  &         \\
  2          & bersk     & berisk  & bersk   \\
  3          & bersk     & berisk  &         \\
  复数现在时 &           &         &         \\
  1          & berumsk   & berimsk & berumsk \\
  2          & berizk    & berizk  & berizk  \\
  3          & berask    & berisk  &         \\
  单数过去时 &           &         &         \\
  1          & bárumk    & bærumk  &         \\
  2          & barzk     & bærisk  &         \\
  3          & barsk     & bærisk  &         \\
  复数过去时 &           &         &         \\
  1          & bárumsk   & bærimsk &         \\
  2          & báruzk    & bærizk  &         \\
  3          & bárusk    & bærisk  &         \\
  不定式     & berask    &         &         \\
  现在分词   & berandisk &         &         \\
  过去分词   & borizk    &         &         \\
\end{longtable}

\subsection{第五强变位法}\label{第五强变位法}

第五强变位法的元音交替模式和第四强变位法基本相同,它们的区别在于词干末尾的辅音,对于除l,r,m,n以外的单辅音结尾的动词要按第五变位法变位。另外第五强变位法过去分词词根的元音一般是e.

\textbf{e -\/- a -\/- á(æ) -\/- e}

五类强动词中包括许多稍不规则的情况:

\begin{enumerate}
  \def\labelenumi{\arabic{enumi})}
  \item
        \textbf{ja-不定式动词}
\end{enumerate}

第五强动词的典型的词根元音是e,但在ja-不定式动词中,元音e \textgreater{}
i保持在现在时变位中。常见的例子是biðja `bid, ask‌',词干bið-j-;sitja
`sit‌', 词干sit-j-,它们的直陈式变形类似于:

\begin{longtable}{lll}
  \toprule
  直陈式 & 现在   & 过去  \\
  \midrule
  \endhead
  \bottomrule
  \endfoot
  1单    & sit    & sat   \\
  2单    & sitr   & sazt  \\
  3单    & sitr   & sat   \\
  1复    & sitjum & sátum \\
  2复    & sitið  & sátuð \\
  3复    & sitja  & sátu  \\
\end{longtable}

上表展示了词干末尾的-j接续辅音和不同类型的元音时的情况,事实上,这里完全符合\ref{半元音的保持性}的规则,读者不难推断出虚拟式以及中动态的情况。

liggja `lie, recline‌' 和þiggja `accept, receive‌'
这两个动词以-ja结尾且辅音是双写辅音,它们的第一基本元的词干是liggj-,
þiggj-,但在过去式中-j-不再出现;单数过去式的词干末尾的双辅音缩短为单辅音,在古诺尔斯语文本中不再写出,同时词干中的元音变为长音(第二基本元lá-,
þá-)。第三、第四基本元保留单辅音,如lág-, þág-; leg-, þeg-.\footnote{这些词本来的形式就如*ligja,-g在i前发生延长发生在原始诺尔斯语晚期,一些i-词干名词中也有这样的现象,如bekkr.
  在这里,可以认为第三、第四基本元是规则的。这个音变似乎没有发生完全,在一些挪威方言中也有不双写g的形式存在。另外,既然j是词干的一部分,四个基本元应当都发生音变才对,但也有可能是j被重解为词尾的一部分,而过去式中j全部脱落使得音变条件消失。}
如þiggja的直陈式变位如下:

\begin{longtable}{lll}
  \toprule
  直陈式 & 现在    & 过去  \\
  \midrule
  \endhead
  \bottomrule
  \endfoot
  1单    & þigg    & þá    \\
  2单    & þiggr   & þátt  \\
  3单    & þiggr   & þá    \\
  1复    & þiggjum & þágum \\
  2复    & þiggið  & þáguð \\
  3复    & þiggja  & þágu  \\
\end{longtable}

\begin{enumerate}
  \def\labelenumi{\arabic{enumi})}
  \setcounter{enumi}{1}
  \item
        \textbf{以-g结尾的动词}
\end{enumerate}

常见动词vega `weigh‌' (词干veg-)
的单数过去式脱去g,并将a变为长元音,如vá,vátt,这类似于第一、第二强变位法中的情况。

vega有两个意思,作为``称重''时按第五变位法变位(\textless PGmc.
*weganą),另一个意思``杀;战斗''继承于PGmc. *wiganą
`fight',这本来是一个第一变位法动词。两个原始日耳曼动词在古诺尔斯语中合并了,但哥特语中仍保留了两个分别的形式(weihan---wigan)。vega在古诺尔斯语中只按第五变位法变位,但其形式可能受到了第一变位法的影响。

原始日耳曼语中以-g结尾的动词只有两个,另一个动词*treganą `suffer;
grieve'在古诺尔斯语中变成了弱动词,因此缺少对以-g结尾动词变位规律的确认。读者可以认为vega是一个不规则动词。

\begin{enumerate}
  \def\labelenumi{\arabic{enumi})}
  \setcounter{enumi}{2}
  \item
        \textbf{不规则动词}
\end{enumerate}

这些动词有时也归类到第四类强动词中,但在历史上是第五类强动词。

1. 常见动词sofa `sleep‌' \textless{} PGmc. swefaną的变位类似于koma:

\begin{longtable}{lll}
  \toprule
  直陈式 & 现在       & 过去   \\
  \midrule
  \endhead
  \bottomrule
  \endfoot
  1单    & søf, sef   & svaf   \\
  2单    & søfr, sefr & svaft  \\
  3单    & søfr, sefr & svaf   \\
  1复    & sofum      & sváfum \\
  2复    & sofið      & sváfuð \\
  3复    & sofa       & sváfu  \\
\end{longtable}

vefa `weave' \textless{} PGmc.
*webaną的过去式变位也和koma类似,但也可以按规则方法变形:

\begin{longtable}{lll}
  \toprule
  直陈式 & 现在  & 过去        \\
  \midrule
  \endhead
  \bottomrule
  \endfoot
  1单    & vef   & óf, vaf     \\
  2单    & vefr  & óft, vaft   \\
  3单    & vefr  & óf, vaf     \\
  1复    & vefum & ófum, váfum \\
  2复    & vefið & ófuð, váfuð \\
  3复    & vefa  & ófu, váfu   \\
\end{longtable}

动词troða
`tread‌'的不定式和过去分词的词干都是o而不是e,但它的过去式变位是规则的。有时把它归到第四类强动词中。

\begin{longtable}{lll}
  \toprule
  直陈式 & 现在   & 过去   \\
  \midrule
  \endhead
  \bottomrule
  \endfoot
  1单    & trøð   & trað   \\
  2单    & trøðr  & tratt  \\
  3单    & trøðr  & trað   \\
  1复    & troðum & tráðum \\
  2复    & troðið & tráðuð \\
  3复    & troða  & tráðu  \\
\end{longtable}

fregna
`ask'和以-g结尾的动词类似,-n-是一个遗留的现在时中缀,以至于单数现在时的-r词尾消失了:

\begin{longtable}{lll}
  \toprule
  直陈式 & 现在    & 过去   \\
  \midrule
  \endhead
  \bottomrule
  \endfoot
  1单    & fregn   & frá    \\
  2单    & fregn   & frátt  \\
  3单    & fregn   & frá    \\
  1复    & fregnum & frágum \\
  2复    & fregnið & fráguð \\
  3复    & fregna  & frágu  \\
\end{longtable}

唯一一个动词sjá `see‌'高度不规则,现在时由sé-, sjá-构成,过去时由sá-构成,
过去分词的词干为sé-,完整的变位如下:

\begin{longtable}{llll}
  \toprule
  sjá        & 直陈       & 虚拟            & 祈使       \\
  \midrule
  \endhead
  \bottomrule
  \endfoot
  单数现在时 &            &                 &            \\
  1          & sé         & sé              &            \\
  2          & sér        & sér             & sé         \\
  3          & sér        & sé, sjái, sjáir &            \\
  复数现在时 &            &                 &            \\
  1          & sjám       & sém             & sjám       \\
  2          & séð, sjáið & séð             & séð, sjáið \\
  3          & sjá        & sé              &            \\
  单数过去时 &            &                 &            \\
  1          & sá         & sæa             &            \\
  2          & sátt       & sæir            &            \\
  3          & sá         & sæi             &            \\
  复数过去时 &            &                 &            \\
  1          & sám        & sæim            &            \\
  2          & sáuð       & sæið            &            \\
  3          & sá, sáu    & sæi             &            \\
  不定式     & sjá        &                 &            \\
  现在分词   & sjándi     &                 &            \\
  过去分词   & sénn       &                 &            \\
\end{longtable}

动词reka `drive‌'代表了典型的五类强动词变位,其主动态如下:

\begin{longtable}{llll}
  \toprule
  五类强动词 & 直陈    & 虚拟  & 祈使  \\
  \midrule
  \endhead
  \bottomrule
  \endfoot
  单数现在时 &         &       &       \\
  1          & rek     & reka  &       \\
  2          & rekr    & rekir & rek   \\
  3          & rekr    & reki  &       \\
  复数现在时 &         &       &       \\
  1          & rekum   & rekim & rekum \\
  2          & rekið   & rekið & rekið \\
  3          & reka    & reki  &       \\
  单数过去时 &         &       &       \\
  1          & rak     & ræka  &       \\
  2          & rakt    & rækir &       \\
  3          & rak     & ræki  &       \\
  复数过去时 &         &       &       \\
  1          & rákum   & rækim &       \\
  2          & rákuð   & rækið &       \\
  3          & ráku    & ræki  &       \\
  不定式     & reka    &       &       \\
  现在分词   & rekandi &       &       \\
  过去分词   & rekinn  &       &       \\
\end{longtable}

中动态如下:

\begin{longtable}{llll}
  \toprule
  五类强动词 & 直陈      & 虚拟    & 祈使    \\
  \midrule
  \endhead
  \bottomrule
  \endfoot
  单数现在时 &           &         &         \\
  1          & rekumk    & rekumk  &         \\
  2          & reksk     & rekisk  & reksk   \\
  3          & reksk     & rekisk  &         \\
  复数现在时 &           &         &         \\
  1          & rekumsk   & rekimsk & rekumsk \\
  2          & rekizk    & rekizk  & rekizk  \\
  3          & rekask    & rekisk  &         \\
  单数过去时 &           &         &         \\
  1          & rákumk    & rækumk  &         \\
  2          & rakzk     & rækisk  &         \\
  3          & raksk     & rækisk  &         \\
  复数过去时 &           &         &         \\
  1          & rákumsk   & rækimsk &         \\
  2          & rákuzk    & rækizk  &         \\
  3          & rákusk    & rækisk  &         \\
  不定式     & rekask    &         &         \\
  现在分词   & rekandisk &         &         \\
  过去分词   & rekizk    &         &         \\
\end{longtable}

\subsection{第六强变位法}\label{第六强变位法}

第六类强动词词根元音为a,元音交替模式如下:

\textbf{a(e) -\/- ó -\/- ó(œ) -- a, e}

过去分词的元音一般是a,如果词尾的辅音是软腭音k或g,a会被前移至e。单数现在时中,a总发生i-变异变为e,因此fara
`go‌'要变为fer, ferr。

由于直陈过去式中都是圆唇元音,因此总是导致v的脱落。例如vaxa `grow, wax‌'
的整个过去式词干都是óx- , 同理有vaða `wade‌'\textgreater óð-.

六类强动词中也有一些特殊情况:

\begin{enumerate}
  \def\labelenumi{\arabic{enumi})}
  \item
        \textbf{以-g结尾的动词}
\end{enumerate}

常见动词draga `drag‌'的单数过去式脱去了-g,由于元音变成了ó ,如dró,
drótt,因此看起来和二类动词的变形很相像。

还有一些动词历史上曾有词尾-g(来自于-h),但是后来在现在时中脱落了,这类动词包括klá
`scratch', slá `slay', flá `flay', þvá `wash'和hlæja
`laugh‌',其中带有长音á的动词都模仿slá. slá和hlæja的直陈式变位如下:

\begin{longtable}{llllll}
  \toprule
  直陈式 & 现在  & 过去   &  & 现在   & 过去   \\
  \midrule
  \endhead
  \bottomrule
  \endfoot
  1单    & slæ   & sló    &  & hlæ    & hló    \\
  2单    & slær  & slótt  &  & hlær   & hlótt  \\
  3单    & slær  & sló    &  & hlær   & hló    \\
  1复    & slám  & slógum &  & hlæjum & hlógum \\
  2复    & sláið & slóguð &  & hlæið  & hlóguð \\
  3复    & slá   & slógu  &  & hlæja  & hlógu  \\
\end{longtable}

它们的过去分词都有-g,且受到-g的影响变成sleginn; hleginn.

注意长音动词接续元音开头的词尾时的元音缩合现象。

\begin{enumerate}
  \def\labelenumi{\arabic{enumi})}
  \setcounter{enumi}{1}
  \item
        \textbf{ja-不定式动词}
\end{enumerate}

ja-不定式动词中-j-的出现导致整个现在时变位全部受i-音变影响,而本来只有单数需要i-变异(古老的词尾所致)。这些动词包括sverja
`swear', hefja `heave; raise', skepja `shape'和kefja
`sink'\footnote{本是kvefja且是一类弱动词,但经类比也可按照六类强动词变位。},他们的变化都类似,参考sverja的变位(另外发生了v的脱落):

\begin{longtable}{lll}
  \toprule
  直陈式 & 现在    & 过去  \\
  \midrule
  \endhead
  \bottomrule
  \endfoot
  1单    & sver    & sór   \\
  2单    & sverr   & sórt  \\
  3单    & sverr   & sór   \\
  1复    & sverjum & sórum \\
  2复    & sverið  & sóruð \\
  3复    & sverja  & sóru  \\
\end{longtable}

动词deyja
`die'也包含增音-j-,但是ey无法发生i-变异。deyja的复数过去式与其他ja-不定式动词有不同,因为它的词干尾没有辅音,使得ó直接与词尾接触,触发元音缩合,故其复数过去式为dóm
\textless{} †dóum, dóð, dó. 以相同方式变位的还是一个不常见动词geyja
`bark'.

除了上面的一些特殊情况外,非常常见的动词standa
`stand‌'在除了现在时以外的地方脱去-n-,d以异构ð保留,其四个基本元如下为stand-,
stóð-, stóð-, stað-.

\begin{quote}
  规则动词fara的完整变格如下:

  主动态:
\end{quote}

\begin{longtable}{llll}
  \toprule
  六类强动词 & 直陈    & 虚拟  & 祈使  \\
  \midrule
  \endhead
  \bottomrule
  \endfoot
  单数现在时 &         &       &       \\
  1          & fer     & fara  &       \\
  2          & ferr    & farir & far   \\
  3          & ferr    & fari  &       \\
  复数现在时 &         &       &       \\
  1          & fǫrum   & farim & fǫrum \\
  2          & farið   & farið & farið \\
  3          & fara    & fari  &       \\
  单数过去时 &         &       &       \\
  1          & fór     & fœra  &       \\
  2          & fórt    & fœrir &       \\
  3          & fór     & fœri  &       \\
  复数过去时 &         &       &       \\
  1          & fórum   & fœrim &       \\
  2          & fóruð   & fœrið &       \\
  3          & fóru    & fœri  &       \\
  不定式     & fara    &       &       \\
  现在分词   & farandi &       &       \\
  过去分词   & farinn  &       &       \\
\end{longtable}

中动态如下所示:

\begin{longtable}{llll}
  \toprule
  六类强动词 & 直陈      & 虚拟    & 祈使    \\
  \midrule
  \endhead
  \bottomrule
  \endfoot
  单数现在时 &           &         &         \\
  1          & fǫrumk    & fǫrumk  &         \\
  2          & fersk     & farisk  & farsk   \\
  3          & fersk     & farisk  &         \\
  复数现在时 &           &         &         \\
  1          & fǫrumsk   & farimsk & fǫrumsk \\
  2          & farizk    & farizk  & farizk  \\
  3          & farask    & farisk  &         \\
  单数过去时 &           &         &         \\
  1          & fórumk    & fœrumk  &         \\
  2          & fórzk     & fœrisk  &         \\
  3          & fórsk     & fœrisk  &         \\
  复数过去时 &           &         &         \\
  1          & fórumsk   & fœrimsk &         \\
  2          & fóruzk    & fœrizk  &         \\
  3          & fórusk    & fœrisk  &         \\
  不定式     & farask    &         &         \\
  现在分词   & farandisk &         &         \\
  过去分词   & farizk    &         &         \\
\end{longtable}

\subsection{第七强变位法}\label{第七强变位法}

第七类强动词的定义是元音交替不符合上述六类的动词,在古诺尔斯语中形成了下列五大元音交替模式,它们和对应的五类变位法的元音交替有一定的联系:

\begin{longtable}{lllll}
  \toprule
  类别                 & 不定式 & 单数过去 & 复数过去 & 过去分词 \\
  \midrule
  \endhead
  \bottomrule
  \endfoot
  1. heita `be called‌' & heita  & hét      & hétum    & heitinn  \\
  2a. auka `increase‌'  & auka   & jók      & jókum    & aukinn   \\
  2b. búa `inhabit‌'    & búa    & bjó      & bjuggum  & búinn    \\
  3. falla `fall‌'      & falla  & fell     & fellum   & fallinn  \\
  4. láta `let‌'        & láta   & lét      & létum    & látinn   \\
  5. blóta `offer‌'     & blóta  & blét     & blétum   & blótinn  \\
\end{longtable}

七类强动词的基本特点是第一和第四基本元的词根元音一致,第二和第三也一致(2b除外)。

\textbf{第1类动词}

第1类动词的不定式词根元音是ei,一共有三个:heita `be called', leika
`play', sveipa `sweep'

\begin{longtable}{llll}
  \toprule
  不定式 & 单数过去 & 复数过去 & 过去分词 \\
  \midrule
  \endhead
  \bottomrule
  \endfoot
  heita  & hét      & hétum    & heitinn  \\
  leika  & lék      & lékum    & leikinn  \\
  sveipa & sveip    & svipum   & sveipinn \\
\end{longtable}

这三个动词中最典型的是leika,它只按照第七强变位法变位;

heita的现在时词干有两种形式,
heit-/heiti-都是可以的,这取决它的意思:1)表示A叫作B;2)表示承诺,第一个意思的现在时词干接弱动词词尾,即按照三类弱动词变形(见\ref{第三弱变位法}):heiti,
heitir, heitir,取第二个意思时按强动词变形:heit, heitr, heitr;

sveipa经常按弱动词变位,其残存的强动词变化形式和第一类强动词发生了混淆。

\textbf{第2类动词}

第2类动词的过去式词干中有双元音jó为特征,其中的子类a的不定式元音是au,其他归为b类,这类动词包括:auka
`increase', ausa `sprinkle', hlaupa `jump', búa `inhabit', hǫggva `hew‌'

\begin{longtable}{lllll}
  \toprule
  不定式 & 单数现在   & 单数过去 & 复数过去                 & 过去分词 \\
  \midrule
  \endhead
  \bottomrule
  \endfoot
  auka   & eyk        & jók      & jókum, jukum             & aukinn   \\
  ausa   & eys        & jós      & jósum, jusum             & ausinn   \\
  hlaupa & hleyp      & hljóp    & hljópum, hljupum         & hlaupinn \\
  búa    & bý         & bjó      & bjoggum, bjuggum, buggum & búinn    \\
  hǫggva & høgg, hegg & hjó      & hjoggum, hjuggum         & hǫggvinn \\
\end{longtable}

复数过去式中,ju和jó经常交替,ju是比较后期的形式。在búa中甚至出现了ju
\textgreater{} u的形式,其复数过去式中的g可能是从一个同根弱动词byggva
`reside'中借来的。

hǫggva的单数现在时值得注意,va-不定式触发了u-变异,因此其现在时词干应该是hagg-v-,但在单数现在时中i-变异的形式是ø,而非预期的e,不过它又在后来的形式中规则化了。

\textbf{第3类动词}

第3类动词相对比较规则,且都非常常见,这主要包括 blanda `mix', ganga
`walk', hanga `hang', falla `fall', halda `hold', falda `fold', fá
`get',它们的变形都发生了类似三类强动词的音变(回顾\ref{第三强变位法}音变规则1,2):

\begin{longtable}{lllll}
  \toprule
  不定式 & 单数现在 & 单数过去 & 复数过去       & 过去分词         \\
  \midrule
  \endhead
  \bottomrule
  \endfoot
  blanda & blend    & blett    & blendum        & blandinn         \\
  ganga  & geng     & gekk     & gengum, gingum & genginn, gingum  \\
  hanga  & heng     & hekk     & hengum         & hanginn          \\
  falla  & fell     & fell     & fell           & fallin           \\
  halda  & held     & helt     & heldum         & haldinn          \\
  falda  & feld     & felt     & feldum         & faldinn          \\
  fá     & fæ       & fekk     & fengum, fingum & fenginn, finginn \\
\end{longtable}

fá的变位和ganga非常一致,它曾经的形式是*fanhaną,词尾的-g也发生了脱落,但在复数过去式中恢复。注意这两个动词复数过去式、过去分词的词根元音发生了抬升,一些很古老的文献中记录了i,但后来变成了e.

\textbf{第4、5类动词}

这两类动词形式规则,它们的区别仅在于不定式的元音是á还是ó,常见动词包括:blása
`blow', gráta `weep', láta `let', ráða `advise', blóta `offer‌'

以láta为例,展示规则的七类强动词的完整变位。其主动态为:

\begin{longtable}{llll}
  \toprule
  七类强动词 & 直陈    & 虚拟  & 祈使  \\
  \midrule
  \endhead
  \bottomrule
  \endfoot
  单数现在时 &         &       &       \\
  1          & læt     & láta  &       \\
  2          & lætr    & látir & lát   \\
  3          & lætr    & láti  &       \\
  复数现在时 &         &       &       \\
  1          & látum   & látim & látum \\
  2          & látið   & látið & látið \\
  3          & láta    & láti  &       \\
  单数过去时 &         &       &       \\
  1          & lét     & léta  &       \\
  2          & lézt    & létir &       \\
  3          & lét     & léti  &       \\
  复数过去时 &         &       &       \\
  1          & létum   & létim &       \\
  2          & létuð   & létið &       \\
  3          & létu    & léti  &       \\
  不定式     & láta    &       &       \\
  现在分词   & látandi &       &       \\
  过去分词   & látinn  &       &       \\
\end{longtable}

中动态为:

\begin{longtable}{llll}
  \toprule
  七类强动词 & 直陈      & 虚拟    & 祈使    \\
  \midrule
  \endhead
  \bottomrule
  \endfoot
  单数现在时 &           &         &         \\
  1          & látumk    & látumkm &         \\
  2          & Læzk      & Látisk  & lázk    \\
  3          & læzk      & látisk  &         \\
  复数现在时 &           &         &         \\
  1          & látumsk   & látimsk & látumsk \\
  2          & Látizk    & látizk  & látizk  \\
  3          & Látask    & látisk  &         \\
  单数过去时 &           &         &         \\
  1          & Létumk    & létumk  &         \\
  2          & Lézk      & létisk  &         \\
  3          & Lézk      & létisk  &         \\
  复数过去时 &           &         &         \\
  1          & létumsk   & létimsk &         \\
  2          & létuzk    & létizk  &         \\
  3          & létusk    & létisk  &         \\
  不定式     & látask    &         &         \\
  现在分词   & látandisk &         &         \\
  过去分词   & látizk    &         &         \\
\end{longtable}

\textbf{复音动词}

这类动词继承了原始印欧语的特点。日耳曼语的过去式主要是由原始印欧语的完成时演变过来的,完成时的构词法是在词根前添加一个重复成分(词根的辅音+元音),在一些古典语言中有清楚的痕迹(连字符是为了更清晰地展示重复成分):

\begin{quote}
  梵 语:现在时bharati `bear, carry' → 完成时 \textbf{ba}-bhāra

  希腊语:现在时leípō `leave' → 完成时\textbf{lé}-loipā
\end{quote}

在日耳曼语中,一些动词的过去式也有这样的特点,称为复音动词(Reduplicated
verb),但复音动词在不同的语言中可能被简化,也可能发生进一步的音变。比较下面几个比较明显的复音动词的基本元:

\begin{quote}
  哥特语:现在时hláupan 单数过去\textbf{haí}-hláup

  古英语:现在时hātan 单数过去hēt \textless{} \textbf{he}-ht
\end{quote}

但是,古诺尔斯语中这两个动词的同源词hlaupa和heita都失去了重复成分,只有三个动词róa
`row‌', sá `sow‌'和snúa
`turn‌'还保有复音动词的特征,但已经发生了比较明显的音变。它们的过去式构成方法是用词首辅音+ø/e+r构成词干,在单数过去式中添加-a,
-ir, -i词尾,复数过去式中仍是-um, -uð,
-u词尾(整个过去式实则添加的是弱动词词尾)。这些动词因其过去式的形态有时被称为ra-动词(现代冰岛语中叫作ri-动词),它们的变位如下所示:

\begin{longtable}{llll}
  \toprule
  不定式     & róa          & sá           & snúa           \\
  \midrule
  \endhead
  \bottomrule
  \endfoot
  单数现在时 &              &              &                \\
  1          & rœ           & sæ           & sný            \\
  2          & rœr          & sær          & snýr           \\
  3          & rœr          & sær          & snýr           \\
  复数现在时 &              &              &                \\
  1          & ró(u)m       & sám          & snúm           \\
  2          & róið         & sáið         & snúið          \\
  3          & róa          & sá           & snúa           \\
  单数过去时 &              &              &                \\
  1          & røra, rera   & søra, sera   & snøra, snera   \\
  2          & rørir, rerir & sørir, serir & snørir, snerir \\
  3          & røri, reri   & søri, seri   & snøri, sneri   \\
  复数过去时 &              &              &                \\
  1          & rørum, rerum & sørum, serum & snørum, snerum \\
  2          & røruð, reruð & søruð, seruð & snøruð, sneruð \\
  3          & røru, reru   & søru, seru   & snøru, sneru   \\
  过去分词   & róinn        & sáinn        & snúinn         \\
\end{longtable}

形似的动词 gróa `grow‌'和gnúa `rub‌'
本来不是复音动词,但它们的变形受类比的影响与róa和snúa一致。

\section{弱动词的变位法}\label{弱动词的变位法}

区别于强动词用元音变换指示过去式,弱动词的过去时标记是添加在词干和人称词尾间的-ði-,例如:

\begin{longtable}{llll}
  \toprule
  词干      & 过去时标记 & 人称标记 & 完整变形              \\
  \midrule
  \endhead
  \bottomrule
  \endfoot
  sigl-i/j- & -ði-       & -a       & siglða `I sailed‌'     \\
  kall-a-   & -ði-       & -r       & kallaðir `you called‌' \\
  lif-i-    & -ði-       & -um      & lifðum `we lived‌'     \\
\end{longtable}

弱动词的词干和强动词稍有不同,我们在后面会具体的介绍。在此之前,我们先熟悉一下变位中常常触发的一些重要的音变:

\begin{enumerate}
  \def\labelenumi{\arabic{enumi})}
  \item
        i-的删去。根据音变规律\ref{元音的音变},后缀ði中的i在元音开头的词尾前前脱落,同时词干元音i或-i/j-在-ði-前脱落:
\end{enumerate}

\begin{quote}
  kall-a- + -ði- + -a \textgreater{} kallaða

  lif-i- + -ði- + -u \textgreater{} lifðu
\end{quote}

\begin{enumerate}
  \def\labelenumi{\arabic{enumi})}
  \setcounter{enumi}{1}
  \item
        u-变异。在u之前,非重读的a变成u,重读的a变成
        ǫ,在许多弱动词中会产生连锁反应:
\end{enumerate}

\begin{quote}
  kall-a- + -ði- + um \textgreater{} *kalluðum \textgreater{} kǫlluðum

  tal-i/j- + -ði- + -uð \textgreater{} *talðuð \textgreater{} tǫlðuð
\end{quote}

\begin{enumerate}
  \def\labelenumi{\arabic{enumi})}
  \setcounter{enumi}{2}
  \item
        -ð + ð- \textgreater{} -dd-:
\end{enumerate}

\begin{quote}
  beiða `ask' \textgreater{} beidda

  eyða `waste' \textgreater{} eydda
\end{quote}

\begin{enumerate}
  \def\labelenumi{\arabic{enumi})}
  \setcounter{enumi}{3}
  \item
        -t, s + ð- \textgreater{} -tt, st:
\end{enumerate}

\begin{quote}
  flytja `move‌' \textgreater{} flutta

  sæta `undergo‌' \textgreater{} sætta
\end{quote}

\begin{enumerate}
  \def\labelenumi{\arabic{enumi})}
  \setcounter{enumi}{4}
  \item
        -p, k, f, l + ð- \textgreater{} -pt, kt, ft,
        lt,但这个变化不总是发生:
\end{enumerate}

\begin{quote}
  þurfa `need‌' \textgreater{} þurfta

  但hafa `have‌' \textgreater{} hafða不变

  mæla `speak‌' \textgreater{} mælta

  但也有 vilja `want‌' \textgreater{} vilda
\end{quote}

\begin{enumerate}
  \def\labelenumi{\arabic{enumi})}
  \setcounter{enumi}{5}
  \item
        辅音简化:-Cdd \textgreater{} -Cd以及-Ctt \textgreater{}
        -Ct,即当上述的规则导致dd/tt出现时,如果它们前面还有一个辅音,则双辅音简化为单辅音
\end{enumerate}

\begin{quote}
  senda `send‌' \textgreater{} *sendda \textgreater{} senda

  skipta `shift‌' \textgreater{} *skiptta \textgreater{} skipta
\end{quote}

弱动词共可分为三类,这是根据词干元音的特征来分的。下面的三个动词krefja
`demand‌', kalla `call‌'和 vaka `wake‌'分属三类弱动词:

\begin{longtable}{lllllll}
  \toprule
  类别 & 不定式          & 三单现在时 & 三单过去时 & 三复过去时 & 三单过去虚拟式 & 过去分词     \\
  \midrule
  \endhead
  \bottomrule
  \endfoot
  I    & krefja `demand‌' & krefr      & krafði     & krǫfðu     & krefði         & krafðr       \\
  II   & kalla `call‌'    & kallar     & kallaði    & kǫlluðu    & kallaði        & kallaðr      \\
  III  & vaka `wake‌'     & vakir      & vakþi      & vǫkþu      & vekþi          & vakat (中性) \\
\end{longtable}

弱动词的分类是一个值得探讨的问题,从上表可以看出,第三人称现在时的形态(-r,
-ar,
-ir)可以用于决定弱动词的种类。有时这和历史演变的情况是矛盾的,不同的学者可能根据不同的标准划分词类。譬如heyra
`hear‌'在过去是第一类弱动词,但古诺尔斯语中它的第三人称现在时是heyrir,应当算作第三类弱动词。因此heyra既可以归为第一类,也可归为第三类弱动词,这视分类者的习惯而定。为了便于理解弱动词的来历,本书仍采用历史语言学的划分方式。

读者可以从此表中发现,弱动词的形式间不存在元音交替的现象,只有元音变异。不同类别的动词发生元音变异的形态不完全相同,见下文详述。

弱动词的变位可以由3个基本元确定,即不定式,单数过去式和过去分词。与强动词不同,这三个基本元在很大程度上是可以根据弱动词的类型互相推断的。

\subsection{弱动词的主动词尾}\label{弱动词的主动词尾}

弱动词的主动词尾如下所示:

\begin{longtable}{llll}
  \toprule
  弱动词     & 直陈          & 虚拟 & 祈使 \\
  \midrule
  \endhead
  \bottomrule
  \endfoot
  单数现在时 &               &      &      \\
  1          & -ø / -a / -i  & -a   &      \\
  2          & -r / -ar/ -ir & -ir  & -    \\
  3          & -r / -ar/ -ir & -i   &      \\
  复数现在时 &               &      &      \\
  1          & -um           & -im  & -um  \\
  2          & -ið           & -ið  & -ið  \\
  3          & -a            & -i   &      \\
  单数过去时 &               &      &      \\
  1          & -ða           & -ða  &      \\
  2          & -ðir          & -ðir &      \\
  3          & -ði           & -ði  &      \\
  复数过去时 &               &      &      \\
  1          & -ðum          & -ðim &      \\
  2          & -ðuð          & -ðið &      \\
  3          & -ðu           & -ði  &      \\
\end{longtable}

这张表中的词尾单数现在时的词尾事实上考虑了词干元音,它们事实上都是-ø,
-r, -r的变体。弱动词和强动词最大的区别在于单数过去式的词尾变成了-a, -ir,
-i.

类似于强动词,弱动词的变形的过程如下所示:

\begin{longtable}{l}
  \toprule
  所有的现在时形式:取不定式词干,加词尾 (需用动词类别区分词尾 -r / -ar/-ir);过去直陈式: 取第二基本元词干,必要时将词干中的元音u-音变,加词尾;现在虚拟式:取第二基本元词干,加词尾;过去虚拟式: 取第二基本元词干,将后元音i-变异,加词尾;现在分词: 取不定式词干,加词尾;过去分词: 取第三基本元,加词尾。 \\
  \midrule
  \endhead
  \bottomrule
  \endfoot
\end{longtable}

将三个基本元标记为 (1), (2), (3), 变形方式如下

\begin{longtable}{llll}
  \toprule
  弱动词     & 直陈                  & 虚拟                  & 祈使                 \\
  \midrule
  \endhead
  \bottomrule
  \endfoot
  单数现在时 &                       &                       &                      \\
  1          & (1) + - / -a / -i     & (1) + -a              &                      \\
  2          & (1) + -r / -ar/ -ir   & (1) + -ir             & (1) + -              \\
  3          & (1) + -r / -ar/ -ir   & (1) + -i              &                      \\
  复数现在时 &                       &                       &                      \\
  1          & (1) + (u-变异) + -um  & (1) + -im             & (1) + (u-变异) + -um \\
  2          & (1) + -ið             & (1) + -ið             & (1) + -ið            \\
  3          & (1) + -a              & (1) + -i              &                      \\
  单数过去时 &                       &                       &                      \\
  1          & (2) + -ða             & (2) + (i-变异) + -ða  &                      \\
  2          & (2) + -ðir            & (2) + (i-变异) + -ðir &                      \\
  3          & (2) + -ði             & (2) + (i-变异) + -ði  &                      \\
  复数过去时 &                       &                       &                      \\
  1          & (2) + (u-变异) + -ðum & (2) + (i-变异) + -ðim &                      \\
  2          & (2) + (u-变异) + -ðuð & (2) + (i-变异) + -ðið &                      \\
  3          & (2) + (u-变异) + -ðu  & (2) + (i-变异) + -ði  &                      \\
  不定式     & (1) + -a              &                       &                      \\
  现在分词   & (1) + -andi           &                       &                      \\
  过去分词   & (3) + -ðr             &                       &                      \\
\end{longtable}

\subsection{弱动词的中动词尾}\label{弱动词的中动词尾}

弱动词的中动态的构成和强动词一致,也是在词尾上添加-sk而已,我们在强动词部分已经讲过了-sk词尾导致的音变,这里不再赘述了。

\begin{longtable}{llll}
  \toprule
  弱动词     & 直陈              & 虚拟   & 祈使       \\
  \midrule
  \endhead
  \bottomrule
  \endfoot
  单数现在时 &                   &        &            \\
  1          & -umk              & -umk   &            \\
  2          & -sk / -ask / -isk & -isk   & -sk / -ask \\
  3          & -sk / -ask / -isk & -isk   &            \\
  复数现在时 &                   &        &            \\
  1          & -umsk             & -imsk  & -umsk      \\
  2          & -izk              & -izk   & -izk       \\
  3          & -ask              & -isk   &            \\
  单数过去时 &                   &        &            \\
  1          & -ðumk             & -ðumk  &            \\
  2          & -ðisk             & -ðisk  &            \\
  3          & -ðisk             & -ðisk  &            \\
  复数过去时 &                   &        &            \\
  1          & -ðumsk            & -ðimsk &            \\
  2          & -ðuzk             & -ðizk  &            \\
  3          & -ðusk             & -ðisk  &            \\
\end{longtable}

中动态的变化模式归纳如下:

\begin{longtable}{llll}
  \toprule
  弱动词     & 直陈                    & 虚拟                    & 祈使                   \\
  \midrule
  \endhead
  \bottomrule
  \endfoot
  单数现在时 &                         &                         &                        \\
  1          & (1) + (u-变异) + -umk   & (1) + (u-变异) + -umk   &                        \\
  2          & (1) + -sk / -ask/ -isk  & (1) + -isk              & (1) + -sk / -ask       \\
  3          & (1) + -sk / -ask/ -isk  & (1) + -isk              &                        \\
  复数现在时 &                         &                         &                        \\
  1          & (1) + (u-变异) + -umsk  & (1) + -imsk             & (1) + (u-变异) + -umsk \\
  2          & (1) + -izk              & (1) + -izk              & (1) + -ið              \\
  3          & (1) + -ask              & (1) + -isk              &                        \\
  单数过去时 &                         &                         &                        \\
  1          & (2) + (u-变异) + -ðumk  & (2) + (u-变异) + -ðumk  &                        \\
  2          & (2) + -ðisk             & (2) + (i-变异) + -ðisk  &                        \\
  3          & (2) + -ðisk             & (2) + (i-变异) + -ðisk  &                        \\
  复数过去时 &                         &                         &                        \\
  1          & (2) + (u-变异) + -ðumsk & (2) + (i-变异) + -ðimsk &                        \\
  2          & (2) + (u-变异) + -ðuzk  & (2) + (i-变异) + -ðizk  &                        \\
  3          & (2) + (u-变异) + -ðusk  & (2) + (i-变异) + -ðisk  &                        \\
  不定式     & (1) + -ask              &                         &                        \\
  现在分词   & (1) + -andisk           &                         &                        \\
  过去分词   & (3) + -zk               &                         &                        \\
\end{longtable}

弱动词的过去式词干是一样的,因此没有强动词中第一人称的不规则现象。

\subsection{第一弱变位法}\label{第一弱变位法}

第一类弱动词的词干元音是-i/j-,和ja-词干名词一样,元音的确定由西弗斯定律支配,即:

\begin{longtable}{l}
  \toprule
  短词干:词干音节只有一个单辅音+不超过一个辅音/一个双元音或长元音 \\
  \midrule
  \endhead
  \bottomrule
  \endfoot
  长词干:词干音节为单辅音+辅音簇/双元音或长元音+任意数量的辅音    \\
  短词干后,词干元音是j;长词干后,词干元音是i.                    \\
\end{longtable}

正如长短ja-词干名词的变格不同,词干元音完全决定了第一类弱动词的变位方法,因此有些语法也把第一弱变位法进一步分成两类。

试比较下面不同类型的长短词干动词的不定式:

\begin{longtable}{llll}
  \toprule
  词干               & 词干类型             & 词干元音 & 不定式                                      \\
  \midrule
  \endhead
  \bottomrule
  \endfoot
  var-i/j- `defend‌'  & 单元音+单辅音 短词干 & -j-      & verja                                       \\
  sigl-i/j- `sail‌'   & 单元音+辅音簇 长词干 & -i-      & sigla                                       \\
  sát-i/j- `undergo‌' & 长元音+单辅音 长词干 & -i-      & sæta                                        \\
  knú-i/j- `knock'   & 单个长元音 短词干    & -j-      & knýja                                       \\
  þrá-i/j- `desire'  & 单个双元音 短词干    & -j-      & þreyja\footnote{弱动词中有一部分动词发生了á
  \textgreater{} ey的i-变异。但从词源上来说,长元音á的确由au缩合得到。}                              \\
\end{longtable}

词干元音-i/j-都触发后元音的i-变异,由于动词的不定式是弱动词的第一基本元,现在时都在第一基本元的基础上添加词尾,因此无论词干长短,弱动词的现在时中一律出现i-变异。

弱动词的第二基本元构成过去式词干,其词干构成和第一基本元(不定式)是密切相关的。具体来说:

\begin{longtable}{l}
  \toprule
  短词干的过去时词干不发生i-变异;长词干的过去式词干发生i-变异(和现在时一致)。 \\
  \midrule
  \endhead
  \bottomrule
  \endfoot
\end{longtable}

读者也可以把这个规律理解为短词干动词现在时的词干元音在过去式中脱落(导致i-变异无法发生),长词干动词的词干元音则在过去时中也保留,如下所示:\footnote{这种理解方法只能帮助读者了解共时问题,但它与历史情况恰恰相反。古诺尔斯语的第一类弱动词过去时词干中的i-变异问题,请参考疑难问题。}

\begin{longtable}{lll}
  \toprule
  基础词干  & 现在时词干 & 过去时词干 \\
  \midrule
  \endhead
  \bottomrule
  \endfoot
  var-i/j-  & ver-j-     & var-       \\
  sigl-i/j- & sigl-i-    & sigl-i-    \\
  sát-i/j-  & sæt-i-     & sæt-i-     \\
  knú-i/j-  & kný-j-     & knú-       \\
  þrá-i/j-  & þrey-j-    & þrá-       \\
\end{longtable}

但根据元音省略规则,过去时词干的词干元音总在-ði-前脱落。因此现在时和过去时的词干差别实际上仅在于i-变异是否发生。

一些值得注意的例外是:

\begin{enumerate}
  \def\labelenumi{\arabic{enumi})}
  \item
        如果长音节词干以-k或-g结尾,词干元音-i-在a/u前变为j,这个规律在ja-词干名词中也有体现:
\end{enumerate}

\begin{longtable}{ll}
  \toprule
  \begin{quote}lág-i/j- `lower‌'\end{quote} & \begin{quote}现在时词干læg-i- + -r \textgreater{} lægir\end{quote} \\
  \midrule
  \endhead
  \bottomrule
  \endfoot
                                           & \begin{minipage}[t]{\linewidth}\raggedright
                                               \begin{quote}
      但有læg-i- + -a \textgreater{} lægja
    \end{quote}
                                             \end{minipage}                         \\
\end{longtable}

\begin{enumerate}
  \def\labelenumi{\arabic{enumi})}
  \setcounter{enumi}{1}
  \item
        某些以-g结尾(但一般不是-k)的短词干动词在现在时词干末尾双写-g:
\end{enumerate}

\begin{longtable}{ll}
  \toprule
  \begin{quote}lag-i/j- `lay‌'\end{quote} & \begin{quote}现在时词干legg-j- + -r \textgreater{} legg\end{quote} \\
  \midrule
  \endhead
  \bottomrule
  \endfoot
\end{longtable}

\begin{quote}
  比较下列以-g结尾的动词:
\end{quote}

\begin{longtable}{lll}
  \toprule
             & 短音节   & 长音节    \\
  \midrule
  \endhead
  \bottomrule
  \endfoot
  词干       & lag-i/j- & talg-i/j- \\
  单数现在时 &          &           \\
  1          & legg     & telgi     \\
  2          & leggr    & telgir    \\
  3          & leggr    & telgir    \\
  复数现在时 &          &           \\
  1          & leggjum  & telgjum   \\
  2          & leggið   & telgið    \\
  3          & leggja   & telgja    \\
  单数过去时 &          &           \\
  1          & lagða    & telgða    \\
  2          & lagðir   & telgðir   \\
  3          & lagði    & telgði    \\
  复数过去时 &          &           \\
  1          & lǫgðum   & telgðum   \\
  2          & lǫgðuð   & telgðuð   \\
  3          & lǫgðu    & telgðu    \\
\end{longtable}

\begin{enumerate}
  \def\labelenumi{\arabic{enumi})}
  \setcounter{enumi}{2}
  \item
        不规则动词。部分长词干动词中的i-变异不规则,这主要是受词根元音后面的辅音的历史音变的影响造成的,我们在此不展开讨论。这类动词只有4个,但都不算罕见,其基本元如下所示:
\end{enumerate}

\begin{longtable}{llll}
  \toprule
  不定式             & 三单现在时  & 三单过去时 & 过去分词 \\
  \midrule
  \endhead
  \bottomrule
  \endfoot
  sœkja/sækja `seek' & sœkir/sækir & sótti      & sóttr    \\
  yrkja `work'       & yrkir       & orti       & ort      \\
  þekkja `know'      & þekkir      & þátti      & þektr    \\
  þykkja `seem'      & þykkir      & þótti      & þóttr    \\
\end{longtable}

弱动词的第三基本元只用在过去分词中。长短词干动词的过去分词也有i-变异的区分,结果和过去时词干一致,即长词干动词发生i-变异,短词干动词不发生i-变异。短词干动词也可以在词干尾加可选的-i,如var\textbf{i}ðr,长词干则没有额外的-i.

动词 verja, varði, variðr/varðr `defend‌', 词干var-i/j-,
是标准的短词干一类弱动词,主动态如下:

\begin{longtable}{llll}
  \toprule
  一类弱动词 & 直陈     & 虚拟   & 祈使   \\
  \midrule
  \endhead
  \bottomrule
  \endfoot
  (var-i/j-) &          &        &        \\
  单数现在时 &          &        &        \\
  1          & ver      & verja  &        \\
  2          & verr     & verir  & ver    \\
  3          & verr     & veri   &        \\
  复数现在时 &          &        &        \\
  1          & verjum   & verim  & verjum \\
  2          & verið    & verið  & verið  \\
  3          & verja    & veri   &        \\
  单数过去时 &          &        &        \\
  1          & varða    & verða  &        \\
  2          & varðir   & verðir &        \\
  3          & varði    & verði  &        \\
  复数过去时 &          &        &        \\
  1          & vǫrðum   & verðim &        \\
  2          & vǫrðuð   & verðið &        \\
  3          & vǫrðu    & verði  &        \\
  不定式     & verja    &        &        \\
  现在分词   & verjandi &        &        \\
  过去分词   & variðr   &        &        \\
\end{longtable}

中动态如下:

\begin{longtable}{llll}
  \toprule
  一类弱动词 & 直陈       & 虚拟     & 祈使     \\
  \midrule
  \endhead
  \bottomrule
  \endfoot
  (var-i/j-) &            &          &          \\
  单数现在时 &            &          &          \\
  1          & verjumk    & verjumk  &          \\
  2          & versk      & verisk   & versk    \\
  3          & versk      & verisk   &          \\
  复数现在时 &            &          &          \\
  1          & verjumsk   & verimsk  & verjumsk \\
  2          & verizk     & verizk   & verizk   \\
  3          & verjask    & verisk   &          \\
  单数过去时 &            &          &          \\
  1          & vǫrðumk    & verðumk  &          \\
  2          & varðisk    & verðisk  &          \\
  3          & varðisk    & verðisk  &          \\
  复数过去时 &            &          &          \\
  1          & vǫrðumsk   & verðimsk &          \\
  2          & vǫrðuzk    & verðizk  &          \\
  3          & vǫrðusk    & verðisk  &          \\
  不定式     & verjask    &          &          \\
  现在分词   & verjandisk &          &          \\
  过去分词   & varizk     &          &          \\
\end{longtable}

动词fella, felldi, felldr `fell (vt.)‌', 词干 fall-i/j-,
是典型的长音节弱动词,主动态如下

\begin{longtable}{llll}
  \toprule
  一类弱动词  & 直陈     & 虚拟    & 祈使   \\
  \midrule
  \endhead
  \bottomrule
  \endfoot
  (fall-i/j-) &          &         &        \\
  单数现在时  &          &         &        \\
  1           & felli    & fella   &        \\
  2           & fellir   & fellir  & fell   \\
  3           & fellir   & felli   &        \\
  复数现在时  &          &         &        \\
  1           & fellum   & fellim  & fellum \\
  2           & fellið   & fellið  & fellið \\
  3           & fella    & felli   &        \\
  单数过去时  &          &         &        \\
  1           & fellda   & fellda  &        \\
  2           & felldir  & felldir &        \\
  3           & felldi   & felldi  &        \\
  复数过去时  &          &         &        \\
  1           & felldum  & felldim &        \\
  2           & fellduð  & felldið &        \\
  3           & felldu   & felldi  &        \\
  不定式      & fella    &         &        \\
  现在分词    & fellandi &         &        \\
  过去分词    & felldr   &         &        \\
\end{longtable}

中动态如下:

\begin{longtable}{llll}
  \toprule
  一类弱动词  & 直陈       & 虚拟      & 祈使     \\
  \midrule
  \endhead
  \bottomrule
  \endfoot
  (fall-i/j-) &            &           &          \\
  单数现在时  &            &           &          \\
  1           & fellumk    & fellumk   &          \\
  2           & fellisk    & fellisk   & fellsk   \\
  3           & fellisk    & fellisk   &          \\
  复数现在时  &            &           &          \\
  1           & fellumsk   & fellimsk  & fellumsk \\
  2           & fellizk    & fellizk   & fellizk  \\
  3           & fellask    & fellisk   &          \\
  单数过去时  &            &           &          \\
  1           & felldumk   & felldumk  &          \\
  2           & felldisk   & felldisk  &          \\
  3           & felldisk   & felldisk  &          \\
  复数过去时  &            &           &          \\
  1           & felldumsk  & felldimsk &          \\
  2           & fellduzk   & felldizk  &          \\
  3           & felldusk   & felldisk  &          \\
  不定式      & fellask    &           &          \\
  现在分词    & fellandisk &           &          \\
  过去分词    & fellzk     &           &          \\
\end{longtable}

\subsection{第二弱变位法}\label{第二弱变位法}

第二弱变位法动词的特征是词干元音-a-,例如kall-a- `call‌', kast-a- `cast‌'.
词干元音-a-存在于各个时态中,但在紧随的元音前脱去。词干元音-a不是重读元音,在含有-u-的词尾前变成u,并且进一步引起词根中重读元音的u-变异。

第二弱变位法动词数量很多,但基本上是最规则的词类。它的三个基本元间没有i-变异的现象,词根元音完全一致,只有少数形式中有明显的u-变异。注意:即便是过去虚拟式当中也不发生i-变异,这是和其他所有强弱动词的重大区别。其原因正是词干元音-a-和-ið-结合,使得过去虚拟式的词尾中出现了不能引起i-变异的双元音(参见\ref{强动词的主动词尾}中对i-变异由来的解释)。

二类弱动词中有两类容易引起问题或混淆的:

\begin{enumerate}
  \def\labelenumi{\arabic{enumi})}
  \item
        \textbf{ja-不定式动词}
\end{enumerate}

有相当一部分动词中词干元音前有-j-,如herja `wage war; harry', bryja
`begin',这些动词仅凭不定式非常无法与短词干的一类弱动词区分开来。这些动词是由名词派生而来,其中的-j-实际上是名词的词干元音(ja-/jō-词干),不过古诺尔斯语中名词的词干元音-j-只在部分形式中出现了,比较herja的原始日耳曼语形式:

\begin{longtable}{llllll}
  \toprule
  ON    & herja    & \textless{} & herr `army' & + & -a    \\
  \midrule
  \endhead
  \bottomrule
  \endfoot
  PGmc. & *harjōną & \textless{} & *harjaz     & + & *-ōną \\
\end{longtable}

这些动词的-j-是词干的一部分,与一类弱动词有根本区别,参考herja的直陈式变位:

\begin{longtable}{lll}
  \toprule
  直陈式 & 现在   & 过去     \\
  \midrule
  \endhead
  \bottomrule
  \endfoot
  1单    & herja  & herjaða  \\
  2单    & herjar & herjaðir \\
  3单    & herjar & herjaði  \\
  1复    & herjum & herjuðum \\
  2复    & herið  & herjuðuð \\
  3复    & herja  & herjuðu  \\
\end{longtable}

\begin{enumerate}
  \def\labelenumi{\arabic{enumi})}
  \setcounter{enumi}{1}
  \item
        \textbf{以-á结尾的动词}
\end{enumerate}

另有一小类动词以长元音-á结尾,如spá `foretell', þrá `desire', fá
`draw'\footnote{注意区分按照七类强动词变形的fá
  `fetch',二者词源不同,弱动词来源于*faihijaną,长元音由ai缩合得到;强动词来源于*fanhaną,长元音来源于-n和-h的脱落。}等。这些动词除了形态略和一般动词不同外,变形实际上没有不规则的情况,参考þrá的直陈式变位:

\begin{longtable}{lll}
  \toprule
  直陈式 & 现在  & 过去   \\
  \midrule
  \endhead
  \bottomrule
  \endfoot
  1单    & þrá   & þráða  \\
  2单    & þrár  & þráðir \\
  3单    & þrár  & þráði  \\
  1复    & þrám  & þráðum \\
  2复    & þráið & þráðuð \\
  3复    & þrá   & þráðu  \\
\end{longtable}

动词kalla, kallaði, kallaðr `call‌', 词干kall-a-,
是标准的第二弱变位法动词,主动态如下:

\begin{longtable}{llll}
  \toprule
  二类弱动词 & 直陈     & 虚拟     & 祈使   \\
  \midrule
  \endhead
  \bottomrule
  \endfoot
  (kall-a-)  &          &          &        \\
  单数现在时 &          &          &        \\
  1          & kalla    & kalla    &        \\
  2          & kallar   & kallir   & kalla  \\
  3          & kallar   & kalli    &        \\
  复数现在时 &          &          &        \\
  1          & kǫllum   & kallim   & kǫllum \\
  2          & kallið   & kallið   & kallið \\
  3          & kalla    & kalli    &        \\
  单数过去时 &          &          &        \\
  1          & kallaða  & kallaða  &        \\
  2          & kallaðir & kallaðir &        \\
  3          & kallaði  & kallaði  &        \\
  复数过去时 &          &          &        \\
  1          & kǫlluðum & kallaðim &        \\
  2          & kǫlluðuð & kallaðið &        \\
  3          & kǫlluðu  & kallaði  &        \\
  不定式     & kalla    &          &        \\
  现在分词   & kallandi &          &        \\
  过去分词   & kallaðr  &          &        \\
\end{longtable}

中动态如下:

\begin{longtable}{llll}
  \toprule
  二类弱动词 & 直陈       & 虚拟       & 祈使     \\
  \midrule
  \endhead
  \bottomrule
  \endfoot
  (kall-a-)  &            &            &          \\
  单数现在时 &            &            &          \\
  1          & kǫllumk    & kǫllumk    &          \\
  2          & kallask    & kallisk    & kallask  \\
  3          & kallask    & kallisk    &          \\
  复数现在时 &            &            &          \\
  1          & kǫllumsk   & kallimsk   & kǫllumsk \\
  2          & kallizk    & kallizk    & kallizk  \\
  3          & kallask    & kallisk    &          \\
  单数过去时 &            &            &          \\
  1          & kǫlluðumk  & kǫlluðumk  &          \\
  2          & kallaðisk  & kallaðisk  &          \\
  3          & kallaðisk  & kallaðisk  &          \\
  复数过去时 &            &            &          \\
  1          & kǫlluðumsk & kallaðimsk &          \\
  2          & kǫlluðuzk  & kallaðizk  &          \\
  3          & kǫlluðusk  & kallaðisk  &          \\
  不定式     & kallask    &            &          \\
  现在分词   & kallandisk &            &          \\
  过去分词   & kallazk    &            &          \\
\end{longtable}

\subsection{第三弱变位法}\label{第三弱变位法}

一小类动词属于第三弱变位法,其特征是词干元音-i-,例如 lif-i- `live‌',
þor-i- `dare‌', vak-i- `wake‌'.
-i-出现在所有现在时中,但在元音前脱去,过去时中没有这个元音。区别于一类弱动词,这里的-i-不会导致i-变异,因为它由PGmc.
*-ai-演变得来(*-ai- \textgreater{} *-e- \textgreater-i-)。

第三弱变位法包括以下一些形态略不规则的动词:

\begin{enumerate}
  \def\labelenumi{\arabic{enumi})}
  \item
        \textbf{以-á结尾的动词}
\end{enumerate}

这类动词主要有两个ná `reach', gá
`heed',它们的变形和第二弱变位法基本一致(除了单数现在时),参考ná的变位:

\begin{longtable}{lll}
  \toprule
  直陈式 & 现在 & 过去  \\
  \midrule
  \endhead
  \bottomrule
  \endfoot
  1单    & nái  & náða  \\
  2单    & náir & náðir \\
  3单    & náir & náði  \\
  1复    & nám  & náðum \\
  2复    & náið & náðuð \\
  3复    & ná   & náðu  \\
\end{longtable}

\begin{enumerate}
  \def\labelenumi{\arabic{enumi})}
  \setcounter{enumi}{1}
  \item
        \textbf{包含i-变异的动词}
\end{enumerate}

古诺尔斯语只有三个动词的现在时词干中有i-变异的痕迹,这是因为在原始日耳曼语中这些动词的词干元音是*-ja-而不是*-ai-.
这三个动词是segja `say', þegja `be silent'和hafa `have'.

segja,
þegja的变形一致,它们的整个现在时词干都发生了i-变异,和第一类弱动词非常相似(但单数现在时的词尾不能用短词干一类弱动词的词尾来解释),例如segja的直陈式为:

\begin{longtable}{lll}
  \toprule
  直陈式 & 现在   & 过去   \\
  \midrule
  \endhead
  \bottomrule
  \endfoot
  1单    & segi   & sagða  \\
  2单    & segir  & sagðir \\
  3单    & segir  & sagði  \\
  1复    & segjum & sǫgðum \\
  2复    & segið  & sǫgðuð \\
  3复    & segja  & sǫgðu  \\
\end{longtable}

hafa的变形和上面的两个动词略有不同,它的复数现在时中没有i-变异:

\begin{longtable}{lll}
  \toprule
  直陈式 & 现在  & 过去   \\
  \midrule
  \endhead
  \bottomrule
  \endfoot
  1单    & hefi  & hafða  \\
  2单    & hefir & hafðir \\
  3单    & hefir & hafði  \\
  1复    & hǫfum & hǫfðum \\
  2复    & hafið & hǫfðuð \\
  3复    & hafa  & hǫfðu  \\
\end{longtable}

在单数现在时中,形如seg, segr; hef, hefr的古老形式也有记载。\footnote{这三个动词在原始语中的形态和在古诺尔斯语的发展颇有争议。现在时中没有-j-的形式可能是由第一类弱动词类比得到。}

我们以动词vaka, vakði `wake‌', 词干vak-i-,
展示标准的第三弱变位法的规则,主动态如下:

\begin{longtable}{llll}
  \toprule
  三类弱动词 & 直陈                                                                                               & 虚拟   & 祈使  \\
  \midrule
  \endhead
  \bottomrule
  \endfoot
  (vak-i-)   &                                                                                                    &        &       \\
  单数现在时 &                                                                                                    &        &       \\
  1          & vaki                                                                                               & vaka   &       \\
  2          & vakir                                                                                              & vakir  & vaki  \\
  3          & vakir                                                                                              & vaki   &       \\
  复数现在时 &                                                                                                    &        &       \\
  1          & vǫkum                                                                                              & vakim  & vǫkum \\
  2          & vakið                                                                                              & vakið  & vakið \\
  3          & vaka                                                                                               & vaki   &       \\
  单数过去时 &                                                                                                    &        &       \\
  1          & vakða                                                                                              & vekða  &       \\
  2          & vakðir                                                                                             & vekðir &       \\
  3          & vakði                                                                                              & vekði  &       \\
  复数过去时 &                                                                                                    &        &       \\
  1          & vǫkðum                                                                                             & vekðim &       \\
  2          & vakðuð                                                                                             & vekðið &       \\
  3          & vakðu                                                                                              & vekði  &       \\
  不定式     & vaka                                                                                               &        &       \\
  现在分词   & vakandi                                                                                            &        &       \\
  过去分词   & vakat (nt.)\footnote{这个动词的过去分词只记录到中性形式。关于过去分词的变形,将在(交叉)中详述。}
             &                                                                                                    &                \\
\end{longtable}

中动态如下:

\begin{longtable}{llll}
  \toprule
  三类弱动词 & 直陈      & 虚拟     & 祈使    \\
  \midrule
  \endhead
  \bottomrule
  \endfoot
  (vak-i-)   &           &          &         \\
  单数现在时 &           &          &         \\
  1          & vǫkumk    & vǫkumk   &         \\
  2          & vakisk    & vakisk   & vakisk  \\
  3          & vakisk    & vakisk   &         \\
  复数现在时 &           &          &         \\
  1          & vǫkumsk   & vakimsk  & vǫkumsk \\
  2          & vakizk    & vakizk   & vakizk  \\
  3          & vakask    & vakisk   &         \\
  单数过去时 &           &          &         \\
  1          & vǫkðumk   & vekðumk  &         \\
  2          & vakðisk   & vekðisk  &         \\
  3          & vakðisk   & vekðisk  &         \\
  复数过去时 &           &          &         \\
  1          & vǫkðumsk  & vekðimsk &         \\
  2          & vakðuzk   & vekðizk  &         \\
  3          & vakðusk   & vekðisk  &         \\
  不定式     & vakask    &          &         \\
  现在分词   & vakandisk &          &         \\
  过去分词   & vakazk    &          &         \\
\end{longtable}

\section{过去-现在混合动词}\label{过去-现在混合动词}

过去-现在混合动词(Preterite-present)指的是该动词的现在时使用强动词的过去词尾,按强动词过去时变位;过去时采取弱动词过去时词尾,按弱动词的过去时变位。造成这个现象的原因是:原始日耳曼语的过去时继承于原始印欧语的完成态,但在这类动词中,完成态演化成了现在时\footnote{造成这种现象的原因还没有共识。可能的解释是,这些动词都是表示状态而非动作的,天生带有完成的含义。},因此原始日耳曼语中过去-现在混合动词的过去时就空缺出来了。对此,这类动词只好采用日耳曼语独立的弱变化结构来构成过去时。

这种新的过去时构成法在这类动词中甚至应用到了不定式上。动词munu `will‌',
skulu `shall‌', 以及 *knega `know, be
able‌'(这个词的不定式没有被直接记录到)有用齿音构成的过去不定式:mundu,
skyldu, knáttu.
这些过去不定式也使用于非限定性的结构,但一般用于主句是过去时的情况。读者可以类别英文中ask
him \textbf{to do}和ask\textbf{ed} him \textbf{to
  do},在英语中只有一个不定式to
do,但在古诺尔斯语中,后一句可以用过去不定式。

由于过去-现在混合动词的过去时按照强变位法变形,其元音变换也有一定的规律,可以大致地和前面的六大规则的强变位法对等起来,例如下面的的第一类动词的现在时中用到了ei和i的交替,和第一强变位法的元音交替有关(不完全一致)。但在古诺尔斯语的十个过去-现在混合动词中,没有一个和第二或第六强变位法的元音交替模式对应。另外,这些动词的含义一般决定了它们没有中动态。

\textbf{第一类}

有两个动词属于这一类,vita `know‌', eiga `have, own‌':

\begin{longtable}{lllll}
  \caption[The First Gradation: Class I]{The First Gradation: Class
    I}\tabularnewline
  \toprule
  第一类混合动词  & \multicolumn{2}{c}{现在} & \multicolumn{2}{c}{过去}                  \\\midrule\endfirsthead\toprule第一类混合动词 &\multicolumn{2}{c}{现在} &\multicolumn{2}{c}{过去} \\
  \midrule
  \endhead
  \bottomrule
  \endfoot
  \textbf{直陈式} & ~                        & ~                        & ~      & ~     \\
  1单             & veit                     & á                        & vissa  & átta  \\
  2单             & veizt                    & átt                      & vissir & áttir \\
  3单             & veit                     & á                        & vissi  & átti  \\
  1复             & vitum                    & eigum                    & vissum & áttum \\
  2复             & vituð                    & eiguð                    & vissuð & áttuð \\
  3复             & vitu                     & eigu                     & vissu  & áttu  \\
  \textbf{虚拟式} & ~                        & ~                        & ~      & ~     \\
  1单             & vita                     & eiga                     & vissa  & ætta  \\
  2单             & vitir                    & eigir                    & vissir & ættir \\
  3单             & viti                     & eigi                     & vissi  & ætti  \\
  1复             & vitim                    & eigim                    & vissim & ættim \\
  2复             & vitið                    & eigið                    & vissið & ættið \\
  3复             & viti                     & eigi                     & vissi  & ætti  \\
  \textbf{祈使式} & ~                        & ~                        & ~      & ~     \\
  2单             & vit                      & eig                      & ~      & ~     \\
  1复             & vitum                    & eigum                    & ~      & ~     \\
  2复             & vituð                    & eiguð                    & ~      & ~     \\
  \textbf{不定式} & vita                     & eiga                     & ~      & ~     \\
  \textbf{分词}   & vitandi                  & eigandi                  & vitaðr & áttr  \\
\end{longtable}

说明:

\begin{enumerate}
  \def\labelenumi{\arabic{enumi})}
  \item
        vita的过去式vissa中的-ss-实际上是由-tt-(\textless{} -t +
        -ð)变化得到,这依旧是日耳曼擦音定律的残留。
  \item
        eiga的变形相比vita不规则一些。其单数现在时á来源于*aig,*aig首先脱去词尾的-g,再发生ai
        \textgreater{} á的缩合。
\end{enumerate}

\begin{quote}
  \textbf{第三类}

  这类动词有三个,其元音交替规律比较规则:

  \textbf{不定式u -\/- 单数现在时a -\/- 复数现在时u -\/- 过去分词u}

  这三个动词是unna `love‌', kunna `know, be able‌', þurfa `need‌':
\end{quote}

\begin{longtable}{lllllll}
  \caption[The Third Gradation: Class III]{The Third Gradation: Class
    III}\tabularnewline
  \toprule
  第三类混合动词  & \multicolumn{3}{c}{现在} & \multicolumn{3}{c}{过去}                                         \\\midrule\endfirsthead\toprule第三类混合动词 &\multicolumn{3}{c}{现在} &\multicolumn{3}{c}{过去} \\
  \midrule
  \endhead
  \bottomrule
  \endfoot
  \textbf{直陈式} & ~                        & ~                        & ~        & ~       & ~      & ~       \\
  1单             & ann                      & kann                     & þarf     & unna    & kunna  & þurfta  \\
  2单             & annt                     & kannt                    & þarft    & unnir   & kunnir & þurftir \\
  3单             & ann                      & kann                     & þarf     & unni    & kunni  & þurfti  \\
  1复             & unnum                    & kunnum                   & þurfum   & unnum   & kunnum & þurftum \\
  2复             & unnuð                    & kunnuð                   & þurfuð   & unnuð   & kunnuð & þurftuð \\
  3复             & unnu                     & kunnu                    & þurfu    & unnu    & kunnu  & þurftu  \\
  \textbf{虚拟式} & ~                        & ~                        & ~        & ~       & ~      & ~       \\
  1单             & unna                     & kunna                    & þurfa    & ynna    & kynna  & þyrfta  \\
  2单             & unnir                    & kunnir                   & þurfir   & ynnir   & kynnir & þyrftir \\
  3单             & unni                     & kunni                    & þurfi    & ynni    & kynni  & þyrfti  \\
  1复             & unnim                    & kunnim                   & þurfim   & ynnim   & kynnim & þyrftim \\
  2复             & unnið                    & kunnið                   & þurfið   & ynnið   & kynnið & þyrftið \\
  3复             & unni                     & kunni                    & þurfi    & ynni    & kynni  & þyrfti  \\
  \textbf{祈使式} & ~                        & ~                        & ~        & ~       & ~      & ~       \\
  2单             & unn                      & kunn                     & -        & ~       & ~      & ~       \\
  1复             & unnum                    & kunnum                   & -        & ~       & ~      & ~       \\
  2复             & unnuð                    & kunnuð                   & -        & ~       & ~      & ~       \\
  \textbf{不定式} & unna                     & kunna                    & þurfa    & ~       & ~      & ~       \\
  \textbf{分词}   & unnandi                  & kunnandi                 & þurfandi & unn(a)t & kunnat &
  þurft                                                                                                         \\
\end{longtable}

\begin{quote}
  说明:
\end{quote}

\begin{enumerate}
  \def\labelenumi{\arabic{enumi})}
  \item
        动词unna和kunna的过去式略不规则,其中没有塞音-ð-的痕迹,但在过去分词中仍有体现。实际上,过去式的-nn-是由*-nþ-变化得到的(参考哥特语过去式kunþa,这个音变还发生在动词finna中,见\ref{第三强变位法}),只是在共时层面上很难发现。
  \item
        þurfa的祈使式没有被记录到。在后面的表格中,一律用``-''表示未被记录到的形式。
\end{enumerate}

\textbf{第四类}

正如三类、四类强动词一样,第三类、第四类的过去-现在混合动词的区别仅在于词干尾的辅音,其元音交替模式是完全一样的。这类动词包括3个,muna
`remember‌', munu `will‌', skulu `shall‌' :

\begin{longtable}{lllllll}
  \caption[The Fourth Gradation: Class IV]{The Fourth Gradation: Class
    IV}\tabularnewline
  \toprule
  第四类混合动词  & \multicolumn{3}{c}{现在} & \multicolumn{3}{c}{过去}                                        \\\midrule\endfirsthead\toprule第四类混合动词 &\multicolumn{3}{c}{现在} &\multicolumn{3}{c}{过去} \\
  \midrule
  \endhead
  \bottomrule
  \endfoot
  \textbf{直陈式} & ~                        & ~                        & ~        & ~      & ~      & ~       \\
  1单             & man                      & mun                      & skal     & munda  & munda  & skylda  \\
  2单             & mant                     & munt                     & skalt    & mundir & mundir & skyldir \\
  3单             & man                      & mun                      & skal     & mundi  & mundi  & skyldi  \\
  1复             & munum                    & munum                    & skulum   & mundum & mundum & skyldum \\
  2复             & munið                    & munuð                    & skuluð   & munduð & munduð & skylduð \\
  3复             & muna                     & munu                     & skulu    & mundu  & mundu  & skyldu  \\
  \textbf{虚拟式} & ~                        & ~                        & ~        & ~      & ~      & ~       \\
  1单             & muna                     & myna                     & skyla    & mynda  & mynda  & skylda  \\
  2单             & munir                    & mynir                    & skylir   & myndir & myndir & skyldir \\
  3单             & muni                     & myni                     & skyli    & myndi  & myndi  & skyldi  \\
  1复             & munim                    & mynim                    & skylim   & myndim & myndim & skyldim \\
  2复             & munið                    & mynið                    & skylið   & myndið & myndið & skyldið \\
  3复             & muni                     & myni                     & skyli    & myndi  & myndi  & skyldi  \\
  \textbf{祈使式} & ~                        & ~                        & ~        & ~      & ~      & ~       \\
  2单             & mun                      & -                        & -        & ~      & ~      & ~       \\
  1复             & munum                    & -                        & -        & ~      & ~      & ~       \\
  2复             & munuð                    & -                        & -        & ~      & ~      & ~       \\
  \textbf{不定式} & muna                     & munu                     & skulu    & -      & mundu  & skyldu  \\
  \textbf{分词}   & munandi                  & -                        & skulandi & munaðr & -      & skyldr  \\
\end{longtable}

说明:

\begin{enumerate}
  \def\labelenumi{\arabic{enumi})}
  \item
        muna和munu本身是同源的,因此在有些形式上必须做出区分。例如muna的过去式munið,
        muna区别于munu所对应的munuð,
        muni;现在虚拟式中前者发生i-变异,后者不发生i-变异。但在约14世纪后,这些形式开始混杂,以至于muna和munu的许多变位在中世纪手稿中混用。例如munu的单数现在时也可以是man,
        mant;虚拟式中也可以不发生i-变异。
  \item
        skulu的直陈过去式和整个虚拟式中都发生了i-变异,但是在现在虚拟式中,i-变异也可以不发生,即skulu的变位中只有过去时全部要发生i-变异。
\end{enumerate}

\textbf{第五类}

这类动词的元音交替也比较规则:

\begin{quote}
  \textbf{不定式e -\/- 单数现在时á -\/- 复数现在时e -\/- 过去分词e}
\end{quote}

这类动词包括两个,mega `be able‌', kná (*kenga), `know, be able‌':

\begin{longtable}{lllll}
  \caption[The Fifth Gradation: Class V]{The Fifth Gradation: Class
    V}\tabularnewline
  \toprule
  第五类混合动词  & \multicolumn{2}{c}{现在} & \multicolumn{2}{c}{过去}                    \\\midrule\endfirsthead\toprule第五类混合动词 &\multicolumn{2}{c}{现在} &\multicolumn{2}{c}{过去} \\
  \midrule
  \endhead
  \bottomrule
  \endfoot
  \textbf{直陈式} & ~                        & ~                        & ~      & ~       \\
  1单             & má                       & kná                      & mátta  & knátta  \\
  2单             & mátt                     & knátt                    & máttir & knáttir \\
  3单             & má                       & kná                      & mátti  & knátti  \\
  1复             & megum                    & knegum                   & máttum & knáttum \\
  2复             & meguð                    & kneguð                   & máttuð & knáttuð \\
  3复             & megu                     & knegu                    & máttu  & knáttu  \\
  \textbf{虚拟式} & ~                        & ~                        & ~      & ~       \\
  1单             & mega                     & knega                    & mætta  & knætta  \\
  2单             & megir                    & knegir                   & mættir & knættir \\
  3单             & megi                     & knegi                    & mætti  & knætti  \\
  1复             & megim                    & knegim                   & mættim & knættim \\
  2复             & megið                    & knegið                   & mættið & knættið \\
  3复             & megi                     & knegi                    & mætti  & knætti  \\
  \textbf{祈使式} & ~                        & ~                        & ~      & ~       \\
  2单             & -                        & -                        & ~      & ~       \\
  1复             & -                        & -                        & ~      & ~       \\
  2复             & -                        & -                        & ~      & ~       \\
  \textbf{不定式} & mega                     & *knega                   & -      & knáttu  \\
  \textbf{分词}   & megandi                  & -                        & mátt   & -       \\
\end{longtable}

说明:

\begin{enumerate}
  \def\labelenumi{\arabic{enumi})}
  \item
        不定式*kenga并没有被记录到,但其过去不定时knáttu并不少见。
\end{enumerate}

\section{不规则动词}\label{不规则动词}

在前面介绍强弱变位法时,我们已经比较详细地介绍了每一类中形态不规则的动词的变形及其来历。这些动词或多或少还可以归类到七类强动词和三类弱动词中,具体来说是valda,vilja和vera三个异态动词。

\begin{enumerate}
  \def\labelenumi{\arabic{enumi}.}
  \item
        valda `cause; dominate‌'
        的现在时词干是vald-,过去分词词干valdin-,现在时系统的变位是规则的,但过去时有明显的不规则。较最早的文献中,单数过去时词干是oll-,复数是ull-,接着加上弱动词的词尾。\footnote{这个动词虽然有强动词的元音交替特征,但它的过去式实际上按弱动词变位。其中,-ll是*-lþ同化得到,因此词干中已经有弱动词的塞音标记。}但后来一般整个过去时词干都变成oll-,最后这个词甚至可以按规则的弱动词变位,且词首的v不脱落,文献中记载到了oldi或者voldi的形式。
  \item
        vilja `want‌'
        的变位和一类弱动词十分相似,但在现在时中词尾-r和-l同化为-ll,即vil + r
        \textgreater{} vill,(注意,其他一类弱动词没有这样的音变,如skilja
        \textgreater{} skil,
        skilr)第二人称单数式有时也写作vilt,类似于过去-现在混合动词。\footnote{vilja的直陈现在时实际上是古诺尔斯语中增补出来的,因此出现了一些不规则现象。在原始语中,现在时中反常的没有直陈式,而只有虚拟式。哥特语中这个动词也没有直陈现在时。}过去时部分规则地按照弱动词变位,齿音以-d-形式实现。它的过去分词是viljat,缺少过去时的标记。另外,这个词也有过去不定式vildu.
  \item
        最不规则的动词是vera `be‌',
        和大多数语言的be动词一样,这是一个异干互补(Suppletion)动词,意味着其词形变化中涉及多个词干:现在时中是*(e)s-
        , 过去时中是*wes-.
        vera中的r是由早期的s变化过来的,因此后期的vera的变形:vera, ert, er,
        var, vart 分别对应早期的vesa, est, es, vas,
        vast\footnote{但是,复数式中并没有记录到含有s的形式。},这一变化大致在1100年左右完成,前者成为主流的用法。vera的变形经常和前后的词合写,这时的vera类似于一种后缀,例如:
\end{enumerate}

\begin{quote}
  nús \textless{} nú es `now is‌',

  þaz \textless{} þat es `that is‌',

  vér(r)óm \textless{} vér erum `we are‌',

  þeir(r)ó \textless{} þeir ero `they are‌',

  emk \textless{} em ek `I am'

  sják \textless{} sjá ek `I may be'
\end{quote}

上述的三个动词的主动态如下,除valda以外的动词没有中动态:

\begin{longtable}{lllllllll}
  \toprule
  不规则动词        & 现在     & 过去               &  & 现在       & 过去   &  & 现在     & 过去       \\
  \midrule
  \endhead
  \bottomrule
  \endfoot
  \textbf{直陈式}   &          &                    &  &            &        &  &          &            \\
  1单               & veld     & olla               &  & vil        & vilda  &  & em       & var, vas   \\
  2单               & veldr    & ollir              &  & vill, vilt & vildir &  & ert, est & vart, vast \\
  3单               & veldr    & olli (oldi, voldi) &  & vill       & vildi  &  & er, es   & var,
  vas                                                                                                   \\
  1复               & vǫldum   & ullum, ollum       &  & viljum     & vildum &  & erum     & várum      \\
  2复               & valdið   & ulluð, olluð       &  & vilið      & vilduð &  & eruð     & váruð      \\
  3复               & valda    & ullu, ollu         &  & vilja      & vildu  &  & eru      & váru       \\
  \textbf{虚拟式}   &          &                    &  &            &        &  &          &            \\
  1单               & valda    & ylla               &  & vilja      & vilda  &  & sjá, sé  & væra       \\
  2单               & valdir   & yllir              &  & vilir      & vildir &  & sér      & værir      \\
  3单               & valdi    & ylli (vyldi)       &  & vili       & vildi  &  & sé       & væri       \\
  1复               & valdim   & yllim              &  & vilim      & vildim &  & sém      & værim      \\
  2复               & valdið   & yllið              &  & vilið      & vildið &  & séð, sét & værið      \\
  3复               & valdi    & ylli               &  & vili       & vildi  &  & sé       & væri       \\
  \textbf{祈使式}   &          &                    &  &            &        &  &          &            \\
  2单               &          &                    &  &            &        &  & ver      &            \\
  2 复              &          &                    &  &            &        &  & verið    &            \\
  \textbf{不定式}   & valda    &                    &  & vilja      & vildu  &  & vera     &            \\
  \textbf{过去分词} & valdandi & valdit             &  & viljandi   & viljat &  &          &
  verit                                                                                                 \\
\end{longtable}

注意vera的现在虚拟式中的许多形式与sjá `see‌'一致,必须按照句意判断。

\chapter{形容词与变格法}\label{形容词与变格法}

\begin{quote}
  \textbf{章节要点:}
\end{quote}

\begin{itemize}
  \item
        \begin{quote}
          形容词性词尾
        \end{quote}
  \item
        \begin{quote}
          形容词的强弱变格法及其含义
        \end{quote}
  \item
        \begin{quote}
          形容词的级
        \end{quote}
  \item
        \begin{quote}
          副词的构成与变格法
        \end{quote}
  \item
        \begin{quote}
          分词的构成与变格法
        \end{quote}
\end{itemize}

\section{形容词的概述}\label{形容词的概述}

形容词最常见的用法有两种:作定语或作表语,但无论是哪一种,形容词的存在都与一个名词密不可分(作定语时,与其所修饰的名词;作表语时,与作主语的名词)。名词的四个基本属性是:格、性、数、特指性,因而形容词的各项属性要和名词保持一一对应。

形容词可以按格、性、数进行变化,这一点与大多数印欧屈折语都是类似的,但在古诺尔斯语中,每个形容词还有强变格和弱变格两种形式。一般来说,当形容词和非特指名词发生关系时,要用其强变格形式;反之,如果句中出现的是特指名词,对应地就要使用弱变格的形容词。因此,形容词的强弱实际上是与名词的特指性相呼应的属性。类似于名词,强变格和弱变格分别对应了一套按格、性、数变化的词尾,但与名词不同的是,名词的强弱是其固有属性,每个名词只能接续一套词尾;而形容词的强弱是一种可变的属性,它在一定程度上反映的是语义的区别,每个形容词都既可以按强变格法变形,也可以按弱变格法变形。

古诺尔斯语的形容词也有``级''的概念,分别是原级、比较级、最高级。前文所述的变形方式是针对形容词的基础形态---原级的,比较级和最高级的变形则略有不同。它们先在词干后添加一个后缀-ar-(比较级)/-ast-(最高级),然后再添加变格词尾。对于比较级而言,其变格有一套专门的词尾,形式上和弱变格词尾非常相似。所有形容词的比较级无论是否修饰了特指名词一律都按弱变格词尾变形,即形容词的比较级不区分强弱。形容词的最高级没有专门的词尾,它和原级一样可以按照强变格和弱变格进行变化。

和许多古代印欧语类似,形容词的词尾和名词的词尾也有很大的相似性。读者可能会联想到,形容词是否也像名词一样区分词干的类型呢?在原始日耳曼语中,词干的类型的确影响形容词的屈折,但在古诺尔斯语中,形容词的变形已经大大规则化了。类似于ja-词干名词这样的情况在形容词中并没有出现,因此,没有必要介绍形容词原来的词干形式。

形容词的词尾不仅用在形容词上,还有一些词类也添加和形容词的词尾,例如动词的现在分词和过去分词、代词(交叉引用)等,这些词在词类上并非严格的形容词。因此,我们把形容词所添加的词尾称为``形容词性词尾'',这些词尾以后将经常用到。

形容词作谓语时,要用其强变格形式。这时候的形容词就是形容词性的,但古诺尔斯语也可以把形容词作名词用,此时的强弱就要根据句义中名词是否特指来决定了。在诗歌中,作定语的形容词几乎都是强的,即便其所修饰的是特指名词,如:
fyrr vil ek kyssa konung ólifðan, en þú blóðogri bryniu kastir `sooner
will I kiss the lifeless king, than you cast off the bloody byrnie‌'.
这种用法并不局限于诗歌,事实上在散文中也不罕见: með fǫður sinn gamlan
`with his old
father‌'.这里反身代词sinn使fǫður成为特指词,但形容词依然使用了强形式。有时在绰号之类中也用强形容词,当然弱形容词更常用一些,比如:
Eiríkr rauðr `Erik the Red‌'. 散文中,在表示呼唤的情况下也常用强形容词:
forða þér, vesall maðr! `save yourself, unfortunate man!‌'

古诺尔斯语中的一些形容词承担了英语中需要使用短语的情况,这些形容词是强的,即便用上了特指冠词/后缀也不例外,如:í
miðjum hauginum `in the middle of the hill‌'; um þueran skálann `straight
across the hall‌'; ǫndorðan vetr `the first part of the winter‌'.
形容词甚至可以修饰代词,如þeir margir `many of them‌'.

诗歌和某些小众风格的文章经常把强形容词作名词用,如: blindr er betri, en
brendr sé `a blind (man) is better off than a burned (man) would be‌';
hvat muntu, ríkr, vinna `what would you tell, powerful (man)‌'; rétt
`Right (vs.
Wrong)‌'。作名词用时,弱形容词和定冠词一道使用是很少见的(我们已经说过,冠词常常省略),在诗歌中也很难见到。散文中这种用法局限于表达专有名词,如:inir
ensku `the English‌'. 这种用法在名词序列中也可以使用,如:inn yngri, inn
ellri `the younger, the older‌'.

在散文中,也可以在人名后加上表褒贬的修饰形容词,这时不加定冠词使用,
如:Hákon góði `Hakon the Good‌'. 类似的用法还有: fyrra sumar `last
summer‌'; á vinstri hlið `on the left side‌'; í næsta hús `in the next
house‌'; við þriðja mann `with the third man‌'.

\section{形容词的强变格法}\label{形容词的强变格法}

形容词的强变格词尾和名词稍有不同,列举在下表。一切以u开头的词尾都是导致前方的元音发生u-变异,某些词尾的u可能已经脱去,但u-变异仍旧存在。这种情况下,另标在表中:

\begin{longtable}{llll}
  \toprule
       & 阳性 & 阴性              & 中性              \\
  \midrule
  \endhead
  \bottomrule
  \endfoot
  单数 &      &                   &                   \\
  N    & -r   & (词干u-变异) + -ø & -t                \\
  A    & -an  & -a                & -t                \\
  G    & -s   & -rar              & -s                \\
  D    & -um  & -ri               & -u                \\
  复数 &      &                   &                   \\
  N    & -ir  & -ar               & (词干u-变异) + -ø \\
  A    & -a   & -ar               & (词干u-变异) + -ø \\
  G    & -ra  & -ra               & -ra               \\
  D    & -um  & -um               & -um               \\
\end{longtable}

一些规律可供参考:

\begin{longtable}{l}
  \toprule
  1. -r是阳性单数主格的标记,这也作为形容词的基本形式。 \\
  \midrule
  \endhead
  \bottomrule
  \endfoot
  2. 复数属格都是-ra,复数与格都是-um.                  \\
  3. 单数属格的词尾和名词类似,但阴性词尾是-rar.        \\
  4. 中性名词不论单数复数,主格和宾格总一样。           \\
\end{longtable}

以sterkr `strong‌', vænn `handsome‌', gamall `old‌', nýr `new‌', frægr
`famous‌', rǫskr `brave‌', fagr `beautiful, fair'
为例,它们的阳性形式如下:

\begin{longtable}{llllllll}
  \toprule
  词干 & sterk-  & væn-  & gamal-  & fagr-  & nýj-    & frægj-  & rǫskv-  \\
  \midrule
  \endhead
  \bottomrule
  \endfoot
  单数 &         &       &         &        &         &         &         \\
  N    & sterkr  & vænn  & gamall  & fagr   & nýr     & frægr   & rǫskr   \\
  A    & sterkan & vænan & gamlan  & fagran & nýjan   & frægjan & rǫskvan \\
  G    & sterks  & væns  & gamals  & fagrs  & nýs     & frægs   & rǫsks   \\
  D    & sterkum & vænum & gǫmlum  & fǫgrum & nýjum   & frægjum & rǫskum  \\
  复数 &         &       &         &        &         &         &         \\
  N    & sterkir & vænir & gamlir  & fagrir & nýir    & frægir  & rǫskvir \\
  A    & sterka  & væna  & gamla   & fagra  & nýja    & frægja  & rǫskva  \\
  G    & sterkra & vænna & gamalla & fagra  & nýr(r)a & frægra  & rǫskra  \\
  D    & sterkum & vænum & gǫmlum  & fǫgrum & nýjum   & frægjum & rǫskum  \\
\end{longtable}

其阴性形式如下:

\begin{longtable}{llllllll}
  \toprule
  词干 & sterk-   & væn-   & gamal-   & fagr-  & nýj-     & frægj-  & rǫskv-  \\
  \midrule
  \endhead
  \bottomrule
  \endfoot
  单数 &          &        &          &        &          &         &         \\
  N    & sterk    & væn    & gǫmul    & fǫgr   & ný       & fræg    & rǫsk    \\
  A    & sterka   & væna   & gamla    & fagra  & nýja     & frægja  & rǫskva  \\
  G    & sterkrar & vænnar & gamallar & fagrar & nýr(r)ar & frægrar &
  rǫskrar                                                                     \\
  D    & sterkri  & vænni  & gamalli  & fagri  & nýrri    & frægri  & rǫskri  \\
  复数 &          &        &          &        &          &         &         \\
  N    & sterkar  & vænar  & gamlar   & fagrar & nýjar    & frægjar & rǫskvar \\
  A    & sterkar  & vænar  & gamlar   & fagrar & nýjar    & frægjar & rǫskvar \\
  G    & sterkra  & vænna  & gamalla  & fagra  & nýr(r)a  & frægra  & rǫskra  \\
  D    & sterkum  & vænum  & gǫmlum   & fǫgrum & nýjum    & frægjum & rǫskum  \\
\end{longtable}

其中性形式如下:

\begin{longtable}{llllllll}
  \toprule
  词干 & sterk-  & væn-  & gamal-  & fagr-  & nýj-    & frægj-  & rǫskv- \\
  \midrule
  \endhead
  \bottomrule
  \endfoot
  单数 &         &       &         &        &         &         &        \\
  N    & sterkt  & vænt  & gamalt  & fagrt  & nýtt    & frægt   & rǫskt  \\
  A    & sterkt  & vænt  & gamalt  & fagrt  & nýtt    & frægt   & rǫskt  \\
  G    & sterks  & væns  & gamals  & fagrs  & nýs     & frægs   & rǫsks  \\
  D    & sterku  & vænu  & gǫmlu   & fǫgru  & nýju    & frægju  & rǫsku  \\
  复数 &         &       &         &        &         &         &        \\
  N    & sterk   & væn   & gǫmul   & fǫgr   & ný      & fræg    & rǫsk   \\
  A    & sterk   & væn   & gǫmul   & fǫgr   & ný      & fræg    & rǫsk   \\
  G    & sterkra & vænna & gamalla & fagra  & nýr(r)a & frægra  & rǫskra \\
  D    & sterkum & vænum & gǫmlum  & fǫgrum & nýjum   & frægjum & rǫskum \\
\end{longtable}

说明:

\begin{enumerate}
  \def\labelenumi{\arabic{enumi})}
  \item
        vænn和gamall中出现了大量的同化现象,对于所有以-r起首的词尾,都触发\ref{辅音的音变}的辅音同化。
  \item
        gamall的代表了典型的多音节形容词的变格,第二个音节的弱读元音a在以元音开头的词尾前省略,但在辅音前则完全保留。古诺尔斯语中多音节的非派生形容词本身很少,除了gamall之外,还有heilagr
        `holy', ǫfugr
        `harsh'等。但是,绝大多数由名词、动词等派生出的形容词都是多音节的,它们不都符合元音省略的规律。例如-ligr,
        - -aðr总是不发生省略,但是-ull, -igr,
        -ugr一般都发生省略。双写辅音后的元音也不发生省略,试比较:
\end{enumerate}

\begin{quote}
  samligr `friendly': sam-lig- + -an \textgreater{} samligan

  auðigr `rich': auð-ig- + -um \textgreater{} auðgum

  minnigr `mindful': minn-ig- + -a \textgreater{} minniga
\end{quote}

\begin{enumerate}
  \def\labelenumi{\arabic{enumi})}
  \setcounter{enumi}{2}
  \item
        部分形容词的词干上有增音-j或-v,它们的出现条件符合\ref{半元音的保持性}中的规律。另外,-j和-v分别触发了整个词干中的i-变异或u-变异,这虽然在共时系统中不造成任何区分,但在一定程度上可以提示rǫskr的词干是包括v的。
  \item
        nýr展示了长音节形容词的变格,添加辅音词尾时,触发了\ref{辅音的音变}的辅音延长。一些早期的文本中nýra也是可行的变格,但后来逐渐都变为更规则的nýrra,但是nýtt中的中性词尾-t始终双写。
  \item
        某些形容词词干本身以-r结尾,如fagr,这时再添加辅音词尾时触发\ref{辅音的音变}的辅音简化。因此其阳性单数主格形式不能区分词干与词尾。
\end{enumerate}

\section{形容词的弱变格法}\label{形容词的弱变格法}

形容词的弱变格法词尾比较简单,它们的单数形式和名词的弱变格词尾完全一致,复数形式虽然与名词有区别,但更加简单且各个性都完全一样:

\begin{longtable}{llll}
  \toprule
       & 阳性 & 阴性 & 中性 \\
  \midrule
  \endhead
  \bottomrule
  \endfoot
  单数 &      &      &      \\
  N    & -i   & -a   & -a   \\
  A    & -a   & -u   & -a   \\
  G    & -a   & -u   & -a   \\
  D    & -a   & -u   & -a   \\
  复数 &      &      &      \\
  N    & -u   & -u   & -u   \\
  A    & -u   & -u   & -u   \\
  G    & -u   & -u   & -u   \\
  D    & -um  & -um  & -um  \\
\end{longtable}

我们仍然使用上面已经提到的几个十分常见的形容词作为例子,比较它们的弱变格阳性形式:

\begin{longtable}{llllllll}
  \toprule
  词干 & sterk-  & væn-  & gamal- & fagr-  & nýj-  & frægj-  & rǫskv- \\
  \midrule
  \endhead
  \bottomrule
  \endfoot
  单数 &         &       &        &        &       &         &        \\
  N    & sterki  & væni  & gamli  & fagri  & ný    & frægi   & rǫskvi \\
  A    & sterka  & væna  & gamla  & fagra  & nýja  & frægja  & rǫskva \\
  G    & sterka  & væna  & gamla  & fagra  & nýja  & frægja  & rǫskva \\
  D    & sterka  & væna  & gamla  & fagra  & nýja  & frægja  & rǫskva \\
  复数 &         &       &        &        &       &         &        \\
  N    & sterku  & vænu  & gǫmlu  & fǫgru  & nýju  & frægju  & rǫsku  \\
  A    & sterku  & vænu  & gǫmlu  & fǫgru  & nýju  & frægju  & rǫsku  \\
  G    & sterku  & vænu  & gǫmlu  & fǫgru  & nýju  & frægju  & rǫsku  \\
  D    & sterkum & vænum & gǫmlum & fǫgrum & nýjum & frægjum & rǫskum \\
\end{longtable}

其阴性形式如下:

\begin{longtable}{llllllll}
  \toprule
  词干 & sterk-  & væn-  & gamal- & fagr-  & nýj-  & frægj-  & rǫskv- \\
  \midrule
  \endhead
  \bottomrule
  \endfoot
  单数 &         &       &        &        &       &         &        \\
  N    & sterka  & væna  & gamla  & fagra  & nýja  & frægja  & rǫskva \\
  A    & sterku  & vænu  & gǫmlu  & fǫgru  & nýju  & frægju  & rǫsku  \\
  G    & sterku  & vænu  & gǫmlu  & fǫgru  & nýju  & frægju  & rǫsku  \\
  D    & sterku  & vænu  & gǫmlu  & fǫgru  & nýju  & frægju  & rǫsku  \\
  复数 &         &       &        &        &       &         &        \\
  N    & sterku  & vænu  & gǫmlu  & fǫgru  & nýju  & frægju  & rǫsku  \\
  A    & sterku  & vænu  & gǫmlu  & fǫgru  & nýju  & frægju  & rǫsku  \\
  G    & sterku  & vænu  & gǫmlu  & fǫgru  & nýju  & frægju  & rǫsku  \\
  D    & sterkum & vænum & gǫmlum & fǫgrum & nýjum & frægjum & rǫskum \\
\end{longtable}

其中性形式如下:

\begin{longtable}{llllllll}
  \toprule
  词干 & sterk-  & væn-  & gamal- & fagr-  & nýj-  & frægj-  & rǫskv- \\
  \midrule
  \endhead
  \bottomrule
  \endfoot
  单数 &         &       &        &        &       &         &        \\
  N    & sterka  & væna  & gamla  & fagra  & nýja  & frægja  & rǫskva \\
  A    & sterka  & væna  & gamla  & fagra  & nýja  & frægja  & rǫskva \\
  G    & sterka  & væna  & gamla  & fagra  & nýja  & frægja  & rǫskva \\
  D    & sterka  & væna  & gamla  & fagra  & nýja  & frægja  & rǫskva \\
  复数 &         &       &        &        &       &         &        \\
  N    & sterku  & vænu  & gǫmlu  & fǫgru  & nýju  & frægju  & rǫsku  \\
  A    & sterku  & vænu  & gǫmlu  & fǫgru  & nýju  & frægju  & rǫsku  \\
  G    & sterku  & vænu  & gǫmlu  & fǫgru  & nýju  & frægju  & rǫsku  \\
  D    & sterkum & vænum & gǫmlum & fǫgrum & nýjum & frægjum & rǫskum \\
\end{longtable}

\section{形容词的比较级和最高级}\label{形容词的比较级和最高级}

绝大多数形容词通过在词干后添加-ar-,进一步添加词尾来形成比较级,最高级的构成也很类似,将-ar-换成-ast-,添加对应的词尾的即可。最高级添加的词尾就是原级的词尾,它可以是强的,也可以是弱的。比较级只添加弱词尾,且与原级略有区别。它们的阴性单数以及整个复数(除与格外)不相同,比较级的词尾是:

\begin{longtable}{llll}
  \toprule
       & 阳性 & 阴性 & 中性 \\
  \midrule
  \endhead
  \bottomrule
  \endfoot
  单数 &      &      &      \\
  N    & -i   & -i   & -a   \\
  A    & -a   & -i   & -a   \\
  G    & -a   & -i   & -a   \\
  D    & -a   & -i   & -a   \\
  复数 &      &      &      \\
  N    & -i   & -i   & -i   \\
  A    & -i   & -i   & -i   \\
  G    & -i   & -i   & -i   \\
  D    & -um  & -um  & -um  \\
\end{longtable}

形容词比较级只添加弱词尾,但并不表示所有的形容词比较级都只修饰特指名词。修饰非特指名词一样可以使用形容词比较级,只不过不使用额外的词尾。形容词hvass
`sharp‌'的比较级如下:

\begin{longtable}{llll}
  \toprule
       & 阳性      & 阴性      & 中性      \\
  \midrule
  \endhead
  \bottomrule
  \endfoot
  单数 &           &           &           \\
  N    & hvassari  & hvassari  & hvassara  \\
  A    & hvassara  & hvassari  & hvassara  \\
  G    & hvassara  & hvassari  & hvassara  \\
  D    & hvassara  & hvassari  & hvassara  \\
  复数 &           &           &           \\
  N    & hvassari  & hvassari  & hvassari  \\
  A    & hvassari  & hvassari  & hvassari  \\
  G    & hvassari  & hvassari  & hvassari  \\
  D    & hvǫssurum & hvǫssurum & hvǫssurum \\
\end{longtable}

形容词的最高级像一般的形容词一样变格,即在插入-ast-后根据形容词的强弱添加强变格法词尾或弱变格法词尾,注意有时受u-变异影响,-ast-会变成-ust-,后者进一步触发词根元音的u-变异,类似于二类弱动词。例如hvass的最高级为:

强变格形式:

\begin{longtable}{llll}
  \toprule
       & 阳性       & 阴性        & 中性       \\
  \midrule
  \endhead
  \bottomrule
  \endfoot
  单数 &            &             &            \\
  N    & hvassastr  & hvǫssust    & hvassast   \\
  A    & hvassastan & hvassasta   & hvassast   \\
  G    & hvassasts  & hvassastrar & hvassasts  \\
  D    & hvǫssustum & hvassastri  & hvǫssustu  \\
  复数 &            &             &            \\
  N    & hvassastir & hvassastar  & hvǫssust   \\
  A    & hvassasta  & hvassastar  & hvǫssust   \\
  G    & hvassastra & hvassastra  & hvassastra \\
  D    & hvǫssustum & hvǫssustum  & hvǫssustum \\
\end{longtable}

弱变格形式:

\begin{longtable}{llll}
  \toprule
       & 阳性       & 阴性       & 中性       \\
  \midrule
  \endhead
  \bottomrule
  \endfoot
  单数 &            &            &            \\
  N    & hvassasti  & hvassasta  & hvassasta  \\
  A    & hvassasta  & hvǫssustu  & hvassasta  \\
  G    & hvassasta  & hvǫssustu  & hvassasta  \\
  D    & hvassasta  & hvǫssustu  & hvassasta  \\
  复数 &            &            &            \\
  N    & hvǫssustu  & hvǫssustu  & hvǫssustu  \\
  A    & hvǫssustu  & hvǫssustu  & hvǫssustu  \\
  G    & hvǫssustu  & hvǫssustu  & hvǫssustu  \\
  D    & hvǫssustum & hvǫssustum & hvǫssustum \\
\end{longtable}

双音节的形容词的比较级一般也会发生省略,这使得词缀-ar-都变成了变成-r,例如:

\begin{quote}
  auðig- + -ar- + -i \textgreater{} auðigri
\end{quote}

但是,形容词的最高级很少缩略-ast-,但有时缩略词干的弱读元音,例如:

\begin{quote}
  auðig- + -ast- + -r \textgreater{} auðgastr
\end{quote}

当然,不发生省略的形式也经常出现,特别是在现代冰岛语中,省略的现象减少了。

另有一部分形容词通过插入-r-或-st-来分别构成比较级和最高级,同时词根元音发生i-变异\footnote{造成i-变异的原因是词尾-r/-st来自于更早的*-iz/*-ist,而一般的-ar/-ast则来自于*-ōz/*-ōst,这两套词尾接续的词干不同。前者似乎可以适用于各类词干,但后者只用于最常见的a-/ō-词干形容词(试比较古诺尔斯语a-词干/ō-词干名词,它们的词干元音都合并为-a,见2.2.3),由于a-/ō-词干形容词的广泛性,*-iz/*-ist逐渐被弃用,因此只留下了非常少的痕迹。}。词尾和插入-ar-和-ast-一致。一些非常常见的例子如下所示:

\begin{longtable}{llll}
  \toprule
  词干           & 原级   & 比较级  & 最高级 (弱,强)     \\
  \midrule
  \endhead
  \bottomrule
  \endfoot
  fá- `few‌'      & fár    & færi    & fæsti, fæstr       \\
  fagr- `fair‌'   & fagr   & fegri   & fegrsti, fegrstr   \\
  há- `high‌'     & hár    & hæri    & hæsti, hæstr       \\
  lág- `low‌'     & lágr   & lægri   & lægsti, lægstr     \\
  sein- `late‌'   & seinn  & seinni  & seinsti, seinstr   \\
  skamm- `short‌' & skammr & skemmri & skemmsti, skemmstr \\
  smá- `small‌'   & smár   & smæri   & smæsti, smæstr     \\
  stór- `big‌'    & stórr  & stœri   & stœrsti, stœrstr   \\
  lang- `long‌'   & langr  & lengri  & lengsti, lengstr   \\
  ung- `young‌'   & ungr   & yngri   & yngsti, yngstr     \\
\end{longtable}

注意:fár, hár,
smár的单数阳性主格词尾-r没有被延长,但是其他以-r开头的词尾都被延长,例如fárri,
fárrar等。

有些形容词将两种方式混合使用,这意味着这类词的比较级和最高级都有两写。例如:

\begin{longtable}{llll}
  \toprule
  \begin{quote}djúpr `deep‌' \textgreater{}\end{quote} & \begin{quote}djúpari,\end{quote}            & \begin{quote}djúpasti,\end{quote}           & \begin{quote}djúpastr\end{quote} \\
  \midrule
  \endhead
  \bottomrule
  \endfoot
  \begin{minipage}[t]{\linewidth}\raggedright
    \begin{quote}
      或
    \end{quote}
  \end{minipage}         & \begin{minipage}[t]{\linewidth}\raggedright
                             \begin{quote}
      dýpri,
    \end{quote}
                           \end{minipage} & \begin{minipage}[t]{\linewidth}\raggedright
                                              \begin{quote}
      dýpsti,
    \end{quote}
                                            \end{minipage} & \begin{minipage}[t]{\linewidth}\raggedright
                                                               \begin{quote}
      dýpstr
    \end{quote}
                                                             \end{minipage}                                                                              \\
  \begin{minipage}[t]{\linewidth}\raggedright
    \begin{quote}
      frægr `famous‌' \textgreater{}
    \end{quote}
  \end{minipage}         & \begin{minipage}[t]{\linewidth}\raggedright
                             \begin{quote}
      frægjari,
    \end{quote}
                           \end{minipage} & \begin{minipage}[t]{\linewidth}\raggedright
                                              \begin{quote}
      frægjasti,
    \end{quote}
                                            \end{minipage} & \begin{minipage}[t]{\linewidth}\raggedright
                                                               \begin{quote}
      frægjastr
    \end{quote}
                                                             \end{minipage}                                                                              \\
  \begin{minipage}[t]{\linewidth}\raggedright
    \begin{quote}
      或
    \end{quote}
  \end{minipage}         & \begin{minipage}[t]{\linewidth}\raggedright
                             \begin{quote}
      frægri,
    \end{quote}
                           \end{minipage} & \begin{minipage}[t]{\linewidth}\raggedright
                                              \begin{quote}
      frægsti,
    \end{quote}
                                            \end{minipage} & \begin{minipage}[t]{\linewidth}\raggedright
                                                               \begin{quote}
      frægstr
    \end{quote}
                                                             \end{minipage}                                                                              \\
\end{longtable}

\section{不规则形容词}\label{不规则形容词}

古诺尔斯语有三类不规则形容词。第一类形容词形态上比较异常,这类形容词有且仅有一个:annarr;第二类形容词的原级、比较级和最高级采用了不同的词干,类似于英语中good---better---best;第三类形容词缺少原级,只有比较级和最高级形式,它们是一些副词的派生词。

\begin{enumerate}
  \def\labelenumi{\arabic{enumi})}
  \item
        \textbf{不规则形容词annarr}
\end{enumerate}

annarr `other, another; second, next‌'
是古诺尔斯语中最不规则的形容词。它不仅在形态上十分特殊,而且缺少比较级和最高级形。annarr只按原级的强变格变形,没有弱变格形式,其强变形同时承担了修饰特指和非特指名词的功能。它的变格中有词干ann-和aðr-的交替,这是因为在古诺尔斯语中,-nn-有时在-r前变为-ð-,另见maðr的变格(参见2.2.6其他辅音词干)。annarr完整的变格形式如下:

\begin{longtable}{llll}
  \toprule
       & 阳性    & 阴性     & 中性    \\
  \midrule
  \endhead
  \bottomrule
  \endfoot
  单数 &         &          &         \\
  N    & annarr  & ǫnnur    & annat   \\
  A    & annan   & aðra     & annat   \\
  G    & annars  & annarrar & annars  \\
  D    & ǫðrum   & annarri  & ǫðrum   \\
  复数 &         &          &         \\
  N    & aðrir   & aðrar    & ǫnnur   \\
  A    & aðra    & aðrar    & ǫnnur   \\
  G    & annarra & annarra  & annarra \\
  D    & ǫðrum   & ǫðrum    & ǫðrum   \\
\end{longtable}

\begin{enumerate}
  \def\labelenumi{\arabic{enumi})}
  \setcounter{enumi}{1}
  \item
        \textbf{异干互补形容词}
\end{enumerate}

这类形容词的原级、比较级和最高级采用了不同的词干,如下所示:

\begin{longtable}{lll}
  \toprule
  原级词干          & 比较级词干   & 最高级词干 \\
  \midrule
  \endhead
  \bottomrule
  \endfoot
  góð- `good‌'       & betr-        & bezt-      \\
  ill-, vánd- `bad‌' & verr-        & verst-     \\
  mikil- `great‌'    & meir-        & mest-      \\
  lítil- `little‌'   & minn-        & minnst-    \\
  marg- `many‌'      & fleir-       & flest-     \\
  gamal - `old‌'     & eldr-, ellr- & elzt-      \\
\end{longtable}

这些形容词基本都有英语的对应,它们在英语中也是异干互补的。异干互补形容词的比较级和最高级实际上是相对规则的,它们的标志一般是更古老的-r-/-st-,-st-有时和齿音合写为-zt-.
另外,比较语言学的研究表明这些形容词的比较级和最高级反而比较容易在其他印欧语中找到同源词,因此,这些形容词过去可能是规则的,但是新产生的形容词取代了它们的原级。

在词形变化上,有一些值得注意的问题:

\begin{enumerate}
  \def\labelenumi{\alph{enumi})}
  \item
        góðr的单数中性主格和宾格有同化现象,另外还有元音缩短:góð- + -t
        \textgreater{} gótt \textgreater{} gott,其他形式是规则的。
  \item
        ill-不能再和-r同化,因此阳性单数主格就是illr,illr的变形完全规则。
  \item
        mikill的变格有一些不规则之处。其单数阳性宾格是mikinn而非†mikilan,单数中性主格和宾格是mikit而非†mikilt.
        另外,作为双音节形容词,mikill也规则地发生弱读元音省略,得到如miklum,
        miklir这样的形式。
  \item
        lítill的变格类似于mikill,也有单数阳性宾格lítinn,单数中性主格和宾格lítit.
        同样地,作为一个双音节形容词,它也符合元音省略的条件,但是在这些发生省略的形式中,长元音í缩短为i,因此有litlum,
        litlir这样的形式。
  \item
        margr的单数中性主格和宾格是mart而非†margt.
  \item
        gamall是双音节形容词,也发生元音缩略。
\end{enumerate}

\begin{enumerate}
  \def\labelenumi{\arabic{enumi})}
  \setcounter{enumi}{2}
  \item
        \textbf{不完全变化形容词}
\end{enumerate}

一些形容词缺少原级,只有比较级和最高级的形式,这类形容词被称为``不完全变化的''(Defective)。它们大多从一些副词派生而来,其语义就是对应副词含义的比较级和最高级。读者可以理解为这些副词``承担''了这类形容词的原级:

\begin{longtable}{lll}
  \toprule
  副词                & 比较级词干      & 最高级词干       \\
  \midrule
  \endhead
  \bottomrule
  \endfoot
  aptr `back'         & aptar-, eptr-   & aptast-, epzt-   \\
  austr `east‌'        & eystr-          & austast-         \\
  fyrr `before‌'       & fyrr-           & fyrst-           \\
  *hinder `behind'    & hindr-          & hinzt-           \\
  inn `in, into‌'      & innr-           & innst-           \\
  niðr `down'         & neðr-           & nezt-            \\
  norðr `north‌'       & norðar-, nyrðr- & norðast-, nyrztr \\
  suðr, sunnr `south' & syðr-           & synst-           \\
  vestr `west'        & vestr-          & vestast-         \\
  út `out'            & ýtr-            & ýzt-             \\
  nær `near'          & nær-            & næst-            \\
\end{longtable}

说明:

\begin{enumerate}
  \def\labelenumi{\arabic{enumi})}
  \item
        hindri,
        hinztr来源于原始日耳曼语副词*hinder,相当于德语hinter,英语hind(参考behind),但是这个词在古诺尔斯语中没有保留为副词,但有意义相关的名词hindr
        `hindrance'.
  \item
        norðr的两种变形有可能从syðri, synstr类比得来。
  \item
        副词suðr和sunnr的两写造成了比较级和最高级词干的不同,其中sunnr是更早的形式。
\end{enumerate}

\section{副词}\label{副词}

副词是一种与形容词密切相关的词类,它大多数是由形容词派生的,但缺少格、性、数的屈折(副词也有级)。本书先介绍副词的构词法,然后介绍一些含义特殊的副词,最后介绍副词的比较级和最高级。

\subsection{副词的构成}\label{副词的构成}

古诺尔斯语只有一小部分词天然属于副词,如mjǫk `very', svá `thus', `so',
þá `then', vel
`well',其他绝大多数副词都由形容词派生而来古诺尔斯语有几种构成方式,最常见的列举如下:

\begin{enumerate}
  \def\labelenumi{\arabic{enumi})}
  \item
        使用强变格单数中性宾格作副词,这种构词法常用于含义最基本的形容词,例如:
\end{enumerate}

\begin{quote}
  mikill \textgreater{} mikit `much, greatly‌'

  lágr \textgreater{} lágt `low, softly‌'

  allr \textgreater{} allt `all the way‌'

  hár \textgreater{} hátt `highly‌'
\end{quote}

\begin{enumerate}
  \def\labelenumi{\arabic{enumi})}
  \setcounter{enumi}{1}
  \item
        添加后缀-a. 这种构词法也是最基本,最普遍的做法,例如:
\end{enumerate}

\begin{quote}
  illr \textgreater{} illa `badly‌'

  gjarn\footnote{gjarn的强变格中,以r开头的词尾与n同化,接着发生辅音简化脱落。因此有*gjarnn
    \textgreater{} gjarn.} \textgreater{} gjarna `eagerly‌'

  许多形容词由-ligr词尾派生,这时用-liga构成副词。-liga有时还加到本身不包含-ligr的形容词上,例如:

  harðligr \textgreater{} harðliga `fiercely‌'

  varligar \textgreater{} varliga `scarcely‌'

  glǫggr \textgreater{} glǫggliga `clearly'

  这些副词有时也脱去-ig-,缩短为harðla, varla等。
\end{quote}

\begin{enumerate}
  \def\labelenumi{\arabic{enumi})}
  \setcounter{enumi}{2}
  \item
        其他格,用形容词或名词的其它格作副词的情况比较少见,有时副词的含义也发生改变。这些副词都是固定的用法:
\end{enumerate}

\begin{quote}
  宾格: megin `side(s)‌' \textless{} vegr `way‌'\footnote{megin的m并不是vegr的一部分,而是前一个词的与格词尾被重解的结果,例如þeim
    megin `on that side' \textless{} *þeim veginn}

  属格: alls `of all, at all‌'; stundar `very, quite‌' \textless{} stund
  `time. hour'

  \begin{quote}
    与格: miklu `much, by far‌'; stórum `hugely‌'; næstum `the last time‌'
    \textless{} næstr
  \end{quote}

  \begin{quote}
    `nearst'; stundum `sometimes‌'.
  \end{quote}
\end{quote}

\subsection{肯定副词与否定副词}\label{肯定副词与否定副词}

相当于现代英语yes和no的副词是já和nei.
其他表否定的副词常用后缀-gi构成,如eigi `not‌'; engi `no, not any‌';
hvergi `nowhere, not at all‌'; aldri `never‌' \textless{} aldregi.
否定动词的副词可以用eigi, ekki或né. 早期诗歌还常用-a,
-at作为动词的否定后缀,例如:

vara `was not' \textless{} var + -a

kannat `knows not' \textless{} kann + -at

vaska `I was not‌' \textless{} vas ek + -a

né有时还可以和这些否定后缀一起使用: sofa né má-k-at `I cannot sleep‌'.

\subsection{方位性副词}\label{方位性副词}

方位性副词通过下列一些后缀构成:

\begin{enumerate}
  \def\labelenumi{\arabic{enumi})}
  \item
        后缀-i. 指示静态的位置:inn `into‌' vs. inni `inside, within‌'
  \item
        后缀-an. 指示从某个位置来:innan `from within‌'
  \item
        后缀-gat/-nig. 指示到某个位置去:hingat/hinnig `to here‌', þangat `to
        there‌'
\end{enumerate}

\begin{quote}
  -an型副词有两个衍生的用法:
\end{quote}

1)和介词fyrir `before, in
front‌'连用作为一个介词性的词组,接续宾格表示``在...方位'':fyrir vestan
valhǫll `in the west of Valhalla'

\begin{quote}
  2)接续一个属格名词,表示``比\ldots 更偏向\ldots'':austan lands `east
  of coast‌'

  方向性副词除了单独使用外,还经常和介词短语连用,详见(交叉引用)。
\end{quote}

\subsection{副词的比较级和最高级}\label{副词的比较级和最高级}

副词的比较级和最高级采用类似于形容词的规则变化,将后缀-ar/-ast或者-r/-st添加在副词的原型后面构成比较级或最高级,用-r/-st构成比较级和最高级时,词根元音也常发生i-变异(但也有不发生i-变异的形式)。有些时候-ar也可以变成-arr,
-ast则变为-arst,例如:

\begin{quote}
  lengi `for a long time' \textgreater{} lengr, lengst

  opt `often' \textgreater{} optar(r), opta(r)st

  framt `forward' \textgreater{} fremr, fremst或framr(r), frama(r)st
\end{quote}

对于用强变格名词/形容词的中性主格/宾格作副词的形容词,它们的比较级/最高级对应名词/形容词的强变格中性主格/宾格的比较级/最高级形式,例如:

skjótr \textgreater{} skjótt `swiftly' \textgreater{} skjótara, skjótast

少数副词的比较级/最高级也会采用异干互补的构词方式,列举如下:

\begin{longtable}{lll}
  \toprule
  原型               & 比较级      & 最高级 \\
  \midrule
  \endhead
  \bottomrule
  \endfoot
  lítt `little‌'      & minnr, miðr & minst  \\
  mjǫk `much‌'        & meir(r)     & mest   \\
  vel `well‌'         & betr        & bezt   \\
  illa `badly‌'       & verr        & verst  \\
  gjarna `willingly‌' & heldr       & helzt  \\
\end{longtable}

总体来说,古诺尔斯语的副词没有像形容词那样明确的规则,一个形容词有时可以派生出多种形式甚至意义不同的副词,副词的比较级和最高级也可能有不同的形式。不过副词的不规则性一般不造成判读的障碍。

\section{分词}\label{分词}

动词的现在分词和过去分词都按形容词变化,分词也可以像形容词一样修饰名词,这种用法和英语非常类似,例如logandi
brandr `burning brand‌'.
但是总的来说,古诺尔斯语更倾向于用从句来改写这种使用分词的情况。

\subsection{现在分词}\label{现在分词}

无论是强动词还是弱动词,现在分词都通过在动词不定式的基础上加-nd-构成。它添加形容词的比较级词尾,即只有弱变化形式。

sofa `sleep‌'的现在分词变形如下所示:

\begin{longtable}{llll}
  \toprule
       & 阳性     & 阴性     & 中性     \\
  \midrule
  \endhead
  \bottomrule
  \endfoot
  单数 &          &          &          \\
  N    & sofandi  & sofandi  & sofanda  \\
  A    & sofanda  & sofandi  & sofanda  \\
  G    & sofanda  & sofandi  & sofanda  \\
  D    & sofanda  & sofandi  & sofanda  \\
  复数 &          &          &          \\
  N    & sofandi  & sofandi  & sofandi  \\
  A    & sofandi  & sofandi  & sofandi  \\
  G    & sofandi  & sofandi  & sofandi  \\
  D    & sofǫndum & sofǫndum & sofǫndum \\
\end{longtable}

分词有时也进一步添加-sk后缀,但这本身是非常少见的用法。

\subsection{过去分词}\label{过去分词}

过去分词相比现在分词来说常见很多,这主要是因为过去分词可以和助动词hafa连用表示完成态,类似于英语的have
done结构。强弱动词的过去分词构成方法并不一致,但它们都可以添加形容词的强变格和弱变格词尾。

强动词的过去分词构成方式是在词干上添加-in,然后添加形容词的词尾。过去分词可以按强变格变化,也可以按弱变格变化,但绝大多数情况下,我们只见到过去分词的强变化形式。以koma
`come‌'的过去分词kominn为例,其强变格为:

\begin{longtable}{llll}
  \toprule
       & 阳性    & 阴性     & 中性    \\
  \midrule
  \endhead
  \bottomrule
  \endfoot
  单数 &         &          &         \\
  N    & kominn  & komin    & komit   \\
  A    & kominn  & komna    & komit   \\
  G    & komins  & kominnar & komins  \\
  D    & komnum  & kominni  & komnu   \\
  复数 &         &          &         \\
  N    & komnir  & komnar   & komin   \\
  A    & komna   & komnar   & komin   \\
  G    & kominna & kominna  & kominna \\
  D    & komnum  & komnum   & komnum  \\
\end{longtable}

说明:

\begin{enumerate}
  \def\labelenumi{\arabic{enumi})}
  \item
        阳性单数宾格词尾为-n而非-an,得到kominn.
  \item
        中性单数主格和宾格为komit.
  \item
        弱读元音i在元音开头的词尾前省略。
\end{enumerate}

弱变格为:

\begin{longtable}{llll}
  \toprule
       & 阳性   & 阴性   & 中性   \\
  \midrule
  \endhead
  \bottomrule
  \endfoot
  单数 &        &        &        \\
  N    & komni  & komna  & komna  \\
  A    & komna  & komnu  & komna  \\
  G    & komna  & komnu  & komna  \\
  D    & komna  & komnu  & komna  \\
  复数 &        &        &        \\
  N    & komnu  & komnu  & komnu  \\
  A    & komnu  & komnu  & komnu  \\
  G    & komnu  & komnu  & komnu  \\
  D    & komnum & komnum & komnum \\
\end{longtable}

弱动词的过去分词主要由第三基本元+齿音-ð-构成。对于弱动词而言,其第三基本元很大程度上与过去时词干类似。过去分词同样有强弱变形的区分,以elska
`love‌'的过去分词elskaðr为例,其强变格形式如下:

\begin{longtable}{llll}
  \toprule
       & 阳性     & 阴性      & 中性     \\
  \midrule
  \endhead
  \bottomrule
  \endfoot
  单数 &          &           &          \\
  N    & elskaðr  & elskuð    & elskat   \\
  A    & elskaðan & elskaða   & elskat   \\
  G    & elskaðs  & elskaðrar & elskaðs  \\
  D    & elskuðum & elskaðri  & elskuðu  \\
  复数 &          &           &          \\
  N    & elskaðir & elskaðar  & elskuð   \\
  A    & elskaða  & elskaðar  & elskuð   \\
  G    & elskaðra & elskaðra  & elskaðra \\
  D    & elskuðum & elskuðum  & elskuðum \\
\end{longtable}

其对应的弱变格如下:

\begin{longtable}{llll}
  \toprule
       & 阳性     & 阴性     & 中性     \\
  \midrule
  \endhead
  \bottomrule
  \endfoot
  单数 &          &          &          \\
  N    & elskaði  & elskaða  & elskaða  \\
  A    & elskaða  & elskuðu  & elskaða  \\
  G    & elskaða  & elskuðu  & elskaða  \\
  D    & elskaða  & elskuðu  & elskaða  \\
  复数 &          &          &          \\
  N    & elskuðu  & elskuðu  & elskuðu  \\
  A    & elskuðu  & elskuðu  & elskuðu  \\
  G    & elskuðu  & elskuðu  & elskuðu  \\
  D    & elskuðum & elskuðum & elskuðum \\
\end{longtable}

\chapter{代词}\label{代词}

\begin{quote}
  \textbf{章节要点}
\end{quote}

\begin{itemize}
  \item
        \begin{quote}
          人称代词及其变格
        \end{quote}
  \item
        \begin{quote}
          物主代词及其变格
        \end{quote}
  \item
        \begin{quote}
          指示代词及其变格
        \end{quote}
  \item
        \begin{quote}
          疑问代词hverr
        \end{quote}
  \item
        \begin{quote}
          常见的不定代词
        \end{quote}
\end{itemize}

\section{人称代词}\label{人称代词}

古诺尔斯语的人称代词系统部分保留了双数。双数仅出现在第一和第二人称代词中,第三人称代词只有单复数之分,但区分性别。人称代词中的变形涉及多个词干,具体如下:

\begin{longtable}{lll}
  \toprule
       & 第一人称 & 第二人称 \\
  \midrule
  \endhead
  \bottomrule
  \endfoot
  单数 &          &          \\
  N    & ek       & þú       \\
  A    & mik      & þik      \\
  G    & mín      & þín      \\
  D    & mér      & þér      \\
  双数 &          &          \\
  N    & vit      & it, þit  \\
  A    & okkr     & ykkr     \\
  G    & okkar    & ykkar    \\
  D    & okkr     & ykkr     \\
  复数 &          &          \\
  N    & vér      & ér, þér  \\
  A    & oss      & yðr      \\
  G    & vár      & yðar     \\
  D    & oss      & yðr      \\
\end{longtable}

古诺尔斯语中动词已经没有双数形式,双数人称代词同样支配动词的复数式。同时,双数代词使用的并不多,并且有和复数合并的趋势。在近现代冰岛语中,双数第一人称逐渐承担了复数的作用。

人称代词作主语时,主格有时以后缀形式粘着在动词后,这在早期的诗歌中尤为常见。第一人称ek失去元音e,以-k的形式添加在动词后,例如:

\begin{quote}
  mælik `I speak‌' \textless{} mæli + ek

  mákat `I cannot‌' \textless{} má + ek + at
\end{quote}

第二人称þú中的þ有时和前面的辅音发生同化现象,例如:

\begin{quote}
  heyrðu `you hear'\textless{} heyr þú

  skaltu `you shall'\textless{} skalt þú

  seldu `you sell'\textless{} sel þú
\end{quote}

这种合写在解读了造成了一些偏差,例如skuluðér(\textless{} skuluð
ér)可以被理解为skuluð þér.
正是由于这个元音,第二人称双数和复数才会出现异体形式þit和þér,它们本来的形式就是it和ér.

第一和第二人称下,间接格同时也可以当作反身代词使用,反身代词本身也不存在主格,因为习惯上只有I
hurt myself, 而不可能有†myself hurt I.
在古老的诗歌中,反身代词也经常以后缀形式黏附在动词后,宾格mik变成-mk,但是与格也常常以同样的形式添加在动词后。即-mk是一种近乎通用的表示宾语(无论与格还是宾格)的办法:

\begin{quote}
  þóttumk \textless{} þótti mér `it seemed to me‌'

  gáfumk íþrótt \textless{} gáf mér íþrótt `gave me skill'
\end{quote}

比较反常的情况是,当这种后缀添加在动词的单数式上时,动词反而要采用对应人称的复数式,这和强动词第三人称单数的中动词尾相呼应(参见\ref{强动词的中动词尾})。

第三人称人称代词虽然没有双数,但区分阴阳中三性。

\begin{longtable}{llll}
  \toprule
       & 阳性     & 阴性     & 中性     \\
  \midrule
  \endhead
  \bottomrule
  \endfoot
  单数 &          &          &          \\
  N    & hann     & hon      & þat      \\
  A    & hann     & hana     & þat      \\
  G    & hans     & hennar   & þess     \\
  D    & honum    & henni    & því, þí  \\
  复数 &          &          &          \\
  N    & þeir     & þær      & þau      \\
  A    & þá       & þær      & þau      \\
  G    & þeir(r)a & þeir(r)a & þeir(r)a \\
  D    & þeim     & þeim     & þeim     \\
\end{longtable}

第三人称不能像上面一样用间接格表示反身代词,相反另有一个专门的反身代词sik。如上面所说,sik没有主格,它既不区分性也不区分数,只按照格变化。

\begin{longtable}{ll}
  \toprule
    & 反身代词 \\
  \midrule
  \endhead
  \bottomrule
  \endfoot
  N & -        \\
  A & sik      \\
  G & sín      \\
  D & sér      \\
\end{longtable}

sik也添加在动词后面,演变为中动态的标记-sk.

\section{物主代词}\label{物主代词}

古诺尔斯语的物主代词一般是形容词性的,但少数情况下也可以作名词用。

物主代词由人称代词的单数属格衍生出来。由于第一、第二人称的人称代词和第三人称人称代词的构成有区别之处,它们对应的物主代词的构词法也有不同。

物主代词的形容词性要求它必须能按性屈折,这样才能与其修饰的名词保持一致。对于第一、第二人称代词而言,人称代词本来不区分性,因此把它的属格原型作为物主代词的词干,接着按照强变格法添加各个性、数、格的词尾,例如ek的属格
mín `of me‌', 构成形容词词干mín-,加单数阳性词尾有*mín- + -r
\textgreater{} minn,注意词干中的长元音在双辅音前缩短。以minn `my‌'
为例,它的完整变格如下:

\begin{longtable}{llll}
  \toprule
       & 阳性  & 阴性   & 中性  \\
  \midrule
  \endhead
  \bottomrule
  \endfoot
  单数 &       &        &       \\
  N    & minn  & mín    & mitt  \\
  A    & minn  & mína   & mitt  \\
  G    & míns  & minnar & míns  \\
  D    & mínum & minni  & mínu  \\
  复数 &       &        &       \\
  N    & mínir & mínar  & mín   \\
  A    & mína  & mínar  & mín   \\
  G    & minna & minna  & minna \\
  D    & mínum & mínum  & mínum \\
\end{longtable}

注意:minn的变格有些许不规则之处,它类似于强动词过去分词的变形。单数阳性宾格以及单数中性主格/宾格都值得注意。

类似地,其余第一、第二人称的物主代词以及反身代词的变化都模仿上述过程,下表给出了它们各个性的单数主格形式:

\begin{longtable}{lllll}
  \toprule
       & 词干     & 阳性     & 阴性  & 中性     \\
  \midrule
  \endhead
  \bottomrule
  \endfoot
  单数 &          &          &       &          \\
  1单  & mín-     & minn     & mín   & mitt     \\
  2单  & þín-     & þinn     & þín   & þitt     \\
  双数 &          &          &       &          \\
  1双  & okkar-   & okkarr   & okkur & okkart   \\
  2双  & ykkar-   & ykkarr   & ykkur & ykkart   \\
  复数 &          &          &       &          \\
  1复  & vár-     & várr     & vár   & várt     \\
  2复  & yð(v)ar- & yð(v)arr & yður  & yð(v)art \\
       &          &          &       &          \\
  反身 & sín-     & sinn     & sín   & sitt     \\
\end{longtable}

说明:

\begin{enumerate}
  \def\labelenumi{\arabic{enumi})}
  \item
        第二人称复数词干yð(v)ar-有两写,如果选择不包含v的词干yðar-,它的变形就和其他双音节形容词一致,弱读元音a在元音开头的词尾前脱落,如yðar-+-um
        \textgreater{} yðrum.
        但是,如果选择包含v的词干yðvar-,则不会有这种现象。虽然yðvar-也有两个音节,但如果脱去了a,半元音v就出现在两个辅音之间,这是古诺尔斯语所不允许的。
  \item
        词干ykkar-和okkar-属于双音节词干,它们规则地适用于双音节形容词的变格法。
\end{enumerate}

第三人称人称代词没有对应的物主代词,只沿用其属格形式就足以表达物主代词的含义,其单数为hans,
hennar, þess复数都是þeira.

\section{指示代词}\label{指示代词}

指示代词一般即可以作名词用也可以作形容词用,它也采取和第三人称代词类似的异干互补系统。古诺尔斯语中的指示代词的用法和英语类似,可以用于空间上、时间上或逻辑上的近指或远指。

\textbf{远指代词sá}

sá有两层含义,它既可以指示较远的事物,相当于英语的`that‌',也可以用作类似于定冠词含义的`the‌',
用于指示已经提到过的事物。对应地,在作名词时,它也可以表示远处的东西或是已经提到过的事物。sá的变格不规则,但还是有明显的形容词词尾的痕迹,如下所示:

\begin{longtable}{llll}
  \toprule
       & 阳性     & 阴性      & 中性     \\
  \midrule
  \endhead
  \bottomrule
  \endfoot
  单数 &          &           &          \\
  N    & sá       & sú        & þat      \\
  A    & þann     & þá        & þat      \\
  G    & þess     & þeir(r)ar & þess     \\
  D    & þeim     & þeir(r)i  & því, þí  \\
  复数 &          &           &          \\
  N    & þeir     & þær       & þau      \\
  A    & þá       & þær       & þau      \\
  G    & þeir(r)a & þeir(r)a  & þeir(r)a \\
  D    & þeim     & þeim      & þeim     \\
\end{longtable}

注意:sá的所有复数形式以及中性的单数形式都和第三人称代词一致,所以在这种情况下,同一个词可能有多种含义(虽然很多情况下,表意是类似的)。

\textbf{近指代词sjá}

sjá是sá的反义词,它指示相对较近的事物,但一般不提示上下文中出现过的事物。sjá的用法和sá类似,也兼具形容词和代词的功能。其变格如下:

\begin{longtable}{llll}
  \toprule
       & 阳性       & 阴性       & 中性   \\
  \midrule
  \endhead
  \bottomrule
  \endfoot
  单数 &            &            &        \\
  N    & sjá, þessi & sjá, þessi & þetta  \\
  A    & þenna      & þessa      & þetta  \\
  G    & þessa      & þessar     & þessa  \\
  D    & þessum     & þessi      & þessu  \\
  复数 &            &            &        \\
  N    & þessir     & þessar     & þessi  \\
  A    & þessa      & þessar     & þessi  \\
  G    & þessa      & þessa      & þessa  \\
  D    & þessum     & þessum     & þessum \\
\end{longtable}

\textbf{代词hinn}

hinn一般不用作近指或远指代词,它一般表示``另一个'',与前文提到的名词形成对比。hinn也有类似于冠词的用法,这种情况下,它和名词的特指后缀-inn(或独立形式inn)表意完全一致。除了中性单数主格/宾格的词尾-tt外,它的变形和inn一致:

\begin{longtable}{llll}
  \toprule
       & 阳性  & 阴性   & 中性  \\
  \midrule
  \endhead
  \bottomrule
  \endfoot
  单数 &       &        &       \\
  N    & hinn  & hin    & hitt  \\
  A    & hinn  & hina   & hitt  \\
  G    & hins  & hinnar & hins  \\
  D    & hinum & hinni  & hinu  \\
  复数 &       &        &       \\
  N    & hinir & hinar  & hin   \\
  A    & hina  & hinar  & hin   \\
  G    & hinna & hinna  & hinna \\
  D    & hinum & hinum  & hinum \\
\end{longtable}

\section{关系代词}\label{关系代词}

古诺尔斯语只有一个关系代词er,
其早期形式为es,后期也用sem表示和er相同的意思。准确来说,这个er应当表达为关系小品词,因为所有的定语从句都可以用er引导(包括英语中需要使用关系副词的情况)。er不可变格,没有人称、数或者性的概念。有时在er的前面加上sá的变格来指示对应的格、性、数,例如sú
er指示阴性单数主格。但这种说法也常常会产生分歧,因为指示代词sá完全可以理解为修饰先行词的形容词,这样sá就是主句的一部分,从而与er在从句中的格、性、数没有任何关系了。

在下面的句型中: ...sverð þat er...,
根据省略号处填补的内容,可以有若干种解读:

1)(that is) the sword, \textbf{which} (bears his name)‌ þat
er整体作引导词,在句中充当主格

2)(that is) the sword, \textbf{by which} (he was killed)‌
er在句中充当与格

3)(that is) the sword, \textbf{of which} (the legacy is well-known)‌
er在句中充当属格

另一个常见的用法是将er与一些副词合用,构成具体的关系副词,例如:

\begin{quote}
  þar `there‌' \textgreater{} þar er `where‌'

  þá `then‌' \textgreater{} þá er `when‌'
\end{quote}

关于er引导的从句,参见(交叉引用)

\section{疑问代词}\label{疑问代词}

基本的疑问代词词干是hverj- `who, which,
what‌',它既是名词性的又是形容词性的。作形容词时,按照强变格法变格,如下所示:

\begin{longtable}{llll}
  \toprule
       & 阳性           & 阴性     & 中性     \\
  \midrule
  \endhead
  \bottomrule
  \endfoot
  单数 &                &          &          \\
  N    & hverr          & hver     & hvert    \\
  A    & hverjan, hvern & hverja   & hvert    \\
  G    & hvers          & hverrar  & hvers    \\
  D    & hverjum, hveim & hverri   & hverju   \\
  复数 &                &          &          \\
  N    & hverir         & hverjar  & hver     \\
  A    & hverja         & hverjar  & hver     \\
  G    & hverra         & hverra   & hverra   \\
  D    & hverjum        & hverrjum & hverrjum \\
\end{longtable}

作名词时形式和形容词一致,但有一些非常常见的两写:

\begin{longtable}{lll}
  \toprule
       & 阳性/阴性      & 中性           \\
  \midrule
  \endhead
  \bottomrule
  \endfoot
  单数 &                &                \\
  N    & hverr          & hvat           \\
  A    & hverjan, hvern & hvat           \\
  G    & hvers, hves(s) & hvers, hves(s) \\
  D    & hverjum, hveim & hví            \\
\end{longtable}

更准确地来说,hverr是对大于三个的物体中的哪一个提问,如果要对两个中的哪一个提问,要用另一个词hvárr
`which of two‌'.
这两个疑问代词反映了古诺尔斯语中残留的双数和复数的区别。hvárr的变形和hverr完全一致,词干为hvár-,见下表:

\begin{longtable}{llll}
  \toprule
       & 阳性   & 阴性    & 中性   \\
  \midrule
  \endhead
  \bottomrule
  \endfoot
  单数 &        &         &        \\
  N    & hvárr  & hvár    & hvárt  \\
  A    & hvárn  & hvára   & hvárt  \\
  G    & hvárs  & hvárrar & hvárs  \\
  D    & hvárum & hvárri  & hváru  \\
  复数 &        &         &        \\
  N    & hvárir & hvárar  & hvár   \\
  A    & hvára  & hvárar  & hvár   \\
  G    & hvárra & hvárra  & hvárra \\
  D    & hvárum & hvárum  & hvárum \\
\end{longtable}

某些疑问代词的形式(或变体)固定下来成为副词形式:

\begin{longtable}{ll}
  \toprule
  疑问副词 & 含义                          \\
  \midrule
  \endhead
  \bottomrule
  \endfoot
  hvaðan   & 从哪里 `whence‌'               \\
  hvar     & 哪里 `where‌'                  \\
  hvert    & 到哪里 `whither‌'              \\
  hvárt    & 是否 `whether (or not)‌'       \\
  hvé      & 怎样 `how‌'                    \\
  hvenær   & 何时 `when‌'                   \\
  hví      & 为何 `why‌'                    \\
  hversu   & 到什么程度 `how‌' (degree)     \\
  hvernig  & 以什么方法 `how, in what way‌' \\
\end{longtable}

这些疑问代词(副词)也可以引导间接疑问句。

\section{不定代词}\label{不定代词}

总体来说,不定代词的变格和强形容词一致,它们大多数本身也是形容词。最常见的不定代词如下所示,它们的词干首先给出:

1. \textbf{all-
  `all‌'},其构词如规则的形容词,它一般都用作强形容词,见下表:

\begin{longtable}{llll}
  \toprule
       & 阳性  & 阴性   & 中性  \\
  \midrule
  \endhead
  \bottomrule
  \endfoot
  单数 &       &        &       \\
  N    & allr  & ǫll    & allt  \\
  A    & allan & alla   & allt  \\
  G    & alls  & allrar & alls  \\
  D    & ǫllum & allri  & ǫllu  \\
  复数 &       &        &       \\
  N    & allir & allar  & ǫll   \\
  A    & alla  & allar  & ǫll   \\
  G    & allra & allra  & allra \\
  D    & ǫllum & ǫllum  & ǫllum \\
\end{longtable}

其主要用法是:

\begin{enumerate}
  \def\labelenumi{\Alph{enumi}.}
  \item
        所有形式都可当作形容词或名词用,表示``所有'';
  \item
        单数形式在许多短语中几乎按副词用,表示``完全'':allr í sundr `all
        asunder';
  \item
        单数中性尤常作不定代词用,类似于英语'everything';
  \item
        allt可当作一个宽泛的副词,表示``完全地;直接地;在所有地方;基本上''等;
  \item
        复数allir单独使用,表示``所有人;一起''。
\end{enumerate}

2. \textbf{sum- `some, (a)
  certain‌'},其构词如规则的强形容词,可以作形容词和代词用。

3. \textbf{ein-
  `one‌'},作不定代词或形容词时区分于ein作数词(交叉引用)的情况。其变格大部分是规则的,只有中性的单数主格和宾格是eitt.

\begin{longtable}{llll}
  \toprule
       & 阳性        & 阴性   & 中性  \\
  \midrule
  \endhead
  \bottomrule
  \endfoot
  单数 &             &        &       \\
  N    & einn        & ein    & eitt  \\
  A    & einn, einan & eina   & eitt  \\
  G    & eins        & einnar & eins  \\
  D    & einum       & einni  & einu  \\
  复数 &             &        &       \\
  N    & einir       & einar  & ein   \\
  A    & eina        & einar  & ein   \\
  G    & einna       & einna  & einna \\
  D    & einum       & einum  & einum \\
\end{longtable}

其主要用法是:

\begin{enumerate}
  \def\labelenumi{\Alph{enumi}.}
  \item
        用作单数时,表示不定代词,表示不特指的某一个;
  \item
        单数或复数都可以表示``单独的'',用法类似于副词:láta einan `let
        alone';
  \item
        einna和其他名词连用,表示强调含义:einna manna bezt `best of all
        single man';
  \item
        eins作副词用,表示``以同一方式'',但常和其他词连用,如eins ok `as if',
        at eins `only';
  \item
        和其他代词、名词连用,如einn hverr `each; some‌'(见下), einn saman
        `together', hverr ok einn, `each and one', né einn `none', fáir einir
        `few'.
\end{enumerate}

4. \textbf{annar- `(an)
  other‌'},其变格参见\ref{不规则形容词}。annat尤其常作为名词用。

5. \textbf{nǫkkur- `any, some; a
  certain‌',}这个词也按强形容词规则变化,但词形非常复杂\footnote{这个词最早的形式是nekkverr,和né
  hverr密切相关。因此,其最早的变格和hverr相同,词干为nekkverj-;后来这个词干脱去-j,按nekkver-变格,同时,元音发生了变化,有nekkvar-,
  nakkver-,
  nakkvar-等形式,受v的影响,a又变为ǫ;v有时也从词干上脱落。最后,这个词在现代冰岛语中还能像双音节形容词一样发生省略。因此这个词记录到的形态非常复杂,有nøkkurr,
  nakkurr, nekkverr, nakkvarr, nǫkkverr, nǫkkvarr, nǫkkr, nǫkkurr等。}。就nǫkkur-这个词干而言,有阳性单数宾格nǫkkurn;中性单数主格/宾格nǫkkut.
有时也用词干nakkvar-,变形如下:

\begin{longtable}{lllllll}
  \toprule
       & \multicolumn{2}{c}{阳性} & \multicolumn{2}{c}{阴性} & \multicolumn{2}{c@{}}{中性}                                   \\
  \midrule
  \endhead
  \bottomrule
  \endfoot
  单数 &                          &                          &                             &          &           &          \\
  N    & nakkvarr                 & nǫkkurr                  &                             & nǫkkur   & nakkvat   & nǫkkut   \\
  A    & nakkvarn                 & nǫkkurn                  & nakkvara                    & nǫkkura  & nakkvat   & nǫkkut   \\
  G    & nakkvars                 & nǫkkurs                  & nakkvarar                   & nǫkkurar & nakkvars  & nǫkkurs  \\
  D    &                          & nǫkkurum                 & nakkvarri                   & nǫkkurri &           & nǫkkuru  \\
  复数 &                          &                          &                             &          &           &          \\
  N    & nakkvarir                & nǫkkurir                 & nakkvarar                   & nǫkkurar &           & nǫkkur   \\
  A    & nakkvara                 & nǫkkura                  & nakkvarar                   & nǫkkurar &           & nǫkkur   \\
  G    & nakkvarra                & nǫkkurra                 & nakkvarra                   & nǫkkurra & nakkvarra &
  nǫkkurra                                                                                                                   \\
  D    &                          & nǫkkurum                 &                             & nǫkkurum &           & nǫkkurum \\
\end{longtable}

表格中的空白处表示没有记录到以词干nakkvar-构成的形式。

这个词没有特别费解的用法,但常用阳性nǫkkurr指代``任何人'',相当于,中性nǫkkurt指代``任何事''。nǫkkurr可以和数词连用,表示``大约''。

6. \textbf{hverj- `each,
  every‌'},按强变格变化,参见\ref{疑问代词}。hverr作不定代词时表示``每一个'',它修饰的名词总要用属格,如gumna
hverr `each man (=every one of men)'.

7. \textbf{ein- + hverj- `each;
  some‌'},einn和hverr连在一起构成不定代词。第二个词干hverj-总是要变格,ein-可能保持ein-不变,也可变格使之与第二个词干一致。例如单数阳性属格可以是einshvers或einhvers.
这个词的有两个含义:

\begin{enumerate}
  \def\labelenumi{\Alph{enumi}.}
  \item
        类似于hverr,表示``每一个'',但语义更强。
  \item
        类似于einn,表示``某一个'',如eina hverja nótt `some night'.
\end{enumerate}

8. \textbf{báð-
  `both‌'},只以复数形式出现,注意中性式的不规则之处。特别地,bæði常作为副词用,构成bæði\ldots ok\ldots{}
`both \ldots{} and \ldots'结构。

\begin{longtable}{llll}
  \toprule
       & 阳性   & 阴性   & 中性   \\
  \midrule
  \endhead
  \bottomrule
  \endfoot
  复数 &        &        &        \\
  N    & báðir  & báðar  & bæði   \\
  A    & báða   & báðar  & bæði   \\
  G    & beggja & beggja & beggja \\
  D    & báðum  & báðum  & báðum  \\
\end{longtable}

9. \textbf{nein- `none, not any‌'},由né + einn得到,其变格参照einn.
neinn虽然是一个表示否定的代词,但是它单独不能表示否定,必须和其他否定副词连用,最常见的是ekki:ekki
neitt `nothing'

10. \textbf{engi `no,
  none‌'},由einn和否定后缀-gi结合得到,其部分形式是不规则的:

\begin{longtable}{llll}
  \toprule
       & 阳性           & 阴性   & 中性           \\
  \midrule
  \endhead
  \bottomrule
  \endfoot
  单数 &                &        &                \\
  N    & engi           & engi   & ekki           \\
  A    & engan, engi    & enga   & ekki           \\
  G    & einskis, engis & engrar & einskis, engis \\
  D    & engum          & engri  & engu           \\
  复数 &                &        &                \\
  N    & engir          & engar  & engi           \\
  A    & enga           & engar  & engi           \\
  G    & engra          & engra  & engra          \\
  D    & engum          & engum  & engum          \\
\end{longtable}

这个词的词性也像nǫkkurr一样多变,它过去常用eing-或øng-词干,有时词干上还有额外的-v,出现-a或-ir前,如
øngvar, engvar;øngvir, engvir.
另外,有时还在主格上添加-nn/-n后缀,有单数阳性主格enginn,单数阴性主格eigin,复数中性主格enginn,其他形式都不添加这个后缀。

engi可以单数使用,相当于英语`none'.

11. \textbf{hvárgi `neither'},这个词的变形也比较多样,参见下表:

\begin{longtable}{llll}
  \toprule
       & 阳性                              & 阴性     & 中性              \\
  \midrule
  \endhead
  \bottomrule
  \endfoot
  单数 &                                   &          &                   \\
  N    & hvárgi, hvárigr, hvárrgi, hvárugr & hvárgi   & hvárki, hvártki   \\
  A    & hvárgan, hvárngan, hvárngi        & hvárga   & hvárki, hvártki   \\
  G    & hvárkis, hvárskis                 & hvárgrar & hvárkis, hvárskis \\
  D    & hvárungi, hvárgum                 & hvárgri  & hvárugi, hvárgu   \\
  复数 &                                   &          &                   \\
  N    & hvárgir, hvárigir                 & hvárgar  & hvárgi            \\
  A    & hvárga                            & hvárgar  & hvárgi            \\
  G    & hvárgra                           & hvárgra  & hvárgra           \\
  D    & hvárgum                           & hvárgum  & hvárgum           \\
\end{longtable}

hvárgi的中性形式常构成这样的搭配,相当于一个副词:

hvárki \ldots{} né `neither \ldots{} nor': hvárki til laga né til úlaga
`be neither friendly nor hostile'

\chapter{数词}\label{数词}

\begin{quote}
  \textbf{章节要点}
\end{quote}

\begin{itemize}
  \item
        \begin{quote}
          20以内的基数词与序数词
        \end{quote}
  \item
        \begin{quote}
          大数词的构成
        \end{quote}
  \item
        \begin{quote}
          数词的变格
        \end{quote}
\end{itemize}

\section{\texorpdfstring{\textbf{基数词与序数词}}{基数词与序数词}}\label{\texorpdfstring{\textbf{基数词与序数词}

下面是古诺尔斯语中基本的数词:

\begin{longtable}{lll}
  \toprule
  数字 & 基数词                & 序数词                             \\
  \midrule
  \endhead
  \bottomrule
  \endfoot
  1    & einn                  & fyrstr                             \\
  2    & tveir                 & annarr                             \\
  3    & þrír                  & þriði                              \\
  4    & fjórir                & fjórði                             \\
  5    & fimm                  & fimmti                             \\
  6    & sex                   & sétti                              \\
  7    & sjau                  & sjaundi                            \\
  8    & átta                  & átti, áttandi                      \\
  9    & níu                   & níundi                             \\
  10   & tíu                   & tíundi                             \\
  11   & ellifu                & ellifti                            \\
  12   & tólf                  & tólfti                             \\
  13   & þrettán               & þrettándi                          \\
  14   & fjórtán               & fjórtándi                          \\
  15   & fimmtán               & fimmtándi                          \\
  16   & sextán                & sextándi                           \\
  17   & sjaután               & sjautándi                          \\
  18   & átján                 & átjándi                            \\
  19   & nítján                & nítjándi                           \\
  20   & tuttugu, tvítján      & tuttugandi, tvítjándi, tuttugasti, \\
  21   & tuttugu ok einn       & tuttugandi ok fyrstr               \\
       & einn ok tuttugu       & fyrstr ok tuttugandi               \\
  22   & tuttugu ok tveir      & tuttugandi ok annarr               \\
  30   & þrír tigir            & þrítugandi                         \\
  31   & þrír tigir ok einn    & þrítugandi ok fyrstr               \\
       & einn ok þrír tigir    & fyrstr ok þrítugandi               \\
  40   & fjórir tigir          & fertugandi                         \\
  50   & fimm tigir            & fimmtugandi                        \\
  60   & sex tigir             & sextugandi                         \\
  70   & sjau tigir            & sjautugandi                        \\
  80   & átta tigir            & áttatugandi                        \\
  90   & níu tigir             & nítugandi                          \\
  100  & tíu tigir             & (títugandi)                        \\
  110  & ellifu tigir          & (ellifutugandi)                    \\
  120  & hundrað               & (hundraðasti)                      \\
  200  & hundrað ok átta tigir & (hundraðasti ok áttatugandi)       \\
  240  & tvau hundrað          &                                    \\
  960  & átta hundrað          &                                    \\
  1200 & þúsund                & (þúsandasti)                       \\
\end{longtable}

括号里的序数词形式是从现代冰岛语借过来的。

有一些情况值得注意:

\begin{enumerate}
  \def\labelenumi{\arabic{enumi})}
  \item
        1-12\footnote{读者可能会好奇为什么11和12的形式也不规则。事实上,ellifu来自于*ainalif,tólf来自于*twalif,它们是由数词1,2加上*-lif
          `left'构成,表示比10多1/2.}的形态是数词变化的基础,需要特别注意。而大于12的数词一般有迹可循。
  \item
        3-12的序数词一般是在词尾上添加-ði/-di/-ti,大于12的序数词一般加-andi/-undi.
  \item
        13-19的基数词是由词尾-tán添加在对应的0-10的基数词上得到的。
  \item
        20以上的整十的基数词由0-10的基数词和tigir构成。tigir是tigr的复数,表示``一组十个''的概念。
  \item
        21-29;31-39等由整十倍的数词和0-10的数词合成,这两个数词哪个在前哪个在后并无影响。
  \item
        hundrað和þúsund与现在的hundred和thousand表意不同,在基督教传入之前,这两个数词在日耳曼语中一般表示的是120/1200。因此200是由120(hundrað)+80(átta
        tigir)表示的。
  \item
        20,100,120的序数词也可以添加-asti词尾。
\end{enumerate}

\section{数词的变形}\label{数词的变形}

大部分基数词是不可变格的。但1-4按照形容词变格,与修饰的名词的格、性、数保持一致。只有ein-有单数和复数,但作复数时表示``单独的'',其它基数词只有复数形式。

\begin{longtable}{llll}
  \toprule
  ein- `one‌' & 阳性  & 阴性   & 中性  \\
  \midrule
  \endhead
  \bottomrule
  \endfoot
  单数       &       &        &       \\
  N          & einn  & ein    & eitt  \\
  A          & einn  & eina   & eitt  \\
  G          & eins  & einnar & eins  \\
  D          & einum & einni  & einu  \\
  复数       &       &        &       \\
  N          & einir & einar  & ein   \\
  A          & eina  & einar  & ein   \\
  G          & einna & einna  & einna \\
  D          & einum & einum  & einum \\
\end{longtable}

\begin{longtable}{llll}
  \toprule
  tveir `two‌' & 阳性     & 阴性     & 中性     \\
  \midrule
  \endhead
  \bottomrule
  \endfoot
  复数        &          &          &          \\
  N           & tveir    & tvær     & tvau     \\
  A           & tvá      & tvær     & tvau     \\
  G           & tveggja  & tveggja  & tveggja  \\
  D           & tveim(r) & tveim(r) & tveim(r) \\
\end{longtable}

\begin{longtable}{llll}
  \toprule
  þrí- `three‌' & 阳性    & 阴性    & 中性    \\
  \midrule
  \endhead
  \bottomrule
  \endfoot
  复数         &         &         &         \\
  N            & þrír    & þrjár   & þrjú    \\
  A            & þrjá    & þrjár   & þrjú    \\
  G            & þriggja & þriggja & þriggja \\
  D            & þrim(r) & þrim(r) & þrim(r) \\
\end{longtable}

\begin{longtable}{llll}
  \toprule
  fjór- `four‌' & 阳性     & 阴性     & 中性     \\
  \midrule
  \endhead
  \bottomrule
  \endfoot
  复数         &          &          &          \\
  N            & fjórir   & fjórar   & fjǫgur   \\
  A            & fjóra    & fjórar   & fjǫgur   \\
  G            & fjǫgurra & fjǫgurra & fjǫgurra \\
  D            & fjórum   & fjórum   & fjórum   \\
\end{longtable}

除此之外,只有tigr, hundrað和þúsund可以变格,其他基数词都不变格。

tigr按照u-词干阳性强名词变格,hundrað按照a-词干中性强名词变格,þúsund按照i-词干性强名词变格:

\begin{longtable}{llll}
  \toprule
       & \multicolumn{3}{c}{数词}                         \\
  \midrule
  \endhead
  \bottomrule
  \endfoot
  词干 & tig-u-                   & hundrað-a & þúsund-i- \\
  单数 &                          &           &           \\
  N    & tigr                     & hundrað   & þúsund    \\
  A    & tig                      & hundrað   & þúsund    \\
  G    & tigar                    & hundraðs  & þúsundar  \\
  D    & tigi                     & hundraði  & þúsund    \\
  复数 &                          &           &           \\
  N.   & tigir                    & hundruð   & þúsundir  \\
  A    & tigu                     & hundruð   & þúsundir  \\
  G    & tiga                     & hundraða  & þúsunda   \\
  D    & tigum                    & hundruðum & þúsundum  \\
\end{longtable}

序数词按照形容词变格,与被修饰的名词保持格、性、数的一致。只有fyrstr既可按照强形容词也可按照弱形容词变格;annarr的变格参见\ref{不规则形容词};3以上的序数词都按弱形容词变格,例如fimmti的变格为:

\begin{longtable}{llll}
  \toprule
       & 阳性    & 阴性    & 中性    \\
  \midrule
  \endhead
  \bottomrule
  \endfoot
  单数 &         &         &         \\
  N    & fimmti  & fimmta  & fimmta  \\
  A    & fimmta  & fimmtu  & fimmta  \\
  G    & fimmta  & fimmtu  & fimmta  \\
  D    & fimmta  & fimmtu  & fimmta  \\
  复数 &         &         &         \\
  N.   & fimmtu  & fimmtu  & fimmtu  \\
  A    & fimmtu  & fimmtu  & fimmtu  \\
  G    & fimmtu  & fimmtu  & fimmtu  \\
  D    & fimmtum & fimmtum & fimmtum \\
\end{longtable}

\chapter{介词}\label{介词}

\begin{quote}
  \textbf{章节重点}
\end{quote}

\begin{itemize}
  \item
        \begin{quote}
          介词的基本结构
        \end{quote}
  \item
        \begin{quote}
          重要介词的用法
        \end{quote}
\end{itemize}

\section{介词的概述}\label{介词的概述}

介词是一类不可变化的词类,它们的词形在各种位置上都保持一致。绝大多数情况下,介词都和名词短语一起构成介词短语,表示各类与时间、地点、方式等形状有关的概念。古诺尔斯语的介词短语结构和英语非常相似。在介词短语中,介词几乎总是出现在名词之前的,这也就是英语\textbf{Pre}position的来源。但偶尔介词也可以出现在名词的后面,这时最好用更准确的说法\textbf{Post}position来称呼这类介词。

介词虽然不能发生变格,但它对接续的名词的格有支配性作用。在英语中,这个现象被称为``介宾结构'',即介词后一律接宾格。在古诺尔斯语中,介词后的名词可以是间接格中的任意一个。同一个介词也可能通过接续不同的格来表达不同的含义,熟悉德语的读者立刻会想到``静三动四''的规则,即表示方位性的介词可以接第三格(即与格)名词表示静态的含义,接第四格(即宾格)名词表示动态的含义,如in+第三/第四格可表示英语中的in/into。在古诺尔斯语中,也有类似的情况。

一些介词也可以单独作副词使用。从严格意义上来说,此时介词的词性已经发生了改变,不过这在语义上并不构成区别。参考下面的句子:

1)He walked into the house

2)He walked in without saying hello

古诺尔斯语中最基本的介词包括á `on', af `off', at `at, to', frá `from', í
`in', með `with', , um `about, in', við
`with',这几个介词无一例外都由原始日耳曼语继承而来。读者也可以立刻发现,它们的英语翻译几乎就是它们的同源词,形态非常类似。还有一类介词由其他实义名词演变而来,如til
`till, to'来源于PGmc. *tila `goal', meðal `between'来源于miðr
`middle'等等。一般来说,第二类介词(即由名词派生出来的介词)往往和第一类介词(即固有的介词,且主要是á,
í, um这三个)连用,如í
meðal,这个词组整体起到介词的作用,读者也可以把这个词组理解为 `in the
middle
of'(=between)。不过,这种词组里的第一类介词并不是必要的,也可以省略。有些第二类介词也总是单独使用,如til.

\section{支配与格和宾格的介词}\label{支配与格和宾格的介词}

相当一部分的介词既可以支配与格又可以支配宾格,名词使用哪个格主要根据语义来决定。一般来说,介词短语表示静态的、方位性质(Location)的概念时,名词用其与格形式;表示动态的、方向性质(Motion)的概念时,则用宾格。当然,除了静态------动态这一区分外,有些介词在表示时间概念的时候接宾格,表示地点概念的时候接与格。

以下是一些常见的可支配与格和宾格的介词:

\begin{enumerate}
  \def\labelenumi{\arabic{enumi}.}
  \item
        á
\end{enumerate}

á是一个典型的支配与格和宾格的介词。

句法

\chapter{限定句}\label{限定句}

章节要点

\section{限定句的定义和结构}\label{限定句的定义和结构}

限定句(Definite
sentence),或简称句子(Sentence),是一串能表述完整含义的词。一般来说,句子包括主语和谓语两个部分,以表达``某人做某事''的基本结构。在古诺尔斯语中,这个谓语部分必须由一个限定动词(Definite
verb)充当,因此我们又把这种句子称为限定句。

限定动词是屈折语中动词的一种形式,它与句子中主语的人称、数等的一致,并体现时态、体貌、语气和语态的特征。由于这类动词的形态受主语的影响,我们说它被主语``限定'',故称为``限定动词''。限定动词和非限定动词相对,后者包括不定式、分词、动名词等。非限定动词不需要与主语的人称、数等语法范畴一致。在一个限定句中,限定动词有且只有一个,但非限定动词可以有多个,并且出现在各种句子成分之中。参考英语中的两个句子:

\begin{quote}
  (1) I \textbf{wanted} to have a cup of coffee.

  (2) To see \textbf{is} to believe.
\end{quote}

在句1中,限定动词是``wanted'',它不仅和主语``I''在人称和数上一致,还表达出过去时。非限定动词是不定式``to
have'',它和后面的名词一起构成了非限定从句,作为wanted的宾语。在句2中,限定动词时``is'',作为系动词连接主语和表语。在此句中,主语和表语都由不定式(to
see \& to believe)承担。

限定句由若干短语构成,短语又由单词构成。我们之前介绍的名词、动词、形容词等都可以构成短语,这些短语在限定句中充当不同的句子成分,如主语、谓语、宾语、状语等。各个短语的构成将在(交叉引用)中详述。

古诺尔斯语有一类特殊的句子不需要主语,这称为无人称结构(Impersonal
structure)。无人称结构常常由一些没有特定的施事者的动词引起,例如与时间流逝、季节变换有关的动词,但有的时候,无人称结构和被动语态紧密联系在一起,详见(交叉引用)。

\section{限定动词}\label{限定动词}

限定动词包含时态和语气的形态标记,并根据主语的人称和数发生屈折(见\ref{动词与变位法})。限定动词是限定句的必有成分,它的语法范畴对整句话的表意都产生影响。本节主要介绍动词的时态、情态和主谓一致性。

\subsection{时态}\label{时态}

时态是用于描述时间参考的语法范畴,在古诺尔斯语中时态由动词屈折体现。我们在\ref{动词的概述}中已经指出,动词严格来说只有两个时态:现在时和过去时。现在时是使用最广泛的时态,除了严格需要表达过去的情况,动词都可以使用现在时。因此,现在时是默认的时态(Unmarked
tense)。过去时则只能在描述过去事件时使用。现在时有两种相对特殊的用法:

1. 将来的表达

将来时没有对应的动词屈折,语义上的将来时可以由形态上的现在时表达,例如:

\begin{longtable}{lllllll}
  \toprule
  þar   & \emph{liggr} & hann & í  & bǫndum & til   & ragnarøkkrs \\
  \midrule
  \endhead
  \bottomrule
  \endfoot
  there & lies         & he   & in & bond   & until & Ragnarök    \\
  \multicolumn{7}{@{}>{\raggedright\arraybackslash}p{(\columnwidth - 12\tabcolsep) * \real{1.0000} + 12\tabcolsep}@{}}{%
  `There he will lie in bonds until the end of the world'}        \\
\end{longtable}

这句话描述的是洛基被诸神绑缚在石头上,直到诸神黄昏的那天才能脱逃的故事。主句中的限定动词liggr是liggja的第三人称单数现在时,但实际上表达的是将来概念,其对应的英语句中一般就使用将来时will
lie.

但古诺尔斯语也并非完全没有专门用于表达将来的说法,利用助动词munu或skulu加上动词不定式也可以表达将来,对应英语的will
do或shall do, 详见交叉引用。故上句中的liggr也可改成mun
liggja或更好的skal liggja.

2. 历史现在时

历史现在时(Historical
present)指的是在叙述过去的事件时,使用现在时时态的情况。在希腊语和拉丁语中也有这样的情况,修辞学家认为这种写作手法能使表述更加生动。古诺尔斯语的散文中也常有历史现在时,特别是在叙事时经常出现现在时和过去时的交替,相邻的两个小句中的时态就可以不一致,例如:

\begin{longtable}{lllll}
  \toprule
  Óláfr & \emph{svarar} & fá  & ok  & \emph{hló} \\
  \midrule
  \endhead
  \bottomrule
  \endfoot
  Olaf  & answers       & few & and & laughed    \\
  \multicolumn{5}{@{}>{\raggedright\arraybackslash}p{(\columnwidth - 8\tabcolsep) * \real{1.0000} + 8\tabcolsep}@{}}{%
  `Olaf said little in reply and laughed'}       \\
\end{longtable}

在连词ok引导的两个简单句中,前面一个使用的是现在时,后面使用的是过去时,虽然表达的都是过去发生的事件。

\subsection{语气}\label{语气}

古诺尔斯语的三个语气是:直陈、虚拟、祈使。

直陈是最基本的语气,用于表达说话者所认定的客观事实。直陈语气是陈述句的基本语气,对此不多加赘述。

虚拟语气表达对非现实的场景的描述,如愿望、情感、可能性等。许多动词接续从句作为补足语时常用虚拟语气,这是动词的语气引起的虚拟语气,将在交叉引用中详述。在主句中,使用虚拟语气的动词常常表达愿望,例如:

\begin{longtable}{llllll}
  \toprule
  Guð                  & \emph{þakki} & yðr  &        &      &              \\
  \midrule
  \endhead
  \bottomrule
  \endfoot
  God                  & thank-SUB    & you  &        &      &              \\
  \multicolumn{3}{@{}>{\raggedright\arraybackslash}p{(\columnwidth - 10\tabcolsep) * \real{0.5315} + 4\tabcolsep}}{%
  `May God thank you'} &              &      &                              \\
  \multicolumn{3}{@{}>{\raggedright\arraybackslash}p{(\columnwidth - 10\tabcolsep) * \real{0.5315} + 4\tabcolsep}}{%
  }                    &              &      &                              \\
  hverr                & er           & eyru & hefir, & hann & \emph{heyri} \\
  whoever              & who          & ears & has    & he   & hear- SUB    \\
  \multicolumn{6}{@{}>{\raggedright\arraybackslash}p{(\columnwidth - 10\tabcolsep) * \real{1.0000} + 10\tabcolsep}@{}}{%
  `He who has ears ought to hear!'}                                         \\
\end{longtable}

祈使语气不在从句中使用,且只有第二人称和复数第一人称形式。最常见的祈使语气用在动词的第二人称上,表示对听者的命令。有时人称词尾附在动词后面:

\begin{longtable}{lllll}
  \toprule
  \emph{lát-tu} & mik & sjá & fyrir & honum \\
  \midrule
  \endhead
  \bottomrule
  \endfoot
  let-IMP-you   & me  & see & for   & him   \\
  \multicolumn{5}{@{}>{\raggedright\arraybackslash}p{(\columnwidth - 8\tabcolsep) * \real{1.0000} + 8\tabcolsep}@{}}{%
  `Let me take care of him'}                \\
\end{longtable}

láttu是动词láta的第二人称单数祈使式和人称词尾þú的合写。

\subsection{主谓一致性}\label{主谓一致性}

限定动词的人称和数与主语的人称和数保持一致,这是主谓一致性的基本原则。有一些特殊情况值得专门说明:

1. 主语是双数的情况

古诺尔斯语的人称代词中仍保留了双数,没有专门的动词双数形式与之对应,因此总是使用动词复数形式进行对应。

\begin{longtable}{llllll}
  \toprule
  þit    & \emph{munuð} & færa  & mér & hǫfuð & hans \\
  \midrule
  \endhead
  \bottomrule
  \endfoot
  you-DU & shall        & bring & me  & head  & his  \\
  \multicolumn{6}{@{}>{\raggedright\arraybackslash}p{(\columnwidth - 10\tabcolsep) * \real{1.0000} + 10\tabcolsep}@{}}{%
  `You two shall bring his head to me'}              \\
\end{longtable}

2. 缺少主格主语的情况

古诺尔斯语中存在一种无人称结构,这类句子中没有显式的主语(以主格标记出来的)。语义上的主语可能以从句承担,但从句作为一个整体而言是没有格标记的;有一些动词也总是不能接施事性论元,详见交叉引用。此时动词一般采用第三人称单数。

A. 以从句为主语的情况:

\begin{longtable}{llll}
  \toprule
  óvíst   & \emph{er} & at & vita  \\
  \midrule
  \endhead
  \bottomrule
  \endfoot
  unclear & is        & to & know  \\
  \multicolumn{4}{@{}>{\raggedright\arraybackslash}p{(\columnwidth - 6\tabcolsep) * \real{1.0000} + 6\tabcolsep}@{}}{%
  `It is difficult to understand'} \\
\end{longtable}

这类句子几乎总是翻译为英语中的it is \ldots{}
结构,it是形式主语,该句的真实主语由后置的从句承担。

B. 动词的无人称用法:

\begin{longtable}{lll}
  \toprule
  en  & er   & \emph{haustaði} \\
  \midrule
  \endhead
  \bottomrule
  \endfoot
  but & when & came-autumn     \\
  \multicolumn{3}{@{}>{\raggedright\arraybackslash}p{(\columnwidth - 4\tabcolsep) * \real{1.0000} + 4\tabcolsep}@{}}{%
  `But when autumn came'}      \\
\end{longtable}

本句中的动词haustaði的原形是hausta,属二类弱动词。hausta由名词haust
`autumn'衍生,作为一个表示季节转变的动词,它没有明显的动作执行者,因此总是用于无人称结构中。

3. 并列主语的情况

并列主语在在多数情况下表达的是复数的概念,虽然有时也可以被理解为一个整体。在古诺尔斯语中,这种情况下并不一定要求用动词的复数式与之对应。动词常与其最邻近的主语保持一致,类似于英语的就近原则,但这也不是严格的规则。总体来说,并列主语对动词数的要求不高。参考下面的例句:

A. 并列主语用单数

\begin{longtable}{lllllll}
  \toprule
  hann & segir & at   & korn & ok  & malt & \emph{var} \\
  \midrule
  \endhead
  \bottomrule
  \endfoot
  he   & says  & that & corn & and & malt & was        \\
  \multicolumn{7}{@{}>{\raggedright\arraybackslash}p{(\columnwidth - 12\tabcolsep) * \real{1.0000} + 12\tabcolsep}@{}}{%
  `He says it was corn and malt'}                      \\
\end{longtable}

B. 并列主语用复数

\begin{longtable}{llllll}
  \toprule
  \emph{sat} & konungr       & ok   & dróttning & í  & hásæti    \\
  \midrule
  \endhead
  \bottomrule
  \endfoot
  sat-3S     & king          & and  & queen     & in & high-seat \\
  ok         & \emph{drukku} & bæði & samt      &    &           \\
  and        & drank-3P      & both & together  &    &           \\
  \multicolumn{6}{@{}>{\raggedright\arraybackslash}p{(\columnwidth - 10\tabcolsep) * \real{1.0000} + 10\tabcolsep}@{}}{%
  `The king and queen sat in high seat and drank together'}      \\
\end{longtable}

本句中的两个动词的数不一样,第一个动词sat是单数式,但drukku是复数式,这进一步说明数的一致性在并列主语的情况下不是特别重要。注意bæði
samt一般理解为副词,和allt
samt等对应,这样bæði就不作第二个分句的代词主语使用。

\section{主语及其变形}\label{主语及其变形}

主语是在句子中带有主格标记的成分。这种定义方式是纯粹形式化的,有时语义上的主语并不带有主格标记。主语在句子结构中的位置同样很复杂,我们首先引入几个概念。

一个典型的句子由一个名词短语(Noun Phrase,NP)和一个动词短语(Verb
Phrase,VP)构成。名词短语作为句子的主语,动词短语作为谓语,对主语进行描述。名词短语和动词短语的内部可以包含各种各样的单词(只要它们能合法地聚合起来),但是名词短语和动词短语本身则有清楚的界限,它们的词性是不同的。包含在动词短语内部的谓语论元称为内部论元(Internal
argument),反之则称为外部论元(External
argument)。例如在下面的句子中:

\begin{quote}
  John reads the book.
\end{quote}

John是句子中的名词短语。reads the
book是句子中的动词短语,它进一步由限定动词reads和名词短语the
book构成。在本句中,John是外部论元,the book是内部论元。

一般而言,外部论元是谓语动词的限定词(Specifier),支配其变位。内部论元是动词的补语。

\subsection{主动句---外部论元作主语}\label{主动句---外部论元作主语}

外部论元作主语是最常见也是最基本的情况。绝大多数主动句中的动词都指派外部论元作为主语。根据动词语义的不同,外部论元可能表示动作的发出者,也可能指示性质、状态或发生非自主变化的主体。在下面的两个句子中,前者的动词是表示动作的,后者是表示所有关系或性质的,它们主语的语义角色有所不同。不过,读者应该不会对主语的判断产生困难。

\begin{longtable}{llllll}
  \toprule
  þá   & reið & \emph{Óðinn} & fyrir & austan  & dyrr \\
  \midrule
  \endhead
  \bottomrule
  \endfoot
  then & rode & Odin-N       & forth & eastern & door \\
  \multicolumn{6}{@{}>{\raggedright\arraybackslash}p{(\columnwidth - 10\tabcolsep) * \real{1.0000} + 10\tabcolsep}@{}}{%
  `Then Odin rode to the eastern door'}               \\
\end{longtable}

\begin{longtable}{llll}
  \toprule
  \emph{Óðinn} & átti & tvá & brœðr    \\
  \midrule
  \endhead
  \bottomrule
  \endfoot
  Odin-N       & had  & two & brothers \\
  \multicolumn{4}{@{}>{\raggedright\arraybackslash}p{(\columnwidth - 6\tabcolsep) * \real{1.0000} + 6\tabcolsep}@{}}{%
  `Odin had two brothers'}             \\
\end{longtable}

\subsection{被动句---内部论元作主语}\label{被动句---内部论元作主语}

内部论元作主语涉及到将动词的补语提升为句子的主语。一个动词可以接续三个间接格中任意一个作为自己的补语(详见交叉引用),这些都属于内部论元。但是,其中只有宾格宾语能提升为主语。涉及到这一转化的共有三种结构:

\begin{enumerate}
  \def\labelenumi{\arabic{enumi}.}
  \item
        被动句
\end{enumerate}

被动句的主语对应相应的主动句的宾格宾语。被动句的目的是抑制相应的主动句中的主语,把句子的主题转移到谓语及其结果上。因此,被动句的主语是谓语的内部论元。

在古诺尔斯语中,被动句由助动词vera/verða+动词的过去分词构成,与英语的be
done结构相似。过去分词在形态学上和形容词属于同一类(参见\ref{过去分词}),因此理论上需要和主语的格、性、数保持一致。但后来,也可以一律使用单数中性主格。

\begin{longtable}{llll}
  \toprule
  Óláfr & var & skírðr            & þar   \\
  \midrule
  \endhead
  \bottomrule
  \endfoot
  Olaf  & was & baptized-M-SING-N & there \\
  \multicolumn{4}{@{}>{\raggedright\arraybackslash}p{(\columnwidth - 6\tabcolsep) * \real{1.0000} + 6\tabcolsep}@{}}{%
  `Olaf was baptized there'}              \\
\end{longtable}

一些动词能接一个宾格宾语,一个与格宾语。最典型的动词是gefa `give',
其与格宾语表示人,宾格宾语表示物,相当于英语中give sb.
sth.的结构。在相应的被动句中,宾格宾语变成主格,而与格宾语保持不变:

\begin{longtable}{llllll}
  \toprule
  var & þeim   & gefit       & ǫl    & at & drekka \\
  \midrule
  \endhead
  \bottomrule
  \endfoot
  was & them-D & given-NEU-N & ale-N & to & drink  \\
  \multicolumn{6}{@{}>{\raggedright\arraybackslash}p{(\columnwidth - 10\tabcolsep) * \real{1.0000} + 10\tabcolsep}@{}}{%
  `Ale was given to them to drink'}                \\
\end{longtable}

在本句中,ǫl从宾语变成了主语,而表示承受者的þeim保持不变,其对应的主动句可以为:

\begin{longtable}{llllll}
  \toprule
  Konungr & gaf  & þeim   & ǫl    & at & drekka \\
  \midrule
  \endhead
  \bottomrule
  \endfoot
  King    & gave & them-D & ale-A & to & drink  \\
  \multicolumn{6}{@{}>{\raggedright\arraybackslash}p{(\columnwidth - 10\tabcolsep) * \real{1.0000} + 10\tabcolsep}@{}}{%
  `The king gave them ale to drink'}            \\
\end{longtable}

\begin{enumerate}
  \def\labelenumi{\arabic{enumi}.}
  \setcounter{enumi}{1}
  \item
        vera+现在分词
\end{enumerate}

vera+现在分词有一种特殊的被动含义,意为该动作是合适的、可能的、必要的意味,相当于英文中should
be
done结构。因此,这种结构中的主语也是由相应动词的宾语变化而来。现在分词在变形上也和形容词同类,与主语的性、数、格保持一致。但考虑到现在分词只有弱变格,其要么以-i结尾,要么以-a结尾。

\begin{longtable}{lllll}
  \toprule
  at & kveldi & er & dagr  & lofandi          \\
  \midrule
  \endhead
  \bottomrule
  \endfoot
  at & knight & is & day-N & prasing-M-SING-N \\
  \multicolumn{5}{@{}>{\raggedright\arraybackslash}p{(\columnwidth - 8\tabcolsep) * \real{1.0000} + 8\tabcolsep}@{}}{%
  `At evening the day should be praised'}     \\
\end{longtable}

\begin{enumerate}
  \def\labelenumi{\arabic{enumi}.}
  \setcounter{enumi}{2}
  \item
        vera+at-不定式
\end{enumerate}

vera+at-不定式有类似于vera+现在分词的作用。

\begin{longtable}{lllllll}
  \toprule
  eru & slíkar & mínar  & at & segja & frá   & honum \\
  \midrule
  \endhead
  \bottomrule
  \endfoot
  are & such-N & mine-N & to & say   & about & him   \\
  \multicolumn{7}{@{}>{\raggedright\arraybackslash}p{(\columnwidth - 12\tabcolsep) * \real{1.0000} + 12\tabcolsep}@{}}{%
  `This is all I have to say about him'}             \\
\end{longtable}

其中slíkar mínar本是segja的宾语,现在移动到了主格位置。

\subsection{主语提升}\label{主语提升}

主语提升(Subject
raising)指的是一个低一级分句中的主语提升到高一级的分句中作主语,原分句中的谓语动词改为不定式作主动词的补语。参见英文中含义相同的两个句子:

\begin{quote}
  1. It seems that he has left.

  2. He seems to have left.
\end{quote}

在句1中,没有出现主语提升的现象,he是从句he has
left的主语,这个从句比主句It seems that \ldots{}
要低一级。句2是句1发生主语抬升后的现象,低一级的从句的主语he``提升''到了主句当中,成为了整个句子的主语,同时原先的谓语has
left变成了不定式to have left作为主动词seems的补足语。

并不是所有谓语动词都支持这样的提升,在古诺尔斯语中,最常见的提升动词是þykkja
`seem',这是一个不规则动词,参考\ref{第一弱变位法}.
这个动词最常见的用法如下所示:

\begin{longtable}{llllllll}
  \toprule
  þá   & þótti  & mér  & þeir   & sœkja  & at & ǫllum & megin \\
  \midrule
  \endhead
  \bottomrule
  \endfoot
  then & seemed & me-D & they-N & attack & at & all   & sides \\
  \multicolumn{8}{@{}>{\raggedright\arraybackslash}p{(\columnwidth - 14\tabcolsep) * \real{1.0000} + 14\tabcolsep}@{}}{%
  (a) `Then it seemed to me that they attacked on all sides'

  (b) `Then I thought they attacked on all sides'}            \\
\end{longtable}

这句话中的主语是þeir,通常þykkja还接续一个体验者,即产生感受的主体,用与格标出,即本句中的mér.
主语的动作用不定式(本句中的sœkja)标记,这个不定式不需要at引导\footnote{注意,本句中的at是和sœkja连用的介词,表示敌意,参考英语中的shout
  to vs shout at; rush to vs rush at.
  以at引导不定式的时候,不定式总紧跟在at后面。},类似于宾格-不定式结构(参见交叉引用)。þykkja可以和句中的主语保持主谓一致,也可以用第三人称单数式,一般是þykkir或þótti,
有时当体验者为第一或第二人称时,还可以用þykki.
故þykkja引起的句式一般为:

\begin{quote}
  \textbf{þykkir/þótti + 体验者-D + {[}主语-N + 不定式 + 其他成分{]}}
\end{quote}

其中以{[}{]}标出的部分为一个非限定性的从句,它和正常的限定句的区别仅在于限定动词变成了不定式。

如果这个从句的谓语部分是由vera引导的主语补足语,如vera+名词短语或形容词短语。则主语补足语的格、性、数要与与整句的主语一致,例如:

\begin{longtable}{llllll}
  \toprule
  torsóttr    & þótta     & ek  & yðr   & næstum & vera \\
  \midrule
  \endhead
  \bottomrule
  \endfoot
  difficult-N & seemed-1S & I-N & you-D & last   & be   \\
  \multicolumn{6}{@{}>{\raggedright\arraybackslash}p{(\columnwidth - 10\tabcolsep) * \real{1.0000} + 10\tabcolsep}@{}}{%
  `You thought I was difficult last time'}              \\
\end{longtable}

注意本句中þótta和主语ek一致,而非采用第三人称形式。这句话的语序和我们前面给出的标准结构略有区别,这是因为作者有意强调形容词torsóttr,
如果转变成我们熟悉的语序,本句为þótta yðr ek vera torsóttr næstum.
但无论如何,torsóttr都以主格形式与ek相对应。

这种结构中的vera有时可以省略:

\begin{longtable}{llllll}
  \toprule
  ǫll & þín  & orðrœða & þykki & mér  & góð    \\
  \midrule
  \endhead
  \bottomrule
  \endfoot
  all & your & talk-N  & seems & me-D & good-N \\
  \multicolumn{6}{@{}>{\raggedright\arraybackslash}p{(\columnwidth - 10\tabcolsep) * \real{1.0000} + 10\tabcolsep}@{}}{%
  `All your talk sounds good to me'}           \\
\end{longtable}

如果句子的主语和体验者恰好是同一个,þykkja用其反身形式þykkjask:

\begin{longtable}{lllll}
  \toprule
  þykkjask    & þeir   & þar   & kenna & Lúsa-Odda   \\
  \midrule
  \endhead
  \bottomrule
  \endfoot
  seem-3P-RFL & they-N & there & know  & Lusa-Oddi-A \\
  \multicolumn{5}{@{}>{\raggedright\arraybackslash}p{(\columnwidth - 8\tabcolsep) * \real{1.0000} + 8\tabcolsep}@{}}{%
  `They think they recognize Lusa-Oddi there'}       \\
\end{longtable}

除þykkja以外,一些动词的反身形式也有相同的意思,如sýnask `appear',
virðask
`deem',它们的用法和þykkja一样,但是这里的-sk并不代表句子的主语和体验者是同一个,相反这些动词只是固定下来的形式,它们依旧可以添加其他与格宾语作为体验者。

\subsection{从句作主语}\label{从句作主语}

作主语的从句是名词性的,这类从句总以引导词at开头(参考交叉引用),相当于英文that.
从句可以是限定性的,也可以是非限定性的。从句作主语时,谓语用第三人称单数形式(另见\ref{主谓一致性})。

\begin{longtable}{lllll}
  \toprule
  hǫrmuligt & er & slíkt  & at & vita \\
  \midrule
  \endhead
  \bottomrule
  \endfoot
  sad-NEU-N & is & such-A & to & know \\
  \multicolumn{5}{@{}>{\raggedright\arraybackslash}p{(\columnwidth - 8\tabcolsep) * \real{1.0000} + 8\tabcolsep}@{}}{%
  `It is sad to know that'}           \\
\end{longtable}

\begin{longtable}{lllll}
  \toprule
  oss  & sýnisk & úmakligt,      & at   & \ldots{} \\
  \midrule
  \endhead
  \bottomrule
  \endfoot
  us-D & seems  & unproper-NEU-N & that & \ldots{} \\
  \multicolumn{5}{@{}>{\raggedright\arraybackslash}p{(\columnwidth - 8\tabcolsep) * \real{1.0000} + 8\tabcolsep}@{}}{%
  `We consider it unproper that \ldots'}           \\
\end{longtable}

在上面的两句中,前者是一个非限定性从句,后者是一个没有写完的限定性从句,at后面的部分可以是任意的限定句。

这种把从句直接指派为论元的用法实际上并不多。更多的情况下,是用代词þat作为主语,把从句作为þat的补足语,它表达的意思与从句主语相同,不过这种用法更符合一般的陈述句的结构:

\begin{longtable}{lllllll}
  \toprule
  þat    & er & upphaf      & þeirar & sǫgu,   & at   & \ldots{} \\
  \midrule
  \endhead
  \bottomrule
  \endfoot
  that-N & is & beginning-N & that-G & story-G & that & \ldots{} \\
  \multicolumn{7}{@{}>{\raggedright\arraybackslash}p{(\columnwidth - 12\tabcolsep) * \real{1.0000} + 12\tabcolsep}@{}}{%
  `The beginning of the story is that \ldots'}                   \\
\end{longtable}

\subsection{无人称结构}\label{无人称结构}

无人称结构,或称无主句,指的是句子中没有主格主语的情况。虽然从句作为主语时严格意义上也是没有主格主语的,但我们在这里不讨论这种情况。句中没有主语有三种可能的情况,一是谓语不指派外部论元;二是某些动词的被动语态;三是主语被省略或作者认为不必要表达。前两种情况是词法上是所要求的,第三种情况则主要受语义或上下文的驱动。下面逐一介绍这三类情况:

\textbf{谓语不指派外部论元}

有几类动词总是不能添加主语:

表示时间流逝、季节转换以及自然发生的事件的动词

这类动词缺乏明确的主体,也没有承受者,因此既不能接主语,也不能接宾语,总是单独使用,以第三人称单数出现。包括hausta
`become autumn', snjófa `snow', fjara `ebb', flæða `flood', dimma `get
dark', birta `get bright, dawn', reka `drift'等:

\begin{longtable}{lllll}
  \toprule
  en  & áðr    & hafði & snjófat & nǫkkut \\
  \midrule
  \endhead
  \bottomrule
  \endfoot
  but & before & had   & snowed  & little \\
  \multicolumn{5}{@{}>{\raggedright\arraybackslash}p{(\columnwidth - 8\tabcolsep) * \real{1.0000} + 8\tabcolsep}@{}}{%
  `But earlier it had snowed a bit'}      \\
\end{longtable}

\begin{longtable}{llll}
  \toprule
  fjarar & nú  & undan & skipinu              \\
  \midrule
  \endhead
  \bottomrule
  \endfoot
  ebbs   & now & under & ship-the             \\
  \multicolumn{4}{@{}>{\raggedright\arraybackslash}p{(\columnwidth - 6\tabcolsep) * \real{1.0000} + 6\tabcolsep}@{}}{%
  `The tide now recedes from under the ship'} \\
\end{longtable}

表示身体状态、感觉或思维过程的动词

这类动词一般接续一个间接格表示感受者,有的动词还能进一步接一个间接格表示感觉的内容。

下列的常见动词只接一个间接格表示感受者:

\begin{longtable}{ll}
  \toprule
  接宾格                   & 接与格                  \\
  \midrule
  \endhead
  \bottomrule
  \endfoot
  þyrsta `be thirsty'      & bregða `startle'        \\
  svengja `be hungry'      & bjóða `find disgusting' \\
  svimra/svima `be dizzy'  & líka `like'             \\
  syfja `be sleepy'        & blæða `bleed'           \\
  verkja `be painful'      & hitna `feel hot'        \\
  langa `desire, long for' &                         \\
\end{longtable}

其基础用法都与下句类似:

\begin{longtable}{llll}
  \toprule
  hana                      & þyrsti  & mjók & \\
  \midrule
  \endhead
  \bottomrule
  \endfoot
  her-A                     & thirsts & muck & \\
  \multicolumn{3}{@{}>{\raggedright\arraybackslash}p{(\columnwidth - 6\tabcolsep) * \real{0.8831} + 4\tabcolsep}}{%
  `She feels very thirsty'} &                  \\
\end{longtable}

有时,这些动词也可以接续介词短语表示原因或感受的对象等,例如:

\begin{longtable}{llllll}
  \toprule
  brá      & þeim   & mjök & \emph{við} & \emph{þessi} & tíðindi \\
  \midrule
  \endhead
  \bottomrule
  \endfoot
  startled & them-D & much & with       & these-A      & news-A  \\
  \multicolumn{6}{@{}>{\raggedright\arraybackslash}p{(\columnwidth - 10\tabcolsep) * \real{1.0000} + 10\tabcolsep}@{}}{%
  `The news startled them a lot'}                                \\
\end{longtable}

langa和líka常接til+属格表示表示期待、喜欢的对象:

\begin{longtable}{lllll}
  \toprule
  mik  & langar & ekki & \emph{til} & \emph{þess} \\
  \midrule
  \endhead
  \bottomrule
  \endfoot
  me-D & longs  & not  & towards    & this        \\
  \multicolumn{5}{@{}>{\raggedright\arraybackslash}p{(\columnwidth - 8\tabcolsep) * \real{1.0000} + 8\tabcolsep}@{}}{%
  `I do not long for this'}                       \\
\end{longtable}

由于介词也只接续间接格(参见交叉引用),句中仍没有主格主语。

以下的常见动词可以接两个简介格表示感受者和感受的内容:

\begin{longtable}{llll}
  \toprule
  动词                 & 感受者 & 感受的内容 & 表意                         \\
  \midrule
  \endhead
  \bottomrule
  \endfoot
  vanta `want'         & 宾格   & 宾格       & sb. want sth.                \\
  dreyma `dream'       & 宾格   & 宾格       & sb. dream about sth.         \\
  gruna `suspect'      & 宾格   & 宾格/属格  & sb. be suspicious about sth. \\
  minna `remember'     & 宾格   & 属格       & sb. remember sth.            \\
  semja `agree on'     & 与格   & 宾格       & sb. agree on sth.            \\
  bresta `lack'        & 宾格   & 宾格       & sb. lack sth.                \\
  skorta `be short of' & 宾格   & 宾格       & sb. lack sth.                \\
\end{longtable}

其用法形如:

\begin{longtable}{llll}
  \toprule
  minnir    & mik  & hinnar & konunnar \\
  \midrule
  \endhead
  \bottomrule
  \endfoot
  remembers & me-A & that-G & woman-G  \\
  \multicolumn{4}{@{}>{\raggedright\arraybackslash}p{(\columnwidth - 6\tabcolsep) * \real{1.0000} + 6\tabcolsep}@{}}{%
  `I remember that woman'}             \\
\end{longtable}

\begin{longtable}{llll}
  \toprule
  skortir & þik   & eigi & metnað  \\
  \midrule
  \endhead
  \bottomrule
  \endfoot
  lacks   & you-A & no   & pride-A \\
  \multicolumn{4}{@{}>{\raggedright\arraybackslash}p{(\columnwidth - 6\tabcolsep) * \real{1.0000} + 6\tabcolsep}@{}}{%
  `You have no lack of pride‌'}     \\
\end{longtable}

表示``缺乏''的动词在语义上和主观的感情或思维过程略有区别,它们反映的是客观事实。不过它们的用法十分相似,因此把它们都归到一类。

上述的这些动词也并非都只有无人称的用法,有时这些动词也能像正常的动词一般使用,如minna也可以指派一个主格主语,这时候它的含义是``提醒'',并有形如英语remind
sb. of
sth.的结构,即提醒的对象用与格(区分于无人称结构中感受者用宾格),提醒的内容用属格。

与事件的``发生、结束''有关的动词可以接两个宾格,一个宾格表示事件发生的对象,另一个宾格表示事件本身。这类动词包括henda
`be caught in, happen', þrjóta `come to end, fail', takask
`succeed'等。有时它们也可以只接一个表示事件的宾格,这时不关注事件发生的对象。参考下面的两句,前者接两个宾格,后者只接表示事件的宾格:

\begin{longtable}{llllllll}
  \toprule
  mik  & hefir & hent     & mart   & til     & afgerða & við     & Guð \\
  \midrule
  \endhead
  \bottomrule
  \endfoot
  me-A & has   & happened & many-A & towards & offence & against & God \\
  \multicolumn{8}{@{}>{\raggedright\arraybackslash}p{(\columnwidth - 14\tabcolsep) * \real{1.0000} + 14\tabcolsep}@{}}{%
  `I have happened to commit many sins against God'}                   \\
\end{longtable}

\begin{longtable}{llll}
  \toprule
  en  & er   & veizluna      & þrýtr             \\
  \midrule
  \endhead
  \bottomrule
  \endfoot
  but & when & banquet-the-A & ends              \\
  \multicolumn{4}{@{}>{\raggedright\arraybackslash}p{(\columnwidth - 6\tabcolsep) * \real{1.0000} + 6\tabcolsep}@{}}{%
  `But when it comes to the end of the banquet'} \\
\end{longtable}

一些形容词有和上述动词相同的表意,它们构成的系表结构也是无人称的:

\begin{longtable}{llll}
  \toprule
  þá   & var & myrkt & mjǫk  \\
  \midrule
  \endhead
  \bottomrule
  \endfoot
  then & was & dark  & muck  \\
  \multicolumn{4}{@{}>{\raggedright\arraybackslash}p{(\columnwidth - 6\tabcolsep) * \real{1.0000} + 6\tabcolsep}@{}}{%
  `The it became very dark'} \\
\end{longtable}

\begin{longtable}{lll}
  \toprule
  mér  & er & kalt \\
  \midrule
  \endhead
  \bottomrule
  \endfoot
  me-D & is & cold \\
  \multicolumn{3}{@{}>{\raggedright\arraybackslash}p{(\columnwidth - 4\tabcolsep) * \real{1.0000} + 4\tabcolsep}@{}}{%
  `I feel cold'}   \\
\end{longtable}

\textbf{无人称被动}

我们在\ref{被动句---内部论元作主语}中提到,被动句的主语来自于相应的主动句的谓语的内部论元,且只有宾格论元可提升为被动句的主格主语。按照这种原则,有两类动词的被动句缺少主语。

不及物动词

不及物动词只指派一个外部论元作为主语,因此根本不存在内部论元可供提升。这类句子在英语中是不能变成被动句的:

\begin{quote}
  主动句:He danced.

  被动句:†is danced.
\end{quote}

但在古诺尔斯语中,不及物动词也能变成被动句,这种被动句是无人称的,且其表意一般是对存在性的判断。要构成不及物动词的被动句,只需把相应主动句的动词改成vera+过去分词。和其他无人称结构一样,vera一般用第三人称单数式。过去分词用中性形式。试比较下面两句:

主动句:

\begin{longtable}{llllll}
  \toprule
  gekk & hann & inn & nǫkkut  & fyrir  & lýsing \\
  \midrule
  \endhead
  \bottomrule
  \endfoot
  went & he-N & in  & shortly & before & dawn   \\
  \multicolumn{6}{@{}>{\raggedright\arraybackslash}p{(\columnwidth - 10\tabcolsep) * \real{1.0000} + 10\tabcolsep}@{}}{%
  `He went in shortly before dawn'}             \\
\end{longtable}

被动句:

\begin{longtable}{llllll}
  \toprule
  var & gengit     & inn & nǫkkut  & fyrir  & lýsing \\
  \midrule
  \endhead
  \bottomrule
  \endfoot
  was & gone-NEU-N & in  & shortly & before & dawn   \\
  \multicolumn{6}{@{}>{\raggedright\arraybackslash}p{(\columnwidth - 10\tabcolsep) * \real{1.0000} + 10\tabcolsep}@{}}{%
  `Someone went in shortly before dawn'}             \\
\end{longtable}

被动句同样不强调动作的主体,其表意侧重``某事发生了''。

不接宾格宾语的动词

\begin{quote}
  许多动词不接续宾格宾语,如bjarga `save', lesa `read', fagna
  `welcome'接与格,leita `seek', bíða `wait
  for'接属格等等(详见交叉引用)。这些动词构成被动句时不能把其非宾格宾语提升为主格主语,因此这类动词也引起无人称结构。要构成被动句,只需要把主动句动词改为vera+过去分词,原先的非宾格宾语保持原先的格不变。试比较:
\end{quote}

主动句

\begin{longtable}{lllll}
  \toprule
  þar   & fǫgnuðu  & þeir   & Þorsteini   & vel  \\
  \midrule
  \endhead
  \bottomrule
  \endfoot
  there & welcomed & they-N & Thorstein-D & well \\
  \multicolumn{5}{@{}>{\raggedright\arraybackslash}p{(\columnwidth - 8\tabcolsep) * \real{1.0000} + 8\tabcolsep}@{}}{%
  `They welcomed Thorstein warmly'}              \\
\end{longtable}

被动句

\begin{longtable}{lllll}
  \toprule
  Þorsteini   & var & þar   & vel  & fagnat         \\
  \midrule
  \endhead
  \bottomrule
  \endfoot
  Thorstein-D & was & there & well & welcomed-NEU-N \\
  \multicolumn{5}{@{}>{\raggedright\arraybackslash}p{(\columnwidth - 8\tabcolsep) * \real{1.0000} + 8\tabcolsep}@{}}{%
  `Thorstein was well received there'}              \\
\end{longtable}

我们在\ref{被动句---内部论元作主语}中还提到了与vera+过去分词类似的被动结构,即vera+现在分词或at-不定式。这类结构有一个特点,即不严格要求把宾格宾语提升为主格主语,下面的句子也是可行的:

\begin{longtable}{lllll}
  \toprule
  nú  & er & at & verja  & \emph{sik}       \\
  \midrule
  \endhead
  \bottomrule
  \endfoot
  now & is & to & defend & oneself-A        \\
  \multicolumn{5}{@{}>{\raggedright\arraybackslash}p{(\columnwidth - 8\tabcolsep) * \real{1.0000} + 8\tabcolsep}@{}}{%
  `Now it is about time to defend oneself'} \\
\end{longtable}

本句话保持宾格sik不变的另一个原因在于反身代词没有主格形式。

\begin{longtable}{lllllll}
  \toprule
  eigi & er & virðandi    & \emph{ásjónir} & manna & í & dómum     \\
  \midrule
  \endhead
  \bottomrule
  \endfoot
  not  & is & considering & appearance-A   & men-G & i & judgement \\
  \multicolumn{7}{@{}>{\raggedright\arraybackslash}p{(\columnwidth - 12\tabcolsep) * \real{1.0000} + 12\tabcolsep}@{}}{%
  `Judge not by one's appearance'}                                 \\
\end{longtable}

\begin{longtable}{llll}
  \toprule
  \emph{þess} & er & fyrst & leitanda  \\
  \midrule
  \endhead
  \bottomrule
  \endfoot
  that-G      & is & first & examining \\
  \multicolumn{4}{@{}>{\raggedright\arraybackslash}p{(\columnwidth - 6\tabcolsep) * \real{1.0000} + 6\tabcolsep}@{}}{%
  `It should be examined first'}       \\
\end{longtable}

和其他无主语句一样,动词使用第三人称单数,现在分词使用-i/-a词尾均可。

主语省略或隐含

上述表示时间流逝、季节转换以及自然发生的事件的动词有一些近义的表达,这些表达里并不涉及典型的无人称动词,但是它们也不需要添加主语:

\begin{longtable}{lllll}
  \toprule
  gerði & nú  & myrkt & af & nótt  \\
  \midrule
  \endhead
  \bottomrule
  \endfoot
  made  & now & dark  & at & night \\
  \multicolumn{5}{@{}>{\raggedright\arraybackslash}p{(\columnwidth - 8\tabcolsep) * \real{1.0000} + 8\tabcolsep}@{}}{%
  `It got dark at night'}          \\
\end{longtable}

如果句子的主语是泛指且谓语的主体并非表意的重心时,在不引起歧义的情况下主语可以省略。当句中的谓语是由情态动词构成的短语时,这种情况尤其常见:

\begin{longtable}{lllll}
  \toprule
  má  & þar   & ekki & stórskipum  & fara   \\
  \midrule
  \endhead
  \bottomrule
  \endfoot
  can & there & not  & big-ships-D & travel \\
  \multicolumn{5}{@{}>{\raggedright\arraybackslash}p{(\columnwidth - 8\tabcolsep) * \real{1.0000} + 8\tabcolsep}@{}}{%
  `One cannot travel there with big ships'} \\
\end{longtable}

\begin{longtable}{lllllll}
  \toprule
  \multicolumn{2}{c}{standi}                                & menn     & upp   & ok    & taki         & hann,   \\
  \midrule
  \endhead
  \bottomrule
  \endfoot
  \multicolumn{2}{@{}>{\raggedright\arraybackslash}p{(\columnwidth - 12\tabcolsep) * \real{0.2577} + 2\tabcolsep}}{%
  stand-SUB-3P}                                             & men-N    & up    & and   & seize-SUB-3P & him-A   \\
  ok                                                        & skal     & hann  & drepa &              &       & \\
  and                                                       & shall-3S & him-A & kill  &              &       & \\
  \multicolumn{6}{@{}>{\raggedright\arraybackslash}p{(\columnwidth - 12\tabcolsep) * \real{0.8675} + 10\tabcolsep}}{%
  `Let men stand up and seize him, and he is to be killed'} &                                                   \\
\end{longtable}

第2句比较复杂,主要是因为hann既是主格又是宾格导致的。注意menn是maðr的复数主格形式,它是动词standi和taki的主语,动词虚拟式表示要求。第一句中taki
hann和第二句中skal
hann的hann是同指的,且都是宾格,这是因为drepa不能作不及物动词使用。第二句中的skal暗示被省略的主语是单数形式,从上下文来看是前一句中menn之中的某一人。

主语省略还可能发生在复合句中,当后一句的主语在第一句中已经提到,并可以不产生歧义地被推断出来时,后一句的主语可以省略。在英语中,只有后一句的主语也是前一句的主语时,后一句的主语才能省略:

He rushed out of the building, ran across the road and disappeared in
the crowd.

†He is reading the book, is about the history of Iceland.

但在古诺尔斯语中,后一句中的主语可以在前一句中充当任何成分:

主格主语:

\begin{longtable}{llllllll}
  \toprule
  Egill & kastar & horninu,  & en  & greip   & sverðit   & ok  & Brá  \\
  \midrule
  \endhead
  \bottomrule
  \endfoot
  Egil  & casts  & horn-the, & but & grasped & sword-the & and & drew \\
  \multicolumn{8}{@{}>{\raggedright\arraybackslash}p{(\columnwidth - 14\tabcolsep) * \real{1.0000} + 14\tabcolsep}@{}}{%
  `Egil cast the horn as he grasped and drew his sword'}              \\
\end{longtable}

直接宾语:

\begin{longtable}{lllllllllll}
  \toprule
  þá                                                               & skar    & Rǫgnvaldr & \multicolumn{2}{c}{hár} & \multicolumn{2}{c}{hans,} & en & \multicolumn{2}{c}{áðr} & hafði         \\
  \midrule
  \endhead
  \bottomrule
  \endfoot
  then                                                             & cut     & Rognvald  &
  \multicolumn{2}{>{\raggedright\arraybackslash}p{(\columnwidth - 20\tabcolsep) * \real{0.1721} + 2\tabcolsep}}{%
  hair-A}                                                          &
  \multicolumn{2}{>{\raggedright\arraybackslash}p{(\columnwidth - 20\tabcolsep) * \real{0.1093} + 2\tabcolsep}}{%
  his}                                                             & but     &
  \multicolumn{2}{>{\raggedright\arraybackslash}p{(\columnwidth - 20\tabcolsep) * \real{0.1197} + 2\tabcolsep}}{%
  before}                                                          & had                                                                                                                      \\
  verit                                                            & úskorit & tíu       & vetr                    &                           &    &                         &       &  &  & \\
  been                                                             & uncut   & ten       & winter                  &                           &    &                         &       &  &  & \\
  \multicolumn{9}{@{}>{\raggedright\arraybackslash}p{(\columnwidth - 20\tabcolsep) * \real{0.8461} + 16\tabcolsep}}{%
  `Then Rognvald cut his hair which had been uncut for ten years'} &         &                                                                                                                \\
\end{longtable}

无人称结构中的与格对象:

\begin{longtable}{lllllll}
  \toprule
  þat & líkaði  & henni & vel  & ok  & þakkaði & honum \\
  \midrule
  \endhead
  \bottomrule
  \endfoot
  it  & pleased & her-D & well & and & thanked & him   \\
  \multicolumn{7}{@{}>{\raggedright\arraybackslash}p{(\columnwidth - 12\tabcolsep) * \real{1.0000} + 12\tabcolsep}@{}}{%
  `She liked it very much and thanked him'}            \\
\end{longtable}

其他情况下,一般不省略主语。

\section{词序}\label{词序}

像很多屈折语一样,古诺尔斯语丰富的形态曲折使得句子成分的判断几乎不依赖于词序(主要是主语和宾语的判断),因而古诺尔斯语的词序往往被当作是``自由''的。不过,有一些规律可供总结。

动词的位置

限定动词是句子的必有成分,因而动词的位置可以作为整句词序的锚点。一般来说,动词位于句中的第二位,这称之为动词第二顺位(V2
Word order)。许多日耳曼语中都有这样的结构,例如英语中的neither
\emph{do} I.
这里的第二顺位指的并不一定是句子中的第二个词,而是第二个句子成分。一个句子成分可以由多个词组成,例如下面的英语句也符合动词第二顺位:

\begin{quote}
  \includegraphics{media/image4.wmf}
\end{quote}

古诺尔斯语中,动词也符合上述的规则。

动词前后的位置

动词前的位置,即句子的第一个成分,在古诺尔斯语中是任意的,这和英语这样强调词序的语言有着根本的区别。句子的首位可以是主语,也可以是宾语,还可能是状语等等。一般来说,移动到句子开头的成分有被强调的作用,这个位置也被称为主题(Topic)。根据主题的不同,整个句子表现出不同的词序:

\textbf{副词}

许多句子副词被提到句首位置,最典型的是þá `then', nú `now', síðan
`since', svá `thus'等。这时,典型的句型是:

\textbf{Adv + V + S + (O)}

句子的主语一般紧随动词之后,之后可接宾语以及其他状语等,例如:

\begin{longtable}{llllll}
  \toprule
  þá   & reið & Óðinn  & fyrir & austan  & dyrr. \\
  \midrule
  \endhead
  \bottomrule
  \endfoot
  Then & rode & Odin-N & forth & eastern & door  \\
  \multicolumn{6}{@{}>{\raggedright\arraybackslash}p{(\columnwidth - 10\tabcolsep) * \real{1.0000} + 10\tabcolsep}@{}}{%
  `Then Odin rode to the eastern door'}          \\
\end{longtable}

否定副词也可以移动到句子首位:

\begin{longtable}{lllllll}
  \toprule
  eigi & munu  & vápn      & eða & viðir   & granda & Baldri   \\
  \midrule
  \endhead
  \bottomrule
  \endfoot
  not  & shall & weapons-N & or  & trees-N & hurt   & Baldur-D \\
  \multicolumn{7}{@{}>{\raggedright\arraybackslash}p{(\columnwidth - 12\tabcolsep) * \real{1.0000} + 12\tabcolsep}@{}}{%
  `No weapon or tree shall hurt Baldur'}                       \\
\end{longtable}

\textbf{主语}

主语出现在句首后,动词后的位置常常被句子副词占据,因此结构类似于:

\textbf{S + V + (Adv) + (O)}

\begin{longtable}{llllll}
  \toprule
  ljós    & brann & í  & skálanum & um         & nóttina   \\
  \midrule
  \endhead
  \bottomrule
  \endfoot
  light-N & burnt & in & hall-the & throughout & night-the \\
  \multicolumn{6}{@{}>{\raggedright\arraybackslash}p{(\columnwidth - 10\tabcolsep) * \real{1.0000} + 10\tabcolsep}@{}}{%
  `A light burnt in the hall throughout the night'}        \\
\end{longtable}

\textbf{宾语}

宾语出现在句首的情况比较罕见,此时动词后一般立刻接主语:

\textbf{O+ V + S + (Adv)}

\begin{longtable}{lllll}
  \toprule
  þik   & vil  & ek & enn  & fregna \\
  \midrule
  \endhead
  \bottomrule
  \endfoot
  you-A & will & I  & more & ask    \\
  \multicolumn{5}{@{}>{\raggedright\arraybackslash}p{(\columnwidth - 8\tabcolsep) * \real{1.0000} + 8\tabcolsep}@{}}{%
  `I wish to ask you more'}         \\
\end{longtable}

\textbf{短语的中心语}

中心语是短语中被修饰语修饰、限制的中心成分。一般来说,一个短语会作为一个整体出现在句子中的某个位置。但在古诺尔斯语中,有时甚至可以把短语拆开,并将其中心语前置:

\begin{longtable}{lllll}
  \toprule
  \emph{styrks} & eiga & ván  & \emph{af} & \emph{Skota-konungi} \\
  \midrule
  \endhead
  \bottomrule
  \endfoot
  support-G     & have & hope & from      & Scots-king           \\
  \multicolumn{5}{@{}>{\raggedright\arraybackslash}p{(\columnwidth - 8\tabcolsep) * \real{1.0000} + 8\tabcolsep}@{}}{%
  `Have faith in the support of the king of the Scots'}          \\
\end{longtable}

本句中styrks af
Skota-konungi这个短语整体作为ván的补足语,但为了强调styrks,把这个短语拆开了。

指示代词前移也比较常见:

\begin{longtable}{lllllll}
  \toprule
  \emph{þau} & skal  & segja & \emph{orð} & \emph{mín} & maðr  & manni \\
  \midrule
  \endhead
  \bottomrule
  \endfoot
  those-A    & shall & say   & words-A    & my         & man-N & man-D \\
  \multicolumn{7}{@{}>{\raggedright\arraybackslash}p{(\columnwidth - 12\tabcolsep) * \real{1.0000} + 12\tabcolsep}@{}}{%
  `Those words of mine shall be told from man to man'}                 \\
\end{longtable}

\textbf{短语的修饰语}

和中心语一样,修饰语也可以移动至句子首位。一般来说,最常见的情况是把名词短语中的形容词前置以构成强调:

\begin{longtable}{llll}
  \toprule
  góðan  & eignum & vér & konung \\
  \midrule
  \endhead
  \bottomrule
  \endfoot
  good-A & have   & we  & king-A \\
  \multicolumn{4}{@{}>{\raggedright\arraybackslash}p{(\columnwidth - 6\tabcolsep) * \real{1.0000} + 6\tabcolsep}@{}}{%
  `We have a good king'}         \\
\end{longtable}

\textbf{空主题}

空主题指的是动词前没有其他的句子成分,即句子本身就以动词开头。按照前述的分类方法,这是一种特殊的情况。造成这种现象的原因主要是句子中缺少某类成分。\ref{无人称结构}中介绍的无人称结构就容易引起空主题:

\begin{longtable}{lllll}
  \toprule
  skal  & hana  & engan & hlut     & skorta \\
  \midrule
  \endhead
  \bottomrule
  \endfoot
  shall & her-A & no-A  & things-A & lack   \\
  \multicolumn{5}{@{}>{\raggedright\arraybackslash}p{(\columnwidth - 8\tabcolsep) * \real{1.0000} + 8\tabcolsep}@{}}{%
  `She shall lack nothing'}                 \\
\end{longtable}

\begin{longtable}{llllll}
  \toprule
  brá      & þeim   & mjök & \emph{við} & \emph{þessi} & tíðindi \\
  \midrule
  \endhead
  \bottomrule
  \endfoot
  startled & them-D & much & with       & these-A      & news-A  \\
  \multicolumn{6}{@{}>{\raggedright\arraybackslash}p{(\columnwidth - 10\tabcolsep) * \real{1.0000} + 10\tabcolsep}@{}}{%
  `The news startled them a lot'}                                \\
\end{longtable}

祈使句有时也可以略去主语,将动词放到句首:

\begin{longtable}{llll}
  \toprule
  trúið          & á  & goð & várt \\
  \midrule
  \endhead
  \bottomrule
  \endfoot
  believe-IMP-2P & on & god & our  \\
  \multicolumn{4}{@{}>{\raggedright\arraybackslash}p{(\columnwidth - 6\tabcolsep) * \real{1.0000} + 6\tabcolsep}@{}}{%
  `Trust our god!'}                \\
\end{longtable}

一种特殊的情况值得注意。用ok
`and‌'作连词连接两句时,它后面总是紧跟着第二个句子的动词:

\begin{longtable}{lllllllll}
  \toprule
  hann & lætr        & Hǫtt & fara           & með  & sér,     & ok  & \emph{gørir} & hann \\
  \midrule
  \endhead
  \bottomrule
  \endfoot
  he   & lets        & Hott & go             & with & himself, & and & did          & he   \\
  þat  & nauðugr     & ok   & \emph{kallaði} & hann & \ldots{} &     &              &      \\
  that & unwillingly & and  & said           & he   & \ldots{} &     &              &      \\
  \multicolumn{9}{@{}>{\raggedright\arraybackslash}p{(\columnwidth - 16\tabcolsep) * \real{1.0000} + 16\tabcolsep}@{}}{%
  `He commanded Hott to go with him; and he did so unwillingly, and said
  \ldots‌'}                                                                                 \\
\end{longtable}

\textbf{疑难问题}

\begin{enumerate}
  \def\labelenumi{\arabic{enumi})}
  \item
        i-变异的失效
\end{enumerate}

我们在\ref{变元音}.
\ref{_Ref117017033})。

无论是u-变异还是i-变异,有两类不规则现象值得注意:

1. 无触发条件,但发生元音变异的;

2. 有触发条件,但不发生元音变异的。

对于u-变异而言,基本只有第1类不规则情况,例如阴性名词的单数主格和宾格、中性名词的复数主格和宾格,这些都是由于词干元音*u脱落导致。由于古诺尔斯语的词尾音节发生了大量的元音、辅音脱落,这种不规则现象在历史语言学的视角下实则并非难题。

i-变异也有许多第1类不规则情况,例如强动词现在时的单数式词尾-r,它本来的形式为*-ir(例如哥特语的词尾-is\footnote{从哥特语的词尾可以看出,这个词尾辅音的性质并不明朗。从古诺尔斯语来看,-r很可能从*-ir演变过来,但其他语言中这个辅音都是嘶音s。在卢恩文中这个词尾是用ᛁᛉ(iz)标记的,但卢恩文在记音上可能并不准确。可以确定的是这个音应该是齿龈辅音的舌冠音,可能是/z\textasciitilde ʒ/之间的略有卷舌或颤音色彩的音。因此,传统的卢恩转写中也常常把它记作ʀ,见下。}就保留了触发i-变异的i)。但是,古诺尔斯语的i-变异还有第2类不规则情况,这类情况和词干音节的长短还有密切的联系。

古诺尔斯语有两类词特别值得注意,它们的词干长短与i-变异的失效有比较明显的对应关系:

1. i-词干名词中,长词干名词一般都发生i-变异,短词干一般都不发生i-变异;

2. 一类弱动词中,长词干的过去式中都发生i-变异,短词干都不发生i-变异。

从共时角度来看,这两类情况中都没有显式地触发i-变异的条件,于是从中我们似乎可以总结出:在缺少触发条件的情况下,古诺尔斯语的长词干一般都发生i-变异,但短词干一般不发生。

为了解释这个现象,传统的观点是区分i-变异的触发条件,并且认为i-变异在历史上多次发生,例如瑞典语言学家柯克(Axel
Kock)在19世纪就提出,古诺尔斯语的i-变异分成三个阶段:

\begin{longtable}{l}
  \toprule
  \begin{enumerate}\def\labelenumi{\Alph{enumi}.}\item  约公元600-700年,长词干受i(但不是j)的影响发生元音变异。\end{enumerate} \\
  \midrule
  \endhead
  \bottomrule
  \endfoot
  \begin{minipage}[t]{\linewidth}\raggedright
    \begin{enumerate}
      \def\labelenumi{\Alph{enumi}.}
      \setcounter{enumi}{1}
      \item
            约公元700-850年,短词干受j或iʀ的影响发生元音变异。
    \end{enumerate}
  \end{minipage}                                                                                                \\
  \begin{minipage}[t]{\linewidth}\raggedright
    \begin{enumerate}
      \def\labelenumi{\Alph{enumi}.}
      \setcounter{enumi}{2}
      \item
            公元900年后,保留下来的i造成长词干和短词干的元音变异。
    \end{enumerate}
  \end{minipage}                                                                                                \\
\end{longtable}

这个说法固然可以解释的通:在A阶段,长词干的i-词干名词发生元音变异,同时长词干的一类动词的整个变位系统中都有音变条件(注意,按西弗斯定律,-i/j-在短词干后变成j,从而没有音变条件)。接着在A、B阶段之间,词中的i发生大量的脱落,消灭了短词干的i-词干名词的触发条件。到了B阶段,短词干的一类弱动词发生i-变异(j的影响),同时许多由-r词尾引起的元音变异也都可以解释(iʀ
\textgreater{} r)。最后,那些保留下来的i/j规则地触发i-变异。

但这个说法的主要问题在于,它完全忽视了i-变异发生的语音学基础。另外,其苛刻的音变条件似乎完全是对着古诺尔斯语的情况拟制出来的,颇有从结论倒推过程的嫌疑。但是,如果我们仔细考察i-变异的发音基础,并尽量简化它的发生条件(不希望区分i和j造成的i-变异)、减少它的发生次数(不希望认定i-变异在历史上多次发生),这反倒造成了巨大的麻烦。

\textbf{i-变异的语音学原理}

元音变异是一种带有同化性质的音变,其目的无疑是为了发音的方便。从类型学上看,这种变化在语言中也并不少见,例如芬兰语中后缀的形式就会根据词根中元音的特性发生变化。具体来说,原则上芬兰语不允许前元音和后元音出现在同一个词中,因此同一个语法功能的后缀拥有两种形式,分别匹配前元音和后元音,例如内格后缀:-ssa/-ssä.
这种音变叫作元音和谐。从发音的角度来看,这两种音变非常相似,只不过元音和谐是顺向的,元音变异是逆向的。

既然i-变异有十分清楚且普适性很强的发音基础,它的触发条件最初也应该是纯粹语音学的,而不可能受到任何形态学的影响。这样,i/j作为发音位置相同的音素,不应该成为区分i-变异条件的基础。相反,i-变异的失效从语音学角度来看只能受两个因素的影响:可能发生变化的元音本身的性质以及它与i之间的阻隔。

\includegraphics[width=2.39729in,height=1.80562in]{media/image5.png}

\begin{enumerate}
  \def\labelenumi{\arabic{enumi})}
  \item
        元音自身性质引起的i-变异失效
\end{enumerate}

由于i-变异是一种同化音变,两个差距越大的元音越具备产生i-变异的条件。因此,在所有后元音中,低元音a是最容易造成i-变异的元音,高元音u则最可能失效。一些日耳曼语中只有a(确切地)发生i-变异的痕迹,例如古撒克逊语和古高地德语。a的i-变异又叫主变异(Primary
Umlaut)。如果主变异本身失效,那么这种现象就值得额外的注意。

\begin{enumerate}
  \def\labelenumi{\arabic{enumi})}
  \setcounter{enumi}{1}
  \item
        辅音阻隔引起的i-变异失效
\end{enumerate}

辅音的阻隔又可以分为两个方面:

A. 辅音长度造成的阻隔。由双辅音或辅音簇造成的阻隔比单辅音大。

B.
发音器官、发音方式造成的阻隔。有研究显示,舌面音最不容易造成阻隔,而唇音和舌根音则反之。发音方式可能也有影响,特别地,三个辅音h,
l, r在古高地德语中甚至能让主变异失效。在古诺尔斯语中,l,
r似乎很容易阻挡i-变异。例如i-词干名词stuldr, sultr,
burðr,但是其中的元音u也是造成阻碍的原因之一。

从发音的角度来看,短词干应该比长词干更容易受到i-变异的影响,因为长词干更有可能引起阻隔。但是,在古诺尔斯语中表现出来的结果确恰恰相反:长词干中i-变异大多保持了下来,而短词干中i-变异则消失了。因此,传统观点所认为的``i-变异(至少在部分时期)只发生在长词干中''是不符合发音原理的。

通过语音学原理,我们必须明确,在最初发生i-变异的时期,无论长词干还是短词干一律都受到影响。有一些十分确切的例子可以证明短词干中确实也发生了i-变异,例如介词gegn
`against‌' \textless{}
*gagin,这种词类不存在形态变换,也不太可能受到类比的影响,因此其元音特性可以反映本质的情况。

\textbf{元音省略和i-变异失效}

长短词干既然都能发生i-变异,其后来产生的不规则情况就与造成i-变异的i/j的脱落密切相关。如果i-变异最早发生在元音省略之前,且在元音省略发生的时刻仍然是普遍、能产的规则,那么,当触发i-变异的元音脱落时,就可以预期这些变元音应该恢复为本来的形式。

以i-词干名词gestr为例,按照上述的想法,应该有这样的音变过程:

\begin{longtable}{lll}
  \toprule
  阶段1 & 无省略、i-变异能产 & *gast + *ir \textgreater{} *gestir \\
  \midrule
  \endhead
  \bottomrule
  \endfoot
  阶段2 & 元音省略           & *gast + *r                         \\
  阶段3 & 变元音恢复         & †gastr                             \\
\end{longtable}

但是记录到的形式却是gestr.
短词干名词,如staðr,似乎的确恢复到了变化前的形式。但令人费解的是,卢恩铭文以及一些其他日耳曼语的证据似乎可以证明\footnote{卢恩文的解读以及定代问题一直饱受争议,也有一些学者认为卢恩文的证据既不能证明,也不能证伪元音省略首先发生在长词干后。},元音省略首先发生在长词干后,随后才扩散到了短词干。如果以此为基础,长词干名词应当比短词干更加缺少维持i-变异的条件,我们应该期待短词干名词中保留元音变异,而长词干中元音变异应该恢复,如†gastr,
†steðr,但古诺尔斯语的情况恰恰相反。

\begin{longtable}{lll}
  \toprule
  时间               & 省略情况                   & 举例                                \\
  \midrule
  \endhead
  \bottomrule
  \endfoot
  不晚于公元600年    & 无任何省略                 & faahidoo, raunijaz, -gastiz, hrazaz \\
  约公元625年前      & 长词干后中间音节元音省略   & waatee \textless{}
  *waat\textbf{ij}ee                                                                    \\
  约公元675年前      & 短词干后中间音节元音省略   & satte \textless{}
  *sat\textbf{i}dee                                                                     \\
  约公元626--700年间 & 长词干后末音节元音省略     & wulfz \textless{}
  *wulf\textbf{a}z                                                                      \\
                     & 短词干后末音节a省略        & ver \textless{} *vir\textbf{a}      \\
  约公元830年        & 短词干后末音节其他元音省略 & sun \textless{}
  *sun\textbf{u}                                                                        \\
\end{longtable}

同样的问题还出现在在一类动词的过去时中(参见\ref{第一强变位法}),长词干中现在时和过去时中的i-变异一以贯之,而短词干中只有现在时保留了i-变异,过去时中i-变异恢复。一类弱动词过去时中长短词干动词i-变异的交替现象在其他日耳曼语中也有保留,这称之为``变异还原''(德语Rückumlaut,英语unumlaut)。在德语中,变异还原和我们预期的情况完全一致,长词干动词还原i-变异而短词干动词保留i-变异,这也与古诺尔斯语的情况完全相反。本质上来说,这两种不规则现象属于同一类问题。

一些可能的解释如下:

\textbf{解释1:i-变异作为自然语音规则的消亡发生在两次元音省略之间。}

i-变异在成文的古诺尔斯语中已经不是一个能产的规则,那么,解决i-变异什么时候开始失去其作为自然的语音规则的性质,转而演变为固定的形态学规则就有可能解决这个问题。一言以蔽之,就是要确定i-变异什么时候失去了能产性。

一种比较合理的假设是,i-变异在长词干发生中间元音省略后、短词干发生中间元音省略前(即约625-675年之间)失效。

\begin{longtable}{lll}
  \toprule
  阶段                            & 长词干                             & 短词干                         \\
  \midrule
  \endhead
  \bottomrule
  \endfoot
  无省略、i-变异能产              & *gast + *ir \textgreater{} *gestir & *stað + *ir
  \textgreater{} *steðir                                                                                \\
  长词干元音省略                  & *gestir \textgreater{} gestr       & *stað + *ir
  \textgreater{} *steðir                                                                                \\
  \textbf{i-变异(从语音上)失效} & ‑‑‑‑‑‑‑‑‑‑‑‑‑‑‑‑‑‑‑‑‑‑‑‑           &
  ‑‑‑‑‑‑‑‑‑‑‑‑‑‑‑‑‑‑‑‑‑‑‑‑‑                                                                             \\
  短词干元音省略                  & gestr                              & *stað + r \textgreater{} staðr \\
\end{longtable}

在长词干元音省略后,当时的共时系统中出现了一些模糊的现象。首先*steðir这样的词还保留着触发i-变异的元音,但在gestr中,则没有i的痕迹。为了解释gestr,最简单的方式就是认为词干中的元音本身就是e,于是,gestr的词干被重构为gest-而非gast-.
与此伴生的是当时古诺尔斯语的使用者开始混淆i-变异的本质。在短词干中,i-变异的原理仍然是清楚的,但在长词干中却因为词干的重构而找不到运用的场景。于是,i-变异好像成为了个别词类中零散的不规则现象。接着,人们不再认为i-变异是一种语音驱动的变化,而把它当作纯粹形态学的规则。随后,到了短词干的中间音节省略的阶段,*stað
+ r就没有任何条件再发生i-变异了,于是成为了现在看到的没有i-变异的形式。

同样的解释也可以运用到一类弱动词的过去时中,请读者先忘记\ref{第一弱变位法}中介绍基础词干、现在时词干和过去时词干的表格,一类弱动词最本质的特征就是词干元音-i/j-,这个词干元音应该保持在各个时态之中,但它也受元音省略的影响先后在长短词干动词中脱落。于是,动词的情况和i-词干名词高度相似,只是名词触发i-变异的是词尾*ir,动词则是词干元音*-i/j-(或者,也可以理解为过去式的变位词尾,如*iða)。元音省略在名词中造成的结果是*ir
\textgreater{} r从而使得短词干没有维持i-变异的条件;在动词中,则是*iða
\textgreater{} að.

但是,这个解释有两个问题存疑:

1)在长词干名词中,词干gast-被重解为gest-,为什么短词干stað-没有被重构为steð-?即,是什么让当时的人坚持形如*steðir的名词的词干是由stað-变化得来,而非更直接的steð-?i-变异在这个时候还没有完全失效,特别是在短词干名词中还有清楚的触发条件,但这只是解释*steðir的必要不充分条件,一定还有某些词形保留了完好的stað-,提醒人们*steðir中的元音不是其本身的形式。i-词干名词的其他形式(我们之前使用的都是单数主格)在卢恩文中记录很少,但却有记录到单数与格的词尾-ē,复数与格的词尾-umz,这些形式中,应该不存在i-变异,所以应该保留了词干不发生i-变异的形式。\footnote{这带来了一个新问题。按照这个说法,gestr的变格中曾经也有一些词干是gast-的形式,这是否会影响gast-
  \textgreater{}
  gest-的重构?从记录到的古诺尔斯语来看,gest-贯穿了所有词形,一个合理的解释是词干重构是基于比较常见的主格(单数主格和复数主格的词尾都有i,它们的词干应该都变成了gest-)进行的,随后类推到了其他的形式中;同时,词干重构前gest-和gast-的交替可能进一步使人对i-变异的原理感到困惑(这些词干是gest-的词形有的没有触发条件),从而用类推消除了这种会令人费解的现象。从这两点来看,词干重构应当是合理的。}

但是,这个问题在动词中没有办法解释,因为动词的词干元音保留在整个变位表中,例如短词干一类弱动词telja在i-变异失效前的过去式应该都是*tel
+ ið\footnote{请读者暂且不要纠结这里是i还是j,因为i和j的交替不仅受西弗斯定律的影响,还和其所处的音节环境有关。(读者也可以认为西弗斯定律的运用条件并不只有前面词干的轻重这一点)}
+
人称词尾的形式。这样,其所有的词形变化一律都受到了i-变异的影响,人们不太可能从中恢复出*tal-.

一种不太可靠的解释是,一类弱动词很多从强动词或名词/形容词\footnote{从动词派生时,表示使役,如强动词falla
  `fall'派生出弱动词fella `cause to fall,
  fell';从名词/形容词派生时,构成表示与名词/形容词相关的动作。}中派生,例如telja派生自名词tal
`talk,
story',这些词中还保留了未发生元音变异的音。但是一类弱动词词尾-ja/-a在古诺尔斯语中已经没有能产性,它的功能已经被二类弱动词取代。因此,当时的人们应该不太可能把一类弱动词和其对应的名词/强动词联系到一起,更不用说从后者中借鉴古老的词干形式了。另外,古诺尔斯语一类弱动词的过去分词也没有发生i-变异,但从过去分词中恢复词干也过于牵强。

2)i-词干名词中有一部分短词干名词发生了i-变异,例如dynr, hlynr, hylr,
hyrr. 还有一些词记录到了发生和不发生i-变异的两写,如hlumr---hlymr,
Hýnir---Húnir, burðr---byrðr,
fúrr---fýrr等。这些名词的词根元音一般都是u,
但u本应该是不容易发生i-变异的元音,特别是在短词干中,这些有元音变异的形式很难解释。

\textbf{解释2:i-变异具有双向可预测性。}

在传统的观点中,i-变异被描述为``由于受到i的影响,后元音向前移动的音变''。这个表述中隐含的一个论点是:在元音变异过程中,基础的、根本的元音是后元音;受影响的、衍生出的元音是前元音。这在历史语言学中自然是无比正确的铁律,因为例如哥特语这样没有发生元音变异的语言可以证明后元音的确是更古老的形式。但是,对于没有语言学知识的使用者而言,他们能否意识到这点呢?

在i-变异最初发生的时期,后元音的前移是纯粹语音驱动的。当时的古诺尔斯语使用者很可能把受i-变异影响的元音和不受影响的元音当作是同一个音位的变体。这样,i-变异就不再被理解为一个由后元音向前元音的单向音变,而应该表述为:

\begin{longtable}{l}
  \toprule
  在i/j前,a, o, u读作/ɛ/ /ø/ /y/在其他情况下,a, o, u读作/a/ /o/ /u/ \\
  \midrule
  \endhead
  \bottomrule
  \endfoot
\end{longtable}

这样,形如/ɛ/和/a/的一对音素就构成了互补分布。i-变异不再是一种单向的变音,而是一种能随着环境改变的、双向可预测的音变。有一些证据表明,这种观点是可行的:

1)古诺尔斯语早期的字母记法表明人们已经意识到了这些互补分布,而且他们还很可能知道这个分布的应用条件。古诺尔斯语最早的语法文献《第一语法论》(成书不晚于12世纪中叶)中,作者给a,
e, i, o,
u这五个基本的元音添加了四个变体,例如,用专门的字母ę表示/ɛ/,并表示这个字母弯钩来自于a,轮廓来自于e,其发音是这两个字母的结合。这个记法和我们描述的规则完全一致。

2)其他日耳曼语中也有i-变异恢复的例子。在古高地德语和中古高地德语中,当词尾的i或e(i-变异的触发元音)脱落时,i-变异的结果也会恢复。例如OHG中ensti失去-i后还记录到了anst的形式;MHG中krefe失去-e后还记录到了kraft的形式。这些例子表明,i-变异应该具有双向可预测性。

从这个观点出发,i-变异的能产性应该至少保持到了短词干中间音节元音省略发生之后。

\begin{longtable}{llll}
  \toprule
  \multicolumn{2}{c}{阶段} & 长词干         & 短词干                                                      \\
  \midrule
  \endhead
  \bottomrule
  \endfoot
  \multirow{4}{=}{\textbf{i-变异}

  \textbf{始终能产}}       & 无省略         & *gastir

  读作/gɛstir/             & *staðir

  读作/stɛðir/                                                                                            \\
                           & 长词干元音省略 & /gɛstir/ \textgreater{} /gɛstr/ & /stɛðir/                  \\
                           &                & \textbf{词干重建为gest-}        & ‑‑‑‑‑‑‑‑‑‑‑‑‑‑‑‑‑‑‑‑‑‑‑‑‑ \\
                           & 短词干元音省略 & gestr                           & staðr

  读作/staðr/                                                                                             \\
\end{longtable}
